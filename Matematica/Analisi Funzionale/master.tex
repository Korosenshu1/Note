\documentclass[11pt, a4paper]{scrartcl}
% Packages
\usepackage[margin=1.25in]{geometry}
\usepackage{index}
\usepackage{amsbsy} % Bold math symbols
\makeindex
\usepackage[utf8]{inputenc}
\usepackage[T1]{fontenc}
\usepackage{tcolorbox}
\tcbuselibrary{theorems}
\tcbuselibrary{skins}
\tcbuselibrary{breakable}
\usepackage{varwidth}
\usepackage{textcomp}
\usepackage{amsmath, amssymb}
\usepackage{esint}
\usepackage{titlesec}
\usepackage{xcolor}
\usepackage{titling}
\usepackage[linktocpage]{hyperref}
\usepackage{pgfplots}
\usepackage{multicol}
\setlength{\columnsep}{2em}
\usepackage{caption}
\usepackage{amsthm}
\usepackage{import}
\usepackage{cancel}
\usepackage{caption}
\usepackage{nicematrix}
\usepackage{mathrsfs}
\usepackage{mathtools}
%\usepackage{parskip}
\usepackage{pythonhighlight}
\usepackage{enumerate}
\usepackage{graphicx}
\usepackage{tikz}
\usepackage[italian]{babel}
% To reset footnote numbering each page
\usepackage[perpage]{footmisc}
\usepackage{setspace}
\setstretch{1.3}


% Titles 
\title{Note di Analisi Funzionale}
\author{Manuel Deodato}
\date{}


% svolgimento
\newenvironment{svolgimento}{\renewcommand\qedsymbol{$\blacksquare$}\begin{proof}[Svolgimento]}{\end{proof}}


%%%%% tcolorbox setup

% Teorema e proposizione
\newtcbtheorem[number within=section]{teorema}{Teorema}
{breakable, top=0.2mm, bottom=0.2mm, boxrule=0mm,arc =.5 mm, colframe=blue!10, coltitle=black, fonttitle=\bfseries, colback=blue!5!white, theorem style=plain apart}{th}

\newtcbtheorem[number within=section]{prop}{Proposizione}
{breakable, top=0.2mm, bottom=0.2mm, boxrule=0mm,arc =.5 mm, colframe=blue!10, coltitle=black, fonttitle=\bfseries, colback=blue!5!white, theorem style=plain apart}{prop}





% Definizione
\definecolor{greendef}{HTML}{b8d8be}

\newtcbtheorem[number within=section]{definizione}{Definizione}
{breakable, top=0.2mm, bottom=0.2mm, boxrule=0mm, arc=.5mm, colframe=greendef, coltitle=black, fonttitle=\bfseries, theorem style = plain apart, colback=greendef!50!white}{def}


% Esempio
\theoremstyle{definition}
\newtheorem{esempio}{Esempio}

%\definecolor{empurple}{HTML}{6e5e89}

%\newtcbtheorem{esempio}{Esempio}{left=0mm,arc=0mm, colframe=empurple!10!white, coltitle=black, fonttitle=\bfseries, theorem style = plain, colback=empurple!20!white, colframe=empurple!90!white, boxrule=1pt, sharp corners, top=.2mm,bottom=.2mm}{es}

\tcolorboxenvironment{esempio}{blanker,breakable,left=5mm,before skip=10pt,after skip=10pt, borderline west={1mm}{0pt}{greendef}}

\numberwithin{esempio}{section}


% Lemma e Corollario
\definecolor{lemcor}{HTML}{a78d8a}

\newtcbtheorem[number within=section]{lemma}{Lemma}{breakable, top=0.2mm, bottom=0.2mm, boxrule=0mm,left=0mm,arc=.5mm, colframe=lemcor!10!white, coltitle=black, fonttitle=\bfseries, theorem style = plain apart, colframe=lemcor!50!white,colback=lemcor!20!white}{lem}
\newtcbtheorem[number within=section]{corollario}{Corollario}{breakable, top=0.2mm, bottom=0.2mm, boxrule=0mm,left=0mm,arc=.5mm, colframe=lemcor!10!white, coltitle=black, fonttitle=\bfseries, theorem style = plain apart, colframe=lemcor!50!white,colback=lemcor!20!white}{cor}



% Osservazione
\theoremstyle{definition}
\newtheorem{obs}{Osservazione}

\definecolor{coloros}{HTML}{6e5e89}

\tcolorboxenvironment{obs}{blanker,breakable,left=5mm,before skip=10pt,after skip=10pt, borderline west={1mm}{0pt}{coloros}}

\numberwithin{obs}{section}

% Nota
\newtheorem{nota}{Nota}

\definecolor{ncol}{HTML}{f9ebbe}

\tcolorboxenvironment{nota}{blanker,breakable,left=5mm,before skip=10pt,after skip=10pt, borderline west={1mm}{0pt}{ncol}}

\numberwithin{nota}{section}

\newtcolorbox{eqbox}[1][]
{
colback=gray!10,
arc=0pt,
boxrule=0pt,
title=#1
}

 \newenvironment{boxenv}[1][]{
    \begin{eqbox}[#1]
    }{
   \end{eqbox}
}


%%%%%%%%%% Medie con integrali multipli
\def\Yint#1{\mathchoice
    {\YYint\displaystyle\textstyle{#1}}%
    {\YYint\textstyle\scriptstyle{#1}}%
    {\YYint\scriptstyle\scriptscriptstyle{#1}}%
    {\YYint\scriptscriptstyle\scriptscriptstyle{#1}}%
      \!\iint}
\def\YYint#1#2#3{{\setbox0=\hbox{$#1{#2#3}{\iint}$}
    \vcenter{\hbox{$#2#3$}}\kern-.51\wd0}}
\def\longdash{{-}\mkern-3.5mu{-}} 
   % consider using "\mkern-7.5mu" if esint package is loaded
\def\tiltlongdash{\rotatebox[origin=c]{15}{$\longdash$}}
\def\fiint{\Yint\tiltlongdash}

\def\Zint#1{\mathchoice
    {\YYint\displaystyle\textstyle{#1}}%
    {\YYint\textstyle\scriptstyle{#1}}%
    {\YYint\scriptstyle\scriptscriptstyle{#1}}%
    {\YYint\scriptscriptstyle\scriptscriptstyle{#1}}%
      \!\iiint}
      \def\tilongdash{\mkern6mu{-}\mkern-4mu{-}\mkern-5mu{-}} 
   % consider using "\mkern-7.5mu" if esint package is loaded
\def\titiltlongdash{\rotatebox[origin=c]{15}{$\tilongdash$}}
\def\fiiint{\Zint\titiltlongdash}

%Captions
\captionsetup[figure]{font=footnotesize,labelfont=footnotesize}
\captionsetup[table]{font=footnotesize,labelfont=footnotesize}
%Titlesec
\titleformat{\section}
{\fontsize{15}{20}\sffamily\scshape}
{\normalfont\color{gray}{\fontsize{20}{20}\selectfont\thesection}}
{0.7em}
{}
\hypersetup{colorlinks,breaklinks, linkcolor=[RGB]{74, 122, 164}}
\definecolor{asdf}{HTML}{4a7aa4}
% Personalizza la formattazione della subsection
\titleformat{\subsection}[block]{\fontsize{12}{20}\bfseries}{\normalfont\thesubsection}{.5em}{}


% Personalizza la formattazione della subsubsection
\titleformat{\subsubsection}[block]{\fontsize{10}{20}\bfseries}{\normalfont\thesubsubsection}{.5em}{}

% Maketitle customization
\renewcommand{\maketitle}{
\begin{center}
{\sffamily
{\fontsize{20}{20}\selectfont\MakeUppercase\thetitle}}

\vspace{0.2in}

{\large\scshape\sffamily\theauthor}
\end{center}
}

%Evaluate symbol
\DeclareMathOperator{\di}{d\!}
\newcommand*\Eval[3]{\left.#1\right\rvert_{#2}^{#3}}

%%%%%%% Numero delle equazioni in formato a.b
\numberwithin{equation}{subsection}
%%%%%

%%%%%%%%%% Personalizzazione numeri lista
\renewcommand{\theenumi}{(\arabic{enumi})}

%%%% Table of contents

\usepackage[titles]{tocloft}

\renewcommand{\cftdot}{}
\usepackage{titletoc}
%\setcounter{tocdepth}{2}

%%%%%%%%%%%%%%%% Toc style

% Personalizzazione scritta indice


% Font
\usepackage[osf]{newpxtext}
\usepackage{sansiwona}



\begin{document}
\maketitle
\newpage
\tableofcontents 
\newpage
\section{Strutture fondamentali in analisi funzionale}

\subsection{Spazi vettoriali e dimensioni di spazi}

Si ricorda che uno spazio vettoriale $V$ su un certo campo $\mathbb{K}$, come $\mathbb{R}$ o $\mathbb{C}$, ha definite due operazioni:
\[
	\begin{split}
		&+ : V \times V \to V \text{ (addizione tra vettori)}\\
		&\cdot  : \mathbb{K} \times V \to V \text{ (prodotto per uno scalare)}
	\end{split}
\] 
Alcuni campi vettoriali sono $\mathbb{R}, \mathbb{R}^2, \mathbb{R}^n,\mathbb{C}, \mathbb{C}^n$ eccetera, ma anche l'insieme delle funzioni continue in un intervallo:
\begin{equation}
	C\left(\left[ 0,1 \right]\right) = \left\{ f : \left[ 0,1 \right] \to \mathbb{C}  \mid  f \text{ \`e continua} \right\} 
\end{equation}
La differenza tra i primi e quest'ultimo \`e la \textit{dimensione}, che non si riferisce alla cardinalit\`a dell'insieme.
\begin{definizione}
	{Dimensione finita}{}
	Uno spazio vettoriale $V$ \`e detto avere dimensione finita se ogni insieme linearmente indipendente di vettori di $V$ \`e finito in termini di cardinalit\`a.
\end{definizione}
\noindent Uno spazio vettoriale infinito-dimensionale, allora, \`e uno che non ha dimensione finita.
L'insieme $C\left(\left[ 0,1 \right] \right) $ ha dimensione infinita perch\'e 
\begin{equation}
	E= \left\{ f_n(x) = x^n  \mid n \in \mathbb{N}\cup \left\{ 0 \right\} \right\} 
\end{equation}
\`e linearmente indipendente e ha cardinalit\`a infinita.
\subsection{Spazi metrici e spazi normati}


\begin{definizione}
	{Norma e spazio normato}{}
	Una norma su uno spazio vettoriale $V$ \`e una funzione
	\begin{equation}
		\left\lVert \cdot  \right\rVert : V \to [0, +\infty)
	\end{equation}
	che soddisfa:
	\begin{enumerate}[(n1).]
		\item $\left\lVert v \right\rVert =0 \iff v = 0 , \ \forall v\in V$;
		\item $\left\lVert \lambda v \right\rVert = \lvert \lambda  \rvert \left\lVert v \right\rVert $, $\forall \lambda \in \mathbb{K},\ \forall v \in V$;
		\item $\forall v_1,v_2 \in V, \ \left\lVert v_1+v_2 \right\rVert \le \left\lVert v_1 \right\rVert + \left\lVert v_2 \right\rVert $.
	\end{enumerate}
	Uno spazio vettoriale $V$ equipaggiato con $\left\lVert \cdot  \right\rVert $ \`e detto \textit{spazio normato}.
\end{definizione}
\begin{definizione}
	{Semi-norma}{}
	Una semi-norma \`e sempre una funzione $\left\lVert \cdot  \right\rVert : V \to [0,+\infty)$ che soddisfa (n2) e (n3), ma non necessariamente (n1).
\end{definizione}
\begin{definizione}
	{Distanza}{}
	Sia $X$ un insieme; la funzione $d:X \times X \to [0,+\infty)$ \`e detta distanza se soddisfa:
	\begin{enumerate}[(d1).]
		\item $\forall x,y \in X, \ d(x,y) =0 \iff x= y$;
		\item $\forall x,y \in X, \ d(x,y) = d(y,x)$;
		\item $\forall x,y,z \in X, \ d(x,z)\le  d(x,y) + d(y,z)$.
	\end{enumerate}
\end{definizione}
\begin{teorema}
	{Metrica indotta da una norma}{}
	Se $\left\lVert \cdot  \right\rVert $ \`e una norma su uno spazio vettoriale $V$, allora 
	\begin{equation}
		d(v,w) := \left\lVert v-w \right\rVert 
	\end{equation}
	\`e una distanza.
	\begin{proof}
		Evidentemente (n1) $\Leftarrow$ (d1). Da (n2), invece, si nota che
		\[
		\left\lVert v-w \right\rVert = \left\lVert (-1) (w-v) \right\rVert = \lvert -1 \rvert \left\lVert w-v \right\rVert = \left\lVert w-v \right\rVert 
		\] 
		quindi soddisfa (d2). 
		Infine, da (n3)	si ottiene (d3) perch\'e:
		\begin{equation*}
			d(v,w) = \left\lVert v-w \right\rVert = \left\lVert v-w +u -u \right\rVert \le \left\lVert v-u \right\rVert + \left\lVert u-w \right\rVert = d(v,u) + d(u,w)
		\end{equation*}
	\end{proof}
\end{teorema}

Prendendo $\mathbb{R}^n$ o $\mathbb{C}^n$, la \textbf{norma Euclidea} \`e data da:
\begin{boxenv}[]
\begin{equation}
	\left\lVert x \right\rVert _2 = \sqrt{\sum_{i=1}^{n} \lvert x_i \rvert ^2} 
\end{equation}
\end{boxenv}
\noindent Un'altra norma che si pu\`o definire su questi spazi \`e:
\begin{boxenv}[]
\begin{equation}
	\left\lVert x \right\rVert _\infty = \max _{1\le i\le n} \lvert x_i \rvert 
\end{equation}
\end{boxenv}
\noindent Pi\`u in generale, si pu\`o definire un'intera famiglia di norme su questi spazi:
\begin{boxenv}[]
\begin{equation}
	\left\lVert x \right\rVert _p = \left(\sum_{i=1}^{n} \lvert x_i \rvert ^p\right) ^{1 / p} , \ 1\le p < \infty
\end{equation}
\end{boxenv}
\noindent Si pu\`o dimostrare anche che per $x$ fissato, mandando $p \to +\infty$, la norma $\left\lVert x \right\rVert _p \to \left\lVert x \right\rVert _\infty$.
\subsection{Alcuni spazi importanti}


Sia $X$ uno spazio metrico; lo spazio 
\begin{boxenv}[]
\begin{equation}
	C_\infty (X) := \left\{ f: X \to \mathbb{C}  \mid f \text{ continua e limitata} \right\} 
\end{equation}
\end{boxenv}
\noindent Per esempio, $C_\infty\left(\left[ 0,1 \right] \right) = C \left(\left[ 0,1 \right] \right) $; inoltre, vale il seguente.
\begin{teorema}
	{}{}
	$C_\infty(X)$ \`e uno spazio vettoriale e 
	\begin{equation}
		\left\lVert  u\right\rVert 	_\infty = \sup _{x \in X} \lvert u(x) \rvert 
	\end{equation}
	\`e una norma su $C_\infty(X)$.
	\begin{proof}
		\`E facile verificare che sia uno spazio vettoriale; si mostra che $\left\lVert u \right\rVert _\infty$ \`e una norma su $C_\infty(X)$.
		Si nota che (n1) e (n2) sono immediate, mentre per (n3) si pu\`o usare la disuguaglianza triangolare del modulo:
\[
|u(x) + v(x)|\le |u(x)| + |v(x)| \le \left\lVert u \right\rVert _\infty + \left\lVert v \right\rVert _\infty
\] 
per cui
\[
\left\lVert u + v \right\rVert _\infty = \sup_{ x \in X} \lvert u(x) +v (x) \rvert \le \left\lVert u \right\rVert _\infty + \left\lVert v \right\rVert _\infty 
\] 
visto che $\left\lVert u \right\rVert _\infty$ \`e un numero.
	\end{proof}
\end{teorema}
\begin{obs}
	Si nota che $u_n \to u$ in $C_\infty(X)$ \`e equivalente a richiedere che $u_n\to u$ uniformemente perch\'e 
	\[
	\left\lVert u_n - u  \right\rVert _\infty \to 0 , \ n\to +\infty \iff\forall \varepsilon >0, \exists N \in \mathbb{N} : \forall n\ge N, \ \forall x \in X, \ \lvert u_n(x) - u(x) \rvert < \varepsilon 
	\] 
	che \`e la definizione di convergenza uniforme in $X$.
\end{obs}
Ora si considera lo spazio
\begin{boxenv}[]
\begin{equation}
	\ell ^p = \left\{ \left\{ a_j \right\}_{j=1} ^\infty \mid \left\lVert a \right\rVert _p < \infty  \right\} 
\end{equation}
\end{boxenv}
dove la norma \`e:
\begin{equation}
	\begin{split}
		&\left\lVert a \right\rVert _p = \left(\sum_{j=1}^{+\infty} \lvert a_j \rvert ^p\right) ^{1/p}, \ 1\le p< \infty\\
		&\left\lVert a \right\rVert _\infty = \sup_{1\le j\le \infty} \lvert a_j \rvert 
	\end{split}
\end{equation}
Ad esempio, la successione $\left\{ 1 / j \right\} _{j=1} ^{\infty} \in \ell ^p$ per $p>1$, ma non per $p=1$ perch\'e non sarebbe convergente.
\begin{center}
	Riprendere da 48:00
\end{center}















\end{document}
