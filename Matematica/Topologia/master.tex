%! TEX program = lualatex
\documentclass[11pt, a4paper]{scrartcl}
% Packages
\usepackage[margin=1.25in]{geometry}
\usepackage{index}
\usepackage{amsbsy} % Bold math symbols
\makeindex
\usepackage[utf8]{inputenc}
\usepackage[T1]{fontenc}
\usepackage{tcolorbox}
\tcbuselibrary{theorems}
\tcbuselibrary{skins}
\tcbuselibrary{breakable}
\usepackage{varwidth}
\usepackage{textcomp}
\usepackage{amsmath, amssymb}
\usepackage{esint}
\usepackage{titlesec}
\usepackage{xcolor}
\usepackage{titling}
\usepackage[linktocpage]{hyperref}
\usepackage{pgfplots}
\usepackage{multicol}
\setlength{\columnsep}{2em}
\usepackage{caption}
\usepackage{amsthm}
\usepackage{import}
\usepackage{cancel}
\usepackage{caption}
\usepackage{nicematrix}
\usepackage{mathrsfs}
\usepackage{mathtools}
%\usepackage{parskip}
\usepackage{pythonhighlight}
\usepackage{enumerate}
\usepackage{graphicx}
\usepackage{tikz}
\usepackage[italian]{babel}
\usepackage{setspace}
\setstretch{1.2}
% To reset footnote numbering each page
\usepackage[perpage]{footmisc}
\usepackage{faktor}

% Titles 
\title{Appunti di\\ \vspace{.1cm} Geometria}
\author{Manuel Deodato}
\date{}


\definecolor{asdf}{HTML}{4a7aa4}
\definecolor{verde}{HTML}{0D8000}


\newtheoremstyle{style1}% name of the style to be used
{15pt}% measure of space to leave above the theorem. E.g.: 3pt
{15pt}% measure of space to leave below the theorem. E.g.: 3pt
{\normalfont}% name of font to use in the body of the theorem
{15pt}% measure of space to indent
{\color{verde}\sffamily\scshape}% name of head font
{}% punctuation between head and body
{ }% space after theorem head; " " = normal interword space
{\thmname{#1}\thmnumber{ #2}{\thmnote{ (#3)}.\ }}

\theoremstyle{style1}
\newtheorem{osservazione}{Osservazione}[section]
\newtheorem{teorema}{Teorema}[section]
\newtheorem{prop}{Proposizione}[section]
\newtheorem{corollario}{Corollario}[teorema]
\newtheorem{lemma}{Lemma}[teorema]
\newtheorem{definizione}{Definizione}[section]
\newtheorem{notazione}{Notazione}[section]
\newtheorem{esempio}{Esempio}[section]
\newtheorem{esercizio}{Esercizio}[section]

\newenvironment{svolgimento}{\renewcommand\qedsymbol{$\blacksquare$}\begin{proof}[Svolgimento]}{\end{proof}}

%% Generic box
\newtcolorbox{eqbox}[1][]
{
colback=gray!10,
arc=0pt,
boxrule=0pt,
title=#1
}

 \newenvironment{boxenv}[1][]{
    \begin{eqbox}[#1]
    }{
   \end{eqbox}
}



%%%%%%%%%% Medie con integrali multipli
\def\Yint#1{\mathchoice
    {\YYint\displaystyle\textstyle{#1}}%
    {\YYint\textstyle\scriptstyle{#1}}%
    {\YYint\scriptstyle\scriptscriptstyle{#1}}%
    {\YYint\scriptscriptstyle\scriptscriptstyle{#1}}%
      \!\iint}
\def\YYint#1#2#3{{\setbox0=\hbox{$#1{#2#3}{\iint}$}
    \vcenter{\hbox{$#2#3$}}\kern-.51\wd0}}
\def\longdash{{-}\mkern-3.5mu{-}} 
   % consider using "\mkern-7.5mu" if esint package is loaded
\def\tiltlongdash{\rotatebox[origin=c]{15}{$\longdash$}}
\def\fiint{\Yint\tiltlongdash}

\def\Zint#1{\mathchoice
    {\YYint\displaystyle\textstyle{#1}}%
    {\YYint\textstyle\scriptstyle{#1}}%
    {\YYint\scriptstyle\scriptscriptstyle{#1}}%
    {\YYint\scriptscriptstyle\scriptscriptstyle{#1}}%
      \!\iiint}
      \def\tilongdash{\mkern6mu{-}\mkern-4mu{-}\mkern-5mu{-}} 
   % consider using "\mkern-7.5mu" if esint package is loaded
\def\titiltlongdash{\rotatebox[origin=c]{15}{$\tilongdash$}}
\def\fiiint{\Zint\titiltlongdash}

%Captions
\captionsetup[figure]{font=footnotesize,labelfont=footnotesize}
\captionsetup[table]{font=footnotesize,labelfont=footnotesize}
%Titlesec
\titleformat{\section}
{\fontsize{18}{20}\sffamily\scshape}
{\normalfont\color{verde}{\fontsize{18}{20}\selectfont\thesection}}
{0.7em}
{}
\titlespacing*{\section}{0pt}{*2}{1cm}
\titlespacing*{\subsection}{0pt}{*5}{.5cm}
\titlespacing*{\subsubsection}{0pt}{*5}{.5cm}

\hypersetup{colorlinks,breaklinks, linkcolor=[RGB]{13,128,0}}

% Personalizza la formattazione della subsection
\titleformat{\subsection}[block]{\centering\fontsize{14}{20}\bfseries}{\normalfont\thesubsection}{.5em}{}


% Personalizza la formattazione della subsubsection
\titleformat{\subsubsection}[block]{\centering\fontsize{12}{20}\bfseries}{\normalfont\thesubsubsection}{.5em}{}

% Maketitle customization
\renewcommand{\maketitle}{
\begin{center}
{\sffamily
{\fontsize{20}{20}\selectfont\MakeUppercase\thetitle}}

\vspace{0.2in}

{\large\scshape\sffamily\theauthor}
\end{center}
}

%Evaluate symbol
\DeclareMathOperator{\di}{d\!}
\newcommand*\Eval[3]{\left.#1\right\rvert_{#2}^{#3}}

%%%%%%% Numero delle equazioni in formato a.b
\numberwithin{equation}{subsection}
%%%%%

%%%%%%%%%% Personalizzazione numeri lista
\renewcommand{\theenumi}{(\arabic{enumi})}

%%%% Table of contents

\usepackage[titles]{tocloft}

\renewcommand{\cftdot}{}
\usepackage{titletoc}
%\setcounter{tocdepth}{2}

%%%%%%%%%%%%%%%% Toc style

% Personalizzazione scritta indice


% Font
\usepackage{helvet}
\renewcommand{\familydefault}{\sfdefault}	
\renewcommand{\operatorname}[1]{\mathop{\mathrm{\textsf{#1}}}}
\usepackage[noSTIXops,scaled=1.1]{newtxsf}

\begin{document}
\maketitle
\vspace{9cm}
\begin{figure}[h!]
	\centering
	\includegraphics[width=.7\columnwidth]{front1.jpeg}
\end{figure}

\newpage
\tableofcontents 
\newpage
\section{Geometria proiettiva}

\subsection{Introduzione agli spazi proiettivi}

\begin{definizione}
	[Spazio proiettivo]
	Sia $V$ uno spazio vettoriale su $\mathbb{K}$; il suo \textit{spazio proiettivo} \`e dato da:
	\[
		\mathbb{P}( V) = \faktor{V \setminus \left\{ 0 \right\} }{\sim}
	\] 
	dove $v\sim w \iff \exists \lambda \in \mathbb{K} : w = \lambda v$.
\end{definizione}
\noindent Dalla definizione, uno spazio proiettivo collassa tutti i vettori di uno spazio vettoriale che appartengono alla stessa retta in un punto.
In questo senso, $\mathbb{P} (V)$ \`e l'insieme delle rette di $V$.
\begin{esempio}
Si nota che $\mathbb{P} \left(\left\{ 0 \right\} \right) = \varnothing / {\sim} = \varnothing$, mentre per $v\neq 0$, si ha:
\[
	\mathbb{P}\left(\operatorname{Span} v\right) = \faktor{\left\{ \lambda v  \mid \lambda \in \mathbb{K}\setminus 0 \right\} }{\sim} = \left\{ \left[ v \right]  \right\} 
\] 
dove $\left[ v \right] $ rappresenta la classe di equivalenza di $v$; questo significa che lo spazio proiettivo dello span di un elemento \`e composto da un solo punto.
\end{esempio}
\begin{definizione}
	[Dimensione di uno spazio proiettivo]
	La dimensione di uno spazio proiettivo \`e
	\[
	\dim_{\mathbb{K}} \mathbb{P}(V) = \dim_{\mathbb{K}} V - 1
	\] 
\end{definizione}
\noindent Intuitivamente, questa definizione \`e dovuta al fatto che gli spazi proiettivi collassano le rette in punti, abbassando di $1$ la dimensione dello spazio vettoriale.

\begin{definizione}
	[Punti, rette e piani proiettivi]
	Si definisce \textit{punto proiettivo} uno spazio proiettivo di dimensione $0$, \textit{retta proiettiva} uno spazio di dimensione $1$ e \textit{piano proiettivo} uno spazio di dimensione $2$.
\end{definizione}
\begin{definizione}
	[Spazio proiettivo standard]
	Sia $\mathbb{K}$ un campo; si definisce lo \textit{spazio proiettivo standard} come
	\[
	\mathbb{P}(\mathbb{K}^{n+1} ) = \mathbb{P}^n(\mathbb{K}) = \mathbb{K}\mathbb{P}^n
	\] 
\end{definizione}
\subsubsection{Trasformazioni proiettive}
Analogamente al caso dei gruppi e degli anelli, si studiano quelle mappe che preservano la struttura di spazio proiettivo.
\begin{definizione}
	[Trasformazione proiettiva]
	Una mappa $f : \mathbb{P}(V)\to \mathbb{P}(W)$ \`e detta \textit{trasformazione proiettiva} se $\exists \varphi : V \to W$ applicazione lineare tale che
	\[
		f([v]) = [\varphi (v)]
	\] 
	In questa definizione, si dice che $f$ \`e \textit{indotta} da $\varphi $ e, talvolta, si scrive che $f = [\varphi ]$.
\end{definizione}
\begin{prop}
	Se $f$ \`e una trasformazione proiettiva indotta da $\varphi $, allora $\varphi $ \`e iniettiva.
	\begin{proof}
		Per assurdo, $\operatorname{ker} \varphi  \neq \left\{ 0 \right\} $ e sia $v \in \operatorname{ker} \varphi \setminus \left\{ 0 \right\} $; allora $f([v]) = [0]$, ma $[0] \not \in \mathbb{P}(W)$ per definizione di spazio proiettivo, quindi $f$ non sarebbe ben definita.
	\end{proof}
\end{prop}
\begin{prop}
	Ogni applicazione lineare iniettiva $\varphi :V\to W$ induce una trasformazione proiettiva $f: \mathbb{P}(V) \to \mathbb{P}(W)$ tramite l'associazione $[v] \mapsto [\varphi (v)]$.
	\begin{proof}
		Se $v\neq 0$, allora $\varphi (v) \neq 0$ perch\'e $\varphi $ \`e iniettiva. 
		Se, invece, $[v] = [w]$, allora, per definizione, $\exists \lambda  \in \mathbb{K}\setminus \left\{ 0 \right\}$
		\[
			[\varphi (v) ] = [\varphi (\lambda w)] = [\lambda \varphi (w)] = [\varphi (w)]
		\] 
		
	\end{proof}
\end{prop}
\begin{prop}
	Tutte le trasformazioni proiettive sono iniettive.
	\begin{proof}
		Sia $f([v]) = f([w])$ e sia $\varphi $ l'applicazione lineare che induce $f$; allora l'uguaglianza si traduce in $[\varphi (v)]=[\varphi (w)]$, ma per come sono definite queste classi di equivalenza, questo vuol dire che $\varphi (v) = \lambda \varphi (w) = \varphi (\lambda w)$.
		Essendo $\varphi $ iniettiva, per\`o, si ottiene che $v = \lambda w$, cio\`e $[v] = [\lambda w]$.
	\end{proof}
\end{prop}
\begin{prop}
	La trasformazione $\operatorname{id} _{\mathbb{P}(V)}$ \`e proiettiva ed \`e indotta da $\operatorname{id} _V$.
	\begin{proof}
		Tale trasformazione deve essere tale per cui $\operatorname{id} _{\mathbb{P}(V)} ([v]) = [v] = [\operatorname{id} _V v]$, quindi \`e indotta da $\operatorname{id} _V$; essendo quest'ultima iniettiva, anche $\operatorname{id}_{\mathbb{P}(V)}  $ \`e iniettiva.
	\end{proof}
\end{prop}
\begin{prop}
	Siano $f : \mathbb{P}(V) \to \mathbb{P}(W)$ e $g:\mathbb{P}(W) \to \mathbb{P}(Z)$ due trasformazioni proiettive; allora $g\circ f:\mathbb{P}(V)\to \mathbb{P}(Z)$ \`e proiettiva.
	\begin{proof}
		Se $\varphi $ induce $f$ e $\psi $ induce $g$, allora $\psi \circ \varphi $ induce $g \circ f$:
		\[
		\left[ \psi \circ \varphi (v) \right] = g \left(\left[ \varphi (v) \right] \right) = g \circ f \left(\left[ v \right] \right) 
		\] 
	\end{proof}
\end{prop}
\noindent Si passa, ora, a caratterizzare gli isomorfismi di spazi proiettivi; il seguente teorema giustificher\`a la definizione di isomorfismo proiettivo.
\begin{teorema}
	Sia $f:\mathbb{P}(V)\to \mathbb{P}(W)$ una trasformazione proiettiva; allora, le seguenti affermazioni sono tutte equivalenti.
	\begin{enumerate}[(a).]
		\item $f$ \`e suriettiva.
		\item $f$ \`e biettiva.
		\item $\dim \mathbb{P}(V) = \dim\mathbb{P}(W)$.
		\item $f$ \`e invertibile e $f^{-1} : \mathbb{P}(W) \to \mathbb{P}(V)$ \`e proiettiva.
	\end{enumerate}
	\begin{proof}
		Il fatto che (a) $\iff$ (b) \`e dato dal fatto che $f$ \`e proiettiva, quindi \`e iniettiva.

Per mostrare che (b) $\Rightarrow $ (c), si prende $\varphi $ che induce $f$ e si fa vedere che \`e suriettiva.
Visto che $\varphi (0) = 0$, basta mostrare che $W\setminus \left\{ 0 \right\} \subset \operatorname{Im} \varphi $.
Sia, dunque, $w \in W \setminus \left\{ 0 \right\} $, quindi $[w] \in \mathbb{P}(W)$; visto che $f$ \`e suriettiva, $\exists [v] \in \mathbb{P}(V) : f([v]) = [w] = [\varphi (v)]$. 
Allora $w = \lambda \varphi (v) = \varphi (\lambda v) \Rightarrow w \in \operatorname{Im} \varphi $.
Questo significa che $\varphi $ \`e un isomorfismo tra $V$ e $W$, per cui
\[
\dim \mathbb{P}(V) = \dim V - 1 = \dim W -1 = \dim \mathbb{P}(W)
\] 
Ora si mostra che (c) $\Rightarrow $ (d), quindi sia $\varphi $ lineare che induce $f$. 
Si sa, dunque, che $\varphi $ \`e iniettiva e che $\dim \mathbb{P}(V) = \dim \mathbb{P}(W)$, il che implica che $\dim V = \dim W$, pertanto $\varphi $ \`e un isomorfismo; in quanto tale, $\varphi ^{-1} $ \`e ben definita ed \`e ancora un isomorfismo di spazi vettoriali.
Rimane da mostrare che $\varphi ^{-1} $ induce $f$; a questo scopo, si nota che:
\[
\begin{split}
	&[\varphi ^{-1} ] f([v]) = [\varphi ^{-1} ][\varphi (v)] = [\varphi ^{-1} \varphi (v)] = [v]\\
	&f[\varphi ^{-1} ]([v]) = f \left([\varphi ^{-1} (v)]\right) =[\varphi \varphi ^{-1} (v)] = [v]
\end{split}
\] 
Infine, (d) $\Rightarrow $ (a) perch\'e, essendo $f$ invertibile, \`e anche suriettiva.
	\end{proof}
\end{teorema}
\begin{definizione}
	[Isomorfismo proiettivo]
	Una trasformazione proiettiva che sia anche suriettiva \`e detta \textit{isomorfismo proiettivo}.
\end{definizione}
\begin{definizione}
[Proiettivit\`a]
Ogni trasformazione proiettiva $f:\mathbb{P}(V) \to \mathbb{P}(V)$ \`e detta \textit{proiettivit\`a}; si indica con $\mathbb{P}GL(V)$ l'insieme delle proiettivit\`a di $V$.
\end{definizione}
\noindent Da questa definizione, si pu\`o notare che ogni proiettivit\`a \`e un isomorfismo perch\'e, se $f$ \`e indotta da $\varphi $, allora vale la formula della dimensione 
\[
	\dim \ker \varphi + \dim \operatorname{Im} \varphi = \dim V \implies \dim \operatorname{Im} \varphi  = \dim V
\] 
Inoltre, si pu\`o mostrare che equipaggiando $\mathbb{P}GL(V)$ con l'operazione di composizione, questo \`e un gruppo.
\begin{osservazione}
	[Punti fissi]
	Sia $f$ una proiettivit\`a indotta da $\varphi $, con $[v]$ punto fisso, cio\`e
	\[
		[v]  = f([v]) = [\varphi (v)]
	\] 
	Allora $\lambda v = \varphi (v)$, cio\`e $v$ \`e un autovettore di $\varphi $, con autovalore $\lambda $; analogamente, se $v$ \`e un autovettore di $\varphi $, allora $[v]$ \`e un punto fisso per lo stesso motivo.
\end{osservazione}
\subsubsection{Sottospazi proiettivi}
\begin{center}
	Riprendere da pagina 7
\end{center}













\newpage
\section{Spazi metrici, topologici e applicazioni continue}
\subsection{Spazi metrici}
\begin{definizione}
	[Spazio metrico]
Sia $X$ un insieme non vuoto; allora $X$ si dice spazio metrico se pu\`o essere equipaggiato con una \textit{distanza}, ossia una funzione $d : X \times X \to \mathbb{R}$ tale che:
\begin{itemize}
	\item $d(x,x') \ge  0 $ e $d(x,x') = 0 \iff x=x'$;
	\item $d(x,x') = d(x',x)$;
	\item $d(x,x'') \le  d(x,x') + d(x',x'')$.
\end{itemize}
\end{definizione}
\noindent Dato uno spazio metrico $(X,d_X)$ e un insieme $Y \subset  X$, si pu\`o definire un sottospazio di $(X,d_X)$ restringendo la distanza al solo $Y$:
\[
d_Y (y,y') := d_X(y,y'), \ \forall y,y' \in Y
\] 
Quindi $(Y,d_Y)$ \`e a sua volta uno spazio metrico, sottospazio di $(X,d_X)$, il quale \`e detto \textit{spazio ambiente} di $Y$.
\subsubsection{Insiemi aperti}
In uno spazio metrico $(X,d)$, si pu\`o definire un \textit{disco aperto} di raggio $r$ e centro $x$ come
\[
B_r(x) := \left\{ x' \in X  \mid d(x,x') < r \right\} 
\] 
\begin{definizione}
	[Insieme aperto]
	Sia $(X,d)$ uno spazio metrico. Un suo sottoinsieme si dice aperto se \`e generato dall'unione di dischi aperti.
\end{definizione}
\subsubsection{Continuit\`a in spazi metrici}
Una funzione $f:\mathbb{R}\to \mathbb{R}$ si dice continua in $x \in \mathbb{R}$ se $\forall \varepsilon , \ \exists \delta (\varepsilon )$ tale che:
\[
\lvert f(x) - f(x') \rvert < \varepsilon, \ \forall \lvert x-x' \rvert < \delta (\varepsilon )
\] 
\`E possibile generalizzare la definizione a spazi metrici usando la metrica definita su di essi.
\begin{definizione}
	[Continuit\`a in spazi metrici]
Sia $f : X\to Y$ un'applicazione, con $(X,d_X) , \ (Y,d_Y)$ spazi metrici. Si dice che $f$ \`e continua in $x \in X$ se $\forall \varepsilon , \ \exists \delta (\epsilon )$ tale che:
\begin{equation}
	d_Y \big(f(x), f(x')\big) < \varepsilon , \ \forall d_X(x,x')< \delta (\varepsilon )
\end{equation}
\end{definizione}
\noindent Usando la nozione di insieme aperto, \`e possibile generalizzare ulteriormente la definizione di continuit\`a al solo concetto di apertura di un insieme.
\begin{teorema}
Un'applicazione $f:X\to Y$ \`e continua $\iff\forall A \subset  Y$ aperto, l'insieme $f^{-1}(A)  $ \`e aperto.
\begin{proof}
Si dimostrano le due implicazioni.	
\begin{itemize}
	\item $(\Rightarrow )$ Si assume che $f$ sia continua. Si prende $f(x) \in A$, con $A\subset Y$ aperto, per qualche $x \in f^{-1} (A)$. Essendo $A$ aperto $\Rightarrow \exists \varepsilon >0 : B_\varepsilon \big(f(x)\big)\subset A$; allo stesso tempo, per continuit\`a di $f$, dato $\varepsilon $ scelto prima, deve esistere $\delta (\varepsilon )$ tale che
		\[
		f\big(B_{\delta (\varepsilon )}(x)\big) \subset B_\varepsilon \big(f(x)\big)
		\] 
	quindi $B_{\delta (\varepsilon )} (x) \subset f^{-1} (A)$. Valendo $\forall x \in f^{-1} (A)\Rightarrow f^{-1} (A)$ \`e aperto perch\'e per ogni suo elemento, esiste una palla tutta contenuta al suo interno.
\item $(\Leftarrow)$ Si assume che $\forall A \subset Y$ aperto, la funzione $f$ sia tale che l'insieme $f^{-1} (A)$ \`e aperto. Per $f(x) \in Y$, esiste $B_\varepsilon \big(f(x)\big) \subset Y$; essendo questo aperto, deve essere aperto anche $f^{-1} \big[B_\varepsilon \big(f(x)\big) \big]$. 
	Dunque, dato $x \in f^{-1} \big[B_\varepsilon \big(f(x)\big) \big] $, $\exists \delta (\varepsilon ) : B_{\delta (\varepsilon )} (x) \subset f^{-1} \big[B_\varepsilon \big(f(x)\big) \big]$, quindi vuol dire che $f\big(B_{\delta (\varepsilon )} (x)\big)\subset B_\varepsilon \big(f(x)\big)$, ossia:
	\[
	d_Y\big(f(x), f(x')\big) < \varepsilon , \ \forall d_X (x,x') < \delta (\varepsilon )
	\] 
Valendo $\forall x \in X$, allora $f$ \`e continua.
\end{itemize}
\end{proof}
\end{teorema}
\noindent Questo permette di parlare di continuit\`a di applicazioni in insiemi su cui non \`e definita una distanza, ma solo i sottoinsiemi aperti.
\subsubsection{Distanze equivalenti}
\begin{definizione}
	[Distanze topologicamente equivalenti]
	Due distanze $d,\overline{d}$ su $X$ si dicono \textit{topologicamente equivalenti} se hanno gli stessi insiemi aperti, cio\`e se generano la stessa topologia.
\end{definizione}
\noindent Se $d(x,y) = r\overline{d}(x,y)$, per $r>0$, si hanno due distanze equivalenti perch\'e, evidentemente, $\forall \varepsilon >0$:
\[
B_\varepsilon (x) = \overline{B}_{r\varepsilon } (x)
\] 
cio\`e le due distanze $d,\overline{d}$ identificano le stesse palle aperte, quindi gli stessi insiemi aperti.
In $\mathbb{R}^n$, le distanze 
\begin{equation}
	\begin{split}
		&d_2(x,x') = \left\lVert x-x' \right\rVert \equiv \sqrt{\sum_{i=1}^{n} (x_i-x'_i)^2} \\
		&d_1(x,x') = \sum_{i=1}^{n} \lvert x-x_i \rvert \\
		&d_{\infty}(x,x') = \max_{i} \left\{ \lvert x_i-x'_i \rvert  \right\} 
	\end{split}
\end{equation}
sono equivalenti e si ha
\begin{equation}
	d_{\infty} (x,x') \le d_2(x,x') \le d_1 (x,x') \le n d_\infty(x,x')
\end{equation}
\begin{proof}
	La prima disuguaglianza \`e giustificata da:
	\[
	d_2 (x,x') = \sqrt{\sum_{i=1}^{n} (x_i -x'_i)^2}  \ge \sqrt{\max_i  \left\{ (x_i-x'_i)^2 \right\} } = \max_i \left\{ \lvert x-x'_i \rvert  \right\} = d_\infty(x,x')
	\] 
	La seconda, invece, \`e vera perch\'e:
	\[
	\left[ d_2(x,x') \right] ^2 = \sum_{i=1}^{n} (x-x'_i)^2 \le \left[ \sum_{i=1}^{n} \lvert x_i-x'_i \rvert  \right] ^2 = \left[ d_1(x,x') \right]^2 
	\] 
L'ultima disuguaglianza \`e immediata.	
\end{proof}
Da questo segue direttamente che\footnote{Apparentemente, la distanza pi\`u grande dovrebbe includere pi\`u elementi, quindi i simboli $\supset$ dovrebbero essere dei $\subset$, invece, avendo fissato il raggio $\varepsilon $, quella che permette di creare la palla pi\`u grande \`e la distanza pi\`u piccola perch\'e \textit{avvicina} i punti tra di loro, quindi pi\`u elementi rientreranno in tale raggio.}
\begin{equation}
	B^{(\infty)} _\varepsilon (x) \supset B^{(2)} _\varepsilon (x) \supset B^{(1)} _{\varepsilon } (x) \supset B^{(\infty)} _{\varepsilon  / n} (x)
\end{equation}
Questo mostra che se $A$ \`e aperto rispetto ad una distanza, lo \`e anche rispetto alle altre.
\subsubsection{Alcuni risultati sulla continuit\`a}
\begin{prop}
Siano $(X,d_X), \ (Y,d_Y)$ due spazi metrici e $f:X\to Y$ un'applicazione. Dato $x \in X$, se esiste costante $M>0$ tale che
\[
d_Y\big(f(x'), f(x)\big) \le Md_X(x',x), \ \forall x ' \in X
\] 
allora $f$ \`e continua in $x$.
\begin{proof}
	Segue direttamente dal fatto che, per ipotesi, definendo $\delta (\varepsilon ) = \varepsilon / M$, si ha $f\big(D_{\delta (\varepsilon )} (x)\big) \subset D_\varepsilon \big(f(x)\big)$.
\end{proof}
\end{prop}
\begin{prop}
	Ogni applicazione lineare $L : \mathbb{R}^n \to \mathbb{R}^m$ \`e continua rispetto alle distanze euclidee.
	\begin{proof}
		Si usa Prop. \ref{prop:c11} applicato alle distanze $d^{(1)} $, che sono topologicamente equivalenti alle distanze euclidee $d^{(2)} $. Inoltre, visto che ogni applicazione costante \`e continua, si esclude che $L$ sia nulla. Si denota con $(a_{ij} )_{1\le i\le m, \ 1\le j\le n} $ la matrice che rappresenta $L$; se $x, x' \in \mathbb{R}^n$ si ha:
		\[
			\begin{split}
				d^{(1)} \big(L(x) , L(x')\big)&= \left\lvert \sum_{j=1}^{n} a_{1j} (x_j-x'_j) \right\rvert + \ldots+ \left\lvert  \sum_{j=1}^{n} a_{mj} (x_j - x'_j) \right\rvert \\
						       &\le \left(\max_j \lvert a_{1j} \rvert+ \ldots +\max_j \lvert a_{mj}  \rvert   \right) \sum_{j=1}^{n} \lvert x_j - x'_j \rvert \le  Mm d^{(1)} (x,x')
			\end{split}
		\] 
		con $M=\max \lvert a_{ij}  \rvert$, che \`e maggiore di $0$ perch\'e $L$ \`e non-nulla. Da Prop. $\ref{prop:c11}$, segue la tesi.
	\end{proof}
\end{prop}
\noindent La precedente proposizione pu\`o essere applicata al caso particolare di applicazioni lineari: le \textbf{proiezioni}. Una proiezione \`e generalmente definita come:
\begin{equation}
	p_i:\mathbb{R}^n \to \mathbb{R}, \ p_i(x) = x_i
\end{equation}
\`E possibile definire, pi\`u in generale, per $1\le i_1 < i_2< \ldots<i_m <n$, la proiezione
\begin{equation}
	p_{i_1,\ldots,i_m}:\mathbb{R}^n \to \mathbb{R}^m, \  p_{i_1,\ldots,i_m} (x) = (x_{i_1} ,\ldots,x_{i_m} )
\end{equation}
che \`e lineare e, quindi, continua.
\subsubsection{Isometrie e omeomorfismi}
\begin{definizione}
	[Isometria]
	Dati $X,Y$ spazi metrici, un'applicazione biettiva $f:X\to Y$ \`e un'isometria se $\forall x,x' \in X$, si ha $d_X(x,x') = d_Y\big(f(x),f(x')\big)$.
\end{definizione}
\noindent Da Prop \ref{prop:c11}, segue che un'isometria \`e un'applicazione continua. Se fra due spazi metrici $X,Y$ esiste un'isometria $f:X\to Y$, gli spazi si dicono \textbf{isometrici}. 

Sono isometrie $\operatorname{Id} : X \to X$, cio\`e l'applicazione identit\`a, l'inversa di un'isometria e la composizione di isometrie. Questo porta al seguente.
\begin{prop}
	Un'isometria fra due spazi metrici \`e una relazione di equivalenza.	
\end{prop}
\begin{definizione}
	[Omeomorfismo]
	Dati $X,Y$ spazi metrici, un'applicazione biettiva $f:X\to Y$ \`e un \textit{omeomorfismo} se la sua inversa e $f$ stessa sono continue.
\end{definizione}
\noindent Ne segue che ogni isometria \`e un omeomorfismo, ma non \`e vero il viceversa. Per esempio, definendo $e^x : \mathbb{R} \to (0,+\infty)$, questa ha un'inversa continua $\log(x) : (0,+\infty) \to \mathbb{R}$, quindi \`e un omeomorfismo, ma non \`e un'isometria perch\'e manda $(-\infty,0]$ in $(0,1]$.
Anche gli omemorfismi definiscono una \textbf{relazione di equivalenza} tra spazi metrici.
\subsection{Spazi topologici}
\begin{definizione}
	[Topologia e spazio topologico]
	Sia $X$ un insieme non-vuoto. Una \textit{topologia} su $X$ \`e una famiglia non-vuota $\tau $ di sottoinsiemi di $X$, chiamati \textit{insiemi aperti della topologia}. Questi soddisfano le seguenti condizioni:
	\begin{itemize}
		\item $\varnothing, \ X$ sono aperti;
		\item l'unione di una qualsiasi famiglia di insiemi aperti \`e un insieme aperto;
		\item l'intersezione di due insiemi aperti \`e un aperto.
	\end{itemize}
	Allora si definisce \textit{spazio topologico} la coppia $(X,\tau )$, dove $X$ \`e detto \textit{supporto} dello spazio topologico e i suoi elementi sono i \textit{punti} dello spazio.
\end{definizione}
\noindent Dato $(X,d)$ spazio metrico, la famiglia degli insiemi aperti rispetto a $d$ \`e una topologia su $X$ indotta da $d$ stessa.
In $\mathbb{R}^n$, si definisce \textbf{topologia euclidea} (o \textbf{naturale}) $\mathcal{E}$ come quella indotta dalla distanza euclidea $d_2$. Su $\mathbb{C}$, la topologia euclidea $\mathcal{E}$ \`e quella indotta da $d(z,w) = \lvert z-w \rvert $; 
questa conclusione si pu\`o ottenere identificando $\mathbb{C}$ con $\mathbb{R}^2$ da $z=x+iy \mapsto (x,y)$ e considerando la distanza euclidea di $\mathbb{R}^2$. 
In modo del tutto analogo, si identifica $\mathbb{C}^n$ con $\mathbb{R}^{2n} $ e la distanza euclidea di $\mathbb{R}^{2n} $ definisce, su $\mathbb{C}^n$, una distanza e, quindi, una topologia che \`e la topologia naturale di $\mathbb{C}^n$, $\mathcal{E}$.
Su un qualunque insieme non-vuoto $X$, si possono sempre definire due topologie:
\begin{itemize}
	\item la \textbf{topologia banale} $\mathcal{B}=\left\{ X, \varnothing \right\} $, con $(X,\mathcal{B})$ \textbf{spazio topologico banale};
	\item la \textbf{topologia discreta} ottenuta prendendo $\tau  = \mathcal{P}(X)$, con $(X,\mathcal{P}(X))$ \textbf{spazio topologico discreto}.
\end{itemize}
\begin{definizione}
	[Spazio metrizzabile]
	Uno spazio topologico $(X,\tau )$ \`e detto \textit{metrizzabile} se si pu\`o definire una distanza su $X$ che induce la topologia $\tau $.
\end{definizione}
\noindent Sia dato $Y$ sottoinsieme non-vuoto di uno spazio metrizzabile $(X,d_X)$; si sa gi\`a che $d_Y$, ottenuta come restrizione di $d_X$ a $Y$, \`e una distanza su $Y$.
In questo caso, la topologia indotta da $d_Y$ su $Y$ si dice \textit{topologia indotta da $X$ su $Y$}. 
Allora, se $y \in Y$: $B_\varepsilon ^{(Y)} (y) = B_\varepsilon ^{(X)} (y) \cap Y$; questo significa che gli aperti di $Y$ sono della forma $A \cap Y$, con $A $ aperto di $X$.





















\end{document}
