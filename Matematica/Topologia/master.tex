%! TEX program = lualatex
\documentclass[12pt]{scrartcl}
% Packages
%\usepackage[margin=1.5in]{geometry}
\usepackage{index}
\usepackage{amsbsy} % Bold math symbols
\makeindex
%\usepackage[utf8]{inputenc}
\usepackage[T1]{fontenc}
\usepackage{tcolorbox}
\tcbuselibrary{theorems}
\tcbuselibrary{skins}
\tcbuselibrary{breakable}
\usepackage{varwidth}
\usepackage{textcomp}
\usepackage{amsmath, amssymb}
\usepackage{esint}
\usepackage{titlesec}
\usepackage{xcolor}
\usepackage{titling}
\usepackage[linktocpage]{hyperref}
\usepackage{pgfplots}
\usepackage{multicol}
\setlength{\columnsep}{2em}
\usepackage{caption}
\usepackage{amsthm}
\usepackage{import}
\usepackage{cancel}
\usepackage{caption}
\usepackage{nicematrix}
\usepackage{mathrsfs}
\usepackage{mathtools}
%\usepackage{parskip}
\usepackage{enumerate}
\usepackage{graphicx}
\usepackage[italian]{babel}
\usepackage{setspace}
\setstretch{1.2}
% To reset footnote numbering each page
\usepackage[perpage]{footmisc}
\usepackage{faktor}
\usepackage{tikz-cd}
\definecolor{mastercolor}{HTML}{92a8d1}
\definecolor{nred}{HTML}{bf0040}


% Titles 
\title{Appunti di\\ \vspace{.3cm} Geometria proiettiva e topologia}
\author{Manuel Deodato}
\date{}




\newtheoremstyle{style}% name of the style to be used
{5pt}% measure of space to leave above the theorem. E.g.: 3pt
{5pt}% measure of space to leave below the theorem. E.g.: 3pt
{\normalfont}% name of font to use in the body of the theorem
%{15pt}% measure of space to indent
{0pt}% measure of space to indent
{\noindent\sffamily\scshape\bfseries}% name of head font
{}% punctuation between head and body
{ }% space after theorem head; " " = normal interword space
{\thmname{#1}\thmnumber{ #2}{\thmnote{ (#3)}.\ }}


\theoremstyle{style}
\newtheorem{esempio}{Esempio}[section]
\newtheorem{definizione}{Definizione}[section]
\newtheorem{prop}{Proposizione}[section]
\newtheorem{teorema}{Teorema}[section]
\newtheorem{lemma}{Lemma}[teorema]
\newtheorem{corollario}{Corollario}[teorema]
\newtheorem{osservazione}{Osservazione}[section]
\newtheorem{notazione}{Notazione}[section]
\newtheorem{esercizio}{Esercizio}[section]





\tcolorboxenvironment{definizione}{blanker,breakable,left=5mm,before skip=10pt,after skip=10pt, borderline west={.5mm}{0pt}{mastercolor}, before upper={\setlength{\parindent}{15pt}}}
\tcolorboxenvironment{lemma}{blanker,breakable,left=5mm,before skip=10pt,after skip=10pt, borderline west={.5mm}{0pt}{mastercolor}, before upper={\setlength{\parindent}{15pt}}}
\tcolorboxenvironment{teorema}{enhanced,blanker,breakable,left=5mm,before skip=10pt,after skip=10pt, borderline west={.5mm}{0pt}{mastercolor}, before upper={\setlength{\parindent}{15pt}}}
\tcolorboxenvironment{corollario}{blanker,breakable,left=5mm,before skip=10pt,after skip=10pt, borderline west={.5mm}{0pt}{mastercolor}, before upper={\setlength{\parindent}{15pt}}}
\tcolorboxenvironment{prop}{blanker,breakable,left=5mm,before skip=10pt,after skip=10pt, borderline west={.5mm}{0pt}{mastercolor}, before upper={\setlength{\parindent}{15pt}}}
\tcolorboxenvironment{esempio}{blanker,breakable,left=5mm,before skip=10pt,after skip=10pt, borderline west={.5mm}{0pt}{mastercolor}, before upper={\setlength{\parindent}{15pt}}}
\tcolorboxenvironment{esercizio}{blanker,breakable,left=5mm,before skip=10pt,after skip=10pt, borderline west={.5mm}{0pt}{mastercolor}, before upper={\setlength{\parindent}{15pt}}}
\tcolorboxenvironment{osservazione}{blanker,breakable,left=5mm,before skip=10pt,after skip=10pt, borderline west={.5mm}{0pt}{mastercolor}, before upper={\setlength{\parindent}{15pt}}}


\newenvironment{svolgimento}{\renewcommand\qedsymbol{$\blacksquare$}\begin{proof}[Svolgimento]}{\end{proof}}




%% Generic box
\newtcolorbox{eqbox}[1][]
{
colback=gray!10,
arc=0pt,
boxrule=0pt,
title=#1
}

 \newenvironment{boxenv}[1][]{
    \begin{eqbox}[#1]
    }{
   \end{eqbox}
}



%Captions
\captionsetup[figure]{font=footnotesize,labelfont=footnotesize}
\captionsetup[table]{font=footnotesize,labelfont=footnotesize}
%Titlesec
\titleformat{\section}
{\fontsize{20}{20}\scshape}
{\color{gray}{\fontsize{50}{20}\selectfont\thesection\hspace{.2cm}\color{gray}{\vrule width 1pt}}}
{0.7em}
{}
\titlespacing*{\section}{0pt}{*2}{1cm}
\titlespacing*{\subsection}{0pt}{*5}{.5cm}
\titlespacing*{\subsubsection}{0pt}{*5}{.5cm}

\hypersetup{colorlinks,breaklinks, linkcolor=[RGB]{146,168,209}}

% Personalizza la formattazione della subsection
\titleformat{\subsection}[block]{\fontsize{15}{20}\bfseries}{\normalfont\thesubsection}{.5em}{}


% Personalizza la formattazione della subsubsection
\titleformat{\subsubsection}[block]{\fontsize{14}{20}\bfseries}{\normalfont\thesubsubsection}{.5em}{}

% Maketitle customization
\renewcommand{\maketitle}{
\begin{center}
{\sffamily
{\fontsize{20}{20}\selectfont\MakeUppercase\thetitle}}

\vspace{0.2in}

{\large\scshape\theauthor}
\end{center}
}

%Evaluate symbol
\DeclareMathOperator{\di}{d\!}
\newcommand*\Eval[3]{\left.#1\right\rvert_{#2}^{#3}}

%%%%%%% Numero delle equazioni in formato a.b
\numberwithin{equation}{subsection}
%%%%%

%%%%%%%%%% Personalizzazione numeri lista
\renewcommand{\theenumi}{(\arabic{enumi})}

%%%% Table of contents

\usepackage[titles]{tocloft}

\renewcommand{\cftdot}{}
\usepackage{titletoc}
%\setcounter{tocdepth}{2}

%%%%%%%%%%%%%%%% Toc style

% Personalizzazione scritta indice


% Font
\renewcommand{\textbf}[1]{\textsf{\bfseries #1}}
\usepackage{fontspec}
\usepackage{unicode-math}
\usepackage[default]{fontsetup}


%%% Hook
\newcommand{\longhookrightarrow}{\lhook\joinrel\longrightarrow}


\begin{document}
\maketitle
\vspace{6cm}
\begin{figure}[h!]
	\centering
	\includegraphics[width=.7\columnwidth]{front1.jpeg}
\end{figure}

\newpage
\tableofcontents 
\newpage
\section{Geometria proiettiva}

\subsection{Introduzione agli spazi proiettivi}

\begin{definizione}
	[Spazio proiettivo]
	Sia $V$ uno spazio vettoriale su $\mathbb{K}$; il suo \textit{spazio proiettivo} \`e dato da:
	\[
		\mathbb{P}( V) = \faktor{V \setminus \left\{ 0 \right\} }{\sim}
	\] 
	dove $v\sim w \iff \exists \lambda \in \mathbb{K} : w = \lambda v$.
\end{definizione}
\noindent Dalla definizione, uno spazio proiettivo collassa tutti i vettori di uno spazio vettoriale che appartengono alla stessa retta in un punto.
In questo senso, $\mathbb{P} (V)$ \`e l'insieme delle rette di $V$.
\begin{esempio}
Si nota che $\mathbb{P} \left(\left\{ 0 \right\} \right) = \varnothing / {\sim} = \varnothing$, mentre per $v\neq 0$, si ha:
\[
	\mathbb{P}\left(\operatorname{Span} v\right) = \faktor{\left\{ \lambda v  \mid \lambda \in \mathbb{K}\setminus 0 \right\} }{\sim} = \left\{ \left[ v \right]  \right\} 
\] 
dove $\left[ v \right] $ rappresenta la classe di equivalenza di $v$; questo significa che lo spazio proiettivo dello span di un elemento \`e composto da un solo punto.
\end{esempio}
\begin{definizione}
	[Dimensione di uno spazio proiettivo]
	La dimensione di uno spazio proiettivo \`e
	\[
	\dim_{\mathbb{K}} \mathbb{P}(V) = \dim_{\mathbb{K}} V - 1
	\] 
\end{definizione}
\noindent Intuitivamente, questa definizione \`e dovuta al fatto che gli spazi proiettivi collassano le rette in punti, abbassando di $1$ la dimensione dello spazio vettoriale.

\begin{definizione}
	[Punti, rette e piani proiettivi]
	Si definisce \textit{punto proiettivo} uno spazio proiettivo di dimensione $0$, \textit{retta proiettiva} uno spazio di dimensione $1$ e \textit{piano proiettivo} uno spazio di dimensione $2$.
\end{definizione}
\begin{definizione}
	[Spazio proiettivo standard]
	Sia $\mathbb{K}$ un campo; si definisce lo \textit{spazio proiettivo standard} come
	\[
	\mathbb{P}(\mathbb{K}^{n+1} ) = \mathbb{P}^n(\mathbb{K}) = \mathbb{K}\mathbb{P}^n
	\] 
\end{definizione}
\subsubsection{Trasformazioni proiettive}
Analogamente al caso dei gruppi e degli anelli, si studiano quelle mappe che preservano la struttura di spazio proiettivo.
\begin{definizione}
	[Trasformazione proiettiva]
	Una mappa $f : \mathbb{P}(V)\to \mathbb{P}(W)$ \`e detta \textit{trasformazione proiettiva} se $\exists \varphi : V \to W$ applicazione lineare tale che
	\[
		f([v]) = [\varphi (v)]
	\] 
	In questa definizione, si dice che $f$ \`e \textit{indotta} da $\varphi $ e, talvolta, si scrive che $f = [\varphi ]$.
\end{definizione}
\begin{prop}
	Se $f$ \`e una trasformazione proiettiva indotta da $\varphi $, allora $\varphi $ \`e iniettiva.
\end{prop}
	\begin{proof}
		Per assurdo, $\operatorname{ker} \varphi  \neq \left\{ 0 \right\} $ e sia $v \in \operatorname{ker} \varphi \setminus \left\{ 0 \right\} $; allora $f([v]) = [0]$, ma $[0] \not \in \mathbb{P}(W)$ per definizione di spazio proiettivo, quindi $f$ non sarebbe ben definita.
	\end{proof}
\begin{prop}
	Ogni applicazione lineare iniettiva $\varphi :V\to W$ induce una trasformazione proiettiva $f: \mathbb{P}(V) \to \mathbb{P}(W)$ tramite l'associazione $[v] \mapsto [\varphi (v)]$.
\end{prop}
	\begin{proof}
		Se $v\neq 0$, allora $\varphi (v) \neq 0$ perch\'e $\varphi $ \`e iniettiva. 
		Se, invece, $[v] = [w]$, allora, per definizione, $\exists \lambda  \in \mathbb{K}\setminus \left\{ 0 \right\}$
		\[
			[\varphi (v) ] = [\varphi (\lambda w)] = [\lambda \varphi (w)] = [\varphi (w)]
		\] 
		
	\end{proof}
\begin{prop}
	Tutte le trasformazioni proiettive sono iniettive.
\end{prop}
	\begin{proof}
		Sia $f([v]) = f([w])$ e sia $\varphi $ l'applicazione lineare che induce $f$; allora l'uguaglianza si traduce in $[\varphi (v)]=[\varphi (w)]$, ma per come sono definite queste classi di equivalenza, questo vuol dire che $\varphi (v) = \lambda \varphi (w) = \varphi (\lambda w)$.
		Essendo $\varphi $ iniettiva, per\`o, si ottiene che $v = \lambda w$, cio\`e $[v] = [\lambda w]$.
	\end{proof}
\begin{prop}
	La trasformazione $\operatorname{Id} _{\mathbb{P}(V)}$ \`e proiettiva ed \`e indotta da $\operatorname{Id} _V$.
\end{prop}
	\begin{proof}
		Tale trasformazione deve essere tale per cui $\operatorname{Id} _{\mathbb{P}(V)} ([v]) = [v] = [\operatorname{Id} _V v]$, quindi \`e indotta da $\operatorname{Id} _V$; essendo quest'ultima iniettiva, anche $\operatorname{Id}_{\mathbb{P}(V)}  $ \`e iniettiva.
	\end{proof}
\begin{prop}
	Siano $f : \mathbb{P}(V) \to \mathbb{P}(W)$ e $g:\mathbb{P}(W) \to \mathbb{P}(Z)$ due trasformazioni proiettive; allora $g\circ f:\mathbb{P}(V)\to \mathbb{P}(Z)$ \`e proiettiva.
\end{prop}
	\begin{proof}
		Se $\varphi $ induce $f$ e $\psi $ induce $g$, allora $\psi \circ \varphi $ induce $g \circ f$:
		\[
		\left[ \psi \circ \varphi (v) \right] = g \left(\left[ \varphi (v) \right] \right) = g \circ f \left(\left[ v \right] \right) 
		\] 
	\end{proof}
\noindent Si passa, ora, a caratterizzare gli isomorfismi di spazi proiettivi; il seguente teorema giustificher\`a la definizione di isomorfismo proiettivo.
\begin{teorema}
	Sia $f:\mathbb{P}(V)\to \mathbb{P}(W)$ una trasformazione proiettiva; allora, le seguenti affermazioni sono tutte equivalenti.
	\begin{enumerate}[(a).]
		\item $f$ \`e suriettiva.
		\item $f$ \`e biettiva.
		\item $\dim \mathbb{P}(V) = \dim\mathbb{P}(W)$.
		\item $f$ \`e invertibile e $f^{-1} : \mathbb{P}(W) \to \mathbb{P}(V)$ \`e proiettiva.
	\end{enumerate}
\end{teorema}
	\begin{proof}
		Il fatto che (a) $\iff$ (b) \`e dato dal fatto che $f$ \`e proiettiva, quindi \`e iniettiva.

Per mostrare che (b) $\Rightarrow $ (c), si prende $\varphi $ che induce $f$ e si fa vedere che \`e suriettiva.
Visto che $\varphi (0) = 0$, basta mostrare che $W\setminus \left\{ 0 \right\} \subset \operatorname{Im} \varphi $.
Sia, dunque, $w \in W \setminus \left\{ 0 \right\} $, quindi $[w] \in \mathbb{P}(W)$; visto che $f$ \`e suriettiva, $\exists [v] \in \mathbb{P}(V) : f([v]) = [w] = [\varphi (v)]$. 
Allora $w = \lambda \varphi (v) = \varphi (\lambda v) \Rightarrow w \in \operatorname{Im} \varphi $.
Questo significa che $\varphi $ \`e un isomorfismo tra $V$ e $W$, per cui
\[
\dim \mathbb{P}(V) = \dim V - 1 = \dim W -1 = \dim \mathbb{P}(W)
\] 
Ora si mostra che (c) $\Rightarrow $ (d), quindi sia $\varphi $ lineare che induce $f$. 
Si sa, dunque, che $\varphi $ \`e iniettiva e che $\dim \mathbb{P}(V) = \dim \mathbb{P}(W)$, il che implica che $\dim V = \dim W$, pertanto $\varphi $ \`e un isomorfismo; in quanto tale, $\varphi ^{-1} $ \`e ben definita ed \`e ancora un isomorfismo di spazi vettoriali.
Rimane da mostrare che $\varphi ^{-1} $ induce $f$; a questo scopo, si nota che:
\[
\begin{split}
	&[\varphi ^{-1} ] f([v]) = [\varphi ^{-1} ][\varphi (v)] = [\varphi ^{-1} \varphi (v)] = [v]\\
	&f[\varphi ^{-1} ]([v]) = f \left([\varphi ^{-1} (v)]\right) =[\varphi \varphi ^{-1} (v)] = [v]
\end{split}
\] 
Infine, (d) $\Rightarrow $ (a) perch\'e, essendo $f$ invertibile, \`e anche suriettiva.
	\end{proof}
\begin{definizione}
	[Isomorfismo proiettivo]
	Una trasformazione proiettiva che sia anche suriettiva \`e detta \textit{isomorfismo proiettivo}.
\end{definizione}
\begin{definizione}
[Proiettivit\`a]
Ogni trasformazione proiettiva $f:\mathbb{P}(V) \to \mathbb{P}(V)$ \`e detta \textit{proiettivit\`a}; si indica con $\mathbb{P}\mathrm{GL} (V)$ l'insieme delle proiettivit\`a di $V$.
\end{definizione}
\noindent Da questa definizione, si pu\`o notare che ogni proiettivit\`a \`e un isomorfismo perch\'e, se $f$ \`e indotta da $\varphi $, allora vale la formula della dimensione 
\[
	\dim \ker \varphi + \dim \operatorname{Im} \varphi = \dim V \implies \dim \operatorname{Im} \varphi  = \dim V
\] 
Inoltre, si pu\`o mostrare che equipaggiando $\mathbb{P}\mathrm{GL} (V)$ con l'operazione di composizione, questo \`e un gruppo.
\begin{osservazione}
	[Punti fissi]
	Sia $f$ una proiettivit\`a indotta da $\varphi $, con $[v]$ punto fisso, cio\`e
	\[
		[v]  = f([v]) = [\varphi (v)]
	\] 
	Allora $\lambda v = \varphi (v)$, cio\`e $v$ \`e un autovettore di $\varphi $, con autovalore $\lambda $; analogamente, se $v$ \`e un autovettore di $\varphi $, allora $[v]$ \`e un punto fisso per lo stesso motivo.
\end{osservazione}
\subsubsection{Sottospazi proiettivi}
Per semplicit\`a di notazione, si introduce la proiezione
\begin{equation}
	\pi : V \setminus \left\{ 0 \right\} \to \mathbb{P}(V)
\end{equation}
che manda $V \setminus \left\{ 0 \right\} $ sul suo quoziente con la relazione di equivalenza.
\begin{definizione}
	[Grassmanniana]
	Sia $V$ uno spazio vettoriale tale che $\dim V = n$ e sia $k \in \left\{ 0,\ldots,n \right\}$; allora la \textit{grassmanniana} $k$ di $V$ \`e l'insieme di tutti i sottospazi di $V$ di dimensione $k$:
	\[
	\mathrm{Gr} _k (V) = \left\{ W \subseteq V  \mid W \text{ spazio vettoriale con } \dim W = k \right\} 
	\] 
Si user\`a, inoltre, la seguente notazione:
\[
\mathrm{Gr}  (k,n) = \mathrm{Gr}  _{k} (\mathbb{K}^{n} )
\] 
\end{definizione}

\begin{definizione}
	[Sottospazio proiettivo]
	Sia $V$ uno spazio vettoriale; un \textit{sottospazio proiettivo} S di $\mathbb{P}(V)$ \`e un sottoinsieme di $\mathbb{P}(V)$ tale che
	\[
	S = \pi (H \setminus \left\{ 0 \right\} )
	\] 
	per qualche $H$ sottospazio vettoriale di $V$.
\end{definizione}
\begin{osservazione}
Dalla definizione, si deduce che uno sottospazio proiettivo \`e esso stesso uno spazio proiettivo, cio\`e $\pi(H \setminus \left\{ 0 \right\} ) = \mathbb{P}(H)$.
\end{osservazione}
\begin{definizione}
	[Iperpiano proiettivo]
	Un iperpiano di $\mathbb{P}(V)$ \`e un sottospazio proiettivo di $\mathbb{P}(V)$ tale che 
	\[
	\dim S = \dim \mathbb{P}(V) - 1
	\] 
\end{definizione}
\begin{teorema}
	Siano $V$ uno spazio vettoriale e $H$ un suo sottospazio.
	Sia $S=\pi(H\setminus\left\{ 0 \right\} ) = \mathbb{P}(H)$ un sottospazio proiettivo di $\mathbb{P}(V)$; allora $\pi^{-1} (S) = H \setminus \left\{ 0 \right\} $.
	In sostanza, si ha una biezione tra i sottospazi vettoriali di $V$ e i sottospazi proiettivi di $\mathbb{P}(V)$.
\end{teorema}
	\begin{proof}
		Si nota che
		\[
			\pi^{-1} (S) = \bigcup_{[v] \in S} \pi^{-1} ([v]) = \bigcup _{v \in H \setminus\left\{ 0 \right\} } \pi^{-1} ([v])
		\] 
		Si dimostrer\`a il teorema per doppia inclusione.
		Avendo $\pi(v) = [v]$, allora $v \in \pi^{-1} ([v])$, perci\`o
		\[
			\bigcup_{v \in H \setminus\left\{ 0 \right\} } \pi^{-1} ([v]) \supseteq H \setminus \left\{ 0 \right\} 
		\] 
		Si nota anche che
		\[
			\pi^{-1} ([v]) = \big\{ w   \mid [w] = [v] \big\} = \big\{ \lambda v  \mid  \lambda \in \mathbb{K}\setminus\left\{ 0 \right\}  \big\} 
		\] 
		pertanto $\forall w \in \pi^{-1} ([v])$, si ha $w = \lambda v \in H \setminus\left\{ 0 \right\} $, da cui segue che
		\[
			\bigcup_{v \in H \setminus\left\{ 0 \right\} } \pi^{-1} ([v]) \subseteq H \setminus\left\{ 0 \right\} 
		\] 
	\end{proof}
\noindent Il teorema appena mostrato permette di concludere che 
\[
\dim \mathbb{P}(H) = \dim H - 1 \iff \dim \pi^{-1} (S) \cup \left\{ 0 \right\}  = \dim S + 1
\] 
Dal punto di vista delle grassmanniane, si ha che
\[
\mathrm{Gr} _k (\mathbb{P}(V)) \cong \mathrm{Gr} _{k+1} (V)
\] 
dove le grassmanniane di uno spazio proiettivo sono ottenute tramite la definizione di dimensione per uno spazio proiettivo.

Si passa, ora, allo studio di somme e intersezioni di sottospazi proiettivi; in particolare, si ha il seguente.
\begin{prop}
	Siano $S_i$, per $i \in I$ un certo insieme di indici, dei sottospazi proiettivi di $\mathbb{P}(V)$; allora l'intersezione $\bigcap_{i \in I} S_i$ \`e ancora un sottospazio di $\mathbb{P}(V)$.
\end{prop}
	\begin{proof}
		Si indicano con $H_i$ i sottospazi vettoriali di $V$ tali che $S_i = \mathbb{P}(H_i)$.
		Per conto diretto, si trova che:
		\[
		\begin{split}
			\bigcap_{i \in I} S_i &= \bigcap _{i\in I} \pi(H_i \setminus\left\{ 0 \right\} ) = \big\{[v]  \mid \forall i \in I, \ [v] \in \pi (H_i \setminus \left\{ 0 \right\} ) \big\}\\
					      &= \big\{[v]  \mid \forall i \in I, \ \exists w_i \in H_i \setminus\left\{ 0 \right\}\text{ t.c. } [w_i] = [v]\big\}\\
					      &=\big\{[v]  \mid \forall i \in I, \ v \in H_i \setminus\left\{ 0 \right\} \big\} = \pi \left(\bigcap_{i \in I} \left(H_i \setminus\left\{ 0 \right\}\right)  \right) \\
					      &= \pi \left[\left(\bigcap_{i \in I} H_i\right) \setminus\left\{ 0 \right\} \right] = \mathbb{P} \left(\bigcap_{i \in I} H_i\right) 
		\end{split}
		\] 
	\end{proof}
\noindent La proposizione appena dimostrata permette di concludere che
\begin{equation}
	\mathbb{P}(V) \cap \mathbb{P}(W) = \mathbb{P}(V \cap W)
\end{equation}
Come nel caso degli spazi vettoriali, l'unione di spazi proiettivi non \`e, in generale, uno spazio proiettivo; l'idea, allora, \`e quella di definire anche in questo caso un concetto di somma.
\begin{definizione}
	[Spazio proiettivo generato]
	Sia $A \subseteq \mathbb{P}(V)$ un sottoinsieme di $\mathbb{P}(V)$.
	A partire da $A$, si pu\`o definire il pi\`u piccolo sottospazio proiettivo di $\mathbb{P}(V)$ contenente $A$ come
	\[
		L(A) = \bigcap \left\{ S \subseteq \mathbb{P}(V)  \mid A\subseteq S \text{ e } S \text{ sottospazio proiettivo} \right\} 
	\]
\end{definizione}
\noindent Si nota che, per definizione, tale intersezione \`e non vuota perch\'e $A \subseteq \mathbb{P}(V)$ e $\mathbb{P}(V)$ \`e un sottospazio proiettivo di se stesso.

In questo modo, si pu\`o trovare lo spazio proiettivo della somma di due sottospazi prendendo 
\begin{equation}
	L(S_1,S_2) = L(S_1\cup S_2)
\end{equation}
\begin{prop}
	Siano $S_1 = \mathbb{P}(H_1)$ e $S_2 = \mathbb{P}(H_2)$ due sottospazi proiettivi di $\mathbb{P}(V)$, con $H_1,H_2$ sottospazi vettoriali di $V$; allora
	\[
	L(S_1,S_2) = \mathbb{P}(H_1+H_2)
	\] 
\end{prop}
	\begin{proof}
		Si procede per doppia inclusione. 
		Si mostra prima che $L(S_1,S_2) \subseteq \mathbb{P}(H_1+H_2)$; per farlo, visto che $H_1 \subseteq H_1 +H_2$, si ha
		\[
		S_1 = \mathbb{P}(H_1) = \pi(H_1\setminus\left\{ 0 \right\} ) \subseteq \pi\big((H_1+H_2) \setminus\left\{ 0 \right\} \big) = \mathbb{P}(H_1+H_2)
		\] 
		In maniera del tutto analoga, si mostra che $S_2 \subseteq \mathbb{P}(H_1+H_2)$.
		Questo significa che $\mathbb{P}(H_1+H_2)$ \`e un sottospazio proiettivo contenente sia $S_1$, che $S_2$, quindi, per la minimalit\`a di $L(S_1,S_2)$, deve valere $L(S_1,S_2) \subseteq \mathbb{P}(H_1+H_2)$.

		Per l'inclusione inversa, visto che $L(S_1,S_2)$ \`e un sottospazio proiettivo, si prende $H$ sottospazio vettoriale di $V$ tale che $L(S_1,S_2) = \mathbb{P}(H)\Rightarrow H = \pi^{-1} (L(S_1,S_2)) \cup \left\{ 0 \right\} $.
		Allora:
		\[
		S_1 \subseteq L(S_1,S_2) \implies H_1 = \pi^{-1} (S_1) \cup \left\{ 0 \right\} \subseteq \pi^{-1} (L(S_1,S_2)) \cup \left\{ 0  \right\} =H
		\] 
		Analogamente si mostra che $H_2\subseteq H$, quindi si ha $H_1+H_2 \subseteq H$, da cui $\mathbb{P}(H_1+H_2) \subseteq \mathbb{P}(H) = L(S_1,S_2)$.
	\end{proof}
\begin{prop}
	Sia $S \subseteq \mathbb{P}(V)$ un sottoinsieme di $\mathbb{P}(V)$ e $f$ una trasformazione proiettiva; allora $f(L(S)) = L(f(S))$.
\end{prop}
	\begin{proof}
		Siano $H = \pi^{-1} (S)$ un sottoinsieme di $V$ e $f = [\varphi ]$.
		Si nota che $f(S) = f(\pi(H)) = \pi(\varphi (H))$, quindi
		\[
		\begin{split}
			f(L(S)) &= f\big[\pi (\operatorname{Span} (H) \setminus \left\{ 0 \right\} )\big] = \pi\big[ \varphi (\operatorname{Span} (H)) \setminus\left\{ 0 \right\} \big]\\
				&= \pi\big[\operatorname{Span} (\varphi (H))\setminus\left\{ 0 \right\}  \big]=\pi\big\{\operatorname{Span} [\pi^{-1} (f(S))] \setminus\left\{ 0 \right\} \big\}\\
				&=L(f(S))
		\end{split}
		\] 
	\end{proof}
\begin{teorema}
	[Formula di Grassmann]
	Siano $S_1,S_2$ due sottospazi proiettivi di $\mathbb{P}(V)$; allora
	\[
	\dim L(S_1,S_2) = \dim S_1 + \dim S_2 - \dim S_1\cap S_2
	\] 
\end{teorema}
	\begin{proof}
		Siano $H_1,H_2$ i sottospazi di $V$ tali che $S_1 = \mathbb{P}(H_1)$ e $S_2 = \mathbb{P}(H_2)$.
		Per la formula di Grassmann, si ha
		\[
		\dim H_1+H_2 = \dim H_1 + \dim H_2 - \dim H_1 \cap H_2
		\] 
		Dal punto di vista degli spazi proiettivi, questa si traduce in:
		\[
			\begin{split}
				&\dim \mathbb{P}(H_1+H_2) +1 = (\dim \mathbb{P}(H_1 ) +1) + (\dim \mathbb{P}(H_2) +1) - \dim \mathbb{P}(H_1\cap H_2) - 1\\
				&\Rightarrow \dim  L(S_1,S_2) = \dim S_1 + \dim S_2 - \dim S_1 \cap S_2
			\end{split}
		\] 
	\end{proof}
\begin{corollario}
	Se $S_1,S_2$ sono sottospazi proiettivi di $\mathbb{P}(V)$ tali che $\dim S_1+\dim S_2 \ge  \dim \mathbb{P}(V)$, allora $S_1 \cap S_2 \neq 0$.
\end{corollario}
	\begin{proof}
		Usando la formula di Grassmann:
		\[
		\dim S_1\cap S_2=\dim S_1 + \dim S_2 - \dim L(S_1,S_2) \ge \dim \mathbb{P}(V) - \dim L(S_1,S_2) \ge 0
		\] 
		Visto che, per convenzione, $\dim \varnothing = -1$, si ha $S_1 \cap S_2 \neq \varnothing$.
	\end{proof}

\begin{teorema}
	Sui piani proiettivi, non esistono \textit{rette parallele}.
	Pi\`u precisamente, dati $r_1,r_2$ sottospazi proiettivi di $\mathbb{P}(V)$, con $\dim \mathbb{P}(V) = 2$ e $\dim r_1=\dim r_2 = 1$, si ha $r_1=r_2$, oppure $r_1\cap r_2 = \left\{ P \right\} $, dove $P$ \`e un punto proiettivo.
\end{teorema}
	\begin{proof}
		Sia $r_1\neq r_2$; allora $\dim L(r_1,r_2) = 2$ e 
		\[
		\dim  r_1 \cap r_2 = \dim r_1 \dim r_2 - \dim L(r_1,r_2) = 1 + 1 - 2 =0
		\] 
		quindi $r_1\cap r_2$ \`e un punto proiettivo.
	\end{proof}
\subsubsection{Riferimenti proiettivi}

L'idea \`e quella di estendere il concetto di indipendenza lineare e base agli spazi proiettivi.

\begin{definizione}
	[Punti indipendenti]
	Siano $P_1, \ldots, P_k \in \mathbb{P}(V)$; questi sono detti \textit{indipendenti} se per $v_1,\ldots,v_k \in V : [v_i] = P_i$, i $v_1,\ldots,v_k$ sono linearmente indipendenti.
\end{definizione}
\noindent \`E anche facile convincersi che tale definizione \`e indipendente dai rappresentati scelti, visto che $v_1,\ldots,v_k$ sono linearmente indipendenti $\iff \lambda _1 v_1,\ldots, \lambda _kv_k$ lo sono.

Questa nozione di indipendenza per gli elementi di $\mathbb{P}(V)$ permette di individuare tutte quelle rette di $V$ che non si possono scrivere in combinazione lineare tra di loro.

Si nota, inoltre, che gli elementi di $\mathbb{P}(V)$ sono, a due a due, sempre indipendenti per costruzione, mentre non \`e vero in generale per un insieme di $k$ punti, con $k>2$.
Infatti, prendendo $V = \mathbb{R}^2$, si osserva che non si potr\`a mai costruire un insieme di punti di $\mathbb{P}(V)$ che sia indipendente e che abbia pi\`u di due elementi perch\'e $\mathbb{R}^2$ \`e ottenuto dallo span di esattamente due elementi indipendenti.
\begin{definizione}
	[Posizione generale]
	Dati $P_1,\ldots,P_k \in \mathbb{P}(V)$, con $V$ spazio vettoriale di dimensione $n+1$, questi sono detti essere in \textit{posizione generale} se ogni loro sottoinsieme di $h\le n+1$ elementi distinti \`e indipendente.
\end{definizione}
\begin{osservazione}
Se $k\le n+1$, allora la nozione di posizione generale coincide con quella di indipendenza, mentre se $k>n+1$, la definizione richiede l'indipendenza di tutte le $(n+1)$-uple di punti.
\end{osservazione}
\begin{definizione}
	[Riferimento proiettivo]
	Sia $V$ uno spazio vettoriale tale che $\mathbb{P}(V)$ \`e $n$-dimensionale.
	Un \textit{riferimento proiettivo} di $\mathbb{P}(V)$ \`e una $(n+2)$-upla di punti in posizione generale.

	L'ultimo punto nel riferimento \`e detto \textbf{punto unit\`a}, mentre gli altri sono detti \textbf{punti fondamentali}.
\end{definizione}
\begin{definizione}
	[Base normalizzata]
	Sia $\mathcal{R} =(P_0,\ldots,P_{n+1} )$ un riferimento di $\mathbb{P}^n(V)$.
	Una \textit{base normalizzata} di $V$ associata a $\mathcal{R} $ \`e una base $\left\{ v_0,\ldots,v_n \right\} $ tale che
	\[
		\forall i \in \left\{ 0,\ldots,n \right\} , \ [v_i] = P_i \text{ e } P_{n+1} = [v_0+\ldots+v_n]
	\] 
\end{definizione}
\begin{teorema}
	Sia $V$ uno spazio vettoriale su $\mathbb{K}$ tale che $\mathbb{P}(V)$ abbia dimensione $n$ e sia $\mathcal{R} =(P_0,\ldots,P_{n+1} )$ un riferimento proiettivo di $\mathbb{P}(V)$.
	Esiste sempre almeno una base normalizzata $\mathcal{B} = \left\{ v_0, \ldots,v_n\right\} $ e, se $\mathcal{B} ' = \left\{ u_0,\ldots,u_n \right\} $ \`e un'altra base, allora $\exists \lambda \in \mathbb{K}\setminus\left\{ 0 \right\} $ tale che $u_i = \lambda v_i$.
\end{teorema}
	\begin{proof}
		Sia $\left\{ w_0, \ldots,w_{n+1}  \right\} \subset V$ un insieme di vettori, con $[w_i] = P_i, \ i=0,\ldots,n+1$.
		Visto che $\mathcal{R}$ \`e un riferimento proiettivo, i $w_0,\ldots,w_n$ sono $n+1$ vettori linearmente indipendenti in $V$ spazio $(n+1)$-dimensionale, quindi sono una base.
		Questo implica che $w_{n+1} = \lambda _0 w_0 + \ldots + \lambda _n w_n$, dove $\lambda _i\neq 0$, altrimenti i $w_0,\ldots,w_{i-1} , w_{i+1} , \ldots, w_{n+1} $ non sarebbero indipendenti e $\mathcal{R} $ non sarebbe un riferimento proiettivo.
		Si prendono, ora, $v_i = \lambda _i w_i, \ i =0,\ldots,n$, mentre $v_{n+1} =w_{n+1} $; in questo modo, $[v_i]=[w_i] = P_i$, mentre $v_{n+1} = \sum_{i=0}^{n} v_i$ per costruzione.
		In questo modo, si \`e costruita $\mathcal{B} =\left\{ v_0,\ldots,v_n \right\} $ base normalizzata associata a $\mathcal{R} $.

		Sia, ora, $\mathcal{B} ' = \left\{ u_0,\ldots,u_n \right\} $ un'altra base normalizzata di $\mathcal{R} $; essendo che $[u_i] = P_i = [v_i]$, allora $\exists \mu _i \in \mathbb{K}\setminus\left\{ 0 \right\} : v_i = \mu _i u_i, \ i =0,\ldots,n+1$.
		Inoltre
		\[
			\begin{split}
				&\mu _{n+1} \sum_{i=0}^{n} u_i = \mu _{n+1} u_{n+1} = v_{n+1} = \sum_{i=0}^{n} v_i = \sum_{i=0}^{n} \mu _i u_i\\
				&\Rightarrow 0 = \sum_{i=0}^{n} (\mu _i-\mu _{n+1} ) u_i
			\end{split}
		\] 
	Visto che gli $u_0,\ldots,u_n$ sono una base, per indipendenza lineare deve valere $\mu _i = \mu _{n+1} , \ \forall i$, cio\`e le due basi coincidono a meno di un fattore invertibile.
	\end{proof}
\begin{osservazione}
Si nota che rispetto all'algebra lineare, in geometria proiettiva non \`e possibile estendere riferimenti proiettivi di sottospazi proiettivi a riferimenti di sottospazi che li estendono.

Sia, infatti, $\mathcal{R} = \left\{ P_0,\ldots,P_{n+1}  \right\}  $ un riferimento di un sottospazio $S $ di $H$; allora si nota che i $P_i$ non sono in posizione generale se visti come punti di $H$ perch\'e se la dimensione aumenta, le $(n+2)$-uple devono essere indipendenti, ma il punto unit\`a \`e scritto come combinazione degli altri.
\end{osservazione}
\noindent Con la teoria sviluppata finora, \`e possibile stabilire un criterio di uguaglianza per individuare trasformazioni proiettive uguali.
\begin{teorema}\label{tpequiv}
	Siano $f,g  : \mathbb{P}(V) \to \mathbb{P}(W)$ due trasformazioni proiettive indotte, rispettivamente, da $\varphi $ e da $\psi $ e sia, inoltre, $\mathcal{R}  $ un riferimento proiettivo di $\mathbb{P}(V)$; allora sono equivalenti le seguenti affermazioni.
	\begin{enumerate}[(a).]
		\item Si trova un coefficiente $ \lambda \in \mathbb{K} \setminus\left\{ 0 \right\} $ tale che $\varphi  = \lambda \psi $.
		\item Le due trasformazioni proiettive sono identiche: $f = g$.
		\item Per ogni punto $P \in \mathcal{R} $, si ha $f(P) = g(P)$.
	\end{enumerate}
\end{teorema}
	\begin{proof}
		Si dimostra che (a) $\Rightarrow $ (b). 
		Per conto diretto:
		\[
			f([v]) =[\varphi (v)] = [\lambda \psi (v)]=[\psi (v)] = g(v)
		\] 
		Ora si mostra che (b) $\Rightarrow $ (c).
Questo, per\`o, \`e ovvio perch\'e $ f(P) = g(P), \ \forall P \in \mathbb{P}(V)$, incluso $\mathcal{R} \subset \mathbb{P}(V)$.

Infine, si mostra (c) $\Rightarrow $ (a).
Per farlo, si prende $\left\{ v_0,\ldots,v_n \right\} $ base normalizzata riferita a $\mathcal{R} $. 
Si sa che
\begin{equation}\label{t1.6rel}
	[\varphi (v_i)]=f(P_i)=f(P_i)=[\psi (v_i)], \ \forall P_i \in \mathcal{R} 
\end{equation}
quindi $\exists \lambda _i \in \mathbb{K}\setminus\left\{ 0 \right\} : \varphi (v_i) = \lambda _i \psi (v_i)$.
Per concludere, si considera cosa succede al punto unit\`a; per la relazione \ref{t1.6rel}, deve valere $\varphi (v_{n+1} )= \lambda _{n+1} \psi (v_{n+1} )$, per qualche $\lambda _{n+1} \in \mathbb{K}\setminus\left\{ 0 \right\} $, ma, al contempo:
\[
		\lambda _{n+1} \sum_{i=0}^{n} \psi (v_i) = \sum_{i=0}^{n} \varphi (v_i)= \sum_{i=0}^{n} \lambda _i\psi (v_i)\implies \sum_{i=0}^{n} (\lambda _i - \lambda _{n+1} ) \psi (v_i) = 0
\] 
Visto che $\psi $ \`e iniettiva, gli $\psi (v_0),\ldots,\psi (v_n)$ sono linearmente indipendenti, quindi $\forall i = 0,\ldots,n , \ \lambda _{n+1} = \lambda _i$, il che vuol dire che $\lambda _{n+1} \psi = \varphi $ sui $v_0,\ldots,v_n$; essendo questi una base, la relazione $\varphi  = \lambda _{n+1} \psi $ vale su tutti i vettori dello spazio.
	\end{proof}
\begin{corollario}
	Si ha
	\[
		\mathbb{P}\mathrm{GL} (V) \cong \faktor{\mathrm{GL} (V)}{N}
	\] 
	con $N = \big\{ \lambda \operatorname{Id}  \mid \lambda \in \mathbb{K}\setminus\left\{ 0 \right\}  \big\}\lhd \mathrm{GL} (V) $.
\end{corollario}
	\begin{proof}
		Si considera la mappa $\varphi \mapsto [\varphi ] : \mathrm{GL} (V) \to \mathbb{P}\mathrm{GL} (V)$.
		Questa mappa \`e un omomorfismo suriettivo di gruppi e, se $[\varphi ]= \operatorname{Id} _{\mathbb{P}(V)}=[\operatorname{Id}_V] $, allora $\varphi  = \lambda \operatorname{Id} _V$ per il teorema \ref{tpequiv} appena mostrato.
		Questo implica che il nucleo di tale omomorfismo \`e proprio $N$, quindi la tesi segue applicando il primo teorema di omomorfismo.
	\end{proof}
\begin{notazione}[Proiettivit\`a standard]
	Le proiettivit\`a di $\mathbb{P}^n(\mathbb{K})$ formano un gruppo indicato da $\mathbb{P}\mathrm{GL} (\mathbb{K}^{n+1} ) = \mathbb{P}\mathrm{GL} _{n+1} (\mathbb{K})$.
	L'$n+1$ come pedice indica la taglia delle matrici che rappresentano le proiettivit\`a, non la dimensione dello spazio su cui agiscono.
\end{notazione}
\begin{teorema}
	[Teorema fondamentale delle trasformazioni proiettive]
	Siano $\mathbb{P}(V)$ e $\mathbb{P}(W)$ due spazi proiettivi su $\mathbb{K}$ tali che $\dim \mathbb{P}(V) = \dim \mathbb{P}(W) = n$.
	Fissati $\mathcal{R} = \left\{P_0,\ldots,P_{n+1}\right\} $ e $\mathcal{R} ' = \{P'_0,\ldots,P'_{n+1} \}$ due riferimenti proiettivi di $\mathbb{P}(V)$ e $\mathbb{P}(W)$ rispettivamente, esiste un'unica trasformazione proiettiva $f:\mathbb{P}(V)\to \mathbb{P}(W)$ tale che $f(P_i) = P_i', \ i=0,\ldots,n+1$. 
\end{teorema}
	\begin{proof}
		L'unicit\`a \`e diretta conseguenza del teorema \ref{tpequiv}. 
		Rimane da mostrare l'esistenza.

		Siano, allora, $\mathcal{B} = \left\{ v_0,\ldots,v_n \right\} $ e $\mathcal{B}' = \left\{ w_0,\ldots,w_n \right\} $ due basi normalizzate associate a $\mathcal{R} $ e $\mathcal{R} '$ rispettivamente e sia $\varphi  : V \to W$ la mappa tale che $\forall i, \ \varphi (v_i) = w_i$; si nota che $\varphi $ \`e iniettiva perch\'e ha rango massimo, quindi si pu\`o prendere $f = [\varphi ]$.
		Per costruzione:
		\[
			f(P_i) = f([v_i]) = [\varphi (v_i)] = [w_i]=P'_i
		\] 
		mentre per i punti fondamentali:
		\[
			f(P_{n+1} ) = f([v_0+\ldots+v_n]) = [\varphi (v_0+\ldots+v_n]) = [w_0+\ldots+w_n] = P'_{n+1} 
		\] 
	\end{proof}

\subsubsection{Coordinate omogenee}
Come per il caso degli spazi vettoriali, \`e utile ricorrere a sistemi di coordinate.
Per capire come definirli, si considera il caso particolare di $\mathbb{P}^n(\mathbb{K})$; questo, per definizione, \`e:
\[
	\mathbb{P}^n(\mathbb{K}) = \faktor{\mathbb{K}^{n+1} \setminus\left\{ 0 \right\} }{\sim} = \left\{ [(x_0,\ldots,x_n)]  \mid \exists x_i \neq 0 \right\} 
\] 
\begin{definizione}
	[Riferimento proiettivo canonico]
	Il \textit{riferimento canonico} (o standard) di $\mathbb{P}^n(\mathbb{K})$ \`e quel riferimento proiettivo che ha, come base normalizzata, la base canonica di $\mathbb{K}^{n+1} $.
\end{definizione}
\noindent In questo caso, si dir\`a che $[(x_0,\ldots,x_n)]$ ha \textit{coordinate omogenee} $[x_0:\ldots :x_n]$ rispetto al riferimento canonico di $\mathbb{P}^n(\mathbb{K})$.
Ora si estende questa definizione a spazi proiettivi generali.
\begin{definizione}
	[Coordinate omogenee]
	Sia $\mathcal{R} = \left\{ P_0,\ldots,P_{n+1}  \right\} $ un riferimento proiettivo di $\mathbb{P}(V)$; dato $P \in \mathbb{P}(V)$, le sue \textit{coordinate omogenee} rispetto a $\mathcal{R} $ sono date da una delle seguenti definizioni.
	\begin{itemize}
		\item Se $f:\mathbb{P}(V)\to \mathbb{P}^n(\mathbb{K})$ \`e l'unico isomorfismo proiettivo che manda $\mathcal{R} $ nel riferimento canonico di $\mathbb{P}^n(\mathbb{K})$, allora le coordinate di omogenee di $P$ sono $f(P)$.
		\item Se $\mathcal{B} = \left\{ v_0, \ldots ,v_n\right\} $ \`e una base normalizzata di $\mathcal{R} $ e $P = [v]$, si considera $v = \sum_{i=0}^{n} a_i v_i$; le coordinate omogenee di $P$ rispetto a $\mathcal{R}$, allora, sono $[a_0:\ldots:a_n]$.
	\end{itemize}
\end{definizione}
\noindent Come per gli spazi vettoriali, \`e possibile rappresentare le trasformazioni proiettive come matrici e i sottospazi proiettivi come i luoghi di zeri di equazioni.
\begin{definizione}
	[Matrice associata]
Sia $f:\mathbb{P}(V) \to \mathbb{P}(W)$ un isomorfismo proiettivo e siano $\mathcal{R} , \mathcal{R} '$ due riferimenti proiettivi di $\mathbb{P}(V)$ e $\mathbb{P}(W)$ rispettivamente, con $\mathcal{B} ,\mathcal{B} '$ le rispettive basi normalizzate.
Se $\varphi $ induce $f$, allora $f$ \`e rappresentata da 
\[
M = M_{\mathcal{B} '} ^{\mathcal{B} } (\varphi ) \in \mathcal{M} (n+1,\mathbb{K})
\] 
\end{definizione}
\begin{osservazione}
	[Prodotto tra matrice e coordinate omogenee]
	Siano $P=[v] \in \mathbb{P}(V)$ e $f:\mathbb{P}(V) \to \mathbb{P}(W)$, con $f=[\varphi ]$.
	Sia $M$ la matrice che rappresenta $f$ e siano $\mathcal{R} ,\mathcal{R} '$ due riferimenti di $\mathbb{P}(V)$ e $\mathbb{P}(W)$, con basi $\mathcal{B} $ e $\mathcal{B}' $.

	Indicando il passaggio a coordinate omogenee rispetto a $\mathcal{R} $ con $[\cdot ]_{\mathcal{R}} $ e il passaggio a coordinate rispetto a $\mathcal{B} $ con $[\cdot ]_{\mathcal{B}} $, si ha:
\begin{equation}
	\begin{split}
		[f(P)]_{\mathcal{R} '} &= [[\varphi (v)]]_{\mathcal{R} '} = \Big[\big[[M[v]_{\mathcal{B} } ]_{\mathcal{B} '} ^{-1} \big]\Big]_{\mathcal{R} '} = \Big[\big[[[M[v]_{\mathcal{B}} ]]_{\mathcal{R} '} ^{-1} \big]\Big]_{\mathcal{R} '} \\
				       &= [M[v]_{\mathcal{B}} ]
	\end{split}
\end{equation}	
\end{osservazione}
\begin{notazione}
	Sia $M$ una matrice e sia $P =[v]\in \mathbb{P}(V)$; prendendo $\mathcal{R} $ un riferimento di $\mathbb{P}(V)$ e $\mathcal{B}$ sua base normalizzata, si pone
	\[
		M[P]_{\mathcal{R}} =[M][P]_{\mathcal{R}}  := \left[ M[v]_{\mathcal{B}}   \right] 
	\] 
\end{notazione}
\begin{definizione}
	[Equazioni cartesiane proiettive]
	Sia $S\subset \mathbb{P}(V)$ un sottospazio proiettivo e $W$ il sottospazio di $V$ tale che $S = \mathbb{P}(W)$.
	Fissato un riferimento $\mathcal{R} $ su $\mathbb{P}(V)$, si individua univocamente una base normalizzata di $V$, pertanto $W$ si esprime come luogo degli zeri di $\dim V - \dim W$ equazioni.
	Queste equazioni si definiscono \textit{equazioni cartesiane} per $S$ rispetto a $\mathcal{R} $.
\end{definizione}
\noindent Si nota che il numero di equazioni cartesiane si pu\`o scrivere anche come
\[
\dim \mathbb{P}(V) - \dim S = \dim V - \dim W
\] 

\subsection{Spazi proiettivi e spazi affini}
\subsubsection{Carte affini}

\begin{definizione}
	[Iperpiano coordinato]
	Dato $\mathbb{P}^n(\mathbb{K})$, si definisce il seguente suo sottoinsieme:
	\[
		H_i = \left\{ [x_0 : \ldots :x_n] \in \mathbb{P}^n(\mathbb{K})  \mid x_i =0  \right\} 
	\] 
	con $i \in \left\{ 0,\ldots,n \right\} $. Tale sottoinsieme \`e noto come l'$i$-esimo \textit{iperpiano coordinato}.
\end{definizione}
\noindent Se considerati come spazi proiettivi, allora
\begin{equation}
	H_i \cong \mathbb{P}^{n-1} (\mathbb{K})
\end{equation}
\begin{definizione}
	[Carta affine -- Insieme]
	Si definisce l'$i$-esima \textit{carta affine} come 
	\[
	U_i = \mathbb{P}^n (\mathbb{K}) \setminus H_i
	\] 
\end{definizione}
\begin{prop}
	Esiste una mappa biettiva tra $U_i$ e $\mathbb{K}^n$, $\forall i=0,\ldots,n$.
\end{prop}

	\begin{proof}
	\end{proof}


















\newpage
\section{Topologia generale}
\subsection{Spazi metrici}
\begin{definizione}
	[Spazio metrico]
Sia $X$ un insieme non vuoto; allora $X$ si dice spazio metrico se pu\`o essere equipaggiato con una \textit{distanza}, ossia una funzione $d : X \times X \to \mathbb{R}$ tale che:
\begin{itemize}
	\item $d(x,x') \ge  0 $ e $d(x,x') = 0 \iff x=x'$;
	\item $d(x,x') = d(x',x)$;
	\item $d(x,x'') \le  d(x,x') + d(x',x'')$.
\end{itemize}
\end{definizione}
\noindent Dato uno spazio metrico $(X,d_X)$ e un insieme $Y \subset  X$, si pu\`o definire un sottospazio di $(X,d_X)$ restringendo la distanza al solo $Y$:
\[
d_Y (y,y') := d_X(y,y'), \ \forall y,y' \in Y
\] 
Quindi $(Y,d_Y)$ \`e a sua volta uno spazio metrico, sottospazio di $(X,d_X)$, il quale \`e detto \textit{spazio ambiente} di $Y$.
In uno spazio metrico $(X,d)$, si pu\`o definire un \textit{disco aperto} di raggio $r$ e centro $x$ come
\[
B_r(x) := \left\{ x' \in X  \mid d(x,x') < r \right\} 
\] 
\begin{definizione}
	[Insieme aperto]
	Sia $(X,d)$ uno spazio metrico. Un suo sottoinsieme si dice aperto se \`e generato dall'unione di dischi aperti.
	
	Equivalentemente, un insieme $\forall \subseteq X$ si dice aperto rispetto alla metrica $d$ se $\forall x \in A, \ \exists \varepsilon >0 $ tale che $B_\varepsilon (x) \subseteq A$.
\end{definizione}
\begin{lemma}
	Le palle aperte sono insiemi aperti relativamente alla metrica che le definisce.
\end{lemma}
	\begin{proof}
		Sia $(X,d)$ uno spazio metrico e $x_0 \in X$; si dimostra che la palla $B_r(x_0) = \left\{ x \in X  \mid d(x_0,x) < r \right\} $ \`e aperto rispetto a $d$.
		Si nota che, $\forall x \in B_r(x_0)$, \`e possibile definire $\delta = r - d(x,x_0)$ tale che $B_\delta (x) \subseteq B_r(x_0)$; infatti tutti i punti di $B_\delta (x)$ sono a distanza minore di $\varepsilon $ da $x$, quindi, per disuguaglianza triangolare:
		\[
			d(x_0,y) \le d(x_0,x) + \underbracket{d(x,y)}_{< \delta }  < r, \ \forall y \in B_\delta (x)
		\] 
		per definizione di $\delta $.
	\end{proof}
\noindent Nello stesso spazio metrico, \`e possibile definire la distanza tra un punto $x \in X$ con un sottoinsieme $A \subseteq X$ come:
\begin{equation}
	d_A(x) = \inf \left\{ d(x,a)  \mid a \in A \right\}   
\end{equation}
\subsubsection{Continuit\`a in spazi metrici}
Una funzione $f:\mathbb{R}\to \mathbb{R}$ si dice continua in $x \in \mathbb{R}$ se $\forall \varepsilon , \ \exists \delta (\varepsilon )$ tale che:
\[
\lvert f(x) - f(x') \rvert < \varepsilon, \ \forall \lvert x-x' \rvert < \delta (\varepsilon )
\] 
\`E possibile generalizzare la definizione a spazi metrici usando la metrica definita su di essi.
\begin{definizione}
	[Continuit\`a in spazi metrici]
Sia $f : X\to Y$ un'applicazione, con $(X,d_X) , \ (Y,d_Y)$ spazi metrici. Si dice che $f$ \`e continua in $x \in X$ se $\forall \varepsilon , \ \exists \delta (\epsilon )$ tale che:
\begin{equation}
	d_Y \big(f(x), f(x')\big) < \varepsilon , \ \forall d_X(x,x')< \delta (\varepsilon )
\end{equation}
Questo si esprime equivalentemente come:
\[
\forall \varepsilon >0, \ \exists \delta > 0 \text{ t.c. } f(B_{d_X} (x,\delta )) \subseteq B_{d_Y} (f(x),\varepsilon )
\] 

\end{definizione}
\noindent Usando la nozione di insieme aperto, \`e possibile generalizzare ulteriormente la definizione di continuit\`a al solo concetto di apertura di un insieme.
\begin{teorema}
Un'applicazione $f:X\to Y$ \`e continua $\iff\forall A \subset  Y$ aperto, l'insieme $f^{-1}(A)  $ \`e aperto.
\end{teorema}
\begin{proof}
Si dimostrano le due implicazioni.	
\begin{itemize}
	\item $(\Rightarrow )$ Si assume che $f$ sia continua. Si prende $f(x) \in A$, con $A\subset Y$ aperto, per qualche $x \in f^{-1} (A)$. Essendo $A$ aperto $\Rightarrow \exists \varepsilon >0 : B_\varepsilon \big(f(x)\big)\subset A$; allo stesso tempo, per continuit\`a di $f$, dato $\varepsilon $ scelto prima, deve esistere $\delta (\varepsilon )$ tale che
		\[
		f\big(B_{\delta (\varepsilon )}(x)\big) \subset B_\varepsilon \big(f(x)\big)
		\] 
	quindi $B_{\delta (\varepsilon )} (x) \subset f^{-1} (A)$. Valendo $\forall x \in f^{-1} (A)\Rightarrow f^{-1} (A)$ \`e aperto perch\'e per ogni suo elemento, esiste una palla tutta contenuta al suo interno.
\item $(\Leftarrow)$ Si assume che $\forall A \subset Y$ aperto, la funzione $f$ sia tale che l'insieme $f^{-1} (A)$ \`e aperto. Per $f(x) \in Y$, esiste $B_\varepsilon \big(f(x)\big) \subset Y$; essendo questo aperto, deve essere aperto anche $f^{-1} \big[B_\varepsilon \big(f(x)\big) \big]$. 
	Dunque, dato $x \in f^{-1} \big[B_\varepsilon \big(f(x)\big) \big] $, $\exists \delta (\varepsilon ) : B_{\delta (\varepsilon )} (x) \subset f^{-1} \big[B_\varepsilon \big(f(x)\big) \big]$, quindi vuol dire che $f\big(B_{\delta (\varepsilon )} (x)\big)\subset B_\varepsilon \big(f(x)\big)$, ossia:
	\[
	d_Y\big(f(x), f(x')\big) < \varepsilon , \ \forall d_X (x,x') < \delta (\varepsilon )
	\] 
Valendo $\forall x \in X$, allora $f$ \`e continua.
\end{itemize}
\end{proof}
\noindent Questo permette di parlare di continuit\`a di applicazioni in insiemi su cui non \`e definita una distanza, ma solo i sottoinsiemi aperti.

Negli spazi metrici, \`e possibile caratterizzare delle mappe che preservano le distanze; queste sono note come \textit{immersioni isometriche}.
\begin{definizione}
	[Immersione isometrica]
	Sia $f:X\to Y$ una mappa tra spazi metrici; questa \`e detta immersione isometrica se
	\[
	d_Y(f(x),f(y)) = d_X(x,y)
	\] 
\end{definizione}
\noindent Un'immersione isometrica deve necessariamente essere iniettiva:
\[
f(x) = f(y) \implies d_Y(f(x),f(y))= 0 = d_X(x,y) \iff x=  y 
\] 
Inoltre, la composizione di due immersioni isometriche \`e ancora un'immersione isometrica e l'identit\`a ne \`e un esempio.
\begin{definizione}
	[Isometria]
	Sia $f:X\to Y$ un'immersione isometrica; allora se $f$ \`e suriettiva, quindi biettiva, \`e detta \textit{isometria}.
\end{definizione}
\noindent Le isometrie formano un gruppo con l'operazione di composizione, che si indica con $\mathrm{Isom} (X)$.
\begin{definizione}
	[Omeomorfismo]
	Dati $X,Y$ spazi metrici, un'applicazione biettiva $f:X\to Y$ \`e un \textit{omeomorfismo} se la sua inversa e $f$ stessa sono continue.
\end{definizione}
\noindent Ne segue che ogni isometria \`e un omeomorfismo, ma non \`e vero il viceversa. Per esempio, definendo $e^x : \mathbb{R} \to (0,+\infty)$, questa ha un'inversa continua $\log(x) : (0,+\infty) \to \mathbb{R}$, quindi \`e un omeomorfismo, ma non \`e un'isometria perch\'e manda $(-\infty,0]$ in $(0,1]$.
Anche gli omemorfismi definiscono una \textbf{relazione di equivalenza} tra spazi metrici.
\begin{definizione}
	[Mappa lipschitziana]
	Siano $(X,d_X)$ e $(Y,d_Y)$ due spazi metrici e sia $f : X \to Y$ una mappa; si dice che $f$ \`e \textit{lipschitizana} se 
	\[
	d_Y(f(p),f(q)) \le kd_X(p,q) ,\ \forall p,q \in X
	\] 
\end{definizione}
\begin{prop}
	Se $f:X\to Y$ \`e lipschitizana, allora \`e continua.
\end{prop}
	\begin{proof}
		Sia $f$ una funzione $k$-lipschitziana; si fissa $\varepsilon > 0 $ e si prende $\delta = \varepsilon / k$, per cui $\forall x' \in X$ tale che $d_X (x,x') < \delta $, si ha
		\[
		d_Y(f(x),f(x')) \le k d_X(x,x') < k \frac{ \varepsilon}{k}  = \varepsilon 
		\] 
		
	\end{proof}
\subsection{Spazi topologici}
Allo scopo di giustificare la definizione e lo studio di spazi topologici, si considera il seguente risultato.
\begin{prop}
	Sia $(X,d)$ uno spazio metrico. Allora:
	\begin{enumerate}[(a).]
		\item $\varnothing$ e $X$ sono aperti;
		\item se $A,B$ sono aperti, allora $A\cap B$ \`e aperto;
		\item se $\left\{ A_i \right\} _{i\in I} $ \`e una famiglia di aperti, allora $\bigcup_{i \in I} A_i$ \`e aperto.
	\end{enumerate}
\end{prop}
	\begin{proof}
		Si divide la dimostrazione nei vari punti.
		\begin{enumerate}[(a).]
			\item $X$ \`e ovviamente aperto, mentre per l'insieme vuoto non ci sono punti per cui bisogna verificare la richiesta, quindi \`e aperto.
			\item Sia $x_0 \in A\cap B$; questo significa che ci sono due palle di raggi $\epsilon _1, \epsilon _2$ interamente contenute in $A$ e $B$ rispettivamente, visto che sono aperti.
				Prendendo $\epsilon = \min \left\{ \epsilon _1,\epsilon _2 \right\} $, si verifica immediatamente che $B_\epsilon (x) \subseteq A\cap B$, visto che \`e interamente contenuta sia in $A$ che $B$.
			\item Evidentemente $\exists j \in I : x_0 \in A_j \implies \exists \epsilon : B_\epsilon (x_0) \subseteq A_j \subseteq \bigcup_{i \in I} A_i$.
		\end{enumerate}
	\end{proof}
\noindent Si nota che l'intersezione arbitraria di aperit pu\`o non risultare aperta, come nel caso della famiglia $B_{1 / n} (0)$ in $\mathbb{R}$.

Dalla precedente proposizione, \`e possibile giustificare la seguente definizione di topologia.
\begin{definizione}
	[Topologia e spazio topologico]
	Sia $X$ un insieme non-vuoto. Una \textit{topologia} su $X$ \`e una famiglia non-vuota $\tau \subseteq \mathcal{P} (x)$, chiamati \textit{insiemi aperti della topologia}. Questi soddisfano le seguenti condizioni:
	\begin{itemize}
		\item $\varnothing, \ X$ sono aperti;
		\item l'unione di una qualsiasi famiglia di insiemi aperti \`e un insieme aperto;
		\item l'intersezione di due insiemi aperti \`e un aperto.
	\end{itemize}
	Allora si definisce \textit{spazio topologico} la coppia $(X,\tau )$, dove $X$ \`e detto \textit{supporto} dello spazio topologico e i suoi elementi sono i \textit{punti} dello spazio.
\end{definizione}
\noindent Dato $(X,d)$ spazio metrico, la famiglia degli insiemi aperti rispetto a $d$ \`e una topologia su $X$ indotta da $d$ stessa.
In $\mathbb{R}^n$, si definisce \textbf{topologia euclidea} (o \textbf{naturale}) $\mathcal{E}$ come quella indotta dalla distanza euclidea $d_2$. Su $\mathbb{C}$, la topologia euclidea $\mathcal{E}$ \`e quella indotta da $d(z,w) = \lvert z-w \rvert $; 
questa conclusione si pu\`o ottenere identificando $\mathbb{C}$ con $\mathbb{R}^2$ da $z=x+iy \mapsto (x,y)$ e considerando la distanza euclidea di $\mathbb{R}^2$. 
\begin{definizione}
	[Topologia discreta]
	Sia $X$ un insieme generico; allora l'insieme $\tau  = \mathcal{P} (X)$ \`e una topologia di $X$, nota col nome di \textit{topologia discreta}.
	Inoltre, si dice che $(X,\tau )$ \`e lo \textit{spazio topologico discreto}.
\end{definizione}
\begin{definizione}
	[Topologia banale]
	Sia $X$ un insieme generico; allora l'insieme $\tau  = \left\{ \varnothing, X \right\} $ definisce una topologia su $X$, nota col nome di \textit{topologia banale}, o \textit{indiscreta}.
	Inoltre, si dice che $(X,\tau )$ \`e lo \textit{spazio topologico banale}.
\end{definizione}
\begin{definizione}
	[Topologia cofinita]
	Sia $X$ un insieme; l'insieme 
	\[
	\tau = \left\{ \varnothing \right\} \cup \left\{ A \subseteq X  \mid \lvert X \setminus A \rvert \in \mathbb{N} \right\} 
	\] 
	\`e una topologia su $X$, detta \textit{topologia cofinita}.
	Questa ha, come chiusi, gli insiemi finiti e tutto lo spazio; quest'ultimo risulta sia chiuso che aperto.
\end{definizione}
\noindent In generale, una topologia non induce una distanza; per esempio, la topologia banale non \`e indotta da alcuna metrica per $\lvert X \rvert \ge 2$ perch\'e se $x_1,x_2\in X$ sono due punti distinti, allora $B(x_1,d(x_1,x_2) /2 ) $ e $B(x_2,d(x_1,x_2)/2)$ sono disgiunte e non-vuote, quindi sono aperte rispetto a questa distanza, ma la topologica banale prevede solo $\varnothing$ e $X$ come aperti.
\begin{definizione}
	[Spazio metrizzabile]
	Uno spazio topologico $(X,\tau )$ \`e detto \textit{metrizzabile} se si pu\`o definire una distanza su $X$ che induce la topologia $\tau $.
\end{definizione}
\begin{definizione}
	[Topologia di sottospazio]
	Sia $(X,d)$ uno spazio metrico e sia $Y \subset X$ un suo sottoinsieme; la topologia di sottospazio \`e l'insieme degli aperti rispetto alla distanza $d|_Y$, rispetto a cui $Y$ \`e uno spazio metrico.
\end{definizione}
\begin{osservazione}
Se $y \in Y$: $B_\varepsilon ^{(Y)} (y) = B_\varepsilon ^{(X)} (y) \cap Y$; questo significa che gli aperti di $Y$ sono della forma $A \cap Y$, con $A $ aperto di $X$.
\end{osservazione}
\noindent Visto che la topologia di uno spazio $(X,\tau )$ permette di individuare gli insiemi aperti di $X$, allora si ha la seguente definizione.
\begin{definizione}
	[Insieme chiuso]
	Sia $(X,\tau )$ uno spazio topologico; si dice che $C \subseteq X$ \`e chiuso se $X \setminus C \in \tau $.
\end{definizione}
\noindent Infine, avendo la possibilit\`a di definire pi\`u topologie per uno spazio topologico $X$, \`e possibile metterle in relazione a seconda degli elementi che contengono.
\begin{definizione}
	[Finezza di una topologia]
	Date $\tau _1, \tau _2$ due topologie dello spazio $X$, si dice che $\tau _1$ \`e pi\`u fine di $\tau _2$ se $\tau _2 \subseteq \tau _1$.
\end{definizione}
\noindent La finezza induce un ordinamento parziale nell'insieme delle topologie di uno spazio topologico, dove la topologia discreta rappresenta la topologia pi\`u fine possibile, mentre la topologia banale quella meno fina.

\begin{definizione}
	[Funzione continua]
	Una mappa fra spazi topologici $f : X \to Y$ si dice \textit{continua} se $\forall A \in \tau _Y$, si ha $f^{-1}(A) \in \tau _X$.
\end{definizione}
\begin{definizione}
	[Omeomorfismo]
	Sia $f:X\to Y$ una mappa continua; si dice che $f$ \`e un \textit{omeomorfismo} se \`e biunivoca e $f^{-1}$ \`e continua.
\end{definizione}
\begin{osservazione}
Gli omeomorfismi e le funzioni continue biettive non coincidono; per esempio, dato $X$ uno spazio topologico con $\tau _1$ e $\tau _2$ due sue topologie tali che $\tau _2 \subsetneq \tau _1$, allora $\operatorname{Id} :(X,\tau _1)\to (X,\tau _2)$ \`e continua, ma $\operatorname{Id} : (X,\tau _2) \to (X,\tau _1)$ no.
\end{osservazione}
\subsubsection{Distanze equivalenti}
\begin{definizione}
	[Distanze topologicamente equivalenti]
	Due distanze $d,\overline{d}$ su $X$ si dicono \textit{topologicamente equivalenti} se generano la stessa topologia.
\end{definizione}
\begin{prop}\label{dfin}
	Siano $d_1,d_2$ due metriche definite nello spazio $X$, che inducono le topologie $\tau _1$ e $\tau _2$ rispettivamente; se $\exists k>0$ tale che $d_1(x,y) \le kd_2(x,y)$, allora $\tau _2$ \`e pi\`u fine di $\tau _1$.
\end{prop}
	\begin{proof}
		Sia $A$ un aperto secondo $d_1$, quindi $\forall x \in A, \ \exists B_{d_1} (x,r) \subseteq A$.
		Ora, se $d_2(x,y) < r / k$, si ha 
		\[
		d_1(x,y) \le k d_2(x,y) < r \implies y \in B_{d_1} (x,r)
		\] 
		da cui si conclude che $B_{d_2} (x,r / k) \subseteq B_{d_1} (x,r) \subseteq A$, e, allora, $A$ \`e aperto anche rispetto a $d_2$, cio\`e $A \in \tau _2$.
		
	\end{proof}
\noindent Se $d(x,y) = r\overline{d}(x,y)$, per $r>0$, si hanno due distanze equivalenti perch\'e, evidentemente, $\forall \varepsilon >0$:
\[
B_\varepsilon (x) = \overline{B}_{r\varepsilon } (x)
\] 
cio\`e le due distanze $d,\overline{d}$ identificano le stesse palle aperte, quindi gli stessi insiemi aperti.
\begin{corollario}
	Siano $d_1,d_2$ due distanze su $X$ per cui $\exists h,k>0$ tali che 
	\[
	d_1(x,y) \le kd_2(x,y) \hspace{1cm} d_2(x,y) \le hd_1(x,y)
	\] 
	$\forall x,y \in X$; allora $d_1$ e $d_2$ sono topologicamente equivalenti.
\end{corollario}
	\begin{proof}
		\`E sufficiente applicare la proposizione \ref{dfin} due volte: dalla prima maggiorazione, si ha $\tau _1 \subseteq \tau _2$, mentre dalla seconda si ha $\tau _2 \subseteq \tau _1$, quindi $\tau _1 = \tau _2$.
	\end{proof}
\noindent Due funzioni come le distanze $d_1,d_2$ della tesi si dicono \textit{bilipschitiziane} fra loro.
\begin{corollario}
In $\mathbb{R}^n$, le distanze 
\begin{equation}
	\begin{split}
		&d_2(x,x') = \left\lVert x-x' \right\rVert \equiv \sqrt{\sum_{i=1}^{n} (x_i-x'_i)^2} \\
		&d_1(x,x') = \sum_{i=1}^{n} \lvert x-x_i \rvert \\
		&d_{\infty}(x,x') = \max_{i} \left\{ \lvert x_i-x'_i \rvert  \right\} 
	\end{split}
\end{equation}
sono equivalenti e si ha
\begin{equation}
	d_{\infty} (x,x') \le d_2(x,x') \le d_1 (x,x') \le n d_\infty(x,x')
\end{equation}
\end{corollario}
\begin{proof}
	La prima disuguaglianza \`e giustificata da:
	\[
	d_2 (x,x') = \sqrt{\sum_{i=1}^{n} (x_i -x'_i)^2}  \ge \sqrt{\max_i  \left\{ (x_i-x'_i)^2 \right\} } = \max_i \left\{ \lvert x-x'_i \rvert  \right\} = d_\infty(x,x')
	\] 
	La seconda, invece, \`e vera perch\'e:
	\[
	\left[ d_2(x,x') \right] ^2 = \sum_{i=1}^{n} (x-x'_i)^2 \le \left[ \sum_{i=1}^{n} \lvert x_i-x'_i \rvert  \right] ^2 = \left[ d_1(x,x') \right]^2 
	\] 
L'ultima disuguaglianza \`e immediata.	
\end{proof}
\noindent Da questo segue direttamente che\footnote{Apparentemente, la distanza pi\`u grande dovrebbe includere pi\`u elementi, quindi i simboli $\supset$ dovrebbero essere dei $\subset$, invece, avendo fissato il raggio $\varepsilon $, quella che permette di creare la palla pi\`u grande \`e la distanza pi\`u piccola perch\'e \textit{avvicina} i punti tra di loro, quindi pi\`u elementi rientreranno in tale raggio.}
\begin{equation}
	B^{(\infty)} _\varepsilon (x) \supset B^{(2)} _\varepsilon (x) \supset B^{(1)} _{\varepsilon } (x) \supset B^{(\infty)} _{\varepsilon  / n} (x)
\end{equation}
Questo mostra che se $A$ \`e aperto rispetto ad una distanza, lo \`e anche rispetto alle altre.
\begin{osservazione}
Non tutte le distanze su uno spazio sono equivalenti. 
Ad esempio, considerando $d_1$ e $d_\infty$ su $C([0,1])$, si ha, per $M = \left\lVert f \right\rVert _\infty$:
\[
	\left\lVert f \right\rVert _1 = \int_{0} ^1 \lvert f \rvert\ d x \le  \int_{0} ^1 M \ dx = M = \left\lVert f \right\rVert _\infty
\] 
Quindi $\tau _1 \subseteq \tau _\infty$.
Si nota anche che l'insieme $B_\infty(0,1)$, cio\`e l'insieme delle funzioni che si discostano dalla funzione identicamente nulla al massimo $1$, risulta aperto per $d_\infty$, ma non per $d_1$.
Infatti, la funzione
\[
f(x) = \begin{cases}
	2 - 2x / \varepsilon &,\ x \in [0,\varepsilon ]\\
	0 &,\ \text{altrimenti}
\end{cases}
\] 
ha integrale $\varepsilon $, ma $\left\lVert f \right\rVert _\infty = 2$, cio\`e $f \not \in B_\infty(0,1)$, quindi $B_1(0,\varepsilon ) \not\subseteq B_\infty(0,1)$.
\end{osservazione}
\begin{definizione}
	[Limitatezza]
	Sia $(X,d)$ uno spazio metrico e $Y \subseteq X$; allora $Y$ \`e detto \textit{limitato} se esistono $x \in X, \ R \in \mathbb{R}$ tali per cui $Y \subseteq B_R(x)$.
\end{definizione}
\begin{prop}
	Sia $(X,d)$ uno spazio metrico; allora esiste una metrica $d'$ su $X$ tale che $d$ e $d'$ sono equivalenti e $d'(x,y)\le 1, \ \forall x,y \in X$.
	Inoltre, $X$ risulta limitato in $(X,d')$.
\end{prop}
	\begin{proof}
		Si definisce 
		\[
		d'(x,y) = \min\left\{ d(x,y) , 1 \right\} 
		\] 
		Per verificare che sono equivalenti, si osserva che:
		\begin{itemize}
			\item $\tau _{d'} \subseteq \tau _d$ perch\'e $d'(x,y) \le 1 \cdot d(x,y)$ per definizione;
			\item $\tau _{d} \subseteq \tau _{d'} $ perch\'e se $A$ \`e un aperto di $(X,d)$, allora $\forall x \in A, \ \exists \varepsilon _x > 0 : B_d(x,\varepsilon _x) \subseteq A$, per cui
				\[
				A = \bigcup _{x \in A} B_d(x,\varepsilon _x)
				\] 
				ma prendendo $\varepsilon _x ' = \min \left\{ \varepsilon _x , 1 \right\} $, si trova che
				\[
				A = \bigcup _{x \in A}  B_d (x,\varepsilon '_x) = \bigcup _{x \in A} B_{d'} (x,\varepsilon '_x)
				\] 
				cio\`e $A$ \`e aperto anche per $\tau _{d'} $.
		\end{itemize}
	\end{proof}
\subsubsection{Chiusura e parte interna}
\begin{definizione}
	[Chiusura]
	Sia $(X,\tau )$ uno spazio topologico e sia $Y\subseteq X$ un suo sottoinsieme.
	Si definisce la \textit{chiusura} di $Y$ in $X$ come il pi\`u piccolo chiuso che contiene $Y$, ossia
	\[
		\overline{Y}=\bigcap_{\substack{Y \subseteq C\\ C \text{ chiuso}}} C
	\] 
\end{definizione}
\begin{definizione}
	[Parte interna]
Sia $(X,\tau )$ uno spazio topologico e $Y\subseteq X$ un suo sottoinsieme.
Si definisce \textit{parte interna} di $Y$ in $X$ come il pi\`u grande aperto contenuto in $Y$, ossia
\[
	\operatorname{Int} Y = \bigcup_{\substack{A \in \tau \\ A\subseteq Y}} A
\] 
\end{definizione}
\noindent Visto che l'intersezione arbitraria di chiusi \`e chiusa e l'unione arbitraria di aperti \`e aperta, le definizioni di parte interna e chiusura sono sensate.

\begin{osservazione}
	Valgono le due seguenti relazioni:
\begin{equation}
	\begin{split}
		&\operatorname{Int} Z = X \setminus \overline{(X\setminus Z)}\\
		&\overline{Z} = X \setminus \operatorname{Int} (X \setminus Z)
	\end{split}
\end{equation}
Inoltre, $Z$ coincide con la sua parte interna se e solo se \`e aperto, mentre coincide con la sua chiusura se e solo se \`e chiuso.
\end{osservazione}
\begin{definizione}
	[Frontiera]
	La \textit{frontiera} di un insieme $Z \subset X$ \`e definita come
	\[
	\partial Z = \overline{Z }\setminus\operatorname{Int} Z
	\] 
\end{definizione}
\begin{definizione}
	[Punti aderenti e di accumulazione]
	Siano $P \in X$ e $Z \subseteq X$. Si dice che:
\begin{itemize}
	\item $P$ \`e \textit{aderente} a $Z$ se $P\in \overline{Z}$;
	\item $P$ \`e \textit{di accumulazione} per $Z$ se $P \in \overline{Z \setminus\left\{ P \right\} }$.
\end{itemize}
\end{definizione}
\begin{prop}
	Sia $X$ uno spazio topologico e $Z \subseteq X$; allora $P \in \overline{Z}$ se e solo se $\forall A \subseteq X$ aperto \`e tale che $P \in A\implies A\cap Z \neq \varnothing$.
\end{prop}
	\begin{proof}
		Da dimostrare\ldots
	\end{proof}










\end{document}
