%! TEX program = lualatex
\documentclass[11pt]{scrartcl}
% Packages
%\usepackage[margin=1.5in]{geometry}
\usepackage{index}
\usepackage{amsbsy} % Bold math symbols
\makeindex
%\usepackage[utf8]{inputenc}
\usepackage{tcolorbox}
\tcbuselibrary{theorems}
\tcbuselibrary{skins}
\tcbuselibrary{breakable}
\usepackage{varwidth}
\usepackage{textcomp}
\usepackage{amsmath,amssymb}
\usepackage{esint}
\usepackage{titlesec}
\usepackage{xcolor}
\usepackage{titling}
\usepackage[linktocpage]{hyperref}
\usepackage{pgfplots}
\usepackage{multicol}
\setlength{\columnsep}{2em}
\usepackage{caption}
\usepackage{amsthm}
\usepackage{import}
\usepackage{cancel}
\usepackage{caption}
\usepackage{nicematrix}
%\usepackage{parskip}
\usepackage{enumerate}
\usepackage{graphicx}
\usepackage[italian]{babel}
\usepackage{setspace}
\setstretch{1.2}
% To reset footnote numbering each page
\usepackage[perpage]{footmisc}
\usepackage{faktor}
\usepackage{tikz-cd}
\hypersetup{colorlinks,breaklinks, linkcolor=[RGB]{133,68,66}}
\definecolor{mastercolor}{HTML}{854442}
\definecolor{nred}{HTML}{bf0040}


% Titles 
\title{Esercizi di Algebra 1}
\author{Manuel Deodato}
\date{}




\newtheoremstyle{style}% name of the style to be used
{5pt}% measure of space to leave above the theorem. E.g.: 3pt
{5pt}% measure of space to leave below the theorem. E.g.: 3pt
{\normalfont}% name of font to use in the body of the theorem
%{15pt}% measure of space to indent
{0pt}% measure of space to indent
{\noindent\bfseries}% name of head font
{}% punctuation between head and body
{ }% space after theorem head; " " = normal interword space
{\thmname{#1}\thmnumber{ #2}{\thmnote{ (#3)}.\ }}


\theoremstyle{style}
\newtheorem{esempio}{Esempio}[section]
\newtheorem{definizione}{Definizione}[section]
\newtheorem{prop}{Proposizione}[section]
\newtheorem{teorema}{Teorema}[section]
\newtheorem{lemma}{Lemma}[teorema]
\newtheorem{corollario}{Corollario}[teorema]
\newtheorem{osservazione}{Osservazione}[section]
\newtheorem{notazione}{Notazione}[section]
\newtheorem{esercizio}{Esercizio}[section]
\newenvironment{svolgimento}{\renewcommand\qedsymbol{$\blacksquare$}\begin{proof}[Svolgimento]}{\end{proof}}




%% Generic box
\newtcolorbox{eqbox}[1][]
{
colback=gray!10,
arc=0pt,
boxrule=0pt,
title=#1
}

 \newenvironment{boxenv}[1][]{
    \begin{eqbox}[#1]
    }{
   \end{eqbox}
}



%Captions
\captionsetup[figure]{font=footnotesize,labelfont=footnotesize}
\captionsetup[table]{font=footnotesize,labelfont=footnotesize}
%Titlesec
\titleformat{\section}
{\fontsize{20}{20}\scshape}
{{\color{mastercolor}\fontsize{30}{20}\selectfont\thesection}}
{0.7em}
{}
\titlespacing*{\section}{0pt}{*2}{1cm}
\titlespacing*{\subsection}{0pt}{*5}{.25cm}
\titlespacing*{\subsubsection}{0pt}{*4.5}{.25cm}

% Personalizza la formattazione della subsection
\titleformat{\subsection}[block]{\fontsize{15}{20}\bfseries}{\thesubsection}{.5em}{}


% Personalizza la formattazione della subsubsection
\titleformat{\subsubsection}[block]{\fontsize{13}{10}\bfseries}{\thesubsubsection}{.5em}{}

% Maketitle customization
\renewcommand{\maketitle}{
\begin{center}
{\sffamily
{\fontsize{20}{20}\selectfont\MakeUppercase{\thetitle}}}

\vspace{0.2in}

{\large\MakeUppercase{\theauthor}}
\end{center}
}

%Evaluate symbol
\DeclareMathOperator{\di}{d\!}
\newcommand*\Eval[3]{\left.#1\right\rvert_{#2}^{#3}}

%%%%%%% Numero delle equazioni in formato a.b
\numberwithin{equation}{subsection}
%%%%%

%%%%%%%%%% Personalizzazione numeri lista
\renewcommand{\theenumi}{(\arabic{enumi})}

%%%% Table of contents

\usepackage[titles]{tocloft}

\renewcommand{\cftdot}{}
\usepackage{titletoc}
%\setcounter{tocdepth}{2}

%%%%%%%%%%%%%%%% Toc style

% Personalizzazione scritta indice


% Font
\renewcommand{\textbf}[1]{\textsf{\bfseries #1}}
\usepackage{fontspec}
\usepackage{unicode-math}
\usepackage[default]{fontsetup}



%%% Hook
\newcommand{\longhookrightarrow}{\lhook\joinrel\longrightarrow}


\begin{document}
\maketitle
\newpage
\tableofcontents
\newpage
\section{Gruppi}
\subsection{Lezione 3 [10-10-2023]}
Si inizia col dimostrare il teorema di Cauchy e il piccolo teorema di Fermat usando le azioni di gruppo.
\begin{teorema}
	[Teorema di Cauchy]
	Sia $G$ un gruppo finito, con $p$ primo tale che $p  \mid \lvert G \rvert $; allora $\exists x \in G : \operatorname{ord}(x) =p$.
\end{teorema}
	\begin{proof}
		Si considera l'azione di $\mathbb{Z} / p\mathbb{Z}$ sull'insieme
		\[
		X = \left\{ (g_1,\ldots,g_p) \in G^p  \ \bigg  \lvert \ \prod_{i=1} ^p g_i = e_G\right\} 
		\] 
		dove l'elemento $i \in \mathbb{Z}/ p\mathbb{Z}$ agisce mandando
		\[
			(g_1,\ldots,g_p)\longmapsto (g_{1+i},g_{2+i}  ,\ldots,g_{p+i} )
		\] 
		dove l'indice di ciascun $g_i$ \`e letto modulo $p$. Ad esempio, l'elemento $1 \in \mathbb{Z}/p\mathbb{Z}$ agisce come 
		\[
			(g_1,\ldots,g_p)\longmapsto (g_{2},g_{3}  ,\ldots,g_{p+1} )
		\] 
		Da questa definizione, \`e facile convincersi un'azione corrisponde ad una rotazione delle componenti di ogni $p$-upla di $X$, pertanto il prodotto restituisce sempre $e_G$, quindi \`e ben definita come biezione di $X$.
		\begin{osservazione}
		Si pu\`o osservare che se $i \in \mathbb{Z} /p \mathbb{Z}$ agisce su una $p$-upla, essendo che 
		\[
		e_G= g_1 \cdots g_p=(g_1 g_2 \cdots g_i) g_{i+1} \cdots g_p \implies g_1 g_2 \cdots g_i = (g_{i+1} \cdots g_p)^{-1}
		\] 
		quindi, a seguito della rotazione tramite $i$, si ha il prodotto
		\[
			(g_{i+1} \cdots g_p) (g_1 g_2 \cdots g_i)= e_G
		\] 
		\end{osservazione}
		Si nota immediatamente che $\lvert X \rvert = n^{p-1} $ perch\'e ogni componente della $p$-upla pu\`o essere scelta arbitrariamente tra gli $n$ elementi di $G$, mentre l'ultima, la $p$-esima, \`e fissata dalla condizione che sia l'inverso del prodotto delle $p-1$ componenti precedenti.
		
		Ora si studiano le orbite dell'azione.
		Per il teorema di orbita-stabilizzatore
		\[
			\lvert \mathrm{Orb} (x) \rvert  \mid \lvert \mathbb{Z}/ p \mathbb{Z} \rvert \implies \lvert \mathrm{Orb} (x)  \rvert  = \left\{ 1,p \right\} 
		\] 
		Le orbite di lunghezza $1$ sono date da tutti gli elementi di $X$ che hanno ogni componente uguale perch\'e sotto rotazione di ogni $i \in \mathbb{Z}/p\mathbb{Z}$ non devono cambiare.
		Un elemento $g \in G$ che ha un corrispondente vettore in $X$ con tutte le componenti uguali deve necessariamente soddisfare
		\[
			e_G=\underbracket{g g \cdots g}_{p \text{ volte}} = g^p \implies \operatorname{ord}(g) \in \left\{ 1,p \right\} 
		\] 
		Un'orbita del genere esiste sicuramente ed \`e data proprio dall'elemento neutro di $G$, $e_{G} $ ed \`e corrispondente proprio a $\operatorname{ord}(g) =1$; poi le altre eventuali orbite del genere sono date dagli elementi di $G$ che hanno ordine $p$.
		L'idea \`e dimostrare che ne esiste almeno uno.
		Ora, visto che le orbite partizionano l'insieme, si ha:
		\[
		\lvert X \rvert  = \bigsqcup_{x \in X} \mathrm{Orb} (x) \implies \lvert X \rvert  = \sum_{x \in \mathcal{R} }^{} \lvert \mathrm{Orb}(x)  \rvert 
		\] 
		dove $\mathcal{R} $ \`e l'insieme dei rappresentanti delle orbite.
		La somma si pu\`o spezzare separando le orbite che hanno lunghezza $1$, da quelle che hanno lunghezza $p$:
		\[
			\lvert X \rvert  = 1 + \left\{ \substack{\displaystyle \text{elementi di}\\\displaystyle  \text{ordine } p} \right\} + p \cdot \# \left\{ \substack{\displaystyle \text{elementi con}\\ \displaystyle \text{orbita lunga } p} \right\}
		\] 
		da cui, passando in modulo $p$, si ottiene che
		\[
		n^{p-1} - 1 \equiv  \# \left\{ \substack{\displaystyle \text{elementi di}\\\displaystyle  \text{ordine } p} \right\} \pmod{p} 
		\] 
		Per assunzione, per\`o, $p  \mid  n$, per cui $n^{p-1} - 1 \equiv -1 \pmod{p} $ e, pertanto
		\[
		 \# \left\{ \substack{\displaystyle \text{elementi di}\\\displaystyle  \text{ordine } p} \right\}\equiv -1 \pmod{p}
		\] 
	Ma questo significa che il numero di elementi di ordine $p$ non \`e nullo perch\'e $0 \not \equiv -1 \pmod{p} $.
	\end{proof}
In maniera del tutto analoga si dimostra il piccolo teorema di Fermat.
\begin{teorema}
	[Piccolo teorema di Fermat]
	Sia $n \in \mathbb{Z}$ un intero non divisibile per $p$; allora $n^{p-1} \equiv 1 \pmod{p} $.
\end{teorema}
	\begin{proof}
		Si considera $G = \mathbb{Z}/ n \mathbb{Z}$, con $p  \nmid n$ e 
		\[
		X = \left\{ (g_1,\ldots,g_p) \in G^p  \mid g_1+ \ldots+ g_p = 0 \right\} 
		\] 
		Allora si considera l'azione di $\mathbb{Z} / p \mathbb{Z}$ su $X$ come sopra e, analogamente, si ha $\lvert X \rvert  = n^{p-1} $.
		Visto che $p \nmid n$, non ci possono essere elementi di ordine $p$ in $G$ e, quindi, vi \`e un'unica orbita di ordine $1$ data dall'elemento neutro $0$.
		Ne segue che:
		\[
		n^{p-1} = \lvert X \rvert  = 1 + p \cdot \# \left\{ \substack{\displaystyle \text{elementi con}\\ \displaystyle \text{orbita lunga } p} \right\} \equiv 1 \pmod{p} 
		\] 
		da cui la tesi.
	\end{proof}
	Il seguente teorema ha come conseguenza il fatto che se un sottogruppo di un gruppo finito ha indice $2$, allora \`e normale.
	\begin{boxenv}[]
	\begin{teorema}
	Sia $G$	un gruppo finito di ordine $n$ e sia $N<G$. 
	Se $[G:N]=p$, con $p$ il pi\`u piccolo primo che divide $n$, allora $N \lhd G$.
	\end{teorema}
	\end{boxenv}
\begin{proof}
	Si considera l'azione di $G$ sul quoziente $G / N$ data da 
	\[
	g' \cdot (gN) = g'gN
	\] 
	Si pu\`o dimostrare facilmente che questa \`e una buona azione e, quindi, si ha un omomorfismo $G \stackrel{\phi }{\longrightarrow } S(G/N) \cong S_p$, visto che $\lvert G / N \rvert = p$ per assunzione. 
Si vuole dimostrare che il suo nucleo coincide con $N$, da cui $N \lhd G$.

Si inizia col notare che $\lvert \mathrm{Im} (\phi ) \rvert  \mid \lvert S(G/N) \rvert = \lvert S_p \rvert =p!$; allo stesso tempo, per il primo teorema di omomorfismo, si ha 
\[
	\frac{G}{\mathrm{Ker} (\phi )}\cong \mathrm{Im} (\phi )\implies \lvert \mathrm{Im} (\phi ) \rvert  \mid \frac{\lvert G \rvert  }{\lvert \mathrm{Ker} (\phi ) \rvert }\implies \lvert \mathrm{Im} (\phi ) \rvert  \mid \lvert G \rvert 
\] 
Visto che $\lvert \mathrm{Im} (\phi ) \rvert  $ deve dividere $p!$, che contiene tutti primi minori o pari a $p$, e deve dividere anche $\lvert G \rvert $, che contiene tutti primi maggiori o uguali a $p$, significa che $\lvert \mathrm{Im} (\phi ) \rvert = \left\{ 1,p \right\} $.
Per\`o non pu\`o essere $|\mathrm{Im} (\phi )| =1$ perch\'e, prendendo $g \in G \setminus N$ e prendendo $n \in N$, si ottengono due mappe $\phi _g,\phi _n \in S(G / N)$ diverse fra loro: $\phi _g (N) = gN \neq N = nN=\phi _n(N)$.
Allora $\lvert \mathrm{Im} (\phi ) \rvert =p = [G : \mathrm{Ker} (\phi )] = [ G : N ]$, quindi $\mathrm{Ker} (\phi )$ e $ N$ hanno stessa cardinalit\`a in un gruppo finito $G$.
Per concludere che $\mathrm{Ker}( \phi) = N$, quindi che $N \lhd G$, \`e sufficiente mostrare un'inclusione; a questo proposito, si nota che se $g \in \mathrm{Ker} (\phi )$, allora $g \cdot N = gN = N \iff g \in N$, cio\`e $\mathrm{Ker} (\phi ) \subseteq N\implies \mathrm{Ker}( \phi ) = N$.
\end{proof}	
	Facendo uso di questo risultato, \`e possibile dimostrare che ogni gruppo di ordine $15$ \`e ciclico.	
\begin{prop}
	Ogni gruppo $G$ di ordine $15$ \`e ciclico.
\end{prop}
\begin{proof}
	Si dimostra tramite i seguenti punti.
	\begin{enumerate}[(a).]
		\item $\exists N \lhd G$ tale che $\lvert N \rvert  = 5$.
		\item $N \subseteq Z(G)$.
		\item $G$ abeliano $\Rightarrow  G $ ciclico.
	\end{enumerate}
	Il punto (a) si dimostra direttamente applicando il teorema di Cauchy e il teorema appena visto; dal primo, si conclude che $\exists g \in G: \langle g \rangle= N < G$ tale che $\lvert N \rvert  = 5=\operatorname{ord}(g)$, mentre dal teorema precedente, visto che $\lvert G \rvert  / \lvert N \rvert  = 3$, che \`e il pi\`u piccolo primo che divide $\lvert G \rvert $, si conclude che $N$ \`e normale in $G$.

	Per il punto (b), $N$ \`e normale in $G$, quindi la mappa 
	\[
\phi : 
		\begin{array}
			{c c c}
			\mathrm{Int} (G)& \longrightarrow &\mathrm{Aut} (N)\\
			\varphi _x &\longmapsto& \varphi _x |_{N} 
		\end{array}
	\] 
	\`e ben definita.
	Allora basta mostrare che che $\mathrm{Im} (\phi ) = \left\{ \mathrm{Id}  \right\} $ per far vedere che $N$ \`e normale.
	Intanto si ricorda che $\mathrm{Int} (G) \cong G / Z(G)$, quindi $\lvert \mathrm{Int} (G) \rvert  \mid 15$; inoltre $\mathrm{Aut} (N) \cong \mathrm{Aut} (\mathbb{Z}/5\mathbb{Z}) \cong (\mathbb{Z}/5\mathbb{Z})^* \cong \mathbb{Z}/4\mathbb{Z}$.
	Ma allora $|\mathrm{Im} (\phi )|  \mid (4,15)  \mid  1$, da cui $\mathrm{Im} (\phi ) = \left\{ \mathrm{Id}  \right\} $ e, quindi, $N \subseteq Z(G)$.

	Infine, per il punto (c), si pu\`o osservare che $\lvert G / Z(G) \rvert = \left\{ 1,3 \right\} $ perch\'e $N \subseteq Z(G) \implies \lvert Z(G) \rvert \ge 5$, da cui $G / Z(G)$ \`e ciclico in entrambi i casi; ricordando che $G / Z(G)$ ciclico $\Rightarrow  G$ abeliano, si conclude la dimostrazione.
\end{proof}
\begin{center}
	Continuare 56:00\ldots
\end{center}
































\end{document}
