% !TeX program = lualatex
\documentclass[12pt]{article}
% Packages
%\usepackage[margin=1.5in]{geometry}
\usepackage{index}
\usepackage{amsbsy} % Bold math symbols
\makeindex
%\usepackage[utf8]{inputenc}
\usepackage[T1]{fontenc}
\usepackage{unicode-math}
\usepackage{fontspec}
\usepackage{tcolorbox}
\tcbuselibrary{theorems}
\tcbuselibrary{skins}
\tcbuselibrary{breakable}
\usepackage{varwidth}
\usepackage{textcomp}
\usepackage{amsmath}
\usepackage{esint}
\usepackage{titlesec}
\usepackage{xcolor}
\usepackage{titling}
\usepackage[linktocpage]{hyperref}
\usepackage{pgfplots}
\usepackage{multicol}
\setlength{\columnsep}{2em}
\usepackage{caption}
\usepackage{amsthm}
\usepackage{import}
\usepackage{cancel}
\usepackage{caption}
\usepackage{nicematrix}
%\usepackage{parskip}
\usepackage{enumerate}
\usepackage{graphicx}
\usepackage[italian]{babel}
\usepackage{setspace}
\setstretch{1.2}
% To reset footnote numbering each page
\usepackage[perpage]{footmisc}
\usepackage{faktor}
\usepackage{tikz-cd}
\hypersetup{colorlinks,breaklinks, linkcolor=[RGB]{133,68,66}}
\definecolor{mastercolor}{HTML}{854442}
\definecolor{nred}{HTML}{bf0040}


% Titles 
\title{Appunti di\\ \vspace{.3cm} Analisi 3}
\author{Manuel Deodato}
\date{}




\newtheoremstyle{style}% name of the style to be used
{5pt}% measure of space to leave above the theorem. E.g.: 3pt
{5pt}% measure of space to leave below the theorem. E.g.: 3pt
{\normalfont}% name of font to use in the body of the theorem
%{15pt}% measure of space to indent
{0pt}% measure of space to indent
{\noindent\bfseries}% name of head font
{}% punctuation between head and body
{ }% space after theorem head; " " = normal interword space
{\thmname{#1}\thmnumber{ #2}{\thmnote{ (#3)}.\ }}


\theoremstyle{style}
\newtheorem{esempio}{Esempio}[section]
\newtheorem{definizione}{Definizione}[section]
\newtheorem{prop}{Proposizione}[section]
\newtheorem{teorema}{Teorema}[section]
\newtheorem{lemma}{Lemma}[teorema]
\newtheorem{corollario}{Corollario}[teorema]
\newtheorem{osservazione}{Osservazione}[section]
\newtheorem{notazione}{Notazione}[section]
\newtheorem{esercizio}{Esercizio}[section]
\newenvironment{svolgimento}{\renewcommand\qedsymbol{$\blacksquare$}\begin{proof}[Svolgimento]}{\end{proof}}




%% Generic box
\newtcolorbox{eqbox}[1][]
{
colback=gray!10,
arc=0pt,
boxrule=0pt,
title=#1
}

 \newenvironment{boxenv}[1][]{
    \begin{eqbox}[#1]
    }{
   \end{eqbox}
}



%Captions
\captionsetup[figure]{font=footnotesize,labelfont=footnotesize}
\captionsetup[table]{font=footnotesize,labelfont=footnotesize}
%Titlesec
\titleformat{\section}
{\fontsize{20}{20}\scshape}
{{\color{mastercolor}\fontsize{30}{20}\selectfont\thesection}}
{0.7em}
{}
\titlespacing*{\section}{0pt}{*2}{1cm}
\titlespacing*{\subsection}{0pt}{*5}{.25cm}
\titlespacing*{\subsubsection}{0pt}{*4.5}{.25cm}

% Personalizza la formattazione della subsection
\titleformat{\subsection}[block]{\fontsize{15}{20}\bfseries}{\thesubsection}{.5em}{}


% Personalizza la formattazione della subsubsection
\titleformat{\subsubsection}[block]{\fontsize{13}{10}\bfseries}{\thesubsubsection}{.5em}{}

% Maketitle customization
\renewcommand{\maketitle}{
\begin{center}
{\sffamily
{\fontsize{20}{20}\selectfont\MakeUppercase\thetitle}}

\vspace{0.2in}

{\large\scshape\theauthor}
\end{center}
}

%Evaluate symbol
\DeclareMathOperator{\di}{d\!}
\newcommand*\Eval[3]{\left.#1\right\rvert_{#2}^{#3}}

%%%%%%% Numero delle equazioni in formato a.b
\numberwithin{equation}{subsection}
%%%%%

%%%%%%%%%% Personalizzazione numeri lista
\renewcommand{\theenumi}{(\arabic{enumi})}

%%%% Table of contents

\usepackage[titles]{tocloft}

\renewcommand{\cftdot}{}
\usepackage{titletoc}
%\setcounter{tocdepth}{2}

%%%%%%%%%%%%%%%% Toc style

% Personalizzazione scritta indice


% Font
\usepackage{microtype}
\renewcommand{\textbf}[1]{\textsf{\bfseries #1}}
\usepackage{newcomputermodern}

%%% Hook
\newcommand{\longhookrightarrow}{\lhook\joinrel\longrightarrow}


\begin{document}
\maketitle
\newpage
\tableofcontents
\newpage
\section{Teoria della misura}
\subsection{Introduzione}
Si cerca un modo per misurare i sottoinsiemi di $\mathbb{R}^d$ (la maggior parte).
La misura di questi corrisponde al volume, per $d=3$, l'area per $d=2$, eccetera.

L'idea principale \`e quella di approssimare i sottoinsiemi dei quali si vuole ottenere la misura tramite l'unione di altri insiemi la cui geometria \`e nota e di cui, dunque, si sa gi\`a calcolare il volume.
Perci\`o, si useranno gli intervalli $[a,b]$ per sottoinsiemi di $\mathbb{R}$, la cui misura \`e data da $b- a$, mentre per il caso generale di $\mathbb{R}^d$, si user\`a il prodotto di intervalli.
\subsubsection{Definizioni preliminari}

Un punto $x \in \mathbb{R}^d$ \`e indicato tramite la $d$-upla $x = (x_1,\ldots,x_d)$ e la sua norma \`e data da
\[
\lvert x \rvert  = \sqrt{x_1^2 + \ldots+ x_d^2} 
\] 
La distanza euclidea che ne deriva tra due punti $x,y \in \mathbb{R}^d$ \`e, quindi, indicata tramite $\lvert x-y \rvert $. 
\begin{definizione}
	[Distanza tra insiemi]
	Siano $E,F \subseteq \mathbb{R}^d$; la loro distanza \`e definita come
	\[
		d(E,F) = \inf_{\substack{x \in E\\ y \in F}} \lvert x - y \rvert 
	\] 
\end{definizione}
Una palla aperta di centro $x$ e raggio $r$ in $\mathbb{R}^d$ si indica con
\[
B_r(x) = \left\{ y \in \mathbb{R}^d  \mid \lvert x-y \rvert <r \right\} 
\] 
\begin{definizione}
	[Insieme aperto]
	Sia $E\subseteq \mathbb{R}^d$; si dice che $E$ \`e \textit{aperto} se $\forall x \in E, \ \exists B_\epsilon (x) \subset E$.
\end{definizione}
Per definizione, un insieme si dice \textit{chiuso} se il suo complementare \`e aperto, mentre \`e \textit{limitato} se pu\`o essere contenuto in una qualche palla aperta.
\begin{definizione}
	[Insieme compatto]
	Un insieme $E \subseteq \mathbb{R}^d$ tale che $E \subseteq \bigcup_{i \in I} U_i$ \`e compatto se $\exists J \subset I$ finito tale che $E \subseteq \bigcup_{j \in J} U_j $.
\end{definizione}
Per il teorema di Heine-Borel, ogni insieme chiuso e limitato \`e anche compatto.

	Un punto $x \in \mathbb{R}^d$ \`e detto \textit{di accumulazione} per $E \subseteq \mathbb{R}^d$ se $\forall r > 0$, la palla $B_r(x)$ contiene almeno un punto di $E$.
	Questo significa che ci sono punti di $E$ arbitrariamente vicini a $x$.
Un punto $x \in E$, invece, \`e detto \textit{isolato} se $\exists r > 0 : B_r(x) \cap E \left\{ x \right\} $.
L'insieme $E$ si dice \textit{perfetto} se non contiene punti isolati.
\subsubsection{Cubi e rettangoli}
\begin{definizione}
	[Rettangolo chiuso]
	Un rettangolo chiuso in $\mathbb{R}^d$ \`e ottenuto dal prodotto di $d$ intervalli unidimensionali chiusi e limitati:
	\[
		R = [a_1,b_1] \times \ldots [a_d,b_d]
	\] 
	con $a_j,b_j, \ j=1,\ldots,d$ sono numeri reali. 
	Questo si pu\`o scrivere come:
	\[
	R = \left\{ (x_1,\ldots,x_d) \in \mathbb{R}^d  \mid  a_j \le x_j \le b_j, \ j =1,...,d\right\} 
	\] 
\end{definizione}
Per questa definizione, il rettangolo \`e, appunto, chiuso e i suoi lati sono paralleli agli assi; il suo volume \`e dato da
\[
\lvert R  \rvert = \prod_{j=1} ^{d} b_j - a_j
\] 
Un rettangolo aperto, invece, \`e ottenuto come prodotto cartesiano di intervalli unidimensionali aperti $(a_1,b_1) \times  \ldots \times (a_d,b_d)$.
\begin{definizione}
	Un'unione di rettangoli \`e detta \textit{quasi disgiunta} se le parti interne di ciascuno di questi rettangoli sono disgiunte.
\end{definizione}
\begin{lemma}
	Se un rettangolo \`e l'unione quasi disgiunta di un numero finito di altri rettangoli, quindi della forma $R = \bigcup_{k=1} ^N R_k$, allora 
	\[
	\lvert R \rvert  = \sum_{k=1}^{N} \lvert R_k \rvert 
	\] 
	\begin{proof}
		\textit{Da scrivere\ldots}
	\end{proof}
\end{lemma}
\begin{lemma}
	Se $R, R_1,\ldots,R_N$ sono rettangoli e $R \subset \bigcup _{k=1} ^N R_k$, allora 
	\[
	\lvert R \rvert \le \sum_{k=1}^{N} \lvert R_k \rvert 
	\] 
	\begin{proof}
		\textit{Da scrivere\ldots}
	\end{proof}
\end{lemma}










\end{document}
