%! TEX program = lualatex
\documentclass[11pt]{article}
% Packages
%\usepackage[margin=1.5in]{geometry}
\usepackage{index}
\usepackage{amsbsy} % Bold math symbols
\makeindex
%\usepackage[utf8]{inputenc}
\usepackage{tcolorbox}
\tcbuselibrary{theorems}
\tcbuselibrary{skins}
\tcbuselibrary{breakable}
\usepackage{varwidth}
\usepackage{textcomp}
\usepackage{amsmath}
\usepackage{esint}
\usepackage{titlesec}
\usepackage{xcolor}
\usepackage{titling}
\usepackage[linktocpage]{hyperref}
\usepackage{pgfplots}
\usepackage{multicol}
\setlength{\columnsep}{2em}
\usepackage{caption}
\usepackage{amsthm}
\usepackage{import}
\usepackage{cancel}
\usepackage{caption}
\usepackage{nicematrix}
%\usepackage{parskip}
\usepackage{enumerate}
\usepackage{graphicx}
\usepackage[italian]{babel}
\usepackage{setspace}
\setstretch{1.2}
% To reset footnote numbering each page
\usepackage[perpage]{footmisc}
\usepackage{faktor}
\usepackage{tikz-cd}
\hypersetup{colorlinks,breaklinks, linkcolor=[RGB]{133,68,66}}
\definecolor{mastercolor}{HTML}{854442}
\definecolor{nred}{HTML}{bf0040}


% Titles 
\title{Appunti di\\ \vspace{.3cm} Analisi 3}
\author{Manuel Deodato}
\date{}




\newtheoremstyle{style}% name of the style to be used
{5pt}% measure of space to leave above the theorem. E.g.: 3pt
{5pt}% measure of space to leave below the theorem. E.g.: 3pt
{\normalfont}% name of font to use in the body of the theorem
%{15pt}% measure of space to indent
{0pt}% measure of space to indent
{\noindent\bfseries}% name of head font
{}% punctuation between head and body
{ }% space after theorem head; " " = normal interword space
{\thmname{#1}\thmnumber{ #2}{\thmnote{ (#3)}.\ }}


\theoremstyle{style}
\newtheorem{esempio}{Esempio}[section]
\newtheorem{definizione}{Definizione}[section]
\newtheorem{prop}{Proposizione}[section]
\newtheorem{teorema}{Teorema}[section]
\newtheorem{lemma}{Lemma}[teorema]
\newtheorem{corollario}{Corollario}[teorema]
\newtheorem{osservazione}{Osservazione}[section]
\newtheorem{notazione}{Notazione}[section]
\newtheorem{esercizio}{Esercizio}[section]
\newenvironment{svolgimento}{\renewcommand\qedsymbol{$\blacksquare$}\begin{proof}[Svolgimento]}{\end{proof}}




%% Generic box
\newtcolorbox{eqbox}[1][]
{
colback=gray!10,
arc=0pt,
boxrule=0pt,
title=#1
}

 \newenvironment{boxenv}[1][]{
    \begin{eqbox}[#1]
    }{
   \end{eqbox}
}



%Captions
\captionsetup[figure]{font=footnotesize,labelfont=footnotesize}
\captionsetup[table]{font=footnotesize,labelfont=footnotesize}
%Titlesec
\titleformat{\section}
{\fontsize{20}{20}\scshape}
{{\color{mastercolor}\fontsize{30}{20}\selectfont\thesection}}
{0.7em}
{}
\titlespacing*{\section}{0pt}{*2}{1cm}
\titlespacing*{\subsection}{0pt}{*5}{.25cm}
\titlespacing*{\subsubsection}{0pt}{*4.5}{.25cm}

% Personalizza la formattazione della subsection
\titleformat{\subsection}[block]{\fontsize{15}{20}\bfseries}{\thesubsection}{.5em}{}


% Personalizza la formattazione della subsubsection
\titleformat{\subsubsection}[block]{\fontsize{13}{10}\bfseries}{\thesubsubsection}{.5em}{}

% Maketitle customization
\renewcommand{\maketitle}{
\begin{center}
{\sffamily
{\fontsize{20}{20}\selectfont\MakeUppercase\thetitle}}

\vspace{0.2in}

{\large\scshape\theauthor}
\end{center}
}

%Evaluate symbol
\DeclareMathOperator{\di}{d\!}
\newcommand*\Eval[3]{\left.#1\right\rvert_{#2}^{#3}}

%%%%%%% Numero delle equazioni in formato a.b
\numberwithin{equation}{subsection}
%%%%%

%%%%%%%%%% Personalizzazione numeri lista
\renewcommand{\theenumi}{(\arabic{enumi})}

%%%% Table of contents

\usepackage[titles]{tocloft}

\renewcommand{\cftdot}{}
\usepackage{titletoc}
%\setcounter{tocdepth}{2}

%%%%%%%%%%%%%%%% Toc style

% Personalizzazione scritta indice


% Font
\renewcommand{\textbf}[1]{\textsf{\bfseries #1}}
\usepackage{fontspec}
\usepackage{unicode-math}
\usepackage[default]{fontsetup}



%%% Hook
\newcommand{\longhookrightarrow}{\lhook\joinrel\longrightarrow}


\begin{document}
\maketitle
\newpage
\tableofcontents
\newpage
\section{Teoria della misura}
\subsection{Introduzione}
	L'obiettivo \`e arrivare a costruire una funzione che permetta di misurare i sottoinsiemi di $\mathbb{R}^d$, o quantomeno la maggior parte, e una conseguente teoria dell'integrazione che abbia un buon comportamento rispetto al passaggio al limite.

	Per ottenere il volume di generici sottoinsiemi di $\mathbb{R}^d$ \`e opportuno partire da oggetti la cui geometria sia nota e \textit{rivestire} tali sottoinsiemi con questi oggetti in modo tale da approssimarne arbitrariamente bene la misura.
	A questo scopo, si definisce il seguente oggetto fondamentale.
	\begin{definizione}
		[Plurintervallo]
	Si definisce \textit{plurintervallo} un sottoinsieme di $I \subseteq \mathbb{R}^d$ tale per cui esistono degli intervalli $I_k \subseteq \mathbb{R}$ tali che
	\[
	I = \prod_{k=1} ^d I_k
	\] 
	dove il prodotto \`e il prodotto cartesiano.
	In altri termini, un plurintervallo $I$ \`e della forma
	\[
	I = \prod_{k=1} ^d (a_k,b_k)
	\] 
	con $-\infty < a_k < b_k <+\infty, \ \forall k$.
	\end{definizione}
	\begin{osservazione}
	Fondamentalmente, un plurintervallo \`e un rettangolo per $d=2$, un parallelepipedo per $d=3$, eccetera.
	\end{osservazione}
	La geometria di questi oggetti \`e nota perch\'e la loro misura\footnote{Cio\`e il loro volume per $d=3$, la loro area per $d=2$, eccetera.} \`e nota ed \`e data da:
	\[
	\lvert I \rvert  = \prod_{k=1} ^d (b_k - a_k) = \prod_{k=1} ^d \lvert I_k \rvert 
	\] 
	Per definire una misura, si parte col definire una misura esterna, cio\`e una funzione $\mu^*  : \mathcal{P} (\mathbb{R}^d) \to [0,+\infty]$ tale che
	\begin{enumerate}[(a).]
		\item $\mu^* (\varnothing)=0$;
		\item se $A \subseteq B \subseteq \mathbb{R}^d$, allora $\mu^* (A) \le \mu^* (B)$;
		\item data $\left\{ E_i \right\} _{i=1} ^{+\infty} $ famiglia numerabile di insiemi, vale
			\[
			\mu^* \left(\bigcup_{i=1} ^{+\infty} E_i\right) \le \sum_{i=1}^{+\infty} \mu^* (E_i)
			\] 
	\end{enumerate}
	Inoltre, si richiede che se $I \subseteq \mathbb{R}^d$ \`e un plurintervallo, allora $\mu^* (I) = \lvert I \rvert $.
	\subsection{Misura esterna}
	Si d\`a la seguente definizione.
	\begin{definizione}
		[Misura esterna di Lebesgue]
	Sia $E \subseteq \mathbb{R}^d$ e sia $S$ un suo ricoprimento, tale che
	\[
	E \subseteq \bigcup_{k=1} ^{+\infty} I_k
	\] 
	con $I_k \subseteq \mathbb{R}^d$ plurintervalli. Sia, inoltre
	\[
	\sigma (S) = \sum_{k=1}^{+\infty} \lvert I_k \rvert 
	\] 
	il volume totale\footnote{Cio\`e si conta anche il volume condiviso tra pi\`u plurintervalli.} del ricoprimento; allora si definisce la \textit{misura esterna} di $E$ come:
	\[
	\mu^* (E) := \inf_S \sigma 
	\] 
	\end{definizione}
Ai fini della teoria, si assume che la frontiera degli insiemi sia a misura nulla, cio\`e si dice che due plurintervalli $I_k, I_j \subseteq \mathbb{R}^d$ \textit{non sono sovrapposti} se
\[
	\mathring{I}_k \cap \mathring{I}_j = \varnothing, \ \text{ per } k \neq j
\] 
\begin{teorema}
	Sia $I \subseteq \mathbb{R}^d$ un plurintervallo; allora $\mu^* (I) = \lvert I \rvert $.
\end{teorema}
	\begin{proof}
		Evidentemente $I$ \`e il pi\`u piccolo ricoprimento di se stesso che, quindi, minimizza $\sigma (S)$, pertanto, per definizione, si ha $\mu^* (I) = \lvert I \rvert $.
	\end{proof}
\begin{teorema}
	Siano $A,B \subseteq \mathbb{R}^d$ tali che $A \subseteq B$; allora $\mu^* (A) \le \mu^* (B)$.
\end{teorema}
	\begin{proof}
		Applicando direttamente la definizione, si nota che:
		\[
		\mu^* (A) = \inf_{S_A} \sigma (S_A) \le \inf_{S_B} \sigma (S_B) = \mu^* (B)  
		\] 
		visto che ogni ricoprimento $S_B$ di $B$ ricopre anche $A$.
	\end{proof}
\begin{corollario}
	Siano $E \subseteq E' \subseteq \mathbb{R}^d$, con $\mu^* (E') = 0$; $\mu^* (E) = 0$.
\end{corollario}
\begin{teorema}
	Sia $E \subseteq \mathbb{R}^d$; allora $\forall \varepsilon >0, \ \exists G \subseteq \mathbb{R}^d$ aperto tale che $E \subset G$ e $\mu^* (G) < \mu^* (E) + \varepsilon $.
\end{teorema}
\begin{proof}
	Sia $\left\{ I_k \right\} _{k=1} ^{+\infty} $ una famiglia numerabile di plurintervalli chiusi di $\mathbb{R}^d$ tali che
	\[
	E \subset \bigcup _{k=1} ^{+\infty}I_k \hspace{1cm} \sum_{k=1}^{+\infty}\lvert I_k \rvert  \le \mu^* (E) + \varepsilon  
	\] 
	Allora si costruiscono dei nuovi intervalli $I_k^*$ tali che $I_k \subset \mathring{I}_k^*$ e $\lvert I_k^* \rvert \le \lvert I_k \rvert + \varepsilon / 2^{k} $; allora il relativo insieme $G$ aperto \`e dato da
	\[
		G = \bigcup_{k=1} ^{+\infty} \mathring{I}_k^*
	\] 
	Infatti
	\[
	\mu^* (G) = \sum_{k=1}^{+\infty} \lvert I_k^* \rvert \le  \sum_{k=1}^{+\infty}\left( \lvert I_k \rvert + \frac{\varepsilon }{2^{k} }\right) \le \mu^* (E)+\varepsilon 
	\] 
	
\end{proof}
\begin{osservazione}
Relativamente al teorema precedente, si notano due cose: intanto fa uso della topologia di $\mathbb{R}^d$ e poi afferma che un generico insieme $E \subseteq \mathbb{R}^d$ \`e approssimabile arbitrariamente bene tramite un aperto $G$.
\end{osservazione}

\begin{teorema}
	Sia $E \subseteq \mathbb{R}^d$; allora $\exists H = \bigcap_{j=1} ^{+\infty} G_j$, con $G_j$ aperti, tale che $E \subset H$ e $\mu ^*(E) = \mu ^*(H)$.
\end{teorema}
	\begin{proof}
		
	\end{proof}
\subsection{Misurabilit\`a}

















\end{document}
