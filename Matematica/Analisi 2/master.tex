\documentclass[10pt, a4paper]{scrartcl}
% Packages
\usepackage[margin=1.25in]{geometry}
\usepackage{index}
\usepackage{stix}
\makeindex
\usepackage[utf8]{inputenc}
\usepackage[T1]{fontenc}
\usepackage{tcolorbox}
\tcbuselibrary{theorems}
\tcbuselibrary{skins}
\tcbuselibrary{breakable}
\usepackage{varwidth}
\usepackage{textcomp}
\usepackage{amsmath, amssymb}
\usepackage{esint}
\usepackage{titlesec}
\usepackage{xcolor}
\usepackage{titling}
\usepackage[linktocpage]{hyperref}
\usepackage{pgfplots}
\usepackage{multicol}
\setlength{\columnsep}{2em}
\usepackage{caption}
\usepackage{amsthm}
\usepackage{import}
\usepackage{cancel}
\usepackage{caption}
\usepackage{nicematrix}
\usepackage{mathrsfs}
\usepackage{mathtools}
%\usepackage{parskip}
\usepackage{pythonhighlight}
\usepackage{enumerate}
\usepackage{graphicx}
\usepackage{tikz}
\usepackage[italian]{babel}

% Titles 
\title{Note di Analisi 2}
\author{Manuel Deodato}
\date{}


% svolgimento
\newenvironment{svolgimento}{\renewcommand\qedsymbol{$\blacksquare$}\begin{proof}[Svolgimento]}{\end{proof}}


%%%%% tcolorbox setup

% Teorema e proposizione
\newtcbtheorem[number within=section]{teorema}{Teorema}
{breakable, top=0.2mm, bottom=0.2mm, boxrule=0mm,arc =.5 mm, colframe=blue!10, coltitle=black, fonttitle=\bfseries, colback=blue!5!white, theorem style=plain apart}{th}

\newtcbtheorem[number within=section]{prop}{Proposizione}
{breakable, top=0.2mm, bottom=0.2mm, boxrule=0mm,arc =.5 mm, colframe=blue!10, coltitle=black, fonttitle=\bfseries, colback=blue!5!white, theorem style=plain apart}{prop}





% Definizione
\definecolor{greendef}{HTML}{b8d8be}

\newtcbtheorem[number within=section]{definizione}{Definizione}
{breakable, top=0.2mm, bottom=0.2mm, boxrule=0mm, arc=.5mm, colframe=greendef, coltitle=black, fonttitle=\bfseries, theorem style = plain apart, colback=greendef!50!white}{def}


% Esempio
\theoremstyle{definition}
\newtheorem{esempio}{Esempio}

%\definecolor{empurple}{HTML}{6e5e89}

%\newtcbtheorem{esempio}{Esempio}{left=0mm,arc=0mm, colframe=empurple!10!white, coltitle=black, fonttitle=\bfseries, theorem style = plain, colback=empurple!20!white, colframe=empurple!90!white, boxrule=1pt, sharp corners, top=.2mm,bottom=.2mm}{es}

\tcolorboxenvironment{esempio}{blanker,breakable,left=5mm,before skip=10pt,after skip=10pt, borderline west={1mm}{0pt}{greendef}}

\numberwithin{esempio}{section}


% Lemma e Corollario
\definecolor{lemcor}{HTML}{a78d8a}

\newtcbtheorem[number within=section]{lemma}{Lemma}{breakable, top=0.2mm, bottom=0.2mm, boxrule=0mm,left=0mm,arc=.5mm, colframe=lemcor!10!white, coltitle=black, fonttitle=\bfseries, theorem style = plain apart, colframe=lemcor!50!white,colback=lemcor!20!white}{lem}
\newtcbtheorem[number within=section]{corollario}{Corollario}{breakable, top=0.2mm, bottom=0.2mm, boxrule=0mm,left=0mm,arc=.5mm, colframe=lemcor!10!white, coltitle=black, fonttitle=\bfseries, theorem style = plain apart, colframe=lemcor!50!white,colback=lemcor!20!white}{cor}



% Osservazione
\theoremstyle{definition}
\newtheorem{obs}{Osservazione}

\definecolor{coloros}{HTML}{6e5e89}

\tcolorboxenvironment{obs}{blanker,breakable,left=5mm,before skip=10pt,after skip=10pt, borderline west={1mm}{0pt}{coloros}}

\numberwithin{obs}{section}

% Nota
\newtheorem{nota}{Nota}

\definecolor{ncol}{HTML}{f9ebbe}

\tcolorboxenvironment{nota}{blanker,breakable,left=5mm,before skip=10pt,after skip=10pt, borderline west={1mm}{0pt}{ncol}}

\numberwithin{nota}{section}



%%%%%%%%%% Medie con integrali multipli
\def\Yint#1{\mathchoice
    {\YYint\displaystyle\textstyle{#1}}%
    {\YYint\textstyle\scriptstyle{#1}}%
    {\YYint\scriptstyle\scriptscriptstyle{#1}}%
    {\YYint\scriptscriptstyle\scriptscriptstyle{#1}}%
      \!\iint}
\def\YYint#1#2#3{{\setbox0=\hbox{$#1{#2#3}{\iint}$}
    \vcenter{\hbox{$#2#3$}}\kern-.51\wd0}}
\def\longdash{{-}\mkern-3.5mu{-}} 
   % consider using "\mkern-7.5mu" if esint package is loaded
\def\tiltlongdash{\rotatebox[origin=c]{15}{$\longdash$}}
\def\fiint{\Yint\tiltlongdash}

\def\Zint#1{\mathchoice
    {\YYint\displaystyle\textstyle{#1}}%
    {\YYint\textstyle\scriptstyle{#1}}%
    {\YYint\scriptstyle\scriptscriptstyle{#1}}%
    {\YYint\scriptscriptstyle\scriptscriptstyle{#1}}%
      \!\iiint}
      \def\tilongdash{\mkern6mu{-}\mkern-4mu{-}\mkern-5mu{-}} 
   % consider using "\mkern-7.5mu" if esint package is loaded
\def\titiltlongdash{\rotatebox[origin=c]{15}{$\tilongdash$}}
\def\fiiint{\Zint\titiltlongdash}

%Captions
\captionsetup[figure]{font=footnotesize,labelfont=footnotesize}
\captionsetup[table]{font=footnotesize,labelfont=footnotesize}
%Titlesec
\titleformat{\section}
{\fontsize{15}{20}\sffamily\scshape}
{\normalfont\color{gray}{\fontsize{20}{20}\selectfont\thesection}}
{0.7em}
{}
\hypersetup{colorlinks,breaklinks, linkcolor=[RGB]{74, 122, 164}}
\definecolor{asdf}{HTML}{4a7aa4}
% Personalizza la formattazione della subsection
\titleformat{\subsection}[block]{\fontsize{12}{20}\bfseries}{\normalfont\thesubsection}{.5em}{}


% Personalizza la formattazione della subsubsection
\titleformat{\subsubsection}[block]{\fontsize{10}{20}\bfseries}{\normalfont\thesubsubsection}{.5em}{}

% Maketitle customization
\renewcommand{\maketitle}{
\begin{center}
{\sffamily
{\fontsize{20}{20}\selectfont\MakeUppercase\thetitle}}

\vspace{0.2in}

{\large\scshape\sffamily\theauthor}
\end{center}
}

%Evaluate symbol
\DeclareMathOperator{\di}{d\!}
\newcommand*\Eval[3]{\left.#1\right\rvert_{#2}^{#3}}

%%%%%%% Numero delle equazioni in formato a.b
\numberwithin{equation}{subsection}
%%%%%

%%%%%%%%%% Personalizzazione numeri lista
\renewcommand{\theenumi}{(\arabic{enumi})}

%%%% Table of contents

\usepackage[titles]{tocloft}

\renewcommand{\cftdot}{}
\usepackage{titletoc}
%\setcounter{tocdepth}{2}

%%%%%%%%%%%%%%%% Toc style

% Personalizzazione scritta indice


% Font
\usepackage[osf]{newpxtext}
\usepackage{sansiwona}



\begin{document}
\maketitle
\newpage
\tableofcontents 
\newpage
\section{Calcolo differenziale in pi\`u variabili}
\subsection{Derivate parziali}

Una funzione di pi\`u variabili $f(x,y):\mathbb{R}^2 \to \mathbb{R}$ pu\`o essere derivata mantenendo fissa una variabile e derivando rispetto all'altra. Questo corrisponde al valutare la variazione di $f$ lungo un asse specifico.
\begin{definizione}
	{Derivata parziale}{}
	Sia $f(x_1,\ldots,x_n) :\mathbb{R}^n \to \mathbb{R}$; la sua derivata parziale rispetto a $x_k$ \`e:
	\begin{equation}
		\frac{\partial f}{\partial x_k}(x_1,\ldots,x_n) = \lim_{h \to 0} \frac{f(x_1,\ldots,x_k + h, \ldots, x_n)-f(x_1,\ldots,x_k,\ldots,x_n)}{h}
	\end{equation}
\end{definizione}
\noindent Il vettore che ha per componenti le derivate di $f$ rispetto a ciascuna delle sue variabili si chiama \textbf{gradiente} e si indica con $\nabla f$.

\subsection{Derivate direzionali}

\`E possibile studiare la variazione di $f$ lungo una particolare direzione individuata dal versore $\hat{n}$. Una retta parallela a $\hat{n}$ e passante per un punto $x$ si individua con $x+t \hat{n}$; fissando i punti $x$ e $\hat{n}$, $g(t) := f(x+t\hat{n})$ \`e una funzione di una variabile e $g'(0)$ \`e la derivata direzionale di $f$ lungo $\hat{n}$:
\begin{equation}
	\frac{\partial f}{\partial \hat{n}} (x) = g'(0) = \lim_{h \to 0} \frac{f(x+h\hat{n}) - f(x)}{h}
\end{equation}
Pi\`u in generale:
\begin{equation}
	g'(t) \overset{\text{def}}{=} \lim_{h \to 0} \frac{g(t+h) - g(t)}{h} = \lim_{h \to 0} \frac{f(x_t+ h \hat{n}) - f(x_t)}{h} \equiv \frac{\partial f}{\partial \hat{n}} (x_t)
\end{equation}
con $x_t = x+t \hat{n}$. 
\begin{obs}
	Conoscendo $\nabla f$, si pu\`o calcolare la derivata direzionale di $f$ come $\nabla f \cdot \hat{n}$.
\end{obs}
\begin{esempio}
	Si calcola la derivata direzionale di $f(x,y) = x^2 y - e^{x+y} $ lungo la direzione $\hat{n} = \left(\frac{1}{2}, \frac{\sqrt{3} }{2}\right) $.
	\begin{svolgimento}
		Si ha 
		\[
		g(t) = f\left(x + \frac{t}{2}, y + \frac{\sqrt{3} }{2}t\right) = \left(x + \frac{t}{2}\right) ^2 \left(y + \frac{\sqrt{3} }{2}t\right) - \exp \left[ x + y + t\left(\frac{1}{2} + \frac{\sqrt{3} }{2}\right)  \right] 
		\] 
		Allora 
		\[
		\frac{\partial f}{\partial \hat{n}} (x,y) = g'(0) = xy + \frac{\sqrt{3} }{2}x^2 - \left(\frac{1}{2}+ \frac{\sqrt{3} }{2} \right) e^{x+y} 
		\] 
		Alternativamente $\nabla f = \left(2xy - e^{x+y} , x^2 - e^{x+y} \right) $, quindi $\partial _{\hat{n}} f = \nabla f \cdot \hat{n} =xy - \frac{1}{2} e^{x+y} +\frac{\sqrt{3} }{2} x^2 - \frac{\sqrt{3} }{2}e^{x+y} = xy + \frac{\sqrt{3} }{2}x^2 - \left(\frac{1}{2}+\frac{\sqrt{3} }{2}\right) e^{x+y}   $.
	\end{svolgimento}
\end{esempio}
\begin{teorema}
	{}{}
	Se $f:A\subset \mathbb{R}^2 \to  \mathbb{R}$ ha un massimo o minimo relativo in $x_0$ interno ad $A$ e se ammette derivata lungo $\hat{n}$ in $x_0$, allora:
	\begin{equation}
		\frac{\partial f}{\partial \hat{n}} (x_0)= 0 
	\end{equation}
	\begin{proof}
		Si prende $g(t) = f(x_0 + t \hat{n})$ che, per costruzione, ha un minimo in $t=0$, quindi $g'(0) = 0$, da cui segue la tesi.
	\end{proof}
\end{teorema}
\noindent In particolare, se $f$ \`e derivabile in $x_0$, tutte le derivate parziali si annullano in quel punto; in questo caso, $x_0$ \`e detto \textbf{punto stazionario}.

\begin{obs}
	Nel caso a una variabile, i punti di massimo/minimo che cadevano sulla frontiera di un insieme erano, solitamente, un numero finito; qua chiaramente non \`e pi\`u cos\`i.
\end{obs}
\begin{esempio}
	Calcolare massimi e minimi di $f(x,y) = (x^2 + y^2 - 1)e^{x+y} $ nel cerchio chiuso centrato nell'origine e di raggio $1$.
	\begin{svolgimento}
		Sul bordo del cerchio $x^2 + y^2 = 1$, quindi $f\equiv 0$. All'interno:
		\[
			\begin{split}
				&f_x = 2x e^{x+y}  + (x^2 + y^2 -1)e^{x+y} \\
				&f_y = 2y e^{x+y} + (x^2 + y^2 - 1) e^{x+y} 
			\end{split}
		\] 
	che si annullano quando 
	\[
	\begin{split}
		&x^2 + y^2 + 2x - 1 = 0\\
		&x^2 + y^2 + 2y - 1 = 0
	\end{split}\Rightarrow 2x - 2y = 0 \Rightarrow x=y
	\] 
Sostituendo $x=y$ nella prima equazione, ad esempio, si ottengono due soluzioni, una sola delle quali appartiene al cerchio; questo corrisponder\`a al punto di minimo della funzione:
\[
f\left(\frac{\sqrt{3} -1}{2}, \frac{\sqrt{3} -1}{2}\right) = (1-\sqrt{3} ) e^{\sqrt{3} -1}  < 0
\] 
\end{svolgimento}
\end{esempio}
\noindent In pi\`u dimensioni vale un analogo del teorema di Lagrange:
\begin{teorema}
	{}{}
	Sia $f(x) : A \subset \mathbb{R}^n \to \mathbb{R}$ e $x_0 \in A$, con $I(x_0,r) \subset A$. Considerando una direzione $\hat{n}$, si definisce $g(s) = f(x_0+ s \hat{n})$ per $\lvert s \rvert <r$. Vale l'analogo del teorema di Lagrange:
	\begin{equation}
		f(x_0+s\hat{n}) - f(x_0) = g(s) - g(0) = s g'(\tau ) = s \frac{\partial f}{\partial \hat{n}} (x_0 + \tau \hat{n})
	\end{equation}
\end{teorema}

\subsection{Derivate successive}

Sia $f$ una funzione per cui esistono le derivate prime e sono anch'esse derivabili; le derivate seconde potranno essere derivate prima rispetto a $x_i$ e poi rispetto a $x_j$ o viceversa. In generale se $f$ \`e una funzione di $m$, si hanno $m^n$ derivate di ordine $n$. Per le derivate seconde miste\footnote{Chiaramente il risultato vale in generale, ma si affronta per funzione di due variabili nel caso delle derivate seconde miste per semplicit\`a.} vale il seguente.
\begin{teorema}
	{Teorema di Schwarz}{}
	Sia $f$ una funzione derivabile in un intervallo $I$ del punto $(x,y)$ e siano queste continue nello stesso intervallo; allora $f_{xy} (x,y) = f_{yx} (x,y)$.
	\begin{proof}
		Siano $h,k \in \mathbb{R}:(x+h,y+k) \in I$ e sia
		\[
		A(h,k) = f(x+h,y+k) - f(x+h,y) - f(x,y+k) + f(x,y)
		\] 
		Prendendo $p(t) = f(t,y+k) - f(t,y)$, si ha $A(h,k) = p(x+h) - p(x)$; per Lagrange:
		\[
			A(h,k) = p'(\xi ) h = \big[f_x(\xi ,y+k) - f_x(\xi ,y)\big]h ,\ x< \xi <x+h
		\] 
		Applicando nuovamente Lagrange, si ha $A(h,k) = f_{yx} (\xi, \eta) hk, \ y<\eta < y+k$. Ripetendo il discorso con $q(t) = f(x+h,t) - f(x,t)$, si trova $A(h,k) = f_{xy} (\sigma ,\tau ) hk$, quindi $f_{yx}(\xi ,\eta) =f_{xy} (\sigma ,\tau ) $, dove $x<\sigma <x+h$ e $y<\tau <y+k$. Prendendo il limite per $h,k\to 0$, risulta $f_{xy} (x,y) = f_{yx} (x,y)$ per continuit\`a delle derivate seconde.
		
	\end{proof}
\end{teorema}
\noindent Come per funzioni di una variabile, vale la formula di Taylor.
\begin{teorema}
	{Formula di Taylor}{}
	Sia $f(x)$ di classe $C^2$ in $A \subset \mathbb{R}^n$ e $x_0$ punto interno ad $A$; in un intorno di $x_0$, allora, si ha:
	\begin{equation}
		f(x) = f(x_0) + \left\langle \nabla f(x_0), x-x_0 \right\rangle + \frac{1}{2} \left\langle Hf(x_0) (x-x_0), x-x_0 \right\rangle + R_2(x;x_0)
	\end{equation}
	con 
	\[
	\lim_{x \to x_0} \frac{R_2(x;x_0)}{\left\lVert x-x_0 \right\rVert ^2} = 0
	\] 
	
\end{teorema}

\subsection{Funzioni differenziabili}


Una funzione derivabile, anche in ogni direzione, non \`e necessariamente continua in pi\`u variabili.
\begin{esempio}
	La funzione $f(x,y) = \begin{cases}
		\frac{xy^2}{x^2 + y^2}& , \ (x,y)\neq 0\\
		0 & , \ (x,y) =0 
	\end{cases}$ ha derivate in ogni direzione nel punto $(0,0)$, ma non \`e continua; prendendo $x_k = (1 / k , 1/k^2)$ per $k\to \infty$, si ha $x_k\to (0,0)$, ma $f(x_k) = \frac{1/k^4}{2 / k^4} \to \frac{1}{2}$.
\end{esempio}

\begin{definizione}{Differenziabilit\`a}{}
	Una funzione $f(x)$ si dice differenziabile in $x_0$ se \`e derivabile in $x_0 $ e se:
	\begin{equation}
		\lim_{x \to x_0} \frac{f(x) - f(x_0) - \langle \nabla f(x_0) , x -x_0 \rangle}{\left\lVert x - x_0 \right\rVert } = 0
	\end{equation}
\end{definizione}
\noindent Questa definizione impone che una funzione sia differenziabile in punto se esiste un piano tangente che la approssima precisamente nel punto stesso.

\begin{teorema}
	{}{}
	Una funzione $f(x)$ differenziabile in $x_0$ \`e continua in $x_0$ ed \`e derivabile in ogni direzione.
	\begin{proof}
		Si mostra che \`e continua:
		\[
		f(x) - f(x_0)  = \frac{f(x) - f(x_0) - \langle \nabla f(x_0) , x-x_0 \rangle}{\left\lVert x-x_0 \right\rVert } \left\lVert x-x_0 \right\rVert + \langle \nabla f(x_0) , x-x_0 \rangle
		\] 
		Per $x\to x_0$ il primo termine di destra va a $0$ per assunzione di differenziabilit\`a e l'altro anche perch\'e diventa un prodotto scalare per $0$, quindi si verifica $\lim_{x \to x_0} f(x) = f(x_0)$.

		Data generica direzione $\hat{v}$ con $x = x_0 + t \hat{v}$, usando ancora definizione di differenziabilit\`a:
		\[
		\lim_{t \to 0} \frac{f(x_0 + t \hat{v})- f(x_0) - \langle \nabla f(x_0), t \hat{v} \rangle}{t} = 0 
		\] 
Visto che $\langle \nabla f(x_0 ) , t \hat{v} \rangle = t \langle \nabla f(x_0) ,\hat{v} \rangle$, si ottiene la tesi.
	\end{proof}
\end{teorema}
\noindent La direzione di massimo incremento di una funzione \`e quella del gradiente. Per mostrarlo, si parte da $x_0$, assumendo che non sia un punto stazionario; si definisce, allora, $\hat{n} = \frac{\nabla f(x_0)}{\left\lVert \nabla f(x_0) \right\rVert} $, da cui:
\[
\frac{\partial f}{\partial \hat{n}} (x_0) = \langle \nabla f(x_0), \hat{n} \rangle = \left\lVert \nabla f(x_0) \right\rVert 
\] 
Prendendo altra direzione generica $\hat{v}$, si ha:
\[
\frac{\partial f}{\partial \hat{v}} (x_0) = \langle \nabla f(x_0) , \hat{v}  \rangle \le  \left\lVert \nabla f(x_0) \right\rVert \left\lVert \hat{v} \right\rVert = \left\lVert \nabla f(x_0) \right\rVert \equiv \frac{\partial f}{\partial \hat{n}} (x_0)
\] 
Dalla definizione di funzione differenziabile il piano $z = f(x_0,y_0) +f_x(x_0,y_0) (x-x_0) + f_y(x_0,y_0) (y-y_0)$ \`e quello che meglio approssima la funzione in $(x_0,y_0)$.

Si \`e concluso che una funzione differenziabile \`e derivabile in ogni direzione, ma una funzione derivabile non \`e differenziabile in generale. Vale, per\`o, il seguente.
\begin{teorema}
	{Teorema del differenziale totale}{}
	Sia $f(x)$ derivabile in $x_0$ e siano le sue derivate continue nello stesso punto; allora $f$ \`e differenziabile in $x_0$.
	\begin{proof}
		Si vuole dimostrare che
		\[
		\lim_{(x,y) \to (x_0,y_0)} \frac{f(x,y) - f(x_0,y_0) - f_x(x_0,y_0) (x-x_0) - f_y(x_0,y_0)(y-y_0)}{\sqrt{(x-x_0)^2 + (y-y_0)^2} } = 0
		\] 
		Si usa il teorema di Lagrange per riscrivere $f(x,y) - f(x_0,y_0)$:
		\[
		\begin{split}
			&f(x,y_0) - f(x_0,y_0) = f_x(\xi ,y_0) (x-x_0),\ x_0<\xi <x\\
			&f(x,y) - f(x,y_0) = f_y(x ,\eta) (y-y_0), \ y_0 < \eta < y\\
			&\Rightarrow f(x,y) - f(x_0,y_0) = f_x(\xi ,y_0) (x-x_0) + f_y(x,\eta) (y-y_0)
		\end{split}
		\] 
	Il limite scritto sopra si riscrive come:
	\[
		\begin{split}
			\lim_{(x,y) \to (x_0,y_0)} \big[f_x(\xi ,y_0) - f_x(x_0,y_0)\big] &\frac{x-x_0}{\sqrt{(x-x_0)^2 + (y-y_0)^2} } + \\ 
											  &+\big[f_y(x,\eta) - f_y(x_0,y_0)\big] \frac{y-y_0}{\sqrt{(x-x_0)^2 + (y-y_0)^2} }
		\end{split}
	\] 
	Essendo le frazioni $\le 1$ e visto che le quantit\`a fra parentesi quadre, questo limite si maggiora con la somma delle parentesi quadre, che tende a $0$ per $(x,y) \to (x_0,y_0)$.
	\end{proof}
\end{teorema}
\subsection{Funzioni composte}
	Data una funzione $x(t) : \mathbb{R}^k \to \mathbb{R}^n$, si definisce, per una generica direzione $v$:
	\begin{equation}
		\frac{\partial x}{\partial v} = \left(\frac{\partial x_1}{\partial v}, \ldots , \frac{\partial x_n}{\partial t}  \right) ^\top
	\end{equation}
Vale il seguente per la derivata della funzione composta.
\begin{teorema}
	{}{}
	Siano $E \subset  \mathbb{R}^k , \ F \subset \mathbb{R}^n$ e $x(t) : E \to F , \ f(x) : F \to \mathbb{R}$ funzioni di classe $C^1$. Allora la funzione composta $g(t) = f(x(t)) : E \to\mathbb{R}$ \`e di classe $C^1$ e per ogni direzione $v$:
	\begin{equation}
		\frac{\partial g}{\partial v} (t) =\left\langle \nabla f\big(x(t)\big), \frac{\partial x}{\partial v} (t) \right\rangle
	\end{equation}
	\begin{proof}
		Si ha $g(t + h v) -g(t) = f\big(x(t+hv)\big) - f\big(x(t)\big) = f\big(x(t) + [x(t+hv) - x(t)]\big) - f\big(x(t)\big)$. Si prende $s = \lVert x(t+hv) - x(t) \rVert $ e la direzione$w = \frac{x(t+hv) - x(t)}{s}$ e si usa il teorema di Lagrange:
		\[
		g(t+hv) - g(t) = f\big(x(t) + s w\big) - f\big(x(t)\big) = s \frac{\partial f}{\partial w} \big(x(t) + \tau w\big) = s \left\langle \nabla f \big(x(t) + \tau  w\big), w \right\rangle
		\] 
	con $0<\tau <s$. Dividendo per $h$ e prendendo il limite $h\to 0$, per definizione $s \to 0$ e, quindi, $\tau  \to 0$, mentre $\frac{x(t+hv) - x(t)}{h}\to \frac{\partial x}{\partial v} (t)$ quindi:
\[
\lim_{h \to 0} \frac{g(t+ hv) -g(t)}{h} = \lim_{h \to 0} \left\langle \nabla f\big(x(t) + \tau  w\big), \frac{x(t+hv) - x(t)}{h} \right\rangle = \left\langle\nabla f\big(x(t)\big), \frac{\partial x}{\partial v} (t)  \right\rangle
\] 
	\end{proof}
\end{teorema}
\noindent Nel caso particolare $k=1$, $x(t)$ \`e una curva e $g(t)$ \`e funzione di una sola variabile con
\[
g'(t) = \sum_{h=1}^{n} \frac{\partial f}{\partial x_h} \big(x(t)\big) x'_h(t) \equiv \Big\langle \nabla f \big(x(t)\big), x'(t) \Big\rangle
\] 
Spesso si prende $x(t) = x + t v $, cio\`e retta passante per $x$ lungo direzione $v$; in questo caso $g'(t) = \nabla f(x+tv) \cdot  v$. Se le derivate seconde sono continue, le derivate prime sono differenziabili e si pu\`o scrivere:
\begin{equation}
	g''(t)	= \sum_{i=1}^{n} v_i \frac{d }{d t} D_i f (x+tv) = \sum_{i=1}^{n} v_i \sum_{j=1}^{n} v_j D_{ij} f(x+tv)
\end{equation}
Indicando con $Hf = \nabla f \nabla ^\top$ la matrice Hessiana di $f$, allora $\sum_{j}^{} v_j D_{ij} f(x+tv) \equiv \left[ Hf(x+tv) v\right]_i $, cio\`e \`e la componente $i$-esima del vettore tra parentesi quadre, essendo $Hf$ una matrice. Allora:
\begin{equation}
	\begin{split}
		&g'(0) = \nabla f(x) \cdot v \\ 
		& g''(0) = \langle Hf(x) v , v \rangle
	\end{split}
\end{equation}
 
\subsection{Massimi e minimi relativi}

Perch\'e una funzione $f$ di pi\`u variabili abbia un punto di massimo o di minimo in $x_0$, \`e condizione necessaria che per ogni direzione $v$, valga $g'(0) = 0$ e $g''(0) \le 0$ o $g''(0) \ge 0$, cio\`e:
\begin{equation}
	\begin{split}
		&\langle Hf (x_0) v , v \rangle \le 0 \text{ punto di massimo}\\
		&\langle Hf (x_0) v , v \rangle \ge 0 \text{ punto di minimo}
	\end{split}
\end{equation}
Allora vale il seguente.
\begin{teorema}
	{}{}
	Sia $f(x)$ una funzione con derivate seconde continue; se in $x_0$, $\nabla f (x_0) = 0$ e la matrice Hessiana \`e tale che $Hf(x_0) > 0$ (definita positiva), allora $x_0$ \`e di minimo relativo per $f$. Se fosse $Hf(x_0) < 0$, $x_0$ sarebbe di massimo relativo.
\end{teorema}
\noindent Possono verificarsi altri due casi:
\begin{itemize}
	\item se $\langle H f(x_0)v ,v \rangle $ assume sia valori positivi che negativi al variare di $v$, si ha un \textbf{punto di sella};
	\item se la matrice Hessiana \`e semidefinita, ma non definita, non si pu\`o concludere niente e bisogna esaminare cosa accade attorno a $x_0$.
\end{itemize}

\newpage 

\section{Calcolo integrale in pi\`u variabili}

\subsection{Integrazione in dimensioni superiori}
Per le definizioni di base, si deve definire cos'\`e un rettangolo.
\begin{definizione}
	{}{}
	Dati due intervalli $[a.b)$ e $[c,d)$, il rettangolo che identificano \`e definito come $R = [a,b) \times  [c,d)$, con $a\le x< b$ e $c\le y<d$.
\end{definizione}
\noindent Si suddividono due intervalli in intervalli pi\`u piccoli, cio\`e $[a,b)$ si suddivide in $n$ sotto-intervalli $I_h = [x_{h-1} ,x_h)$, con $x_0 =a , \ldots x_n = b$ e $[c,d)$ in $m$ sotto-intervalli $J_k = [y_{k-1} ,y_k)$. Allora il rettangolo sar\`a suddiviso in $n\times m$ sotto-rettangoli $R_{hk}  = I_h \times J_k$.

Una funzione semplice $\varphi (x)$ \`e una funzione che assume un valore costante su ogni sotto-rettangolo e che vale $0$ fuori da $R$. Indicando con $\lambda _{hk}$ il valore costante che assume in $R_{hk} $:
\begin{equation}
	\varphi (x)= \sum_{h=1}^{n} \sum_{k=1}^{m} \lambda _{hk} \chi _{R_{hk} } (x)
\end{equation}
con $\chi _D$ funzione caratteristica del dominio $D$. L'integrale di funzioni simili \`e dato da:
\begin{equation}
	\int \varphi (x) \ dxdy = \sum_{h=1}^{n} \sum_{k=1}^{m} \lambda _{hk} m(R_{hk} )= \sum_{h=1}^{n} \sum_{k=1}^{m} \lambda _{hk} m(I_h) m(J_k) = \sum_{h=1}^{n} \sum_{k=1}^{m} \lambda _{hk} (x_h -x_{h-1} ) (y_k - y_{k-1} )
\end{equation}
\`E necessario dare anche la definizione di supporto di una funzione:
\begin{definizione}
	{}{}
	Il supporto di una funzione $f$ \`e la chiusura dell'insieme in cui $f\neq 0$, cio\`e:
	\begin{equation}
		\operatorname{supp} (f) = \overline{\left\{ x : f(x) \neq 0 \right\} }
	\end{equation}
\end{definizione}
\noindent Infine, si indica con $\mathscr{S}^+(D)$ la classe delle funzioni semplici $\varphi $ che maggiorano $f$ in $D$ e $\mathscr{S}^-(D)$ la classe delle funzioni semplici $\psi $ che minorano $f$ in $D$; da questo, si ha la seguente definizione di integrale di Riemann.
\begin{definizione}
	{Integrazione di funzioni a supporto compatto}{}
	Sia $f$ una funzione a supporto compatto, con $\operatorname{supp} (f) \subset K$; $f$ \`e integrabile secondo Riemann se:
	\begin{equation}
		\sup_{\psi \in \mathscr{S}^-(K)}  \int \psi  \ dx dy = \inf_{\varphi \in \mathscr{S}^+(K)}  \int \varphi \ dxdy
	\end{equation}
	dove
	\begin{equation}
		\begin{split}
			&\int_{*} f(x) \ dx =  \inf_{\varphi \in \mathscr{S}^+(K)}  \int \varphi \ dxdy \ \text{ integrale inferiore} \\
			&\int^{*} f(x) \ dx =  \sup_{\psi  \in \mathscr{S}^-(K)}  \int \psi  \ dxdy \ \text{ integrale superiore} \\
		\end{split}
	\end{equation}
	La condizione di integrabilit\`a si pu\`o esprimere come:
	\begin{equation}
		\int_{*} f(x) \ dx= \int^{*} f(x) \ dx
	\end{equation}
\end{definizione}
\begin{obs}
	Anche per pi\`u variabili, \`e condizione sufficiente e necessaria perch\'e $f$ a supporto compatto sia integrabile che $\forall \varepsilon >0$, esistono funzioni semplici $\varphi ,\psi $ tali che:
	\begin{equation}
		\int \varphi  \ dxdy - \int \psi  \ dxdy < \varepsilon 
	\end{equation}
\end{obs}

















\end{document}
