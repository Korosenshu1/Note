\documentclass[10pt, a4paper]{scrartcl}
% Packages
\usepackage[margin=1.25in]{geometry}
\usepackage{index}
\usepackage{stix}
\makeindex
\usepackage[utf8]{inputenc}
\usepackage[T1]{fontenc}
\usepackage{tcolorbox}
\tcbuselibrary{theorems}
\tcbuselibrary{skins}
\tcbuselibrary{breakable}
\usepackage{varwidth}
\usepackage{textcomp}
\usepackage{amsmath, amssymb}
\usepackage{esint}
\usepackage{titlesec}
\usepackage{xcolor}
\usepackage{titling}
\usepackage[linktocpage]{hyperref}
\usepackage{pgfplots}
\usepackage{multicol}
\setlength{\columnsep}{2em}
\usepackage{caption}
\usepackage{amsthm}
\usepackage{import}
\usepackage{cancel}
\usepackage{caption}
\usepackage{nicematrix}
\usepackage{mathrsfs}
\usepackage{mathtools}
%\usepackage{parskip}
\usepackage{pythonhighlight}
\usepackage{enumerate}
\usepackage{graphicx}
\usepackage{tikz}
\usepackage[italian]{babel}

% Titles 
\title{Note di Analisi 2}
\author{Manuel Deodato}
\date{}


% svolgimento
\newenvironment{svolgimento}{\renewcommand\qedsymbol{$\blacksquare$}\begin{proof}[Svolgimento]}{\end{proof}}


%%%%% tcolorbox setup

% Teorema e proposizione
\newtcbtheorem[number within=section]{teorema}{Teorema}
{breakable, top=0.2mm, bottom=0.2mm, boxrule=0mm,arc =.5 mm, colframe=blue!10, coltitle=black, fonttitle=\bfseries, colback=blue!5!white, theorem style=plain apart}{th}

\newtcbtheorem[number within=section]{prop}{Proposizione}
{breakable, top=0.2mm, bottom=0.2mm, boxrule=0mm,arc =.5 mm, colframe=blue!10, coltitle=black, fonttitle=\bfseries, colback=blue!5!white, theorem style=plain apart}{prop}





% Definizione
\definecolor{greendef}{HTML}{b8d8be}

\newtcbtheorem[number within=section]{definizione}{Definizione}
{breakable, top=0.2mm, bottom=0.2mm, boxrule=0mm, arc=.5mm, colframe=greendef, coltitle=black, fonttitle=\bfseries, theorem style = plain apart, colback=greendef!50!white}{def}


% Esempio
\theoremstyle{definition}
\newtheorem{esempio}{Esempio}

%\definecolor{empurple}{HTML}{6e5e89}

%\newtcbtheorem{esempio}{Esempio}{left=0mm,arc=0mm, colframe=empurple!10!white, coltitle=black, fonttitle=\bfseries, theorem style = plain, colback=empurple!20!white, colframe=empurple!90!white, boxrule=1pt, sharp corners, top=.2mm,bottom=.2mm}{es}

\tcolorboxenvironment{esempio}{blanker,breakable,left=5mm,before skip=10pt,after skip=10pt, borderline west={1mm}{0pt}{greendef}}

\numberwithin{esempio}{section}


% Lemma e Corollario
\definecolor{lemcor}{HTML}{a78d8a}

\newtcbtheorem[number within=section]{lemma}{Lemma}{breakable, top=0.2mm, bottom=0.2mm, boxrule=0mm,left=0mm,arc=.5mm, colframe=lemcor!10!white, coltitle=black, fonttitle=\bfseries, theorem style = plain apart, colframe=lemcor!50!white,colback=lemcor!20!white}{lem}
\newtcbtheorem[number within=section]{corollario}{Corollario}{breakable, top=0.2mm, bottom=0.2mm, boxrule=0mm,left=0mm,arc=.5mm, colframe=lemcor!10!white, coltitle=black, fonttitle=\bfseries, theorem style = plain apart, colframe=lemcor!50!white,colback=lemcor!20!white}{cor}



% Osservazione
\theoremstyle{definition}
\newtheorem{obs}{Osservazione}

\definecolor{coloros}{HTML}{6e5e89}

\tcolorboxenvironment{obs}{blanker,breakable,left=5mm,before skip=10pt,after skip=10pt, borderline west={1mm}{0pt}{coloros}}

\numberwithin{obs}{section}

% Nota
\newtheorem{nota}{Nota}

\definecolor{ncol}{HTML}{f9ebbe}

\tcolorboxenvironment{nota}{blanker,breakable,left=5mm,before skip=10pt,after skip=10pt, borderline west={1mm}{0pt}{ncol}}

\numberwithin{nota}{section}



%%%%%%%%%% Medie con integrali multipli
\def\Yint#1{\mathchoice
    {\YYint\displaystyle\textstyle{#1}}%
    {\YYint\textstyle\scriptstyle{#1}}%
    {\YYint\scriptstyle\scriptscriptstyle{#1}}%
    {\YYint\scriptscriptstyle\scriptscriptstyle{#1}}%
      \!\iint}
\def\YYint#1#2#3{{\setbox0=\hbox{$#1{#2#3}{\iint}$}
    \vcenter{\hbox{$#2#3$}}\kern-.51\wd0}}
\def\longdash{{-}\mkern-3.5mu{-}} 
   % consider using "\mkern-7.5mu" if esint package is loaded
\def\tiltlongdash{\rotatebox[origin=c]{15}{$\longdash$}}
\def\fiint{\Yint\tiltlongdash}

\def\Zint#1{\mathchoice
    {\YYint\displaystyle\textstyle{#1}}%
    {\YYint\textstyle\scriptstyle{#1}}%
    {\YYint\scriptstyle\scriptscriptstyle{#1}}%
    {\YYint\scriptscriptstyle\scriptscriptstyle{#1}}%
      \!\iiint}
      \def\tilongdash{\mkern6mu{-}\mkern-4mu{-}\mkern-5mu{-}} 
   % consider using "\mkern-7.5mu" if esint package is loaded
\def\titiltlongdash{\rotatebox[origin=c]{15}{$\tilongdash$}}
\def\fiiint{\Zint\titiltlongdash}

%Captions
\captionsetup[figure]{font=footnotesize,labelfont=footnotesize}
\captionsetup[table]{font=footnotesize,labelfont=footnotesize}
%Titlesec
\titleformat{\section}
{\fontsize{15}{20}\sffamily\scshape}
{\normalfont\color{gray}{\fontsize{20}{20}\selectfont\thesection}}
{0.7em}
{}
\hypersetup{colorlinks,breaklinks, linkcolor=[RGB]{74, 122, 164}}
\definecolor{asdf}{HTML}{4a7aa4}
% Personalizza la formattazione della subsection
\titleformat{\subsection}[block]{\fontsize{12}{20}\bfseries}{\normalfont\thesubsection}{.5em}{}


% Personalizza la formattazione della subsubsection
\titleformat{\subsubsection}[block]{\fontsize{10}{20}\bfseries}{\normalfont\thesubsubsection}{.5em}{}

% Maketitle customization
\renewcommand{\maketitle}{
\begin{center}
{\sffamily
{\fontsize{20}{20}\selectfont\MakeUppercase\thetitle}}

\vspace{0.2in}

{\large\scshape\sffamily\theauthor}
\end{center}
}

%Evaluate symbol
\DeclareMathOperator{\di}{d\!}
\newcommand*\Eval[3]{\left.#1\right\rvert_{#2}^{#3}}

%%%%%%% Numero delle equazioni in formato a.b
\numberwithin{equation}{subsection}
%%%%%

%%%%%%%%%% Personalizzazione numeri lista
\renewcommand{\theenumi}{(\arabic{enumi})}

%%%% Table of contents

\usepackage[titles]{tocloft}

\renewcommand{\cftdot}{}
\usepackage{titletoc}
%\setcounter{tocdepth}{2}

%%%%%%%%%%%%%%%% Toc style

% Personalizzazione scritta indice


% Font
\usepackage[osf]{newpxtext}
\usepackage{sansiwona}



\begin{document}
\maketitle
\newpage
\tableofcontents 
\newpage
\section{Calcolo differenzialeciao in pi\`u variabili}
\subsection{Derivate parziali}

Una funzione di pi\`u variabili $f(x,y):\mathbb{R}^2 \to \mathbb{R}$ pu\`o essere derivata mantenendo fissa una variabile e derivando rispetto all'altra. Questo corrisponde al valutare la variazione di $f$ lungo un asse specifico.
\begin{definizione}
	{Derivata parziale}{}
	Sia $f(x_1,\ldots,x_n) :\mathbb{R}^n \to \mathbb{R}$; la sua derivata parziale rispetto a $x_k$ \`e:
	\begin{equation}
		\frac{\partial f}{\partial x_k}(x_1,\ldots,x_n) = \lim_{h \to 0} \frac{f(x_1,\ldots,x_k + h, \ldots, x_n)-f(x_1,\ldots,x_k,\ldots,x_n)}{h}
	\end{equation}
\end{definizione}
\noindent Il vettore che ha per componenti le derivate di $f$ rispetto a ciascuna delle sue variabili si chiama \textbf{gradiente} e si indica con $\nabla f$.

\subsubsection{Derivate direzionali}

\`E possibile studiare la variazione di $f$ lungo una particolare direzione individuata dal versore $\hat{n}$. Una retta parallela a $\hat{n}$ e passante per un punto $x$ si individua con $x+t \hat{n}$; fissando i punti $x$ e $\hat{n}$, $g(t) := f(x+t\hat{n})$ \`e una funzione di una variabile e $g'(0)$ \`e la derivata direzionale di $f$ lungo $\hat{n}$:
\begin{equation}
	\frac{\partial f}{\partial \hat{n}} (x) = g'(0) = \lim_{h \to 0} \frac{f(x+h\hat{n}) - f(x)}{h}
\end{equation}
Pi\`u in generale:
\begin{equation}
	g'(t) \overset{\text{def}}{=} \lim_{h \to 0} \frac{g(t+h) - g(t)}{h} = \lim_{h \to 0} \frac{f(x_t+ h \hat{n}) - f(x_t)}{h} \equiv \frac{\partial f}{\partial \hat{n}} (x_t)
\end{equation}
con $x_t = x+t \hat{n}$. 
\begin{obs}
	Conoscendo $\nabla f$, si pu\`o calcolare la derivata direzionale di $f$ come $\nabla f \cdot \hat{n}$.
\end{obs}
\begin{esempio}
	Si calcola la derivata direzionale di $f(x,y) = x^2 y - e^{x+y} $ lungo la direzione $\hat{n} = \left(\frac{1}{2}, \frac{\sqrt{3} }{2}\right) $.
	\begin{svolgimento}
		Si ha 
		\[
		g(t) = f\left(x + \frac{t}{2}, y + \frac{\sqrt{3} }{2}t\right) = \left(x + \frac{t}{2}\right) ^2 \left(y + \frac{\sqrt{3} }{2}t\right) - \exp \left[ x + y + t\left(\frac{1}{2} + \frac{\sqrt{3} }{2}\right)  \right] 
		\] 
		Allora 
		\[
		\frac{\partial f}{\partial \hat{n}} (x,y) = g'(0) = xy + \frac{\sqrt{3} }{2}x^2 - \left(\frac{1}{2}+ \frac{\sqrt{3} }{2} \right) e^{x+y} 
		\] 
		Alternativamente $\nabla f = \left(2xy - e^{x+y} , x^2 - e^{x+y} \right) $, quindi $\partial _{\hat{n}} f = \nabla f \cdot \hat{n} =xy - \frac{1}{2} e^{x+y} +\frac{\sqrt{3} }{2} x^2 - \frac{\sqrt{3} }{2}e^{x+y} = xy + \frac{\sqrt{3} }{2}x^2 - \left(\frac{1}{2}+\frac{\sqrt{3} }{2}\right) e^{x+y}   $.
	\end{svolgimento}
\end{esempio}
\begin{teorema}
	{}{}
	Se $f:A\subset \mathbb{R}^2 \to  \mathbb{R}$ ha un massimo o minimo relativo in $x_0$ interno ad $A$ e se ammette derivata lungo $\hat{n}$ in $x_0$, allora:
	\begin{equation}
		\frac{\partial f}{\partial \hat{n}} (x_0)= 0 
	\end{equation}
	\begin{proof}
		Si prende $g(t) = f(x_0 + t \hat{n})$ che, per costruzione, ha un minimo in $t=0$, quindi $g'(0) = 0$, da cui segue la tesi.
	\end{proof}
\end{teorema}
\noindent In particolare, se $f$ \`e derivabile in $x_0$, tutte le derivate parziali si annullano in quel punto; in questo caso, $x_0$ \`e detto \textbf{punto stazionario}.

\begin{obs}
	Nel caso a una variabile, i punti di massimo/minimo che cadevano sulla frontiera di un insieme erano, solitamente, un numero finito; qua chiaramente non \`e pi\`u cos\`i.
\end{obs}
\begin{esempio}
	Calcolare massimi e minimi di $f(x,y) = (x^2 + y^2 - 1)e^{x+y} $ nel cerchio chiuso centrato nell'origine e di raggio $1$.
	\begin{svolgimento}
		Sul bordo del cerchio $x^2 + y^2 = 1$, quindi $f\equiv 0$. All'interno:
		\[
			\begin{split}
				&f_x = 2x e^{x+y}  + (x^2 + y^2 -1)e^{x+y} \\
				&f_y = 2y e^{x+y} + (x^2 + y^2 - 1) e^{x+y} 
			\end{split}
		\] 
	che si annullano quando 
	\[
	\begin{split}
		&x^2 + y^2 + 2x - 1 = 0\\
		&x^2 + y^2 + 2y - 1 = 0
	\end{split}\Rightarrow 2x - 2y = 0 \Rightarrow x=y
	\] 
Sostituendo $x=y$ nella prima equazione, ad esempio, si ottengono due soluzioni, una sola delle quali appartiene al cerchio; questo corrisponder\`a al punto di minimo della funzione:
\[
f\left(\frac{\sqrt{3} -1}{2}, \frac{\sqrt{3} -1}{2}\right) = (1-\sqrt{3} ) e^{\sqrt{3} -1}  < 0
\] 
\end{svolgimento}
\end{esempio}

\begin{center}
	\textbf{Riprendere da pagina 7, teorema di Lagrange} 
\end{center}












\end{document}
