%! TEX program = lualatex
\documentclass[12pt]{scrartcl}
% Packages
%\usepackage[margin=1.5in]{geometry}
\usepackage{index}
\usepackage{amsbsy} % Bold math symbols
\makeindex
%\usepackage[utf8]{inputenc}
\usepackage[T1]{fontenc}
\usepackage{tcolorbox}
\tcbuselibrary{theorems}
\tcbuselibrary{skins}
\tcbuselibrary{breakable}
\usepackage{varwidth}
\usepackage{textcomp}
\usepackage{amsmath,amssymb}
\usepackage{esint}
\usepackage{titlesec}
\usepackage{xcolor}
\usepackage{titling}
\usepackage[linktocpage]{hyperref}
\usepackage{pgfplots}
\usepackage{multicol}
\setlength{\columnsep}{2em}
\usepackage{caption}
\usepackage{amsthm}
\usepackage{import}
\usepackage{cancel}
\usepackage{caption}
\usepackage{nicematrix}
\usepackage{mathrsfs}
\usepackage{mathtools}
%\usepackage{parskip}
\usepackage{pythonhighlight}
\usepackage{enumerate}
\usepackage{graphicx}
\usepackage[italian]{babel}
\usepackage{setspace}
\setstretch{1.2}
% To reset footnote numbering each page
\usepackage[perpage]{footmisc}
\usepackage{faktor}
\usepackage{tikz-cd}
\definecolor{mastercolor}{HTML}{a67c00}
\definecolor{nred}{HTML}{bf0040}


% Titles 
\title{Riassunti di Algebra}
\author{}
\date{}




\newtheoremstyle{style}% name of the style to be used
{5pt}% measure of space to leave above the theorem. E.g.: 3pt
{5pt}% measure of space to leave below the theorem. E.g.: 3pt
{\normalfont}% name of font to use in the body of the theorem
%{15pt}% measure of space to indent
{0pt}% measure of space to indent
{\noindent\bfseries}% name of head font
{}% punctuation between head and body
{ }% space after theorem head; " " = normal interword space
{\thmname{#1}\thmnumber{ #2}{\thmnote{ (#3)}.\ }}


\theoremstyle{style}
\newtheorem{esempio}{Esempio}[section]
\newtheorem{definizione}{Definizione}[section]
\newtheorem{prop}{Proposizione}[section]
\newtheorem{teorema}{Teorema}[section]
\newtheorem{lemma}{Lemma}[teorema]
\newtheorem{corollario}{Corollario}[teorema]
\newtheorem{osservazione}{Osservazione}[section]
\newtheorem{notazione}{Notazione}[section]
\newtheorem{esercizio}{Esercizio}[section]





\tcolorboxenvironment{definizione}{blanker,breakable,left=5mm,before skip=10pt,after skip=10pt, borderline west={.5mm}{0pt}{mastercolor}, before upper={\setlength{\parindent}{15pt}}}
\tcolorboxenvironment{lemma}{blanker,breakable,left=5mm,before skip=10pt,after skip=10pt, borderline west={.5mm}{0pt}{mastercolor}, before upper={\setlength{\parindent}{15pt}}}
\tcolorboxenvironment{teorema}{enhanced,blanker,breakable,left=5mm,before skip=10pt,after skip=10pt, borderline west={.5mm}{0pt}{mastercolor}, before upper={\setlength{\parindent}{15pt}}}
\tcolorboxenvironment{corollario}{blanker,breakable,left=5mm,before skip=10pt,after skip=10pt, borderline west={.5mm}{0pt}{mastercolor}, before upper={\setlength{\parindent}{15pt}}}
\tcolorboxenvironment{prop}{blanker,breakable,left=5mm,before skip=10pt,after skip=10pt, borderline west={.5mm}{0pt}{mastercolor}, before upper={\setlength{\parindent}{15pt}}}
\tcolorboxenvironment{esempio}{blanker,breakable,left=5mm,before skip=10pt,after skip=10pt, borderline west={.5mm}{0pt}{mastercolor}, before upper={\setlength{\parindent}{15pt}}}
\tcolorboxenvironment{esercizio}{blanker,breakable,left=5mm,before skip=10pt,after skip=10pt, borderline west={.5mm}{0pt}{mastercolor}, before upper={\setlength{\parindent}{15pt}}}
\tcolorboxenvironment{osservazione}{blanker,breakable,left=5mm,before skip=10pt,after skip=10pt, borderline west={.5mm}{0pt}{mastercolor}, before upper={\setlength{\parindent}{15pt}}}


\newenvironment{svolgimento}{\renewcommand\qedsymbol{$\blacksquare$}\begin{proof}[Svolgimento]}{\end{proof}}




%% Generic box
\newtcolorbox{eqbox}[1][]
{
colback=gray!10,
arc=0pt,
boxrule=0pt,
title=#1
}

 \newenvironment{boxenv}[1][]{
    \begin{eqbox}[#1]
    }{
   \end{eqbox}
}



%%%%%%%%%% Medie con integrali multipli
\def\Yint#1{\mathchoice
    {\YYint\displaystyle\textstyle{#1}}%
    {\YYint\textstyle\scriptstyle{#1}}%
    {\YYint\scriptstyle\scriptscriptstyle{#1}}%
    {\YYint\scriptscriptstyle\scriptscriptstyle{#1}}%
      \!\iint}
\def\YYint#1#2#3{{\setbox0=\hbox{$#1{#2#3}{\iint}$}
    \vcenter{\hbox{$#2#3$}}\kern-.51\wd0}}
\def\longdash{{-}\mkern-3.5mu{-}} 
   % consider using "\mkern-7.5mu" if esint package is loaded
\def\tiltlongdash{\rotatebox[origin=c]{15}{$\longdash$}}
\def\fiint{\Yint\tiltlongdash}

\def\Zint#1{\mathchoice
    {\YYint\displaystyle\textstyle{#1}}%
    {\YYint\textstyle\scriptstyle{#1}}%
    {\YYint\scriptstyle\scriptscriptstyle{#1}}%
    {\YYint\scriptscriptstyle\scriptscriptstyle{#1}}%
      \!\iiint}
      \def\tilongdash{\mkern6mu{-}\mkern-4mu{-}\mkern-5mu{-}} 
   % consider using "\mkern-7.5mu" if esint package is loaded
\def\titiltlongdash{\rotatebox[origin=c]{15}{$\tilongdash$}}
\def\fiiint{\Zint\titiltlongdash}

%Captions
\captionsetup[figure]{font=footnotesize,labelfont=footnotesize}
\captionsetup[table]{font=footnotesize,labelfont=footnotesize}
%Titlesec
\titleformat{\section}
{\fontsize{20}{20}\scshape}
{\color{mastercolor}{\fontsize{30}{20}\selectfont\thesection\hspace{.2cm}\color{mastercolor}{\vrule width 1pt}}}
{0.7em}
{}
\titlespacing*{\section}{0pt}{*2}{1cm}
\titlespacing*{\subsection}{0pt}{*5}{.5cm}
\titlespacing*{\subsubsection}{0pt}{*5}{.5cm}

\hypersetup{colorlinks,breaklinks, linkcolor=[RGB]{166,124,0}}

% Personalizza la formattazione della subsection
\titleformat{\subsection}[block]{\centering\fontsize{14}{20}\bfseries}{\normalfont\color{mastercolor}\S\thesubsection}{.5em}{}


% Personalizza la formattazione della subsubsection
\titleformat{\subsubsection}[block]{\centering\fontsize{12}{20}\bfseries}{\normalfont\color{mastercolor}\S\thesubsubsection}{.5em}{}

% Maketitle customization
\renewcommand{\maketitle}{
\begin{center}
{\sffamily
{\fontsize{20}{20}\selectfont\MakeUppercase\thetitle}}

\vspace{0.2in}

{\large\scshape\theauthor}
\end{center}
}

%Evaluate symbol
\DeclareMathOperator{\di}{d\!}
\newcommand*\Eval[3]{\left.#1\right\rvert_{#2}^{#3}}

%%%%%%% Numero delle equazioni in formato a.b
\numberwithin{equation}{subsection}
%%%%%

%%%%%%%%%% Personalizzazione numeri lista
\renewcommand{\theenumi}{(\arabic{enumi})}

%%%% Table of contents

\usepackage[titles]{tocloft}

\renewcommand{\cftdot}{}
\usepackage{titletoc}
%\setcounter{tocdepth}{2}

%%%%%%%%%%%%%%%% Toc style

% Personalizzazione scritta indice


% Font
\usepackage{fontspec}
\usepackage{unicode-math}
\usepackage{kpfonts}



\newcommand{\longhookrightarrow}{\lhook\joinrel\longrightarrow}
\begin{document}
\maketitle
\tableofcontents 
\newpage




\section{Teoria dei gruppi}
\subsection{Automorfismi e azioni}
\begin{prop}
	Dato un gruppo $G$, si ha che $\operatorname{Int} G \lhd \operatorname{Aut} G$ e $\operatorname{Int} G \cong G / Z(G)$.
\end{prop}
\begin{definizione}
	[Azione]
	Un'azione di $G$ gruppo su $X$ insieme \`e un omomorfismo
	\[
		\gamma :
	\begin{array}
		{c c c}
		G & \xrightarrow{\qquad}& S(X) = \left\{ f : X \to X | f \text{ biettiva} \right\} \\
		g & \xmapsto{\qquad} & \psi _g : \psi _g(x) = g \cdot x
	\end{array}
	\] 
\end{definizione}
\noindent Cio\`e un'azione di $G$ permette di identificare un modo in cui un elemento del gruppo pu\`o agire (tramite una permutazione) sull'insieme $X$.
\begin{boxenv}[]
Un'azione di gruppo \`e ben definita se:
\begin{enumerate}[(a).]
	\item $e \cdot x  = x, \ \forall  x \in X$, con $e \in G$ identit\`a;
	\item $h\cdot (g\cdot x) = (hg)\cdot x$, per $g,h \in G$ e $x \in X$.
\end{enumerate}
\end{boxenv}
\noindent Relativamente ad un'azione $\gamma : G \to S(X)$, si definiscono:
\begin{itemize}
	\item \textbf{orbita:} dato $x \in X$, la sua orbita \`e l'insieme $\operatorname{Orb} x = \left\{ g\cdot x  \mid g \in G \right\} $;
	\item \textbf{stabilizzatore:} dato $x \in X$, il suo stabilizzatore \`e l'insieme 
	\[
	\operatorname{Stab} x = \left\{ g \in G  \mid g\cdot x = x \right\}<G
	\] 
\end{itemize}
\noindent Le orbite partizionano $X$, visto che $x \sim_\gamma y \iff \operatorname{Orb} x = \operatorname{Orb} y$, quindi:
\begin{boxenv}[]
\[
\lvert X \rvert = \sum_{x \in \mathcal{R}}^{} \lvert \operatorname{Orb} x \rvert  
\] 
\end{boxenv}
\begin{lemma}
	[Orbita-stabilizzatore]
	Esiste una biezione $\operatorname{Orb} x \longrightarrow G / \operatorname{Stab} x$ definita da $g\cdot x \longmapsto g \operatorname{Stab} x$.
\end{lemma}
\noindent Per $X = G$ e $\gamma : G \longrightarrow \operatorname{Int} G \subset S(G) $ si ha l'azione per coniugio.
Le orbite sono le \textbf{classi di coniugio} $\operatorname{Cl} (x)$ e gli stabilizzatori sono detti \textbf{centralizzatori} $Z(x)$.
Per il lemma orbita-stabilizzatore, si ha $\lvert G \rvert = \lvert \operatorname{Cl}( x) \rvert \lvert Z(x) \rvert $.

Si pu\`o far agire $G$ su $X = \left\{ H \le G \right\} $ con $g\cdot H = g H g^{-1}$.
In questo caso, le orbite non hanno un nome particolare, ma gli stabilizzatori si dicono \textbf{normalizzatori} $N_G(H)$.
In questo senso, $H \lhd G \iff N_G(H) = G$.
Questo significa che $N_G(H)$ contiene tutti i generatori $g_1,\ldots,g_n$ di $G$, quindi $g_i H g_i^{-1} = H,\ \forall i$.

Dall'azione per coniugio, si ottiene la \textbf{formula delle classi di coniugio}:
\begin{boxenv}[]
\[
\lvert G \rvert = \lvert Z(G) \rvert + \sum_{x \in \mathcal{R} \setminus Z(G)}^{} \frac{\lvert G \rvert }{\lvert Z(x) \rvert }
\] 

\end{boxenv}

\subsection{I p-gruppi}
\begin{definizione}
	Un $p$-gruppo \`e un gruppo $G$ di ordine $p^n$ per qualche $n\in \mathbb{N}$.
\end{definizione}
\begin{prop}
	Il centro di un $p$-gruppo \`e non-banale.
\end{prop}
\begin{prop}
	Un gruppo di ordine $p^2$ \`e abeliano.
\end{prop}
\begin{teorema}
	Ogni $p$-gruppo $G$ di ordine $p^n$ ha sottogruppi $G_k$ di ordine $p^k, \ k=0,\ldots,n$ tali che
	\[
	\left\{ e \right\}=G_0 \lhd G_1 \lhd \ldots \lhd G_{n-1} \lhd G_n=G
	\] 
	
\end{teorema}

\subsection{Teoremi di Cauchy e Cayley}
\begin{teorema}
	[Cauchy]
	Sia $p$ un primo e $G$ un gruppo finito; se $p  \mid |G|$, allora $G$ ha un elemento di ordine $p$.
\end{teorema}
\begin{teorema}
	[Cayley]
	Ogni gruppo $G$ \`e isomorfo a un sottogruppo di $S(G)$.
	Se $\lvert G \rvert = n$, allora $G\longhookrightarrow S_n$.
\end{teorema}
\subsection{Commutatore e gruppo derivato}
\begin{definizione}
	[Derivato]
	Dato $G$ gruppo, si definisce il derivato come
	\[
		G '=[G:G] := \langle [g,h]  \mid g,h \in G \rangle 
	\] 
	cio\`e \`e il pi\`u piccolo sottogruppo di $G$ contenente tutti i commutatori.
\end{definizione}
\noindent Le sue propriet\`a sono le seguenti:
\begin{itemize}
	\item $G'  =\left\{ e \right\} \iff G$ abeliano;
	\item $G' \lhd G$;
	\item $G'$ caratteristico in $G$;
	\item se $H \lhd G$ \`e tale che $G / H$ \`e abeliano, allora $G' \subset H$.
\end{itemize}
\begin{prop}
	Sia $G$ un gruppo e $G'$ il suo derivato.
	Allora $G_{\text{ab}} = G / G'$ \`e abeliano ed \`e il pi\`u grande quoziente abeliano di $G$.
\end{prop}
\subsection{Il gruppo diedrale}
\begin{prop}
	Tutti gli elementi di $D_n$ si scrivono come $\sigma \rho ^i$, oppure come $\rho ^i$, per $i=0,\ldots,n-1$.
\end{prop}
\begin{prop}
	In $D_n$, il numero di elementi di ordine $k$ \`e dato da:
	\[
	\begin{cases}
		n+1&,\ \text{ se }k = 2, \ n \text{ pari}\\
		n &,\ \text{ se }k=2, \ n \text{ dispari}\\
		\phi (k) &,\ \text{ se }k \mid n\\
		0&,\ \text{altrimenti}
	\end{cases}
	\] 
\end{prop}
\noindent Di seguito, si riportano tutte le caratteristiche riguardanti la struttura di $D_n$.
\begin{itemize}
	\item \textbf{Sottogruppi.} Un sottogruppo di $D_n$ pu\`o essere composto da sole rotazioni, caso in cui coincide con un sottogruppo di $\mathbb{Z}/n\mathbb{Z}$, oppure ha, in egual numero, rotazioni e riflessioni, caso in cui \`e isomorfo a $D_m$, per qualche $m$.
	\item \textbf{Sottogruppi normali.} Visto che $[D_n:C_n] = 2$, allora $C_n \lhd D_n$.
		Ogni sottogruppo di $C_n$ \`e caratteristico in $C_n$ perch\'e unico, quindi \`e automaticamente normale in $D_n$.
		Se $n$ \`e pari, si pu\`o definire $H = \langle \rho ^2 \rangle \sqcup \tau \langle \rho ^2 \rangle$, per cui $[D_n :H ] = 2 \Rightarrow H \lhd D_n$.
		In questo caso, sottogruppi di questa forma sono $\langle \rho ^2 , \sigma  \rangle$ e $\langle \rho ^2 , \sigma \rho  \rangle$.
		Se $n$ \`e dispari, invece, un sottogruppo normale contenente una riflessione, le deve contenere tutte, quindi coincide con $D_n$.
	\item \textbf{Sottogruppi caratteristici.} Per $n\ge 3$, $C_n$ e i suoi sottogruppi di ordine $d> 2, \ d \mid n$ sono gli unici ad essere sempre caratteristici.
		Per gli $n$ pari, $\langle \rho ^2 ,\sigma  \rangle$ e $\langle \rho ^2 ,\sigma \rho  \rangle$ non sono caratteristici perch\'e $\tau :D_n\to D_n$ con $\tau (\rho ) = \rho $ e $\tau (\sigma ) = \sigma \rho $	\`e un automorfismo ben definito che scambia i due sottogruppi.
	\item \textbf{Centro.} Se $n$ \`e dispari, $Z(D_n) = \left\{ e \right\} $, mentre, se $n$ \`e pari, $Z(D_n) = \left\{ e , \rho ^{n / 2}  \right\} \cong \mathbb{Z}/2\mathbb{Z}$.
	\item \textbf{Quozienti.} Questi sono in corrispondenza biunivoca con i sottogruppi normali.
		In generale, si ha $D_n / \langle \rho ^m \rangle\cong D_m$.
		Per $n$ pari, invece, i quozienti relativi a $\langle \rho ^2,\sigma  \rangle$ e $\langle \rho ^2 ,\sigma \rho  \rangle$ hanno indice due, quindi sono isomorfi a $\mathbb{Z}/2\mathbb{Z}$.
	\item \textbf{Automorfismi.} Un automorfismo di $D_n$ \`e della forma
		\[
			\gamma: 
		\begin{array}
			{c c c}
			D_n &\longrightarrow& D_n\\
			\rho & \longmapsto & \rho ^i \\
			\sigma &\longmapsto &\sigma \rho ^j
		\end{array}\ , \hspace{1cm} \operatorname{gcd}(i,n)=1
		\] 
		Allora $\lvert \operatorname{Aut} (D_n) \rvert = n \phi (n)$.
\end{itemize}
\begin{boxenv}[]
\centering $D_n \cong \mathbb{Z}/ n\mathbb{Z}\rtimes _\varphi \mathbb{Z}/ 2\mathbb{Z}$
\end{boxenv}
\subsection{Il gruppo simmetrico}

\begin{prop}
	Ogni $k$-ciclo ha $k$ scritture equivalenti.
\end{prop}

\begin{prop}
	I cicli di una permutazione di $S_n$ sono le orbite degli elementi di $X = \left\{ 1,\ldots,n \right\} $ formate dall'azione indotta da tale permutazione.
\end{prop}

\begin{corollario}
	$S_n$ \`e generato dai cicli.
\end{corollario}
\begin{prop}
	Ogni permutazione si scrive come composizione di trasposizioni.
\end{prop}
\noindent L'applicazione \textbf{segno} \`e definita da
\[
\operatorname{sgn} :
\begin{array}
	{c c c}
	S_n & \longrightarrow & \left\{ \pm 1 \right\} \\
	\sigma & \longmapsto & \displaystyle \prod_{1\le i<j\le n}  \frac{\sigma (i) - \sigma (j)}{i - j}
\end{array}
\] 
ed \`e un omomorfismo di gruppi.
Vale $-1$ sulle trasposizioni; infatti, restituisce la parit\`a del numero di trasposizioni presenti nella decomposizione di una permutazione.
Il suo nucleo coincide con $A_n\lhd S_n$.
\begin{teorema}
	Due permutazioni id $S_n$ sono coniugate se e solo se hanno lo stesso tipo di decomposizione in cicli disgiunti.
\end{teorema}
\noindent Di seguito, la caratterizzazione di $S_n$ e dei suoi elementi.
\begin{itemize}
	\item \textbf{Numero di un certo tipo di permutazioni con precisa decomposizione.} 
		In $S_n$, il numero complessivo di $k$-cicli \`e ottenuto tramite 
		\[
		\binom{n}{k}(k-1)!
		\] 
		Volendo cercare quante permutazioni con una precisa decomposizione in cicli disgiunti ci sono, si procede come da esempio. 
		In $S_{12}$, il numero di permutazioni date dalla composizione di due $3$-cicli e tre $2$-cicli \`e
		\[
			\binom{12}{3}\frac{3!}{3}\binom{9}{3}\frac{3!}{3}\binom{6}{3}\frac{2!}{2}\binom{4}{3}\frac{2!}{2}\binom{2}{3}\frac{2!}{2}\frac{1}{3!2!}
		\] 
		Questo si generalizza nella seguente formula:
		\[
			\frac{n!}{\prod_{k\ge 1} \left[k^{m_k} (m_k!)\right] }
		\] 
		con $m_k$ numero di $k$-cicli.
	\item \textbf{Ordine di una permutazione.} 
Un $k$-ciclo ha ordine $k$; se una permutazione \`e composta da $\ell $ cicli disgiunti $\sigma _i$, allora il suo ordine \`e 
\[
\operatorname{lcm} \big(\operatorname{ord}(\sigma _1),\ldots,\operatorname{ord}(\sigma _\ell )  \big)
\] 
	\item \textbf{Centralizzatore di una permutazione.} Sapendo che due permutazioni sono coniugate se e solo se hanno lo stesso tipo di decomposizione in cicli disgiunti, si sa calcolare $\lvert \operatorname{Cl} (\sigma ) \rvert $ tramite la formula al primo punto. 
		Per orbita-stabilizzatore, si ha $\lvert Z(\sigma ) \rvert \lvert \operatorname{Cl} (\sigma ) \rvert = n!$, quindi si pu\`o calcolare $\lvert Z(\sigma ) \rvert $.
\end{itemize}
\begin{prop}
	Per la formula delle classi, $\lvert Z_{S_n} (\sigma ) \rvert \lvert \operatorname{Cl} _{S_n} (\sigma ) \rvert = n!$ e $\lvert Z_{A_n} (\sigma ) \rvert \lvert \operatorname{Cl} _{A_n} (\sigma ) \rvert = n!/2$, con:
	\[
	Z_{A_n} (\sigma ) = Z_{S_n} (\sigma )\cap A_n
	\] 
	Per la stessa formula, nel passare da $\operatorname{Cl} _{S_n} (\sigma )$ a $\operatorname{Cl} _{A_n} (\sigma )$ e da $Z_{S_n} (\sigma )$ a $Z_{A_n} (\sigma )$, uno dei due dimezza di ordine, mentre l'altro rimane invariato.
\end{prop}
\begin{prop}
	Dato $H < S_n$, allora o $H \subset A_n$, quindi $\lvert H\cap A_n \rvert = \lvert H \rvert $, oppure $\lvert H\cap A_n \rvert = \lvert H \rvert / 2$.
\end{prop}
\begin{prop}
	I $3$-cicli sono tutti coniugati in $A_n$, per $n\ge 5$.
\end{prop}
\begin{prop}
	I $5$-cicli in $A_5$ NON sono tutti coniugati.
\end{prop}
\begin{prop}
	$A_4$ non ha sottogruppi di ordine $6$.
\end{prop}
\begin{teorema}
	$A_n$ \`e semplice $\forall n \ge 5$.
\end{teorema}
\begin{boxenv}[]
\centering $S_n \cong A_n \rtimes \langle \tau  \rangle$, con $\tau $ trasposizione
\end{boxenv}
\subsection{I quaternioni}
Il gruppo \`e definito come $Q_8 = \langle i,j \mid i^4 = 1, \ i^2 = j^2 , \ ij=j^3 i \rangle$.
$i^4 = 1$ e $i^2 = j^2$, allora $j^4=1$, quindi $\operatorname{ord}(j)  \mid 4$. 
Poi $\operatorname{ord}(j^2) = \operatorname{ord}(i^2) = 2$, quindi $\operatorname{ord}(j) =4$.
Allora $Q_8$ ha due gruppi ciclici di ordine $4$: $\langle i \rangle$ e $\langle j \rangle$, con $\langle i \rangle\cap \langle j \rangle=\left\{ 1,i^2=j^2 \right\} $.
Visto che $\langle i \rangle,\langle j \rangle<Q_8$ e $\lvert \langle i \rangle\langle j \rangle \rvert = 8$, allora 
\[
Q_8 = \langle i \rangle\langle j \rangle=\left\{ 1,i, i^2 ,i^3,j,j^3 ,ij,i^3j \right\} 
\] 
visto che $\langle i \rangle, \langle j \rangle\lhd Q_8$ (hanno indice $2$).
\begin{osservazione}
$Q_8$ non \`e abeliano: $ij = j^3 i = j^{-1}i \neq ji$.
\end{osservazione}
\noindent Di seguito, la caratterizzazione strutturale del gruppo.
\begin{itemize}
	\item \textbf{Sottogruppi.} 
		$\langle i \rangle, \langle j \rangle\lhd Q_8$ perch\'e hanno indice $2$.
		Anche $\langle i^2 \rangle=\langle j^2 \rangle\lhd Q_8$ perch\'e $i^2$ (quindi $j^2$) commuta con i generatori.
	\item \textbf{Centro.} Si ha $\langle i^2 \rangle=Z(Q_8)$ perch\'e $\langle i^2 \rangle$ ha ordine $2$, quindi contenuto in $Z(Q_8)$; al contempo, $\lvert Z(Q_8) \rvert \in \left\{ 2,4,8 \right\} $, ma, se non fosse $2$, $Q_8$ sarebbe abeliano.
	\item \textbf{Elementi.} Prendendo $k = ij$ e $i^2 = -1$, si ha 
		\[
		Q_8 = \left\{ \pm 1 , \pm i, \pm j , \pm k \right\} 
		\] 
		Si ha $i^2 = -1 \Rightarrow  i^3 = -i \Rightarrow i^3 j = -ij = -k$.
		Quindi: $ij = k , \ jk = i, \ ki = j$ e $ji = -k, \ ik = -j $ e $kj = -i$.
		Infine, $k^2 = (ij)^2 = ijij=i^2$, quindi $\operatorname{ord}(k) =4$.
		In questi termini, $\langle -1 \rangle=Z(Q_8)$.
	\item \textbf{Sottogruppi normali e caratteristici.} Per quanto detto, $\langle -1 \rangle= Z(Q_8)$ quindi \`e caratteristico e, in particolare, normale.
		Invece $\langle i \rangle,\langle j \rangle,\langle k \rangle\lhd Q_8$, ma non sono caratteristici.
		Allora ogni sottogruppo di $Q_8$ \`e normale.
	\item \textbf{Prodotto semi-diretto.} Si nota che $Q_8$ non si pu\`o ottenere come prodotto semi-diretto perch\'e ogni coppia di sottogruppi non si interseca mai solo in $1$, ma anche $-1$.
\end{itemize}





























\subsection{Prodotti diretti}
\begin{teorema}
	[Decomposizione diretta]
Sia $G$ un gruppo e siano $H ,K \lhd G$; se $HK = G$ e $H\cap K = \left\{ e \right\} $, allora $G \cong H \times K$.
\end{teorema}
\begin{corollario}
In un prodotto diretto, i fattori commutano fra loro.
\end{corollario}
\begin{corollario}
	Se $G = H \times K$, allora $Z(H\times K) \cong Z(H) \times Z(K)$, visto che $Z(H) \times \left\{ e_k \right\} $ e $\left\{ e_H \right\} \times Z_k$ sono sottogruppi normali di $Z(H\times K)$.
	Questo implica che
	\[
	\operatorname{Int} (H\times K) \cong \frac{H\times K}{Z(H\times K)}\cong H / Z(H) \times K / Z(K) \cong \operatorname{Int} (H) \times \operatorname{Int} (K)
	\] 
\end{corollario}
\begin{teorema}
	Si ha $\operatorname{Aut} (H \times K) \cong \operatorname{Aut} (H) \times \operatorname{Aut} (K)$ se e soltanto se $H \times \left\{ e_K \right\} $ e $\left\{ e_H \right\} \times K$ sono caratteristici in $H \times K$.
	Altrimenti $ \operatorname{Aut} (H) \times \operatorname{Aut} (K)\longhookrightarrow  \operatorname{Aut} (H \times K)$.
\end{teorema}
\begin{corollario}
	Sia $G = H \times K$, con $\lvert H \rvert =n$ e $\lvert K \rvert = m$; se $\operatorname{gcd}(n,m) =1$, allora $H\times \left\{ e_K \right\} ,\ \left\{ e_H \right\} \times  K$ sono caratteristici in $G$.
\end{corollario}
\subsection{Prodotti semi-diretti}
\begin{definizione}
	[Prodotto semi-diretto]
	Siano $H,K$ due gruppi e $\gamma : K \to  \operatorname{Aut} (H)$ un omomorfismo tale che $\gamma(k)=\gamma_k \in \operatorname{Aut} (H)$.
	Allora si definisce $H \rtimes_\gamma K$ il gruppo $H\times K$ la cui operazione di gruppo \`e definita da
	\[
		(h,k) * (h',k') = \Big(h \gamma_{k} (h'),k k'\Big)
	\] 
	Il prodotto diretto \`e dato da $\gamma(K) = \operatorname{Id} _H$.
\end{definizione}
\begin{prop}
	Si considera $H \rtimes _\gamma K$ e si definiscono $\overline{H}= H \times \left\{ e_K \right\} $ e $\overline{K} = \left\{ e_H \right\} \times K$.
	Per costruzione, $\overline{K},\overline{H} \lhd H \times K$, mentre:
	\begin{itemize}
		\item $\overline{H}\lhd H \rtimes_\gamma K $ sempre;
		\item $\overline{K} \lhd H \rtimes _\gamma K \iff $ il prodotto \`e diretto.
	\end{itemize}
\end{prop}
\begin{teorema}
	[Decomposizione semi-diretta]
Sia $G$ un gruppo e siano $H \lhd G$ e $K < G$. 
Se $HK = G$ e $H\cap K = \left\{ e \right\} $, allora $G\cong H \rtimes _\gamma K$, con $\gamma : K \to \operatorname{Aut} (H)$	e $\gamma(k) = khk^{-1}$.
\end{teorema}
\subsection{Teorema di struttura per gruppi abeliani finiti}
\begin{definizione}
	[$p$-torsione]
	Dato un gruppo abeliano finito $G$, se ne definisce la $p$-componente 
	\[
	G(p) := \left\{ g \in G  \mid \operatorname{ord}(g) =p^k , k\in \mathbb{N} \right\} 
	\] 
\end{definizione}
\begin{prop}
	La $p$-torsione $G(p)$ di un gruppo $G$ abeliano finito \`e un sottogruppo caratteristico.
\end{prop}

\begin{teorema}
	Se $G$ \`e un gruppo abeliano di ordine $\lvert G \rvert = n = p_1^{e_1} \cdots p_s^{e_s}$, con $p_i$ primi diversi fra loro, allora 
	\[
	G \cong G(p_1) \times \ldots \times G(p_s)
	\] 
\end{teorema}
\begin{lemma}
	Sia $G$ un $p$-gruppo e $x_1 \in G$ elemento di ordine massimo.
	Dato anche $\overline{x}\in G / \langle x_1 \rangle$, $\exists y \in \pi^{-1}_{\langle x_1 \rangle} (\overline{x})$ tale che $\operatorname{ord}_G(y) = \operatorname{ord}_{G / \langle x_1 \rangle} (\overline{x}) $.
\end{lemma}

\begin{teorema}
Se $G$ \`e un $p$-gruppo abeliano, allora esistono unici $r_1,\ldots,r_t \in \mathbb{N}$ tali che 
\[
G \cong \mathbb{Z}/p^{r_1} \mathbb{Z} \times  \ldots \times  \mathbb{Z}/p_{r_t} \mathbb{Z}
\] 
con $r_1\ge r_2\ge \ldots \ge r_t$.
\end{teorema}
\begin{teorema}
	[Teorema di struttura]
	Sia $G$ un gruppo abeliano finito; allora $G$ si decompone univocamente come
	\[
	G \cong \mathbb{Z}/n_1\mathbb{Z}\times \ldots\times \mathbb{Z}/n_s\mathbb{Z}
	\] 
	dove $n_{i+1}  \mid n_i, \ \forall i = 1,\ldots,s-1$.
\end{teorema}

\subsection{Teoremi di Sylow}
Per i seguenti teoremi, si considera un gruppo finito $G$ di ordine $\lvert G \rvert =p^nm$, con $p$ primo e $\operatorname{gcd}(m,p) =1$.
\begin{teorema}
	[I teorema]
	Dato $\alpha \in \mathbb{N}$, con $0\le \alpha \le n$, allora $\exists H<G$ di ordine $\lvert H \rvert =p^\alpha $.
\end{teorema}
\begin{teorema}
	[II teorema]
	Ogni $p$-gruppo di $G$ \`e contenuto in un $p$-Sylow.
	Inoltre, due qualunque $p$-Sylow di $G$ sono coniugati.
\end{teorema}
\begin{teorema}
	[III teorema]
	Dato $n_p$ il numero di $p$-Sylow di $G$, si ha che $n_p  \mid \lvert G \rvert $ e $n_p \equiv 1 \operatorname{mod} p $.
	In particolare, si avr\`a $n_p  \mid  m$.
\end{teorema}


















\subsection{Risultati sulle classificazioni}
\begin{itemize}
	\item \textbf{Classificazione dei gruppi di ordine 6.} 

		Si ha $\lvert G \rvert = 2 \cdot 3$, quindi sono presenti, per Cauchy, un sottogruppo $P_2$ di ordine $2$ e un sottogruppo $P_3$ di ordine $3$.
		Visto che $P_3$ ha indice $2$ \`e normale in $G$. 
		Inoltre, $P_3\cap P_2 = \left\{ e \right\} $ perch\'e gli altri elementi di un gruppo hanno ordine coprimo con l'ordine dell'altro gruppo.
		Questo implica che $P_2P_3 = G$, quindi $G \cong P_3 \rtimes _\phi P_2$, con 
		\[
		\phi : \mathbb{Z}/2\mathbb{Z} \longrightarrow \operatorname{Aut} \mathbb{Z}/3\mathbb{Z} \cong (\mathbb{Z}/3\mathbb{Z})^* \cong \mathbb{Z}/2\mathbb{Z}
		\] 
		Ne segue che ci sono due possibili omomorfismi: $\phi (1) = 0$, che corrispnde al prodotto diretto $G\cong \mathbb{Z}/3\mathbb{Z} \times \mathbb{Z}/2\mathbb{Z}\cong \mathbb{Z}/6\mathbb{Z}$, oppure $\phi (1) = 2 \in (\mathbb{Z}/3\mathbb{Z})^*$.
		Riguardo l'ultimo caso, notando che $2 \equiv -1 \operatorname{mod} 3 $, si conclude che l'ultimo omomorfismo consiste nel prodotto per $-1$.
		Pertanto, dati $(a,b), (c,d) \in \mathbb{Z}/3\mathbb{Z} \rtimes _\phi \mathbb{Z}/2\mathbb{Z}$, il prodotto \`e definito da:
		\[
			(a,b)*(c,d) = \big(a+(-1)^b c, b + d\big)
		\] 
	Per finire, si nota che, per $a \in (\mathbb{Z}/3\mathbb{Z})^*$:
	\[
		(0,1)(a,0)(0,1)^{-1}=(0,1)(a,0)(0,1) = (-2a,0) = (a,0)^{-1}
	\] 
	Quindi, $G$ soddisfa la presentazione di $S_3$, per cui $G \cong S_3$.
	Si conclude che se $G$ \`e un gruppo di ordine $6$, le possibilit\`a sono:
	\begin{boxenv}[]
	\[
	\mathbb{Z}/6\mathbb{Z} \hspace{1cm} S_3
	\] 
	\end{boxenv}
	rispettivamente nel caso abeliano e non-abeliano.
	\item \textbf{Classificazione dei gruppi di ordine pq.} 
		Se $p=q$, $G$ \`e abeliano e $G \cong \mathbb{Z}/p\mathbb{Z} \times \mathbb{Z}/p\mathbb{Z}$ oppure $G\cong \mathbb{Z}/p^2 \mathbb{Z}$.

		Se $q>p$ e $p \not  \mid q-1$, allora si pu\`o avere solo $G \cong \mathbb{Z}/pq\mathbb{Z}$; altrimenti si ha un prodotto semi-diretto non-banale, unico a meno di isomorfismo.
	\item \textbf{Classificazione dei gruppi di ordine 12.} 
		\begin{boxenv}[]
		\[
		\mathbb{Z}/12 \mathbb{Z} \hspace{1cm}\mathbb{Z}/2\mathbb{Z}\times \mathbb{Z}/6\mathbb{Z} \hspace{1cm}A_4 \hspace{1cm}\mathbb{Z}/3\mathbb{Z} \rtimes _\varphi \mathbb{Z}/4\mathbb{Z} \hspace{1cm} D_6
		\] 
		
		\end{boxenv}
		
	\item \textbf{Classificazione dei gruppi di ordine 8.} 
	\begin{boxenv}[]
	\[
	\mathbb{Z}/8\mathbb{Z} \hspace{1cm}\mathbb{Z}/4\mathbb{Z} \times \mathbb{Z}/2\mathbb{Z} \hspace{1cm} (\mathbb{Z}/2\mathbb{Z})^3 \hspace{1cm}D_4 \hspace{1cm} Q_8
	\] 
	\end{boxenv}
	\item \textbf{Classificazione dei gruppi di ordine 30.} 
		\begin{boxenv}[]
		\[
		\mathbb{Z}/30\mathbb{Z} \hspace{1cm} D_{15} \hspace{1cm} D_5 \times \mathbb{Z}/3\mathbb{Z} \hspace{1cm}D_3 \times \mathbb{Z}/5\mathbb{Z}
		\] 
		\end{boxenv}
\end{itemize}


\subsection{Risultati vari sui gruppi}

\begin{prop}
	$G / Z(G)$ ciclico $\iff G$ abeliano.
\end{prop}
\begin{prop}
	Se $H,K < G$, allora $HK < G \iff HK = KH$; in questo caso, $\lvert HK \rvert = \frac{\lvert H \rvert \lvert K \rvert }{\lvert H\cap K \rvert }$.
\end{prop}
\begin{prop}
	Se $H,K \lhd G$, con $H\cap K = \left\{ e \right\}$, allora $hk = kh, \ \forall h \in H, \ \forall k \in K$.
\end{prop}
\begin{prop}
	Sia $H<G$ con $[G:H]= 2$; allora $H\lhd G$.
\end{prop}
\begin{prop}
	Siano $H\lhd G$ e $K$ sottogruppo caratteristico di $H$; allora $K\lhd G$.
\end{prop}
\begin{prop}
	Sia $H<G$, con $\lvert H \rvert =2$; allora $H$ \`e normale se e solo se $H < Z(G)$.
\end{prop}
\begin{teorema}
Se $H < G$ abeliano, allora $\operatorname{Hom} (G,H) \longleftrightarrow \operatorname{Hom} (G / G' , H)$.
\end{teorema}
\begin{teorema}
$S'_n = A_n$.
\end{teorema}
\begin{teorema}
$Z(S_n) = \left\{ e \right\} $, per $n > 2$.
\end{teorema}
\begin{lemma}
	[Normalizzatore-Centralizzatore]
	Dato $H< G$, si ha:
	\begin{enumerate}[(a).]
		\item $Z_G(H) \lhd N_G(H)$;
		\item $N_G (H) / Z_G(H) \longhookrightarrow \operatorname{Aut} (H)$.
	\end{enumerate}
\end{lemma}
\begin{teorema}
	[Isomorfismo di prodotti semi-diretti]
Siano $N,H$ due gruppi e $\varphi  : H \to \operatorname{Aut} (N)$. 
Se $f \in \operatorname{Aut} (H)$, allora:
\[
N \rtimes _\varphi H \cong N \rtimes _{\varphi \circ f} H
\] 
\end{teorema}


\newpage
\section{Teoria degli anelli}
\subsection{Propriet\`a di base}
\begin{prop}
	Sia $A$ un anello commutativo con identit\`a; allora:
	\begin{enumerate}[(a).]
		\item $(A^*,\cdot )$ \`e un gruppo abeliano;
		\item $A^* \cap D(A) = \varnothing$;
		\item se $A$ \`e finito, allora $A = D(A) \cup A^*$.
	\end{enumerate}
\end{prop}
\noindent \textit{Dimostrazione (c).} $A^*, D(A) \subseteq A$, quindi $A^* \cup D(A) \subseteq A$. 
Viceversa, per vedere che $A\subseteq A^* \cup D(A)$, si nota che se $x \in D(A)$, la tesi \`e versa, mentre se $x \in A \setminus D(A)$, allora si pu\`o definire 
\[
\varphi _x : 
\begin{array}
	{c c c}
	A & \longrightarrow & A\\
	a & \longmapsto & xa
\end{array}
\] 
con $\operatorname{Ker} \varphi _x = \left\{ a \in A  \mid xa = 0 \right\} $. 
Per\`o $x \not \in D(A)$, quindi $\operatorname{Ker} \varphi _x = \left\{ 0 \right\} $; usando che $A$ \`e finito, $\varphi _x$ \`e iniettiva e, quindi, anche suriettiva, per cui $1 \in \operatorname{Im} \varphi _x$.
Questo significa che $\exists \overline{a} \in A$ per cui $x\overline{a} = 1$, quindi $x \in A^*$.
\qed
\begin{definizione}
	[Ideale]
	Dato $A$ anello, $I \subseteq A$ \`e un \textit{ideale} se
	\begin{enumerate}[(a).]
		\item $(I,+) < (A,+)$;
		\item per ogni $ a \in A$, si ha $ aI \subset I$ e $Ia \subset I$.
	\end{enumerate}
\end{definizione}
\begin{definizione}
	[Ideale generato]
	Dato $A$ anello e $S=\left\{ s_1,\ldots,s_n \right\}  \subset A$, l'ideale \textit{generato} da $S$ \`e:
	\[
	\langle S \rangle := \left\{ \sum_{i=1}^{n} a_is_i\ \Bigg \lvert\ a_i \in A\right\} 
	\] 
\end{definizione}
\begin{prop}
Dato $A$ anello e $I,J \subseteq A$ due suoi ideali, le seguenti operazioni producono altri ideali:
\begin{enumerate}[(a).]
	\item $I\cap J$;
	\item $I+J:= \langle I,J \rangle= \left\{ i+j  \mid i \in I , \j \in J \right\} $;
	\item $IJ = \left\{ \sum_{k=1}^{n} i_kj_k  \mid n\ge 1, \ i_k \in I, \ j_k \in J \right\} $;
	\item $\sqrt{I} = \left\{ x \in A  \mid \exists n \in \mathbb{N} : x^n \in I \right\} $;
	\item $\left(I:J\right) = \left\{ x \in A  \mid xJ \subseteq I \right\} $.
\end{enumerate}
\end{prop}
\begin{prop}
	$A$ anello e $I,J$ ideali; in generale, $IJ \subseteq I\cap J$, mentre $IJ = I \cap J$ se e solo se $ I+ J = A$.
\end{prop}
\begin{prop}
	$I \subset A$ \`e un ideale proprio se e solo se $I \cap A^* = \varnothing$.
\end{prop}
\begin{corollario}
$A$ \`e un campo se e solo se i suoi unici ideali sono $\left\{ 0 \right\} $ e $A$.
\end{corollario}
\subsection{Omomorfismi e quoziente}
\begin{prop}
	Gli ideali di un anello $A$ sono tutti e soli i nuclei degli omomorfismi da $A$.
\end{prop}
\begin{teorema}
	[I teorema di omomorfismo]
	Sia $f : A \to B$ un omomorfismo; allora esiste un unico omomorfismo iniettivo $ \varphi : A / \mathrm{Ker} (f) \to B$ tale che $f = \varphi \circ \pi$, ossia con $\operatorname{Im} f = \operatorname{Im} \varphi $.
\end{teorema}
\begin{teorema}
	[II teorema di omomorfismo]
	Dati $I,J \subset A$ due ideali, con $I \subset J$, allora $J / I$ \`e un ideale di $A / I$ e 
	\[
	\frac{A / I}{J / I} \cong A / J
	\] 
\end{teorema}
\begin{teorema}
	[III teorema di omomorfismo]
	Sia $I \subset A$ ideale e $B \subset A$ sottoanello; allora:
	\[
	\frac{B+I}{I}\cong \frac{B}{B\cap I}
	\] 
\end{teorema}
\begin{lemma}
	Sia $f : A \to  B$ un omomorfismo; allora:
	\begin{enumerate}[(a).]
		\item $\forall J$ ideale di $B$, si ha che $f^{-1}(J)$ \`e un ideale di $A$;
		\item se $f$ \`e suriettiva, allora $\forall I$ ideale di $A$, si ha che $f(I)$ \`e un ideale di $B$.
	\end{enumerate}
\end{lemma}
\begin{teorema}
	[Teorema di corrispondenza]
	Sia $I$ un ideale di $A$ e $\pi$ la proiezione al quoziente $A / I$. 
	Tale proiezione induce una corrispondenza biunivoca tra gli ideali di $A / I$ e gli ideali di $A$ che contengono $I$.
\end{teorema}
\begin{teorema}
	[Teorema cinese del resto]
	Sia $A$ un anello commutativo con unit\`a e $I,J$ due suoi ideali; allora 
	\[
	f:
	\begin{array}
		{c c c}
		A & \longrightarrow & A / I \times A / J\\
		a & \longmapsto & (a + I , a+ J )
	\end{array}
	\] 
	\`e un omomorfismo, con $\operatorname{Ker} f = I \cap J$.
	Inoltre, vale $I + J = A \iff f$ \`e suriettiva; in questo caso:
	\[
	A / IJ \cong A / I \times A / J
	\] 
\end{teorema}
\subsection{Ideali primi e ideali massimali}
\begin{definizione}
	[Maggiorante]
	Dato $(\mathcal{F} ,\le )$ un insieme parzialmente ordinato e $X \subset \mathcal{F} $ un sottoinsieme, si dice che $M \in \mathcal{F} $ \`e un maggiorante per $X$ se, $\forall A \in X$, $ \ A\le M$.
\end{definizione}
\begin{definizione}
	[Elemento massimale]
	Dato $(\mathcal{F} ,\le )$, si ha $A \in \mathcal{F} $  elemento massimale per $\mathcal{F} $ se, $\forall B \in \mathcal{F} : A\le B$, si ha $A = B$.
\end{definizione}
\begin{definizione}
	[Massimo]
	$A \in \mathcal{F} $ \`e detto \textit{massimo} per $\mathcal{F} $ se, $\forall B \in \mathcal{F} $, si ha $B \le A$.
\end{definizione}
\begin{definizione}
	[Catena]
	Una catena di $\mathcal{F} $ \`e un suo sottoinsieme totalmente ordinato.
\end{definizione}
\begin{definizione}
	[Insieme induttivo]
	Si dice che $\mathcal{F} $ \`e induttivo se ogni sua catena ammette un maggiorante al suo interno.
\end{definizione}
\begin{lemma}
	[Lemma di Zorn]
	Se $(\mathcal{F} ,\le )$ \`e un insieme parzialmente ordinato e induttivo, allora contiene elementi massimali.
\end{lemma}
\begin{definizione}
	[Ideale primo]
	Un $I$ ideale proprio di $A$ anello, si dice \textit{primo} se
	\[
	xy \in I \implies x \in I \ \text{ oppure }\ y \in I, \ \forall x,y \in A
	\] 
\end{definizione}
\begin{definizione}
	[Ideale massimale]
	$I \subsetneq A$ \`e detto \textit{massimale} se \`e un elemento massimale della famiglia $\mathcal{F} $ di tutti gli ideali propri di $A$, cio\`e se e solo se $\forall J \subsetneq A : I \subseteq J \implies I = J$.
\end{definizione}
\begin{prop}
Ogni anello unitario ammette ideali massimali.
\end{prop}
\begin{prop}
	[Propriet\`a degli ideali massimali]
	Dato $A$ anello, si ha che
	\begin{enumerate}[(a).]
		\item ogni ideale proprio di $A$ \`e contenuto in un ideale massimale;
		\item ogni elemento non-invertibile di $A$ \`e contenuto in un ideale massimale.
	\end{enumerate}
\end{prop}
\begin{prop}
	[Caratterizzazione degli ideali primi e massimali]
	Sia $A$ un anello e $I \subsetneq A$ un suo ideale proprio; allora:
	\begin{enumerate}[(a).]
		\item $I$ \`e primo se e solo se $A / I$ \`e un dominio;
		\item $I$ \`e massimale se e solo se $A / I $ \`e un campo;
		\item $A$ \`e un dominio se e solo se $(0)$ \`e un ideale primo;
		\item $A$ \`e un campo se e solo se $(0)$ \`e un ideale massimale;
		\item $I$ massimale $\implies I$ primo.
	\end{enumerate}
\end{prop}
\begin{prop}
La biezione tra ideali data da $\pi : A \to A / I$ preserva ideali primi e massimali (contenenti $I$).
\end{prop}
\subsection{Anello delle frazioni}
\begin{definizione}
	[Parte moltiplicativa]
	Dati $A$ dominio (commutativo e con identit\`a) e $S \subset A$, si dice che $S$ \`e una \textit{parte moltiplicativa} di $A$ se:
	\begin{enumerate}[(a).]
		\item $0 \not\in S$;
		\item $1 \in S$;
		\item $S$ \`e chiuso sotto moltiplicazione, cio\`e, dati $x,y \in S$, allora $xy \in S$.
	\end{enumerate}
\end{definizione}
\begin{definizione}
	[Insieme delle frazioni]
	Dato un dominio $A$ e data $S$ una sua parte moltiplicativa, si definisce l'\textit{insieme delle frazioni} come
	\[
		S^{-1}A = \faktor{\left\{ \frac{a}{s}\ \Bigg \lvert \ a \in A, \ s \in S\right\}}{\mathrm{\sim} }
	\] 
	dove $a / s \sim b/t \iff at = bs$.
\end{definizione}
\begin{prop}
L'applicazione 
\[
f: 
\begin{array}
	{c c c}
	A &\longrightarrow & S^{-1}A \\
	a & \longmapsto & a / 1
\end{array}
\] 
\`e un omomorfismi iniettivo, quindi $S^{-1}A$ \`e un'estensione di $A$.
\end{prop}
\begin{prop}
Sia $A$ un dominio e $S = A \setminus \left\{ 0 \right\} $ una sua parte moltiplicativa; allora l'anello delle frazioni $S^{-1}A$ \`e il pi\`u piccolo campo contenente $A$.
\end{prop}
\begin{definizione}
	[Localizzato]
	Dato un dominio $A $ e $P \subset A$ un suo ideale primo, considerando $S = A \setminus P$, si definisce $S ^{-1} A = A_p$ come il localizzato di $A$ a $P$.
\end{definizione}
\begin{osservazione}
$A_p$ \`e un anello locale, ossia ha un unico ideale massimale.
\end{osservazione}
\noindent Di seguito, alcune caratteristiche di $S^{-1}A$.
\begin{itemize}
	\item \textbf{Invertibili.} Sono tutti gli $a / s \in S^{-1} A$ tali che $s / a \in S^{-1}A$, ossia quelli tali che $\exists b \in A : ab \in S$.
	\item \textbf{Ideali.} Dato $I \subset A$, si costruisce l'insieme $S^{-1}I = \left\{ x / s \in S^{-1}A  \mid x \in I , \ s \in S \right\} $.
		Per questo, vale la seguente proposizione.
\end{itemize}
\begin{prop}
Sia $I \subset A $ e $S ^{-1}A$ l'anello delle frazioni di $A$. Allora:
\begin{enumerate}[(a).]
	\item $S^{-1}I$ \`e un ideale di $S^{-1}A$;
	\item per ogni ideale $J \subset S^{-1}A$, si trova un ideale $I \subset A$ tale che $J = S^{-1}I$;
	\item $S^{-1}I$ \`e proprio se e solo se $I \cap S = \varnothing$;
	\item dato $P$ ideale primo, allora $S ^{-1} P$ \`e un ideale primo di $S^{-1}A$.
\end{enumerate}
\end{prop}
\subsection{Divisibilit\`a nei domini}
\begin{definizione}
	[Divisibilit\`a] Siano $a,b \in A$ dominio, con $a \neq 0$; allora $a  \mid b \iff \exists c \in A : b = ac$.
\end{definizione}
\begin{osservazione}
$a  \mid b \iff \langle b \rangle\subseteq \langle a \rangle$; infatti, $ca= b \Rightarrow b \in \langle a \rangle \Rightarrow \langle b \rangle\subseteq \langle a \rangle$.
\end{osservazione}
\begin{definizione}
[Elemento associato]
Dati $a,a' \in A$ dominio, si dicono \textit{associati} se vale una delle seguenti, equivalenti, condizioni:
\begin{enumerate}[(a).]
	\item $a  \mid  a' $ e $a'  \mid  a$;
	\item $\exists u \in A^*$ tale che $a = u a'$;
	\item $\langle a \rangle= \langle a' \rangle$.
\end{enumerate}
\end{definizione}
\begin{definizione}
	[MCD] Per $a,b \in A$ dominio non entrambi nulli, $d$ \`e un MCD se valgono entrambe le seguenti condizioni:
	\begin{enumerate}[(a).]
		\item $d \mid a $ e $d  \mid  b$;
		\item $\forall x \in A$ tale che $x  \mid a$ e $x \mid b$, si ha $x  \mid  d$.
	\end{enumerate}
\end{definizione}
\begin{prop}
	Dati $a,b \in A$, si dice che $d$ e $d'$ sono loro MCD se e solo se $d \sim d'$.
\end{prop}
\begin{definizione}
	[Elemento primo]	
$x \in A\setminus (A^* \cup \left\{ 0 \right\} )$ \`e primo se, $\forall a,b \in A$, si ha $x  \mid ab \implies x  \mid a$ oppure $x | b$.
\end{definizione}
\begin{definizione}
	[Elemento irriducibile]
	$x \in A \setminus (A^* \cup \left\{ 0 \right\} )$ \`e irriducibile se, $\forall a,b \in A$, vale $x = ab \Rightarrow a \in A^*$ oppure $b \in A^*$.
\end{definizione}
\begin{prop}
Se $x \in A$ dominio \`e primo, allora \`e irriducibile.
\end{prop}
\begin{prop}
	[Caratterizzazione elementi primi e irriducibili]
	Sia $x \in A$ dominio. Allora:
	\begin{enumerate}[(a).]
		\item $x$ \`e primo $\iff \langle x \rangle$ \`e un ideale primo non-nullo;
		\item $x$ \`e irriducibile $\iff \langle x \rangle$ \`e un ideale massimale nell'insieme degli ideali principali.
	\end{enumerate}
\end{prop}
\subsection{ED, PID e UFD}
\subsubsection{ED}


\begin{definizione}
	[ED] 
	$A$ dominio \`e \textit{euclideo} se si pu\`o definire una mappa $d : A \setminus \left\{ 0 \right\}  \to   \mathbb{N}$ tale che:
	\begin{enumerate}[(a).]
		\item $d(x) \le  d(xy) , \ \forall x,y \in A \setminus\left\{ 0 \right\} $;
		\item $\forall x \in A , \ \forall y \in A \setminus\left\{ 0 \right\} $, si trovano $q,r \in A$ tali che $x = yq + r$, con $d(r) < d(y)$ oppure $r=0$.
	\end{enumerate}
\end{definizione}
\begin{prop}
	[Algoritmo di Euclide]
	In $A$ dominio euclideo, per ogni coppia $a,b \in A$, esiste un MCD ottenuto tramite algoritmo di Euclide.
\end{prop}
\begin{prop}
	Gli elementi di grado minimo di $A$ dominio euclideo coincidono con gli elementi di $A^*$.
\end{prop}
\begin{prop}
	Tutti gli ideali di $A$ dominio euclideo sono principali e generati da un elemento di grado minimo nell'ideale in questione.
\end{prop}
\subsubsection{PID}
\begin{definizione}
	[PID]
	Un dominio $A$ \`e a \textit{ideali principali} se ogni suo ideale \`e principale.
\end{definizione}
\begin{prop}
Se $A$ \`e un PID, i suoi unici ideali primi sono $\langle 0 \rangle$ e quelli massimali.
\end{prop}
\begin{prop}
	[MCD nei PID]
	Dati $x,y \in A$ PID non entrambi nulli, si ha $\langle x,y \rangle = \langle d \rangle$, con $d = (x,y)$. 
\end{prop}
\subsubsection{UFD}
\begin{definizione}
	[UFD]
	$A$ dominio \`e a \textit{fattorizzazione unica} se ogni elemento $x \in A \setminus (A^* \cup \left\{ 0 \right\} )$ si decompone univocamente in irriducibili, a meno di prodotto per un'unit\`a.
\end{definizione}
\begin{prop}
	Dati $a,b \in A$ UFD non entrambi nulli, esiste sempre un loro MCD.	
\end{prop}
\begin{teorema}
	[Caratterizzazione degli UFD]
	$A$ dominio \`e un UFD se e solo se sono soddisfatte entrambe le seguenti condizioni:
\begin{enumerate}[(a).]
	\item ogni irriducibile \`e primo;
	\item ogni catena discendente di divisibilit\`a \`e stazionaria, cio\`e data $\left\{ a_i \right\} _{i\ge 0} \subset A$, con $a_{i+1}  \mid a_i$, allora $\exists n_0 \in \mathbb{N}$ tale che $a_i \sim a _{n_0} , \ \forall i \ge n_0$.
\end{enumerate}
\end{teorema}
\begin{corollario}
Se $A$ \`e un PID, allora \`e un UFD.
\end{corollario}
\begin{boxenv}[]
	\centering $\text{ED}\implies \text{PID} \implies \text{UFD}$
\end{boxenv}

\subsection{Anelli di polinomi}
\begin{definizione}
	[Contenuto]
	Dato $f(x) \in A[x]$, con $A$ UFD e $f(x) =\sum_{i=0}^{n} a_i x^i$, si definisce il \textit{contenuto} di $f(x)$ come l'MCD dei suoi coefficienti:
	\[
	c\big(f(x)\big) = \operatorname{gcd}(a_0,\ldots,a_n) 
	\] 
\end{definizione}
\begin{definizione}
	[Elemento primitivo]
	$f(x) \in A[x]$, con $A$ UFD, \`e \textit{primitivo} se $c (f(x)) \sim 1$.
\end{definizione}
\begin{lemma}
	[Lemma di Gauss]
	Dati $f(x) , g(x) \in A[x]$, allora:
	\[
	c\big(f(x) g(x)\big) = c(f(x)) c(g(x))
	\] 
\end{lemma}
\begin{corollario}
	Dati $f(x) , g(x) \in A[x]$, con $c(f(x)) = 1$ e $f(x)  \mid g(x)$ in $K[x]$, con $K$ campo dei quozienti di $A$, allora $f(x)  \mid g(x)$ in $A[x]$.
\end{corollario}
\begin{corollario}
	Dato $f(x) \in A[x]$, con $f(x) = g(x) h(x)$ in $K[x]$ (con $K$ campo dei quozienti di $A$) e $\deg g(x), \deg h(x) \ge 1$ (quindi $f$ riducibile in $K[x]$), allora $\exists \delta \in K^*$ tale che $g_1(x) = \delta g(x) \in A[x]$ e $h_1(x) = \delta ^{-1}h(x) \in A[x]$, per cui $f(x) = g_1(x) h_1(x)$ in $A[x]$.
\end{corollario}
\begin{teorema}
	Gli irriducibili di $A[x]$, con $A$ UFD, soddisfano una tra le seguenti condizioni:
	\begin{enumerate}[(a).]
		\item $f(x) \in A$ e irriducibile in $A$;
		\item $f(x) \in A[x]$, con $\deg f(x) \ge 1$, $c(f(x)) = 1$ e $f(x)$ irriducibile in $K[x]$.
	\end{enumerate}
\end{teorema}
\begin{teorema}
	Se $A$ \`e un UFD, allora $A[x]$ \`e un anello a fattorizzazione unica.
\end{teorema}
\begin{corollario}
	Se $A$ \`e un UFD, allora $A[x_1,\ldots,x_n]$ \`e un anello a fattorizzazione unica.
\end{corollario}
\begin{prop}
	[Eisenstein]
	Sia $A$ un UFD e $f(x) \in A[x]$ primitivo, con $f(x) = \sum_{i=0}^{n} a_i x^i$. 
	Dato $p \in A$ un primo tale che
	\begin{enumerate}[(a).]
		\item $p \nmid a_n$,
		\item $p  \mid a_i$, $\forall i =0,\ldots,n-1$,
		\item $p^2 \nmid a_0$;
	\end{enumerate}
	allora $f(x)$ \`e irriducibile in $A[x]$ e in $K[x]$.
\end{prop}

\subsection{Risultati vari sugli anelli}
\begin{teorema}
Dato $A$ un anello qualsiasi e dati $I,J$ due suoi ideali, allora 
\[
\sqrt{IJ} = \sqrt{I \cap J} =\sqrt{I} \cap \sqrt{J} 
\] 

\end{teorema}
\begin{prop}
	Sia $M$ un ideale massimale di $\mathbb{Z}[x]$ tale che $M \cap \mathbb{Z}$ contiene un primo $p$; allora $M = (p,f(x))$, con $f(x) \operatorname{mod} p$ irriducibile in $\mathbb{F}_p[x]$.
\end{prop}
\newpage


























\section{Teoria dei campi}
\subsection{Estensioni di campi}


\begin{definizione}
	[Elementi algebrici e trascendenti]
	Dato $K$ campo e $L$ una sua estensione, $\alpha \in L$ \`e algebrico su $K$ se $\exists  f(x) \in K[x] \setminus \left\{ 0 \right\} $ tale che $f(\alpha )= 0$. 
	Altrimenti, $\alpha $ \`e detto trascendente su $K$.
\end{definizione}
\begin{definizione}
	[Grado di un'estensione]
	Data $L / K$ estensione, il suo grado si indica con $[L:K] = \dim _K L $ ed \`e la dimensione di $L$ come spazio vettoriale su $K$.
\end{definizione}
\begin{prop}
Sia $L / K$ un'estensione, con $\alpha  \in L$; allora:
\[
	[K(\alpha ) : K] = \begin{cases}
		+\infty &,\ \alpha \text{ trascendente}\\
		\deg \mu_\alpha (x) &,\ \alpha \text{ algebrico}
	\end{cases}
\] 
con $\mu _\alpha (x)$ polinomio minimo di $\alpha$ in $K[x]$.
\end{prop}
\begin{prop}
	[Formula della torre]
	Sia data la torre di estensioni $K \subset F \subset L$; allora $L / K$ \`e finita se e solo se $L/F$ e $F / K$ sono finite e vale 
	\[
		[L:K]=[L:F][F:K]
	\] 
\end{prop}
\begin{definizione}
	[Estensione composta]
	Sia $\Omega $ un campo e $L,M \subset \Omega $; allora $LM=L(M)=M(L)$ \`e il pi\`u piccolo sottocampo di $\Omega $ contenente $L $ e $M$.
	Se $M,L$ sono estensioni finitamente generate, cio\`e $L = K(\alpha _1,\ldots,\alpha _n)$ e $M = K(\beta _1,\ldots,\beta _m)$, vale
	\[
	LM = K(\alpha _1,\ldots,\alpha _n,\beta _1,\ldots,\beta _m)
	\] 
\end{definizione}
\begin{prop}
	Si considerano le due torri $K \subset L \subset FL$ e $K \subset F \subset  FL$, con $[L:K] = m$ e $[F:K] = n$; allora $[FL:K] = d < +\infty$ e $[m,n] \mid d$.
\end{prop}
\[
\begin{tikzcd}
	& FL &\\
	L & & F \\
	  & K &
	\arrow[from=1-2, to=3-2, no head, "\displaystyle d"]
        \arrow[from=1-2, to=2-1, no head]
        \arrow[from=1-2, to=2-3, no head]
	\arrow[from=3-2, to=2-1, no head, "\displaystyle m"]
	\arrow[from=3-2, to=2-3, no head, "\displaystyle n"']
\end{tikzcd}
\] 
\begin{definizione}
[Estensione algebrica]
$L / K$ \`e algebrica se $\forall  \alpha  \in L$, $\alpha $ \`e algebrico su $K$.
\end{definizione}
\begin{prop}
Ogni estensione finita \`e algebrica.
\end{prop}
\begin{prop}
Data estensione $L / K$, allora $A = \left\{ \alpha \in L  \mid \alpha  \text{ algebrico su } K \right\} $ \`e un campo e un'estensione algebrica di $K$.
\end{prop}
\begin{prop}
$L / K$ \`e un'estensione finitamente generata da algebrici, cio\`e $L=K(\alpha _1,\ldots,\alpha _n)$, se e solo se $L / K$ \`e finita.
\end{prop}
\begin{teorema}
	[Caratterizzazione delle estensioni algebriche]
	Valgono i due seguenti punti.
	\begin{enumerate}[(a).]
		\item Data la torre $K \subset L \subset F$, $F / K$ \`e algebrica se e solo se $F / L$ e $L / K$ sono algebriche.
		\item Date due estensioni $L / K$ e $M / K$, queste sono algebriche se e solo se $LM / K$ \`e algebrica.
	\end{enumerate}
\end{teorema}
\subsection{Chiusura algebrica}
\begin{definizione}
	[Campo algebricamente chiuso]
	$\Omega $ \`e detto algebricamente chiuso se ogni $f(x) \in \Omega [x]$ non costante ha almeno una radice in $\Omega $.	
\end{definizione}
\begin{definizione}
	[Chiusura algebrica]
	$\Omega / K$ \`e una chiusura algebrica di $K$ se valgono i due seguenti punti:
	\begin{enumerate}[(a).]
		\item $\Omega $ \`e algebricamente chiuso;
		\item $\Omega / K$ \`e un'estensione algebrica.
	\end{enumerate}
\end{definizione}
\begin{teorema}
	[Esistenza e unicit\`a della chiusura]
	Dato $K$ campo, allora esiste sempre una sua chiusura algebrica, che \`e unica a meno di isomorfismo.
\end{teorema}
\begin{definizione}
[Campo di spezzamento]
Dato $f(x) \in K[x]$, con $\deg f(x) \ge 1$, e date $\alpha _1 ,\ldots,\alpha _n \in \overline{K}$ e sue radici, se ne definisce il campo di spezzamento su $K$ come il sotto campo di $\overline{K}$ dato da $K(\alpha _1,\ldots,\alpha _n)$.
\end{definizione}
\begin{prop}
Sia $K$ un campo e $\alpha \in \overline{K}$. 
Se $k$ \`e il numero di radici distinte di $\mu _\alpha (x) $ in $\overline{K}$, allora 
\[
\exists \varphi _1,\ldots, \varphi _k : K(\alpha ) \longhookrightarrow \overline{K}
\] 
estensioni dell'immersione $K \longhookrightarrow \overline{K}$ data dall'identit\`a, con $\varphi _i|_K = \operatorname{Id} _K$.
\end{prop}
\begin{teorema}
	[Criterio della derivata]
	Sia $f(x) \in K[x]$; allora $f(x)$ ha radici multiple in $\overline{K}$ se e solo se $\operatorname{gcd}(f(x),f'(x)) \neq 1$.
	Se $f$ \`e irriducibile in $K[x]$, allora $f$ ha radici multiple se e solo se $f'(x) = 0$.
\end{teorema}
\begin{definizione}
	[Campo perfetto]
	$K$ \`e perfetto se ogni irriducibile di $K[x]$ ha derivata non-nulla.
\end{definizione}
\begin{prop}
	Sia $\alpha \in \overline{K}$, con $[K(\alpha ) : K] = n$.
	Allora, per ogni $\varphi : K \longhookrightarrow \overline{K}$
	\[
	\exists \varphi _1 ,\ldots,\varphi _n : K(\alpha )\longhookrightarrow \overline{K}
	\] 
	con $\varphi _i|_K = \varphi, \ \forall i$.
\end{prop}
\begin{corollario}
$E/K$ estensione di grado $n$; allora $\forall \varphi : K \longhookrightarrow \overline{K}$, si trovano esattamente $n$ immersioni $\varphi _1,\ldots,\varphi _n : E \longhookrightarrow \overline{K}$, con $\varphi _i|_K = \varphi $.
\end{corollario}
\begin{definizione}
	[Elementi coniugati]
	Per $\alpha \in \overline{K}$, i suoi coniugati su $K$ sono le radici del suo polinomio minimo su $K$.
\end{definizione}
\begin{definizione}
	[Estensione separabile]
	$K \subset L$ estensione algebrica \`e separabile se il polinomio minimo di ogni suo elemento \`e separabile, cio\`e se ha tutte radici distinte in un campo di spezzamento.
\end{definizione}
\begin{teorema}
	[Teorema dell'elemento primitivo]
	Sia $K$ un campo e $E / K$ un'estensione finita e separabile; allora $E / K$ \`e semplice, cio\`e $\exists \gamma\in E : E = K(\gamma)$.
\end{teorema}
\subsection{Estensioni normali}
\begin{definizione}
	[Estensione normale]
	$F / K$ estensione algebrica \`e normale se $\forall \varphi  : F \longhookrightarrow \overline{K}$, con $\varphi |_K = \operatorname{Id} _K$, si ha $\varphi (F) = F$.
\end{definizione}
\begin{prop}
	Sia $F / K$ algebrica e finita.
	Allora le seguenti affermazioni sono equivalenti:
	\begin{enumerate}[(a).]
		\item $F / K$ \`e normale;
		\item ogni irriducibile $f(x) \in K[x]$ che ha una radice in $F$, le ha tutte in $F$;
		\item $F$ \`e il campo di spezzamento su $K$ di una famiglia di polinomi di $K[x]$.
	\end{enumerate}
\end{prop}
\begin{prop}
Ogni estensione di grado $2$ \`e normale in caratteristica diversa da $2$.
\end{prop}
\begin{prop}
Siano $F / K$ e $L / K$ due estensioni normali di $K$ nella chiusura $\overline{K}$; allora anche $FL / K$ e $(F\cap L) / K$ sono normali.
\end{prop}
\begin{prop}
Data la torre $K \subset F \subset L$ nella chiusura $\overline{K}$, se $L / K$ \`e normale, allora $L / F$ \`e normale.
\end{prop}

\subsection{Teoria di Galois}
\begin{definizione}
	[Estensione di Galois]
Un'estensione $E / K$ \`e di Galois se e solo se \`e normale e separabile.
\end{definizione}
\begin{definizione}
	[Gruppo di Galois]
	Dato 
	\[
		\operatorname{Aut} _K E = \left\{ \varphi : E \stackrel{\mathrm{\sim}}{\longrightarrow} E \ \big\lvert \ \varphi |_K = \operatorname{Id} _K\right\} 
	\] 
	l'insieme delle immersioni $\left\{ \varphi : E \longhookrightarrow \overline{K}  \mid \varphi |_K = \operatorname{Id} _K \right\} $ che fissano $E$ (perch\'e $E \ K$ \`e normale) con immagine in $E$, si definisce 
	\[
	\operatorname{Gal} (E / K) := \left(\mathrm{Aut} _K E , \circ\right) 
	\] 
\end{definizione}
\begin{osservazione}
Visto che il numero di immersioni $E \longhookrightarrow \overline{K}$ coincide con il grado dell'estensione, si ha:
\[
	\lvert \operatorname{Gal} E / K \rvert = [E:K]
\] 
\end{osservazione}


\begin{prop}
	Sia $f(x) \in K[x]$ irriducibile di grado $n$; se $F$ \`e il suo campo di spezzamento su $K$, allora $n  \mid [F:K]  \mid n!$ e $\operatorname{Gal} F / K \longhookrightarrow S_n$.
\end{prop}
\begin{osservazione}
L'azione di $\operatorname{Gal} F / K$ sull'insieme delle radici di $f(x)$ \`e fedele e transitiva.
\end{osservazione}



\subsubsection{Gruppo di Galois di $\mathbb{F}_{q^d} / \mathbb{F}_q$}

\begin{prop}
L'estensione $\mathbb{F}_{q^{d} } / \mathbb{F}_q$, con $q = p^r$ e $p$ primo, \`e normale.
\end{prop}
\begin{corollario}
Tutte le estensioni di campi finiti sono normali.
\end{corollario}
\begin{definizione}
	[Automorfismo di Frobenius]
	Si definisce come
	\[
	\phi : 
	\begin{array}
		{c c c}
		\mathbb{F}_{q^d} & \stackrel{\sim}{\longrightarrow} & \mathbb{F}_{q^d} \\
		x & \longmapsto & x^q
	\end{array}
	\] 
\end{definizione}
\begin{teorema}
Il gruppo di Galois di $\mathbb{F}_{q^d} /\mathbb{F}_q$ \`e generato dall'automorfismo di Frobenius.
\end{teorema}

\subsubsection{Teorema di corrispondenza di Galois}

\begin{definizione}
Sia $L / K$ un'estensione di Galois finita e $H < \operatorname{Gal}  L / K$; allora si definisce 
\[
L^H := \operatorname{Fix} (H) = \left\{ \alpha \in L  \mid \varphi (\alpha ) = \alpha , \ \forall \varphi  \in H \right\} \subseteq L
\] 
\end{definizione}
\begin{prop}
$L^H$ \`e un sottocampo.
\end{prop}
\begin{lemma}
Sia $L / M$ di Galois e $ H\le \operatorname{Gal} L / M$; allora
\[
M = L^H \iff H = \operatorname{Gal} L / M
\] 
\end{lemma}
\begin{lemma}
Sia $L / K$ di Galois e $H < \operatorname{Gal} L / K$. 
Per $\sigma  \in \operatorname{Gal} L / K$, si ha $L^{\sigma H\sigma ^{-1}} = \sigma (L^H)$.
\end{lemma}
\begin{teorema}
	[Teorema di corrispondenza di Galois]
	Data $L / K$ di Galois, l'insieme delle sottoestensioni di $L / K$ e l'insieme dei sottogruppi di $\operatorname{Gal} L / K$ sono in corrispondenza biunivoca; inoltre, $H \lhd \operatorname{Gal} L / K \iff L^H / K $ \`e normale e, in tal caso:
	\[
	\operatorname{Gal} L^H / K \cong \frac{\operatorname{Gal} L / K}{\operatorname{Gal} L/L^H}
	\] 
\end{teorema}
\begin{prop}
	[Propriet\`a della corrispondenza]
	Siano $H,S \le \operatorname{Gal} L / K$. 
	Allora valgono i seguenti punti:
	\begin{enumerate}[(a).]
		\item $H \le S \iff L^H \supseteq L^S$;
		\item $L^{H\cap S} = L^H L^S$ (composto dei due campi);
		\item $L^{\langle S,H \rangle} = L^H \cap L^S$.
	\end{enumerate}
\end{prop}

\subsection{Risultati vari sui campi}
\begin{teorema}
	[Campo di spezzamento di $x^n - 1$ su $\mathbb{F}_p$.]
Dato $n = p^km$, con $(m,p)=1$, il campo di spezzamento di $x^n - 1$ su $\mathbb{F}_p$ \`e $\mathbb{F}_{p^d} $, con $d$ ordine moltiplicativo di $p$ modulo $n$, cio\`e \`e il minimo valore positivo che soddisfa
\[
p^x \equiv 1\pmod{m} 
\] 
\end{teorema}

\begin{prop}
$\operatorname{Gal} \big(\mathbb{Q}(\zeta_n) / \mathbb{Q}\big) \cong (\mathbb{Z}/n\mathbb{Z})^\times$.
\end{prop}
\begin{prop}
Date le torri $F \subset K \subset KL$ e $F \subset L \subset KL$, se $K / F$ di Galois, allora:
\begin{itemize}
	\item $KL / L$ \`e di Galois;
	\item $\operatorname{Gal} (KL / L ) \cong \operatorname{Gal}(K / L\cap K) $.
\end{itemize}
\end{prop}
\begin{corollario}
Siano date le torri $F \subset K \subset KL$ e $F \subset L \subset KL$, con $K / F$ di Galois e $K \cap L = F$; allora:
\[
	[KL : F]=[K:F][L:F]
\] 
\end{corollario}
\begin{teorema}
	[Biquadratiche]
	Dato $p(x) = x^4 + a x^2 + b \in \mathbb{Q}[x]$ irriducibile e dato $K$ il suo campo di spezzamento su $\mathbb{Q}$, allora:
	\begin{enumerate}[(a).]
		\item $\operatorname{Gal} (K / \mathbb{Q})\cong \mathbb{Z}/2\mathbb{Z} \times \mathbb{Z}/2\mathbb{Z}$ se $b$ \`e un quadrato su $\mathbb{Q}$;
		\item $\operatorname{Gal} (K/\mathbb{Q}) \cong \mathbb{Z}/4\mathbb{Z}$ se $b\Delta = b(a^2-4b)$ \`e un quadrato su $\mathbb{Q}$;
		\item $\operatorname{Gal} (K / \mathbb{Q}) \cong D_4$ altrimenti.
	\end{enumerate}
\end{teorema}
\begin{prop}
Sia $K$ un campo di caratteristica diversa da $2$ e siano $a,b \in K^\times$; allora $K (\sqrt{a} ) = K(\sqrt{b} ) \iff a / b $ \`e un quadrato in $K \iff ab$ \`e un quadrato in $K$.
\end{prop}
\begin{teorema}
Sia $p$ un primo dispari; allora l'unica sottoestensione quadratica di $\mathbb{Q}(\zeta_p) / \mathbb{Q}$ \`e:
\begin{enumerate}[(a).]
	\item $\mathbb{Q}(\sqrt{p} )$, se $p\equiv 1 \pmod{4} $;
	\item $\mathbb{Q}(\sqrt{-p} )$, se $p\equiv 3 \pmod{4} $.
\end{enumerate}
\end{teorema}



\newpage

\section{Esercizi}
\subsection{Esercizi su gruppi 2}
\begin{esercizio}
Sia $G = \mathbb{Z}/4\mathbb{Z}\rtimes _\phi \mathbb{Z}/4\mathbb{Z}$ definito da $\phi (1) = - 1 \in (\mathbb{Z}/4\mathbb{Z})^* \cong \operatorname{Aut} (\mathbb{Z}/4\mathbb{Z})$.
\begin{enumerate}[(a).]
	\item Per ogni intero $n$, contare gli elementi di ordine $n$ in $G$.
	\item Dimostrare che $Z(G) \cong \mathbb{Z}/2\mathbb{Z} \times \mathbb{Z}/2\mathbb{Z}$.
	\item Calcolare $G'$ e la classe di isomorfismo di $G_\text{ab} := G / G'$.
\end{enumerate}
\end{esercizio}
\begin{svolgimento}
	Si divide lo svolgimento nei vari punti.
	\begin{enumerate}[(a).]
		\item I possibili $n$ sono $1,2,4,8,16$.

			Per $n=1$, si ha evidentemente l'identit\`a $(0,0)$.

			Per $n=2$, si osserva che:
			\[
				(a,b)^2 = (0,0) \iff \Big(a+(-1)^b a , 2b\Big) = (0,0)
			\] 
			Conviene dividere i casi in cui $b$ \`e pari o dispari.
			Se $b$ pari (cio\`e $b=0,2$), allora il quadrato \`e pari a $(2a,2b)$ e questo coincide con $(0,0)$ se e soltanto se $a,b \in \left\{ 0,2 \right\} $. Escludendo l'identit\`a stessa, ci sono tre possibilit\`a: $(2,0), \ (2,2), \ (0,2)$.
			Se $b$ \`e dispari, invece, il quadrato \`e pari a $(0,2b)$; questo risulterebbe pari a $(0,0)$ se $b\equiv 0 \operatorname{mod} 2 $, ma questo \`e impossibile perch\'e si \`e assunto $b$ dispari.

			Per $n=4$, invece, si impone $(a,b)^4 = (0,0)$, cio\`e:
			\[
				\Big(a + (-1)^b a , 2b\Big)\Big(a+(-1)^b a, 2b\Big) =\begin{cases}
					 (0,4b) \equiv (0,0) \mod{4}&,\ b \text{ dispari} \\
					 (4a,4b)\equiv (0,0) \mod{4}&,\ b \text{ pari}
				\end{cases}
			\] 
			Questo conteggio permette di concludere che tutti gli elementi di $G$ che non sono di ordine $1$ o $2$ sono di ordine $4$.
			Visto che l'identit\`a e gli elementi di ordine $2$ sono quattro in totale, si conclude che quelli di ordine $4$ sono $12$.
		\item Per il lemma orbita-stabilizzatore, $\lvert Z(G) \rvert  \mid \lvert G \rvert $, quindi le possibili cardinalit\`a sono $1,2,4,8,16$.
			$G$ \`e un $p$-gruppo, quindi $1$ non \`e ammissibile; inoltre, $8$ e $16$ non sono possibili in quanto $G$ risulterebbe abeliano, che \`e assurdo.
			Allora $\lvert Z(G) \rvert \in \left\{ 2,4 \right\} $.
			Tuttavia, neanche $\lvert Z(G) \rvert =2$ \`e possibile perch\'e $Z(G)$ contiene tutti gli elementi di ordine $2$; infatti, dato $(a,b) \in G$ con $a,b\equiv 0 \operatorname{mod} 2 $, si ha:
			\[
				\begin{split}
					&(c,d) (a,b) = \Big(c + (-1)^d a , d + b\Big)\\
					&(a,b) (c,d) = \Big(a + (-1)^b c , b+d \Big) = \Big(a + c , b+d \Big)
				\end{split}
			\] 
			Questi coincidono per ogni elemento $(c,d) \in G$ se e solo se $a+c = c - a$; per\`o si \`e assunto $a \equiv 0 \operatorname{mod} 2 $, quindi verifica $a \equiv -a \operatorname{mod} 4 $ e, allora, $(c,d) (a,b) = (a,b)(c,d) , \ \forall (c,d) \in G$.
			Se ne conclude che $\lvert Z(G) \rvert = 4$, dove tre elementi sono id ordine $2$ e l'ultimo \`e l'identit\`a.
			Essendo un gruppo di ordine $4$ per forza abeliano, il teorema di struttura assicura che $Z(G) \cong \mathbb{Z}/ 4 \mathbb{Z}$, oppure $Z(G) \cong \mathbb{Z}/ 2 \mathbb{Z} \times \mathbb{Z}/2\mathbb{Z}$; per quanto appena detto sugli ordini degli elementi di $Z(G)$, l'unica possibilit\`a \`e proprio quella richiesta: $Z(G) \cong \mathbb{Z}/2\mathbb{Z} \times \mathbb{Z}/2\mathbb{Z}$.
		\item Per costruire $G'$, si nota che i quozienti 
			\[
			G / Z(G) \hspace{1cm} \frac{G}{\mathbb{Z}/4\mathbb{Z} \times \left\{ 0 \right\} }
			\] 
		sono abeliani (visto che il quoziente ha cardinalit\`a $4$), quindi 
		\[
		G ' \subseteq Z(G) \cap \Big(\mathbb{Z}/4\mathbb{Z}\times  \left\{ 0 \right\} \Big) \cong \Big(\mathbb{Z}/2\mathbb{Z} \times \mathbb{Z}/2\mathbb{Z}\Big)\cap \Big(\mathbb{Z}/4\mathbb{Z}\times  \left\{ 0 \right\} \Big) = \left\{ (0,0), (1,0) \right\} 
		\] 
		Quindi $\lvert G' \rvert = \left\{ 1,2 \right\} $; visto che $G$ non \`e abeliano, $\lvert G' \rvert = 2$ e, quindi, $G ' =\left\{ (0,0), (2,0) \right\} $, dato che $Z(G)$ contiene gli elementi di $G$ di ordine $2$.
		In questo modo, $G_{\text{ab}} $ ha cardinalit\`a $8$ ed \`e abeliano, quindi le classi di isomorfismo possibili, per il teorema di struttura, sono le seguenti:
		\[
		\big(\mathbb{Z}/2\mathbb{Z}\big)^3 \hspace{1cm} \mathbb{Z}/2\mathbb{Z} \times \mathbb{Z}/4\mathbb{Z} \hspace{1cm} \mathbb{Z}/8\mathbb{Z}
		\] 
		Per\`o $G_{\text{ab}} $ ha elementi di ordine $4$ e non ha elementi di ordine $8$, quindi l'unica possibilit\`a rimanente \`e $G_\text{ab}\cong \mathbb{Z}/2\mathbb{Z} \times  \mathbb{Z}/4\mathbb{Z}$.
	\end{enumerate}
\end{svolgimento}
\begin{esercizio}
Determinare il pi\`u piccolo $m$ tale che esiste un sottogruppo di $S_m$ isomorfo a $A_5 \times \mathbb{Z}/6\mathbb{Z}$.
\end{esercizio}
\begin{svolgimento}
	Si deve trovare un $m$ compatibile con il fatto che gli elementi di $A_5$ e quelli di $\mathbb{Z}/6$ devono commutare.
	Gli elementi di $A_5$ sono tutte le permutazioni pari che agiscono fedelmente su $5$ elementi; pertanto, $A_5$ contiene elementi di ordine $5$, mentre $\mathbb{Z}/6$ contiene elementi di ordine $6$.
	Mentre un elemento di ordine $5$ in $S_m$ \`e per forza un $5$-ciclo, un elemento di ordine $6$ si pu\`o costruire componendo un $2$-ciclo con un $3$-ciclo, quindi deve avere a disposizione almeno $5$ elementi su cui agire.
	Per fare in modo che elementi di ordine $5$ e $6$ commutino, si deve far in modo che agiscano su elementi distinti; per quanto appena detto, quindi, il pi\`u piccolo $m$ che permette ci\`o \`e $m=10$.
	Per dimostrarlo, si fa prima vedere che $S_9$ non sarebbe sufficiente e poi si d\`a un esempio concreto in $S_{10}$.

	Sia, allora, $\varphi : A_5 \times \mathbb{Z}/6\mathbb{Z} \to S_9$.
	Dato $\sigma \in A_5$, $\varphi (\sigma ,0)$ deve avere ordine $5$ perch\'e $\varphi $ sia un isomorfismo, quindi deve essere necessariamente un $5$-ciclo; inoltre, $\varphi (e, 1)$ deve avere ordine $6$ e commutare con $\varphi (\sigma ,0)$.
	A questo punto, si osserva che 
	\[
		\lvert \operatorname{Cl}_{S_9}  (\varphi (\sigma ,0)) \rvert  = \binom{9}{5}4!=\frac{9!}{4!5!} = \frac{9!}{5!}\implies \lvert Z_{S_9} (\varphi (\sigma ,0) \rvert =\frac{9!}{9! / 5!}=5!
	\] 
Data $\varphi (\sigma ,0) = (1 \ 2 \ 3 \ 4\ 5)$, prendendo  
\[
H:=\langle (1 \ 2 \ 3 \ 4\ 5) \rangle \hspace{1cm}  K:= \left\{ \text{permutazioni di } S_9 \text{ che fissano } \left\{ 1 \ 2 \ 3\ 4\ 5 \right\}  \right\} \cong S_4
\] 
questi sono sottogruppi di $\mathbb{Z}_{S_9} (\varphi (\sigma ,0))$, commutano, si intersecano nell'identit\`a e sono tali che $\lvert HK \rvert = 5!$, quindi, per il teorema di decomposizione diretta, si pu\`o scrivere che
\[
	Z_{S_9} (1 \ 2 \ 3 \ 4 \ 5) \cong \mathbb{Z}/5 \times S_4
\] 
Ma a questo punto sorge un assurdo: non pu\`o esistere alcun elemento di ordine $6$ nel centralizzatore, quindi questo isomorfismo non va bene.

Ora si mostra che \`e possibile costruire un isomorfismo in $S_{10}$.
Siano $H$ il sottogruppo di $S_{10}$ delle permutazioni pari agiscono su $\left\{ 1 \ 2 \ 3 \ 4\ 5 \right\} $ e $ \tau=  (6 \ 7 )(8 \ 9 \ 10)\in S_{10}$, quindi $H \cong A_5$ e $K= \langle \tau  \rangle\cong \mathbb{Z}/6$. 
Si nota che $H$ e $K$ commutano e hanno intersezione banale perch\'e agiscono su elementi differenti, quindi, per il teorema di decomposizione diretta, esiste un sottogruppo di $S_{10}$ isomorfo a $ H \times K \cong A_5 \times \mathbb{Z}/6$.
\end{svolgimento}






























\subsection{Esercizi su anelli 2}
\begin{esercizio}
Siano $I = (4, 3x + 1)$ e $J = (3, x^2 + 1)$, ideali dell’anello $\mathbb{Z}[x]$. Contare gli ideali massimali di $\mathbb{Z}[x]/IJ$.
\end{esercizio}
\begin{svolgimento}
	Si vuole utilizzare il teorema cinese del resto per scrivere $\mathbb{Z}[x] / IJ \cong \mathbb{Z}[x] / I \times \mathbb{Z}[x] / J$.
	Per farlo, si osserva che $1 \in I+J$ perch\'e $4 \in I$ e $3 \in J$, quindi $1 = 4-3 \in I+J$, da cui $I + J = \mathbb{Z}[x]$. 
	Questo assicura l'isomorfismo voluto.
	Ora si nota che 
	\[
		\mathbb{Z}[x] / (4,3x+1) \cong  \mathbb{Z}_4[x] / (3x+1)
	\] 
	dove $(3x+1)$ nell'espressione finale \`e l'ideale in $\mathbb{Z}_4[x]$; per questo motivo, visto che $3$ \`e un'unit\`a in $\mathbb{Z}_4$, si ha che $(3x+1) = (x+3) \subset \mathbb{Z}_4[x]$.
	Ne segue che $\mathbb{Z}[x] / (4,3x+1)\cong\mathbb{Z}_4[x] / (x+3)\cong \mathbb{Z}_4$.
	Per l'altro fattore, si ha:
	\[
		\mathbb{Z}[x] / (3,x^2+1) \cong \mathbb{F}_3[x] / (x^2+1) \cong \mathbb{F}_9
	\] 
	perch\'e $x^2 + 1$ \`e irriducibile in $\mathbb{F}_3$ e il campo risultante \`e quello con $9$ elementi.
	Se ne conclude che $\mathbb{Z}[x] /IJ \cong \mathbb{Z}_4\times \mathbb{F}_9$.
	Ora, $\mathbb{F}_9$ ha, come unico ideale proprio, quello banale, mentre $\mathbb{Z}_4$ ha solo $(2)$ come ideale massimale.
	Questo significa che gli ideali massimali di $\mathbb{Z}[x] /IJ$ sono due e sono dati da $(2) \times \mathbb{F}_9$ e $\mathbb{Z}_4 \times \left\{ 0 \right\} $.
\end{svolgimento}

\begin{esercizio}
	Sia $A = \mathbb{Z}[x,y] / (y^2 +1)$.
	\begin{enumerate}[(a).]
		\item Contare il numero di omomorfismi da $A$ in $\mathbb{F}_7$ e $\mathbb{F}_{49} $.
		\item Contare gli ideali di $A$ che contengono $(7,x^2+1)$.
	\end{enumerate}
\end{esercizio}
\begin{svolgimento}
	Per il punto (a), si contano quanti omomorfismi da $\mathbb{Z}[x,y] / (y^2+1)$ in $\mathbb{F}_7$ e $\mathbb{F}_{49} $ sono possibili, perch\'e la scelta di dove mappare $x$ \`e totalmente arbitraria, visto che non ci sono vincoli.
	Per $y$, invece, \`e necessario che un omomorfismo $\varphi : \mathbb{Z}[x,y] / (y^2+1) \to \mathbb{F}_7, \mathbb{F}_{49}  $ rispetti la condizione 
\[
\varphi (y)^2 + 1 =0 \implies \varphi (y)^2 = -1
\] 
Questo si analizza studiando se, nel rispettivo campo di arrivo, \`e possibile che $-1$ sia un quadrato.
\begin{enumerate}[(i).]
	\item Caso per $\mathbb{F}_7$.

		In questo campo, $-1 \equiv 6 \operatorname{mod} 7  $, quindi \`e necessario risolvere $a^2\equiv 6 \operatorname{mod} 7 , \ a \neq 0$.
		Visto che $a^6 \equiv 1 \operatorname{mod} 7 $, allora deve risultare
		\[
		(a^2)^3 \equiv 1 \pmod{7} \implies 6^3 \equiv 1 \pmod{7} 
		\] 
		Ma $ 6^3 =  216 = 210 + 6\equiv 6 \pmod{7} $, quindi non c'\`e alcun omomorfismo in $\mathbb{F}_7$ che possa soddisfare tale relazione.
	\item Caso per $\mathbb{F}_{49} $.

	Questa volta, la relazione da soddisfare \`e $a^{48} \equiv 1 \operatorname{mod} 49 $, ossia $(-1)^{24} \equiv 1 \operatorname{mod} 49  $, che \`e verificato perch\'e $24\equiv 0 \operatorname{mod} 2$.
	Ora, visto che $-1$ \`e un quadrato in $\mathbb{F}_{49} $, ci saranno due elementi di $\mathbb{F}_{49} $ che soddisfano $t^2 = - 1$, quindi ci sono due scelte per $\varphi (y)$. 
	Al contempo, ci sono $49$ possibili scelte per $\varphi (x)$, per un totale di $98$ omomorfismi.
\end{enumerate}
Per il punto (b), invece, contare gli ideali di $A$ che contengono $(7,x^2+1)$ \`e equivalente a contare gli ideali di $A / (7,x^2+1)$.
Allora: 
\[
	A / (7,x^2+1) \cong \frac{\mathbb{Z}[x,y] / (y^2+1)}{(7,x^2+1)}
\] 
Usando il secondo teorema di omomorfismo, si ha $A/J \cong \frac{A / I }{J / I}$, quindi:
\[
	\frac{\mathbb{Z}[x,y] / (y^2+1)}{(7,x^2+1)} \cong \frac{\mathbb{Z}[x,y] / (y^2+1)}{(7,x^2+1,y^2+1) / (y^2+1)}\cong \mathbb{Z}[x,y] / (7,x^2+1,y^2+1) \cong\frac{\mathbb{F}_7[x,y]}{(x^2+1,y^2+1)}
\] 
Ora, usando il fatto che $x^2+1$ \`e irriducibile in $\mathbb{F}_7[x]$ e che, detta $\alpha $ una sua radice, si ha $\mathbb{F}_7[x] / (x^2+1) \cong \mathbb{F}_7[\alpha ]\cong \mathbb{F}_{49} $
\[
	\frac{\mathbb{F}_7[x,y]}{(x^2+1,y^2+1)} \cong \frac{\big(\mathbb{F}_7[x] / (x^2+1)\big)[y]}{(y^2+1)}\cong \frac{\mathbb{F}_{49}[y]}{(y^2+1)}
\] 
In $\mathbb{F}_{49} [y]$, il polinomio $y^2 + 1 = (y-\alpha )(y+\alpha )$, con $(y+\alpha )-(y-\alpha )=2\alpha \in \mathbb{F}_{49} ^\times$, quindi $\langle y-\alpha  \rangle + \langle y+\alpha  \rangle = \mathbb{F}_{49} [y]$ e si pu\`o applicare il teorema cinese del resto:
\[
	\frac{\mathbb{F}_{49}[y]}{(y^2+1)} \cong \frac{\mathbb{F}_{49} [y]}{(y+\alpha )} \times \frac{\mathbb{F}_{49} [y]}{(y-\alpha )}\cong \mathbb{F}_{49} \times \mathbb{F}_{49} 
\] 
I suoi ideali, allora, sono dati da $\left\{ 0 \right\} \times \left\{ 0 \right\} , \ \left\{ 0 \right\} \times \mathbb{F}_{49} , \ \mathbb{F}_{49} \times \left\{ 0 \right\} $ e $\mathbb{F}_{49} \times \mathbb{F}_{49} $, pertanto ci sono $4$ ideali in $\mathbb{Z}[x,y] / (y^2+1)$ che contengono $(7,x^2+1)$.
\end{svolgimento}
\begin{esercizio}
	Sia $A = \mathbb{Z}[i]$ e siano $I = (x-2-i), \ J = (x-2+i)$ due ideali di $A[x]$.
	\begin{enumerate}[(a).]
		\item Dimostrare che $I\cap J$ \`e principale.
		\item Dimostrare che esiste un unico ideale massimale $M $ in $A[x]$ che contiene $I+J$.
		\item Dimostrare che $I+J$ non \`e principale.
	\end{enumerate}
\end{esercizio}
\begin{svolgimento}
Per il punto (a), visto che i polinomi $x-2-i$ e $x-2+i$ hanno radici distinte e si \`e in un UFD, un elemento $f \in I \cap J$ deve essere diviso sia da $x-2-i$, che da $x-2+i$, quindi deve essere diviso dal prodotto.
Se ne conclude immediatamente che ogni elemento di $I \cap J$ \`e diviso da $(x-2-i)(x-2+1)$ perch\'e un elemento di $A[x]$ diviso sia da $x-2+i$ che da $x-2-i$, per quanto appena detto, deve essere diviso dal prodotto. 
Quindi $I \cap J = \big((x-2+i)(x-2-i)\big)= (x^2 - 4x + 5)$.

Per il punto (b)
\end{svolgimento}





































\end{document}
