%! TEX program = lualatex
\documentclass[12pt, a4paper]{scrartcl}
% Packages
\usepackage[margin=1.25in]{geometry}
\usepackage{index}
\usepackage{amsbsy} % Bold math symbols
\makeindex
\usepackage[utf8]{inputenc}
\usepackage[T1]{fontenc}
\usepackage{tcolorbox}
\tcbuselibrary{theorems}
\tcbuselibrary{skins}
\tcbuselibrary{breakable}
\usepackage{varwidth}
\usepackage{textcomp}
\usepackage{amsmath, amssymb}
\usepackage{esint}
\usepackage{titlesec}
\usepackage{xcolor}
\usepackage{titling}
\usepackage[linktocpage]{hyperref}
\usepackage{pgfplots}
\usepackage{multicol}
\setlength{\columnsep}{2em}
\usepackage{caption}
\usepackage{amsthm}
\usepackage{import}
\usepackage{cancel}
\usepackage{caption}
\usepackage{nicematrix}
%\usepackage{mathrsfs}
\usepackage{mathtools}
%\usepackage{parskip}
\usepackage{pythonhighlight}
\usepackage{enumerate}
\usepackage{graphicx}
\usepackage{tikz}
\usepackage{tikz-cd}
\usepackage[italian]{babel}
% To reset footnote numbering each page
\usepackage[perpage]{footmisc}
\usepackage{setspace}
\setstretch{1.2}
\usepackage{faktor}

% Titles 
\title{Appunti di Algebra 2}
\author{Manuel Deodato}
\date{}


% svolgimento
\newenvironment{svolgimento}{\renewcommand\qedsymbol{$\blacksquare$}\begin{proof}[Svolgimento]}{\end{proof}}


%%%%% tcolorbox setup

% Teorema e proposizione
\newtcbtheorem[number within=section]{teorema}{Teorema}
{breakable, top=0.2mm, bottom=0.2mm, boxrule=0mm,arc =.5 mm, colframe=blue!10, coltitle=black, fonttitle=\bfseries, colback=blue!5!white, theorem style=plain apart, before upper={\setlength{\parindent}{15pt} \noindent}}{th}

\newtcbtheorem[number within=section]{prop}{Proposizione}
{breakable, top=0.2mm, bottom=0.2mm, boxrule=0mm,arc =.5 mm, colframe=blue!10, coltitle=black, fonttitle=\bfseries, colback=blue!5!white, theorem style=plain apart, before upper={\setlength{\parindent}{15pt} \noindent}}{prop}





% Definizione
\definecolor{greendef}{HTML}{b8d8be}

\newtcbtheorem[number within=section]{definizione}{Definizione}
{breakable, top=0.2mm, bottom=0.2mm, boxrule=0mm, arc=.5mm, colframe=greendef, coltitle=black, fonttitle=\bfseries, theorem style = plain apart, colback=greendef!50!white, before upper={\setlength{\parindent}{15pt} \noindent}}{def}


% Esempio
\theoremstyle{definition}
\newtheorem{esempio}{Esempio}

%\definecolor{empurple}{HTML}{6e5e89}

%\newtcbtheorem{esempio}{Esempio}{left=0mm,arc=0mm, colframe=empurple!10!white, coltitle=black, fonttitle=\bfseries, theorem style = plain, colback=empurple!20!white, colframe=empurple!90!white, boxrule=1pt, sharp corners, top=.2mm,bottom=.2mm}{es}

\tcolorboxenvironment{esempio}{blanker,breakable,left=5mm,before skip=10pt,after skip=10pt, borderline west={1mm}{0pt}{greendef}}

\numberwithin{esempio}{section}


% Lemma e Corollario
\definecolor{lemcor}{HTML}{a78d8a}

\newtcbtheorem[number within=section]{lemma}{Lemma}
{breakable, top=0.2mm, bottom=0.2mm, boxrule=0mm, arc=.5mm, colframe=lemcor!10!white, coltitle=black, fonttitle=\bfseries, theorem style = plain apart, colframe=lemcor!50!white,colback=lemcor!20!white, before upper={\setlength{\parindent}{15pt} \noindent}}{lem}

\newtcbtheorem[number within=section]{corollario}{Corollario}
{breakable, top=0.2mm, bottom=0.2mm, boxrule=0mm, arc=.5mm, colframe=lemcor!10!white, coltitle=black, fonttitle=\bfseries, theorem style = plain apart, colframe=lemcor!50!white,colback=lemcor!20!white, before upper={\setlength{\parindent}{15pt} \noindent}}{cor}



% Osservazione
\theoremstyle{definition}
\newtheorem{osservazione}{Osservazione}

\definecolor{coloros}{HTML}{6e5e89}

\tcolorboxenvironment{osservazione}{blanker,breakable,left=5mm,before skip=10pt,after skip=10pt, borderline west={1mm}{0pt}{coloros}}

\numberwithin{osservazione}{section}

% Nota
\newtheorem{nota}{Nota}

\definecolor{ncol}{HTML}{f9ebbe}

\tcolorboxenvironment{nota}{blanker,breakable,left=5mm,before skip=10pt,after skip=10pt, borderline west={1mm}{0pt}{ncol}, before upper={\setlength{\parindent}{15pt} \noindent}}

\numberwithin{nota}{section}

%% Generic box
\newtcolorbox{eqbox}[1][]
{
colback=gray!10,
arc=0pt,
boxrule=0pt,
title=#1
}

 \newenvironment{boxenv}[1][]{
    \begin{eqbox}[#1]
    }{
   \end{eqbox}
}



%%%%%%%%%% Medie con integrali multipli
\def\Yint#1{\mathchoice
    {\YYint\displaystyle\textstyle{#1}}%
    {\YYint\textstyle\scriptstyle{#1}}%
    {\YYint\scriptstyle\scriptscriptstyle{#1}}%
    {\YYint\scriptscriptstyle\scriptscriptstyle{#1}}%
      \!\iint}
\def\YYint#1#2#3{{\setbox0=\hbox{$#1{#2#3}{\iint}$}
    \vcenter{\hbox{$#2#3$}}\kern-.51\wd0}}
\def\longdash{{-}\mkern-3.5mu{-}} 
   % consider using "\mkern-7.5mu" if esint package is loaded
\def\tiltlongdash{\rotatebox[origin=c]{15}{$\longdash$}}
\def\fiint{\Yint\tiltlongdash}

\def\Zint#1{\mathchoice
    {\YYint\displaystyle\textstyle{#1}}%
    {\YYint\textstyle\scriptstyle{#1}}%
    {\YYint\scriptstyle\scriptscriptstyle{#1}}%
    {\YYint\scriptscriptstyle\scriptscriptstyle{#1}}%
      \!\iiint}
      \def\tilongdash{\mkern6mu{-}\mkern-4mu{-}\mkern-5mu{-}} 
   % consider using "\mkern-7.5mu" if esint package is loaded
\def\titiltlongdash{\rotatebox[origin=c]{15}{$\tilongdash$}}
\def\fiiint{\Zint\titiltlongdash}

%Captions
\captionsetup[figure]{font=footnotesize,labelfont=footnotesize}
\captionsetup[table]{font=footnotesize,labelfont=footnotesize}
%Titlesec
\titleformat{\section}
{\fontsize{20}{20}\sffamily\scshape}
{\color{gray}{\fontsize{25}{20}\selectfont\thesection}}
{0.7em}
{}
\hypersetup{colorlinks,breaklinks, linkcolor=[RGB]{74, 122, 164}}
\definecolor{mastercolor}{HTML}{4a7aa4}
% Personalizza la formattazione della subsection
\titleformat{\subsection}[block]{\fontsize{15}{20}\bfseries}{\normalfont\thesubsection}{.5em}{}


% Personalizza la formattazione della subsubsection
\titleformat{\subsubsection}[block]{\fontsize{13}{20}\bfseries}{\normalfont\thesubsubsection}{.5em}{}

% Maketitle customization
\renewcommand{\maketitle}{
\begin{center}
{\sffamily
{\fontsize{20}{20}\selectfont\MakeUppercase\thetitle}}

\vspace{0.2in}

{\large\scshape\sffamily\theauthor}
\end{center}
}

%Evaluate symbol
\DeclareMathOperator{\di}{d\!}
\newcommand*\Eval[3]{\left.#1\right\rvert_{#2}^{#3}}

%%%%%%% Numero delle equazioni in formato a.b
\numberwithin{equation}{subsection}
%%%%%

%%%%%%%%%% Personalizzazione numeri lista
\renewcommand{\theenumi}{(\arabic{enumi})}

%%%% Table of contents

\usepackage[titles]{tocloft}

\renewcommand{\cftdot}{}
\usepackage{titletoc}
%\setcounter{tocdepth}{2}

%%%%%%%%%%%%%%%% Toc style

% Personalizzazione scritta indice


% Font
\usepackage{fontspec}
\usepackage{unicode-math}
\usepackage{kpfonts}



\begin{document}
\maketitle
\newpage
\tableofcontents 
\newpage
\section{Teoria degli anelli}
\subsection{Nozioni di base}


\begin{definizione}{Anello}{}
	Un insieme $A$ si dice \textit{anello} se \`e dotato di due operazioni, una somma $+$ e un prodotto $\cdot $, tali che:
	\begin{enumerate}[(a).]
		\item $(A,+)$ \`e un gruppo abeliano;
		\item il prodotto \`e associativo, ossia $(ab)c=a(bc), \ \forall a,b,c \in A$;
		\item prodotto e somma soddisfano la distributivit\`a, ossia $a(b+c) = ab + ac$ e $(a+b)c=ac+bc, \ \forall a,b,c \in A$.
	\end{enumerate}
\end{definizione}
\noindent Si parla di anello commutativo con unit\`a se, rispettivamente, il prodotto \`e commutativo e se ha l'elemento neutro.
\begin{boxenv}[]
Tutti gli anelli trattati saranno commutativi con identit\`a.
\end{boxenv}
\begin{definizione}
	{Omomorfismo di anelli}{}
	Dati due anelli $A,B$, si dice che $f:A\to B$ \`e un omomorfismo se:
	\begin{enumerate}[(a).]
		\item $f(a+b) = f(a)+f(b), \ \forall a,b \in A$;
		\item $f(ab) = f(a)f(b), \ \forall a,b \in A$;
		\item $f(1_A) = 1_B$.
	\end{enumerate}
\end{definizione}
\noindent Un \textit{sottoanello} $B$ di un anello $A$ \`e un sottogruppo additivo, chiuso rispetto al prodotto e contenente l'unit\`a.
\begin{definizione}
	{Ideale}{}
	Sia $A$ un anello e $I \subseteq A$ un sottoinsieme; questo \`e detto \textit{ideale} se:
\begin{enumerate}[(a).]
	\item \`e un sottogruppo additivo;
	\item ha la propriet\`a di assorbimento, cio\`e se $\forall i \in I, \ \forall a \in A$, si ha $ai \in A$.
\end{enumerate}
\end{definizione}
\begin{osservazione}
Lavorando con anelli commutativi, la richiesta $ai \in A$, oppure $ia \in A$ \`e equivalente e individua lo stesso elemento.
\end{osservazione}
\noindent Dato $S \subseteq A$ un sottoinsieme, si indica con $(S)$ il pi\`u piccolo ideale di $A$ che contiene $S$.
Visto che si lavora con anelli commutativi con identit\`a, se $S = \left\{ s_1,\ldots,s_n \right\} $, allora:
\[
	(S) = \left\{ \sum_{i=1}^{n} a_i s_i \ \Bigg\lvert \ a_i \in A \right\} 
\] 
Se $S$ \`e un insieme finito, allora $(S)$ si dice \textit{finitamente generato}; inoltre, se $S = s$, allora $(S)=(s)$ ed \`e detto \textit{principale}.

Dato un ideale $I$ di un anello $A$, l'insieme $A / I$ si pu\`o dotare di una struttura di anello tramite le operazioni
\[
	(a + I)+(b+I)=(a+b) + I \hspace{1cm} (a+I)(b+I) = ab + I 
\] 
Se $f:A\to B$ \`e un omomorfismo di anelli, il suo nucleo, $\operatorname{Ker} f$, \`e un ideale di $A$.
\begin{teorema}
	{I teorema di omomorfismo}{}
	Sia $f:A\to B$ un omomorfismo di anelli e $I \subseteq A$ un ideale di $A$, con $I \subseteq \operatorname{Ker} f$.
	Data la proiezione $\pi : A \to A / I$, allora esiste ed \`e unica $ \widetilde{f}:A / I \to B$ tale che $\widetilde{f}\circ \pi = f$, che risulta iniettiva se e solo se $I = \operatorname{Ker} f$.
	Per $I = \operatorname{Ker} f$, quindi, si ha $A / I \cong \operatorname{Im} f$.
\end{teorema}
\begin{teorema}
	{II teorema di omomorfismo}
	Sia $A$ un anello e siano $I,J \subseteq A$ due ideali, con $I \subseteq J$.
	Ne segue che $J / I$ \`e un ideale di $A / I$ e $(A / I)(J / I) \cong A / J$.
\end{teorema}
\begin{teorema}
	{Teorema di corrispondenza}{}
	Dato $A$ un anello e $I \subseteq A$ un suo ideale, la mappa $\pi:A\to A / I$ induce una corrispondenza biunivoca tra gli ideali di $A$ che contengono $I$ e gli ideali di $A / I$.
\end{teorema}
\noindent  Dato un anello $A$, un suo ideale proprio $P$ \`e detto \textit{primo} se
\[
\forall a,b \in A : ab \in P \implies a \in P \text{ oppure } b \in P
\] 
Inoltre, un ideale proprio $M$ di $A$ \`e detto \textit{massimale} se 
\[
\forall I \text{ ideale di } A : M \subseteq I \implies I = M \text{ oppure } I = A
\] 

\begin{definizione}
	{Spettro di un anello}{}
	Sia $A$ un anello; si definisce il suo \textit{spettro} come:
	\[
	\operatorname{Spec} A := \left\{ P \subset A  \mid P \text{ ideale primo di } A \right\} 
	\] 
\end{definizione}
\noindent Sulla stessa linea della precedente definizione, si definisce anche 
\begin{equation}
	\operatorname{Max} A := \left\{ M \subset A  \mid M \text{ \`e un ideale massimale di } A \right\} 
\end{equation}
Si ricorda che $P$ \`e un ideale primo se e solo se $A / I$ \`e un dominio, mentre $M$ \`e massimale se e solo se $A / M$ \`e un campo.
Ne segue che gli ideali massimali sono anche ideali primi.

Si ricorda, inoltre, che ogni anello non-banale ammette almeno un ideale massimale, quindi almeno un ideale primo.
Allora, ogni ideale proprio e, in particolare, ogni elemento non-invertibile \`e contenuto in un ideale massimale.
\begin{definizione}
	{Dimensione di Krull}{}
	Sia $A$ un anello; la sua \textit{dimensione di Krull} \`e definita come
	\[
	\dim A := \sup \left\{ k  \mid P_0 \subsetneq P_1 \subsetneq \ldots \subsetneq P_k , \ P_i \in \operatorname{Spec} A\right\} 
	\] 
	cio\`e \`e la massima lunghezza di una catena di ideali primi di $A$.
\end{definizione}
\begin{esempio}
	\ 

	\begin{itemize}
		\item Se $k$ \`e un campo, il suo unico ideale primo \`e $(0)$ ed \`e massimale, quindi $\dim k = 0$.
		\item Se $A = \mathbb{Z}$ (o un qualunque altro PID), i primi di $A$ sono $(0)$ e i massimali, pertanto la pi\`u lunga catena \`e data da $(0) \subseteq M$, per quale $M$ ideale massimale. 
			Allora $\dim A = 1$.
	\end{itemize}
\end{esempio}
\subsection{Ideali}



















\end{document}
