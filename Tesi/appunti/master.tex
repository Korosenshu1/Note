\documentclass[11pt, a4paper]{scrartcl} % Packages
\usepackage{sansiwona}
\usepackage[margin=1.5in]{geometry}
\usepackage{index}
\makeindex
\usepackage[utf8]{inputenc}
\usepackage[T1]{fontenc}
\usepackage{varwidth}
\usepackage{amsmath, amssymb, amsbsy}
\usepackage{esint}
\usepackage{titlesec}
\usepackage{xcolor}
\usepackage{titling}
\usepackage{braket}
\usepackage{tensor}
\usepackage[linktocpage]{hyperref}
\usepackage{pgfplots}
\usepackage{multicol}
\setlength{\columnsep}{2em}
\usepackage{caption}
\usepackage{amsthm}
\usepackage{import}
\usepackage{cancel}
\usepackage{caption}
\usepackage{tcolorbox}
\usepackage{nicematrix}
\usepackage{mathtools}
\usepackage{enumerate}
\usepackage{graphicx}
\usepackage{lipsum}
\usepackage[italian]{babel}
% To reset footnote numbering each page
\usepackage[perpage]{footmisc}
\usepackage{setspace}
\setstretch{1.5}

%Captions
\captionsetup[figure]{font=footnotesize,labelfont=footnotesize}
\captionsetup[table]{font=footnotesize,labelfont=footnotesize}
%Titlesec
\titleformat{\section}
{\fontsize{15}{20}\sffamily\scshape}
{\normalfont\color{gray}{\fontsize{20}{20}\selectfont\thesection}}
{0.7em}
{}
\hypersetup{colorlinks,breaklinks, linkcolor=[RGB]{74, 122, 164}}

\newcommand\vertarrowbox[3][6ex]{%
  \begin{array}[t]{@{}c@{}} #2 \
  \left\uparrow\vcenter{\hrule height #1}\right.\kern-\nulldelimiterspace\
  \makebox[0pt]{\scriptsize#3}
  \end{array}%
}
\definecolor{asdf}{HTML}{4a7aa4}
% Personalizza la formattazione della subsection
\titleformat{\subsection}[block]{\fontsize{13}{20}\bfseries}{\normalfont\thesubsection}{.5em}{}


% Personalizza la formattazione della subsubsection
\titleformat{\subsubsection}[block]{\fontsize{12}{20}\bfseries}{\normalfont\thesubsubsection}{.5em}{}

% Maketitle customization
\renewcommand{\maketitle}{
\begin{center}
{\sffamily
{\fontsize{20}{20}\selectfont\MakeUppercase\thetitle}}

\vspace{0.2in}

{\large\scshape\sffamily\theauthor}
\end{center}
}

% Titles 
\title{Divergenza dello sviluppo perturbativo per l'oscillatore anarmonico}
\author{Manuel Deodato}
\date{}



%Evaluate symbol
\DeclareMathOperator{\di}{d\!}
\newcommand*\Eval[3]{\left.#1\right\rvert_{#2}^{#3}}

%%%%%%% Numero delle equazioni in formato a.b
\numberwithin{equation}{section}
%%%%%

%%%%%%%%%% Personalizzazione numeri lista
\renewcommand{\theenumi}{(\arabic{enumi})}

%%%%%%%%%% Medie con integrali multipli
\def\Yint#1{\mathchoice
    {\YYint\displaystyle\textstyle{#1}}%
    {\YYint\textstyle\scriptstyle{#1}}%
    {\YYint\scriptstyle\scriptscriptstyle{#1}}%
    {\YYint\scriptscriptstyle\scriptscriptstyle{#1}}%
      \!\iint}
\def\YYint#1#2#3{{\setbox0=\hbox{$#1{#2#3}{\iint}$}
    \vcenter{\hbox{$#2#3$}}\kern-.51\wd0}}
\def\longdash{{-}\mkern-3.5mu{-}} 
   % consider using "\mkern-7.5mu" if esint package is loaded
\def\tiltlongdash{\rotatebox[origin=c]{15}{$\longdash$}}
\def\fiint{\Yint\tiltlongdash}

\def\Zint#1{\mathchoice
    {\YYint\displaystyle\textstyle{#1}}%
    {\YYint\textstyle\scriptstyle{#1}}%
    {\YYint\scriptstyle\scriptscriptstyle{#1}}%
    {\YYint\scriptscriptstyle\scriptscriptstyle{#1}}%
      \!\iiint}
      \def\tilongdash{\mkern6mu{-}\mkern-4mu{-}\mkern-5mu{-}} 
   % consider using "\mkern-7.5mu" if esint package is loaded
\def\titiltlongdash{\rotatebox[origin=c]{15}{$\tilongdash$}}
\def\fiiint{\Zint\titiltlongdash}


%%%% Table of contents

\usepackage[titles]{tocloft}

\renewcommand{\cftdot}{}
\usepackage{titletoc}
%\setcounter{tocdepth}{2}

%%%%%%%%%%%%%%%% Toc style

% Personalizzazione scritta indice




% Ambienti
\newtheoremstyle{style1}% name of the style to be used
{15pt}% measure of space to leave above the theorem. E.g.: 3pt
{15pt}% measure of space to leave below the theorem. E.g.: 3pt
{\normalfont}% name of font to use in the body of the theorem
{}% measure of space to indent
{\sffamily\scshape\bfseries}% name of head font
{}% punctuation between head and body
{ }% space after theorem head; " " = normal interword space
{\thmname{#1}\thmnumber{ #2}{\thmnote{ (#3)}}.\newline}

\newtheoremstyle{style2}% name of the style to be used
{15pt}% measure of space to leave above the theorem. E.g.: 3pt
{15pt}% measure of space to leave below the theorem. E.g.: 3pt
{\normalfont}% name of font to use in the body of the theorem
{}% measure of space to indent
{\sffamily\scshape\bfseries}% name of head font
{}% punctuation between head and body
{ }% space after theorem head; " " = normal interword space
{\thmname{#1}\thmnumber{ #2}{\thmnote{~--- #3}}.\ }


\theoremstyle{style2}
\newtheorem{osservazione}{Osservazione}[section]

\theoremstyle{style1}
\newtheorem{teorema}{Teorema}[section]
\newtheorem{corollario}{Corollario}[teorema]
\newtheorem{lemma}{Lemma}[teorema]
\newtheorem{definizione}{Definizione}[section]
\newtheorem{notazione}{Notazione}[section]
\newtheorem{esempio}{Esempio}[section]
\newtheorem{esercizio}{Esercizio}[section]

\renewcommand\qedsymbol{$\blacksquare$}

\newenvironment{svolgimento}{\renewcommand\qedsymbol{$\spadesuit$}\begin{proof}[Svolgimento]}{\end{proof}}

%% Generic box
\newtcolorbox{eqbox}[1][]
{
colback=gray!10,
arc=0pt,
boxrule=0pt,
title=#1
}

 \newenvironment{boxenv}[1][]{
    \begin{eqbox}[#1]
    }{
   \end{eqbox}
}






% Font
\usepackage{mathpazo}
\usepackage{ebgaramond}
%\usepackage{cmupint}
%%%


%%%%%%%%%%%%%%%%%%%%%%%%%%%%%%%%%%%%%%%%%%%%%%%%%%%%%%%%%%%%%%%%%%%%%%%%

\begin{document}
\maketitle
\newpage
\tableofcontents
\newpage
\section{Un esempio di divergenza}
Molte serie perturbative in meccanica quantistica sono divergenti; l'origine di questa divergenza appare connessa con altri aspetti, apparentemente sconnessi, fra cui:
\begin{itemize}
	\item la possibili\`a di ricostruire il risultato esatto da uno sviluppo asintotico;
	\item l'analiticit\`a della struttura del problema in esame;
	\item la stabilit\`a del sistema.
\end{itemize}
Si vuole studiare un esempio concreto per capire come approcciare il problema.
A tal proposito, si usa, come sistema esempio, un oscillatore anarmonico descritto da
\begin{equation}
	\hat{H} = \frac{1}{2}\hat{p}^2 + \frac{1}{2}\hat{x}^2 + \frac{1}{2}g\hat{x}^4
\end{equation}
dove si sono scelti $m=1,\omega =1$.
Esprimendo $\hat{x}^4$ tramite gli operatori di creazione e distruzione, ci si convince che gli elementi di matrice della perturbazione possono connettere solamente stati che hanno $|\Delta  n| \le 4$.
L'espansione perturbativa dell'energia \`e della forma $E = 1/2 + \sum_{}^{} g^n E_n$, mentre la funzione d'onda del fondamentale imperturbato \`e della forma $\psi _0 \propto e^{-  x^2 / 2} $.
Visto che la funzione d'onda dell'$n$-esimo stato \`e un polinomio di grado $n$ moltiplicato per $\psi _0$, si cerca soluzione della forma $B(x) e^{-x^2 / 2} $ all'equazione differenziale
\begin{equation}
\left(	-\frac{1}{2} \frac{d ^2}{d x^2}  + \frac{1}{2}x^2  +\frac{1}{2}gx^4  \right) \psi  = E \psi 
\end{equation}
Sostituendo l'\textit{ansatz}, si trova
\[
\frac{d ^2B}{d x^2} - 2x \frac{d B}{d x}  - gx^4 B + (2E - 1) B = 0
\] 
Essendo interessati particolarmente allo sviluppo dello stato fondamentale e visto che $\hat{P}_a$ commuta con la perturbazione, lo stato fondamentale, che originariamente \`e pari, rimane pari; allora avr\`a senso la seguente scelta ({\color{red}perch\'e?}): 
\[
B(x) = \sum_{k=0}^{+\infty} g^k B_k(x) \hspace{1cm} B_k(x) = \sum_{j=0}^{2k} A_{kj} x^{2j} 
\] 
con $A_{k0}=1 $.
Le energie si potranno scrivere come $2E = \sum_{k=0}^{+\infty} \epsilon _k g^k$, dove $\epsilon _0 = 1$.
La condizione di normalizzazione per queste funzioni d'onda \`e $B_k(0) = 1$. 
Sostituendo nell'equazione trovata per $B$, si ottiene:
\[
	\begin{split}
		&B''_k - 2B'_k - x^4 B_{k-1} + \sum_{s=0}^{k} \epsilon _s B_{k-s} - B_k = 0\\
		&\begin{split}
			\Rightarrow \sum_{\ell =1}^{2k} 2\ell (2\ell -1) A_{k,\ell } x^{2\ell -1} &-2 \sum_{j=1}^{2k} 2j A_{k,j} x^{2j}  - \sum_{\ell =0}^{2k-2} A_{k-1,\ell } x^{2\ell +4} \\
			&- \sum_{j=0}^{2k} A_{k,j} x^{2j}  + \sum_{s=0}^{k} \epsilon _s \sum_{j=0}^{2k-2s} A_{k-s,j} x^{2j} =0
		\end{split}
	\end{split}
\] 
Si adotta la convenzione per cui $A_{k,j} = 0$ per $j > 2k$; cos\`i facendo, sostituendo nella prima somma $\ell = j +1$ e nella terza $\ell = j-2$, si trova che:
\begin{equation*}
	\sum_{j=0}^{2k} x^{2j}\left[ (2j+2)(2j+1) A_{k,j+1} -4j A_{k,j}  - A_{k-1,j-2} -A_{k,j} + \sum_{s=0}^{k} \epsilon _s A_{k-s,j}  \right]  = 0
\end{equation*}
Tutti i coefficienti di questo polinomio devono essere nulli. 
Usando che $\epsilon _0 = 1$, si trova che il termine $s=0$ cancella $-A_{k,j} $:
\begin{equation}
	(2j+2)(2j+1) A_{k,j+1} - 4j A_{k,j} - A_{k-1,j-2} + \sum_{s=1}^{k} \epsilon _s A_{k-s,j} =0
\end{equation}
Inoltre, il termine per $s=k$ nella somma ha coefficiente $A_{0,j} $, che \`e pari a $1$ per $j=0$ e nullo altrimenti; da questo, si ottiene una relazione ricorsiva per le energie $\epsilon _k$, quando siano noti i coefficienti $A_{k,1} $:
\begin{equation}
	\epsilon _k = -2A_{k,1} - \sum_{s=1}^{k-1} \epsilon _s
\end{equation}
Considerando, invece, $j\neq 0 $:
\begin{equation*}
	(2j+2)(2j+1) A_{k,j+1} - 4j A_{k,j} - A_{k-1,j-2} + \sum_{s=1}^{k-1} \epsilon _s A_{k-s,j} =0
\end{equation*}
Si nota che in questa espressione, il termine $\epsilon _k$ non compare pi\`u.
Si prende $j=2k$, per cui $A_{k,2k+1} =0$. 
Questo permette di trovare una relazione ricorsiva per $A_{k,j} $:
\begin{equation}
	A_{k,j} = \frac{1}{4j} \left[ (2j+2)(2j+1)A_{k,j+1} -A_{k-1,j-2} +\sum_{s=1}^{k-1} \epsilon _s A_{k-s,j}  \right] 
\end{equation}
con $j=2k,2k-1, \ldots, 1$.

Trovati gli $A_{k,1} $, si possono determinare le energie $\epsilon _k$; facendolo, si osserva che questi termini, di segno alterno, aumentano molto velocemente.

\section{Origine della divergenza}
Si considera lo studio della serie dell'energia relativa ad uno stato fondamentale (nello specifico, quello dell'oscillatore appena trattato); in generale, questa \`e data da:
\begin{equation}
	E(g) = E_0 + \sum_{}^{} E_n g^n
\end{equation}
Quando questa serie \`e divergente come nel caso trattato sopra (in cui $E_n \sim n!$), questo vuol dire che $E(g)$ non \`e analitica in $g=0$.
La motivazione per questa non-analiticit\`a \`e proposta da Dyson e si basas sul fatto che, se fosse $E(g)$ analitica in $g=0$, esisterebbe un dominio di convergenza attorno a tale punto, sia questo $\lvert g \rvert < R$, e, in questo dominio, la somma della serie dovrebbe riprodurre esattamente $E(g)$.
Tuttavia, per $g<0$, l'Hamiltoniano non \`e limitato dal basso, quindi la funzione $E(g)$ non pu\`o essere analitica. 

Nel caso dell'oscillatore armonico imperturbato (quindi armonico), {\color{red}una particella nel fondamentale pu\`o decadere tramite effetto tunnel attraverso la barriera di potenziale (?)} e tale stato pu\`o essere unicamente metastabile; questo stesso effetto non pu\`o essere descritto tramite una teoria perturbativa, che mantiene gli stati in $\mathbb{L}^2$. 
Allora, nel potenziale perturbato
\[
V(x) = \frac{x^2 + gx^4 }{2}
\] 
nel caso di $g< 0$, l'effetto tunnel \`e responsabile per la non-analiticit\`a della funzione $E(g)$.
Si afferma, inoltre, che la parte immaginaria di $E(g)$ per $g<0$, che restituisce lo spessore di linea e il tempo di dimezzamento, deve essere collegata con il comportamento divergente dei coefficienti perturbativi $E_n$.
Questa relazione, si ottiene di seguito come relazione di dispersione.
\section{Analiticit\`a del dominio}
La mancanza di analiticit\`a si pu\`o intendere come segue.
Sia $\hat{H}_0$ l'Hamiltoniano imperturbato al quale \`e associato un certo dominio $D(H_0)\subset \mathbb{L}^2$ dato da tutte quelle funzioni per cui $\hat{H}_0$ risulta autoaggiunto.

Per esempio, il dominio deve essere tale che $\forall \psi \in D(H_0)$
\[
\int dx \lvert \psi (x) \rvert ^2 x^2 < \infty
\] 
cio\`e il valore d'aspettazione del potenziale deve esistere.
Sia, esempio, $\psi _0\sim 1 / x^2$; per questa, si ha:
\[
\int dx \lvert \psi _0(x) \rvert ^2 x^2 < \infty \hspace{1cm} \int dx \lvert \psi _0(x) \rvert ^2 x^4 \sim \int dx \frac{1}{x^4} x^4 \to \infty
\] 
In questo caso, la funzione $\psi _0$ appartiene al dominio di $\hat{H}_0$, ma $\psi _0 \not \in D(H)$.
Questo discorso serve per dire che esistono stati perfettamente ammissibili dal punto di vista dall'Hamiltoniano imperturbato, ma che non lo sono da quello di $\hat{H}$ perch\'e, indipendentemente da quanto si prenda $g$ piccolo, la perturbazione non potr\`a mai essere considerata piccola.
Questo discorso \`e motivato formalmente dal seguente teorema.
\begin{teorema}
	[Kato-Rellich]
Sia $H (g) $ una famiglia di operatori con $g \in S \subset  \mathbb{C}$ e valgano i seguenti punti:
\begin{enumerate}[(a).]
	\item $D\big(H(g)\big)$ \`e indipendente da $g$;
	\item $\forall \psi  \in D\big(H(g)\big)$, l'elemento di matrice $\braket{\psi |H(g)|\psi } $ \`e una funzione analitica di $g$ in $S$.
\end{enumerate}
Allora $\forall g_0 \in S$ e per ogni autovalore isolato $E(g_0)$ relativo a $\hat{H}(g_0)$, esiste un intorno $V_{g_0} $ tale che $\hat{H}(g)$ ha un autovalore unico e isolato $E(g)$ in un intorno di $E(g_0)$.
La funzione $E(g)$ \`e analitica in $V_{g_0} $ ed esiste una funzione $\psi _g$, analitica in $g$, tale che
\[
\hat{H}(g) \psi _g = E(g) \psi _g
\] 
Inoltre, la serie di Taylor di $E(g)$ coincide con lo sviluppo perturbativo per $\hat{H}(g)$.
\end{teorema}
Questo teorema permette di concludere che, soddisfatte le sue condizioni, lo sviluppo perturbativo restituisce il risultato esatto.
Una condizione sufficiente per la validit\`a dei due punti del teorema \`e data dal seguente.
\begin{teorema}
	[Kato]
	Se esistono due numeri $a,b$ tali che $\forall \psi \in D(\hat{H}_0), \ \forall \psi \in D(V)$ risulta soddisfatta 
	\begin{equation}
		\left\lVert \hat{V}\psi  \right\rVert \le a \left\lVert \hat{H}_0 \psi  \right\rVert + b \left\lVert \psi  \right\rVert 
	\end{equation}
	allora le condizioni del teorema di Kato-Rellich sono soddisfatte.
\end{teorema}
Nel caso dell'oscillatore anarmonico, il problema sorge nella condizione (a) del teorema di Kato-Rellich: di fatto, la funzione $E(g)$ non pu\`o essere espansa in $g=0$, anche se \`e espandibile $\forall g > 0$.
Quello che si verifica \`e che, man mano che si approccia lo zero, il raggio di convergenza dello sviluppo diventa sempre pi\`u piccolo fino a risultare nullo.

Nel caso specifico dell'oscillatore anarmonico, molte propriet\`a relative all'analiticit\`a di $E(g)$ possono essere ottenute tramite le seguenti considerazioni. 
Si prende la trasformazione $U(\lambda ) \psi (x) = \lambda ^{1 / 2} \psi \big(\lambda x\big)$, che si nota essere unitaria e tale che per un Hamiltoniano della forma $\hat{H}= \hat{p}^2 / 2 + \alpha  \hat{x}^2 / 2 + g \hat{x}^4 / 2$, soddisfa:
\begin{equation}
	\hat{U}(\lambda ) \hat{H}(\alpha ,g) \hat{U}(\lambda ^{-1} ) = \lambda ^{-2}  \hat{H}(\alpha \lambda ^4, g \lambda ^6)
\end{equation}
Dovendo gli autovalori rimanere invariati sotto trasformazioni unitarie, prendendo $\lambda = g ^{ 1/6} $, si ha:
\begin{equation}\label{wer}
	E_n(1,g) = g^{1 / 3} E_n(g^{-2 / 3} , 1)
\end{equation}
Questa relazione permette di concludere che la singolarit\`a per $g = 0$ di $E(g)$ \`e cubica; inoltre, per il criterio di Kato, l'espansione
\[
E_n(g) = g^{1 / 3} \sum_{k}^{} a_k g^{ - 2k / 3} 
\]
\`e convergente e, asintoticamente, si ha $E_n(g) \sim g^{ 1/3} $.
Si pu\`o mostrare che la superficie di Riemann associata \`e a tre fogli, con un branch point in $g=0$, come suggerito da eq. \ref{wer}, e che la funzione $E(g)$ risulta analitica nel primo foglio con $\lvert \operatorname{arg} (g) \rvert < \pi $.
\end{document}
