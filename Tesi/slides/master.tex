\documentclass[10pt]{beamer}
\usepackage{beamer}
\usepackage[T1]{fontenc}
\usepackage[utf8]{inputenc} 
\usepackage{graphicx}
\usepackage{svg}
\usepackage{amsmath, amssymb, amsbsy}
\usepackage{tensor}
\usepackage{amsthm}
\usepackage{cancel}
\usepackage{braket}
\usepackage{mathtools}
\usepackage{newtxmath}
\usepackage[italian]{babel}
\newtheorem{teorema}{Teorema}
% default color is dark
\usepackage{setspace}
\setstretch{1.1}
\usepackage{tcolorbox}
%% Generic box
\newtcolorbox{eqbox}[1][]
{
colback=gray!10,
arc=0pt,
boxrule=0pt,
title=#1
}

 \newenvironment{boxenv}[1][]{
    \begin{eqbox}[#1]
    }{
   \end{eqbox}
}

\def\colortheme{dark} % preset themes available are lighten, light, dark
\def\alertcolor{dark} % preset themes available are lighten, light, dark
\def\upperbar{\true} % i want upper index/navigation bar, \true or \false
\def\bottomsectionbar{\true} % i want bottom bar with Title and Frame/Slide number, \true or \false
\def\bottomtitlebar{\true} % i want bottom bar with Section and Institute, \true or \false

% % %
\if\upperbar\false
    \setbeamertemplate{headline}{}
\fi
% % %

\title{Divergenza delle serie perturbative}
\author{%
  \begin{tabular}{c @{\hspace{5cm}} c}
	  \textit{\small Laureando}  & \textit{\small Relatore}   \\
	  Manuel Deodato & Claudio Bonati
  \end{tabular}
}

\institute
{
  Università di Pisa
}
\date{}

\begin{document}

\firstpage % optional. First page

\footnotesize
\begin{frame}
	\frametitle{Introduzione}
In meccanica quantistica, molti problemi non si risolvono esattamente $\to$ si risolvono perturbativamente, scrivendo $\hat{H}=\hat{H}_0 + \lambda \hat{V}$, con $\hat{H}_0$ noto e $\lambda \ll 1$.

Cos\`i facendo, energie e stati si sviluppano in serie:
\[
\begin{split}
	&E= E^{(0)} + \lambda E^{(1)}  + \lambda ^2 E^{(2)}  + \ldots\\
	& \ket{\psi }  = \ket{\psi } ^{(0)} + \lambda \ket{\psi  } ^{(1)} + \lambda ^2 \ket{\psi } ^{(2)}  +\ldots
\end{split}
\] 
Potrebbe sembrare che la condizione di \textit{perturbazione piccola}, definita dalla richiesta $\lambda \ll 1$, assicuri la validit\`a dello sviluppo, ma questo non \`e vero in generale: in molti casi, le serie perturbative divergono.

L'obiettivo \`e di capire cosa causa questa divergenza e trovare delle condizioni per cui la convergenza \`e assicurata; a tale scopo, si considerer\`a il caso specifico dell'oscillatore armonico perturbato da un potenziale quartico come riferimento per il caso generale.
\end{frame}
\begin{frame}
	\frametitle{L'oscillatore anarmonico}
Particella 1D in potenziale (prendendo $m=1,\omega =1$)
\begin{equation*}
	\hat{V} = \frac{1}{2}\hat{x}^2 + \frac{1}{2}g \hat{x}^4
\end{equation*}
L'energia del fondamentale \`e della forma $E = 1/2 + \sum_{}^{} g^n E^{(n)} $. Per studiare lo sviluppo, si nota che le funzioni d'onda degli stati eccitati dell'oscillatore armonico si scrivono come il prodotto di un polinomio per $e^{-x^2/2} $, quindi si cercano soluzioni all'equazione di Schr\"odinger della forma $B(x) e^{-x^2 / 2} $.
Se $\ket{n}$, $\ket{m} $ sono due autostati di $\hat{H}_0$:
\begin{equation*}
	\braket{n |\hat{x}^4|m} \neq 0 \iff \begin{cases}
		\Delta n= |n-m| \le  4 \hspace{.25cm}\\
		\pi_n = \pi_m
	\end{cases}
\end{equation*}
Si parte dal fondamentale, quindi $m\le 4$ e $\pi_m= +1$; il polinomio $B(x)$ si pu\`o scrivere come:
\begin{equation*}
	B(x) = \sum_{k=0}^{+\infty} g^k B_k(x) \hspace{1cm} B_k(x)= \sum_{j=0}^{2k} A_{k,j} x^{2j} 
\end{equation*}
Dall'equazione di Schr\"odinger, si trovano delle relazioni ricorsive che permettono di determinare i coefficienti dello sviluppo dell'energia; tramite calcolo numerico, si vede che questi hanno un andamento del tipo $E^{(n)}  \sim n!$.
\end{frame}


  \begin{frame}
\frametitle{Origine della divergenza e scaling di Symanzik}

  Divergenza $\Rightarrow $ non analiticit\`a di $E(g)$ in un intorno di $g=0$. 
  \begin{minipage}{0.45\textwidth}
    	\centering
    	\includegraphics[width=\columnwidth]{f1.pdf}
  \end{minipage}%
  \hfill
  \begin{minipage}{0.5\textwidth}
Per $g<0$, il potenziale \`e $V= \frac{1}{2}x^2 - \frac{1}{2}\lvert g \rvert x^4$ e lo stato fondamentale del sistema non esiste pi\`u: la particella pu\`o fuoriuscire dalla buca di potenziale per effetto tunnel, quindi lo stato in cui si trova pu\`o solo essere metastabile.
Questo \`e, quindi, responsabile della non-analiticit\`a per $g<0$.
  \end{minipage}
{\hspace{2cm}}
  Nel caso dell'oscillatore anarmonico, i problemi di analiticit\`a delle autoenergie di $\hat{H}$ possono essere evidenziati considerando la trasformazione unitaria $\hat{U}(\lambda ) \psi (x) = \lambda ^{1 / 2} \psi (\lambda x)$ che, su $\hat{H}(\alpha ,g) = \hat{p}^2 / 2 + \alpha  \hat{x}^2 / 2 + g \hat{x}^4 / 2$, agisce come:
  \begin{equation*}
  	\hat{U}(\lambda ) \hat{H}(\alpha ,g) \hat{U}(\lambda ^{-1} ) = \lambda ^{-2} \hat{H}(\alpha  \lambda ^4, g \lambda ^6)
  \end{equation*}
Ponendo $\lambda  = g^{- 1 / 6} $, si ottiene una forma analitica per le autoenergie $E_n$ in termini di una serie convergente:
  \[
  E_n(1,g) = g^{1 / 3} E_n(g^{-2 / 3} , 1) \implies E_n(g) = g^{1 /3 }  \sum_{k}^{} a_k g^{-2k/3}
\] 
  \end{frame} 
   \begin{frame}
	   \frametitle{Analiticit\`a del dominio}
	   $D(\hat{H}) \subset L^2$ \`e caratterizzato dalle funzioni che rendono $\hat{H}$ autoaggiunto; in particolare, il valore medio del potenziale deve esistere. Se $\psi \sim 1/x^2$ (per $x$ grandi):
	   \begin{equation*}
	   	\int dx \ \lvert \psi (x) \rvert ^2 x^2 < \infty \hspace{1cm} \int dx\ \lvert \psi (x) \rvert ^2 x^4 \sim \int dx\  \frac{1}{x^4}x^4 \to \infty
	   \end{equation*}
	   Allora $\psi \in D(\hat{H}_0),\ \psi \not \in D(\hat{H})$ perch\'e il valore medio del potenziale perturbato diverge.

	   $\Rightarrow $ Per quanto $g$ sia piccolo, la perturbazione non pu\`o mai essere considerata tale.

	   \begin{boxenv}[]
	   \textbf{\textit{Teorema di Kato-Rellich.} } 

		   Sia $\hat{H}(g)$ una famiglia di operatori con $g \in S \subset \mathbb{C}$, con $S$ aperto, tale che:
		   \begin{enumerate}[1]
		   	\item $D(\hat{H}(g))$ \`e indipendente da $g$;
			\item $\forall \psi \in D(\hat{H}(g))$, la funzione $\braket{\psi |\hat{H}(g)|\psi } $ \`e analitica per $g \in S$.
		   \end{enumerate}  
		   Allora $\forall g_0\in S, \forall E(g_0)$ autovalore isolato di $\hat{H}(g_0)$, esiste un intorno $I_{g_0} $ tale che $\hat{H}(g)$ ha un unico autovalore isolato $E(g)$; in questo intorno, $E(g)$ \`e analitica e esiste $\psi _g$ anch'essa analitica e tale che $\hat{H}(g) \psi _g = E(g) \psi _g$.
	   \end{boxenv}

	Per l'oscillatore con $\hat{V} = \frac{1}{2} \hat{x}^2 + \frac{1}{2}g \hat{x}^4$ non \`e verificato il punto (1) del teorema di Kato-Rellich $\longrightarrow$ il dominio dipende da $g$.
   \end{frame} 
%   \begin{frame}
%		Le due condizioni del teorema di Kato-Rellich possono essere verificate tramite la condizione posta dal seguente.\pause
%   	\begin{teorema}
%		[Teorema di Kato]
%Siano $a,b \in \mathbb{R}$ tali che $\forall \psi \in D(\hat{H}_0) \cap D(\hat{V})$, cio\`e $\forall \psi  \in D(\hat{H})$:\pause
%\begin{equation*}
%	\lVert \hat{V}\psi  \rVert \le  a \lVert \hat{H}_0 \psi  \rVert + b \lVert \psi  \rVert 
%\end{equation*}
% 	\end{teorema}\pause
%   \end{frame}
   \begin{frame}
	   \frametitle{Conclusioni}
	   Il problema della divergenza \`e, quindi, legato alla presenza di effetti che alterano gli stati del sistema imperturbato, come l'effetto tunnel.
	   Matematicamente, questi effetti si manifestano nella differenza tra i domini dell'Hamiltoniano imperturbato e quello perturbato: nel caso specifico dell'oscillatore anarmonico, $D(\hat{H}(g))$ non \`e indipendente da $g$. 
	   
	   In generale, la convergenza dello sviluppo perturbativo pu\`o essere verificata dal teorema di Kato-Rellich.

	   \vspace{.25cm}
	   Si nota, per\`o, che questo non rende vano lo sviluppo: qualora la serie fosse asintotica, cio\`e soddisfa
\[
\left\lvert f(z) - \sum_{k=0}^{N} f_k z^k \right\rvert \le  C_{N+1} \lvert z \rvert ^{N+1} , \ \forall N
\] 
	   in un dominio $D \subset \mathbb{C}$ e con $f(z)$ analitica in $D$, come nel caso dell'oscillatore anarmonico, i primi termini dello sviluppo fornirebbero una buona approssimazione per $g$ relativamente piccolo.
   \end{frame}
\end{document}
