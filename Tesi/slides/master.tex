\documentclass[10pt]{beamer}
\usepackage{beamer}
\usepackage[english]{babel}
\usepackage[T1]{fontenc}
\usepackage[utf8]{inputenc} 
\usepackage{graphicx}
\usepackage{svg}
\usepackage{amsmath, amssymb, amsbsy}
\usepackage{tensor}
\usepackage{amsthm}
\usepackage{cancel}
\usepackage{braket}
\usepackage{mathtools}
\usepackage{newtxmath}


% default color is dark

\def\colortheme{dark} % preset themes available are lighten, light, dark
\def\alertcolor{dark} % preset themes available are lighten, light, dark
\def\upperbar{\true} % i want upper index/navigation bar, \true or \false
\def\bottomsectionbar{\true} % i want bottom bar with Title and Frame/Slide number, \true or \false
\def\bottomtitlebar{\true} % i want bottom bar with Section and Institute, \true or \false

% % %
\if\upperbar\false
    \setbeamertemplate{headline}{}
\fi
% % %

\title{Divergenza delle serie perturbative}
\author{Manuel Deodato}
\institute
{
  Università di Pisa
}
\date{}

\begin{document}

\firstpage % optional. First page

\footnotesize
\begin{frame}
	\frametitle{L'oscillatore anarmonico}
Particella 1D in potenziale 
\begin{equation*}
	\hat{V} = \frac{1}{2}\hat{x}^2 + \frac{1}{2}g \hat{x}^4
\end{equation*}
Perturbativamente, le energie del fondamentale sono della forma $E = 1/2 + \sum_{}^{} g^n E_n$ e le funzioni d'onda degli stati eccitati si scrivono come $B(x) e^{-x^2 / 2} $.
Se $\ket{n} , \ket{m} $ sono due autostati di $\hat{H}_0$:
\begin{equation*}
	\braket{n |\hat{x}^4|m} \neq 0 \iff \begin{cases}
		\Delta n= |n-m| \le  4\\
		\pi_n = \pi_m
	\end{cases}
\end{equation*}
Si parte dal fondamentale, quindi $m\le 4$ e $\pi_m= +1$; il polinomio $B(x)$ si pu\`o scrivere come:
\begin{equation*}
	B(x) = \sum_{k=0}^{+\infty} g^k B_k(x) \hspace{1cm} B_k(x)= \sum_{j=0}^{2k} A_{k,j} x^{2j} 
\end{equation*}
con $B_k(0)=1$ come scelta di normalizzazione $\Rightarrow A_{k,0} =1$.
Inserendo nell'equazione di Schr\"odinger, si trovano le seguenti relazioni ricorsive per determinare le energie:
\begin{equation*}
		\epsilon _k = - 2A_{k,1}  - \sum_{s=1}^{k-1} \epsilon _s; \ A_{k,j} = \frac{1}{4j} \left[ (2j+2)(2j+1) A_{k,j+1} - A_{k-1,j-2} + \sum_{s=1}^{k-1} \epsilon _s A_{k-s,j}  \right] 
\end{equation*}
Tramite calcolo numerico, si vede che queste sono divergenti con $E_n \sim n!$.
\end{frame}


  \begin{frame}
\frametitle{Origine della divergenza e scaling di Symanzik}
Divergenza $\Rightarrow $ non analiticit\`a di $E(g)$ in un intorno di $g=0$. 
Per $g<0$, il potenziale \`e
\begin{equation*}
	V= \frac{1}{2}x^2 - \frac{1}{2}\lvert g \rvert x^4
\end{equation*}
  \begin{minipage}{0.45\textwidth}
    	\centering
    	\includegraphics[width=\columnwidth]{f1.pdf}
  \end{minipage}%
  \hfill
  \begin{minipage}{0.5\textwidth}
	  Per $g<0$, la particella si trova in una buca di potenziale $\to$ per effetto tunnel vi pu\`o fuoriuscire e liberarsi.

	  Questo non \`e descrivibile perturbativamente.
  \end{minipage}
  \begin{center}
	  \color{red} Scrivere formula effetto tunnel
  \end{center}
  Per evidenziare la divergenza, si considera la trasformazione unitaria $\hat{U}(\lambda ) \psi (x) = \lambda ^{1 / 2} \psi (\lambda x)$ che, su $\hat{H}(\alpha ,g) = \hat{p}^2 / 2 + \alpha  \hat{x}^2 / 2 + g \hat{x}^4 / 2$, agisce come:
  \begin{equation*}
  	\hat{U}(\lambda ) \hat{H}(\alpha ,g) \hat{U}(\lambda ^{-1} ) = \lambda ^{-2} \hat{H}(\alpha  \lambda ^4, g \lambda ^6)
  \end{equation*}
  Quindi, per $\lambda  = g^{- 1 / 6} $:
  \[
  E_n(1,g) = g^{1 / 3} E_n(g^{-2 / 3} , 1) \implies E_n(g) = g^{1 /3 }  \sum_{k}^{} a_k g^{-2k/3} \sim g^{1 / 3} 
  \] 
  
  \end{frame} 
   \begin{frame}
	   \frametitle{Analiticit\`a del dominio}
	   $D(\hat{H}) \subset L^2$ \`e caratterizzato dalle funzioni che rendono $\hat{H}$ Hermitiano; in particolare, il valore medio del potenziale deve esistere.
	   Se $\psi \sim 1/x^2$:
	   \begin{equation*}
	   	\int dx \ \lvert \psi (x) \rvert ^2 x^2 < \infty \hspace{1cm} \int dx\ \lvert \psi (x) \rvert ^2 x^4 \sim \int dx\  \frac{1}{x^4}x^4 \to \infty
	   \end{equation*}
	   Allora $\psi \in D(\hat{H}_0),\ \psi \not \in D(\hat{H})$ perch\'e il valore medio del potenziale perturbato diverge.

	   $\Rightarrow $ Per quanto $g$ sia piccolo, la perturbazione non pu\`o mai essere considerata tale.

	   Questo si generalizza nel seguente.
	   \begin{theorem}
		   [Teorema di Kato-Rellich]
		   Sia $\hat{H}(g)$ una famiglia di operatori con $g \in S \subset \mathbb{C}$ tale che:
		   \begin{enumerate}[1]
		   	\item $D(\hat{H}(g))$ \`e indipendente da $g$;
			\item $\forall \psi \in D(\hat{H}(g))$, la funzione $\braket{\psi |\hat{H}(g)|\psi } $ \`e analitica per $g \in S$.
		   \end{enumerate}  
		   Allora $\forall g_0\in S, \forall E(g_0)$ autovalore isolato di $\hat{H}(g_0)$, esiste un intorno $I_{g_0} $ tale che $\hat{H}(g)$ ha un unico autovalore isolato $E(g)$; in questo intorno, $E(g)$ \`e analitica e esiste $\psi _g$ anch'essa analitica e tale che $\hat{H}(g) \psi _g = E(g) \psi _g$.
	   \end{theorem}
   \end{frame} 
   \begin{frame}
	   Le due condizioni del teorema di Kato-Rellich possono essere verificate tramite la condizione posta dal seguente.
   	\begin{theorem}
		[Teorema di Kato]
Siano $a,b \in \mathbb{R}$ tali che $\forall \psi \in D(\hat{H}_0) \cap D(\hat{V})$, cio\`e $\forall \psi  \in D(\hat{H})$:
\begin{equation*}
	\lVert \hat{V}\psi  \rVert \le  a \lVert \hat{H}_0 \psi  \rVert + b \lVert \psi  \rVert 
\end{equation*}
   	\end{theorem}
	Per l'oscillatore con $\hat{V} = \frac{1}{2} \hat{x}^2 + \frac{1}{2}g \hat{x}^4$ non \`e verificato il punto (1) del teorema di Kato-Rellich $\longrightarrow$ il dominio dipende da $g$.
   \end{frame}
\end{document}
