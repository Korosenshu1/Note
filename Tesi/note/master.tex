\documentclass[10pt, a4paper]{scrartcl}
% Packages
\usepackage[margin=1.25in]{geometry}
\usepackage{index}
\makeindex
\usepackage[utf8]{inputenc}
\usepackage[T1]{fontenc}
\usepackage{tcolorbox}
\tcbuselibrary{theorems}
\tcbuselibrary{skins}
\tcbuselibrary{breakable}
\usepackage{varwidth}
\usepackage{textcomp}
\usepackage{amsmath, amssymb}
\usepackage{esint}
\usepackage{titlesec}
\usepackage{xcolor}
\usepackage{titling}
\usepackage[linktocpage]{hyperref}
\usepackage{pgfplots}
\usepackage{multicol}
\setlength{\columnsep}{2em}
\usepackage{caption}
\usepackage{amsthm}
\usepackage{import}
\usepackage{cancel}
\usepackage{caption}
\usepackage{nicematrix}
\usepackage{mathrsfs}
\usepackage{mathtools}
%\usepackage{parskip}
\usepackage{pythonhighlight}
\usepackage{enumerate}
\usepackage{graphicx}
\usepackage{tikz}
\usepackage[italian]{babel}

% Titles 
\title{Appunti di EDP}
\author{Manuel Deodato}
\date{}

%%%%% tcolorbox setup

% Teorema e proposizione
\NewTcbTheorem[number within=section]{teorema}{Teorema}
{breakable, top=0.2mm, bottom=0.2mm, boxrule=0mm,arc =.5 mm, colframe=blue!10, coltitle=black, fonttitle=\bfseries, colback=blue!5!white, theorem style=plain apart}{th}

\NewTcbTheorem[number within=section]{prop}{Proposizione}
{breakable, top=0.2mm, bottom=0.2mm, boxrule=0mm,arc =.5 mm, colframe=blue!10, coltitle=black, fonttitle=\bfseries, colback=blue!5!white, theorem style=plain apart}{prop}





% Definizione
\definecolor{greendef}{HTML}{b8d8be}

\NewTcbTheorem[number within=section]{definizione}{Definizione}
{breakable, top=0.2mm, bottom=0.2mm, boxrule=0mm, arc=.5mm, colframe=greendef, coltitle=black, fonttitle=\bfseries, theorem style = plain apart, colback=greendef!50!white}{def}


% Esempio
\theoremstyle{definition}
\newtheorem{esempio}{Esempio}

%\definecolor{empurple}{HTML}{6e5e89}

%\NewTcbTheorem{esempio}{Esempio}{left=0mm,arc=0mm, colframe=empurple!10!white, coltitle=black, fonttitle=\bfseries, theorem style = plain, colback=empurple!20!white, colframe=empurple!90!white, boxrule=1pt, sharp corners, top=.2mm,bottom=.2mm}{es}

\tcolorboxenvironment{esempio}{blanker,breakable,left=5mm,before skip=10pt,after skip=10pt, borderline west={1mm}{0pt}{greendef}}

\numberwithin{esempio}{section}


% Lemma e Corollario
\definecolor{lemcor}{HTML}{a78d8a}

\NewTcbTheorem[number within=section]{lemma}{Lemma}{breakable, top=0.2mm, bottom=0.2mm, boxrule=0mm,left=0mm,arc=.5mm, colframe=lemcor!10!white, coltitle=black, fonttitle=\bfseries, theorem style = plain apart, colframe=lemcor!50!white,colback=lemcor!20!white}{lem}
\NewTcbTheorem[number within=section]{corollario}{Corollario}{breakable, top=0.2mm, bottom=0.2mm, boxrule=0mm,left=0mm,arc=.5mm, colframe=lemcor!10!white, coltitle=black, fonttitle=\bfseries, theorem style = plain apart, colframe=lemcor!50!white,colback=lemcor!20!white}{cor}



% Osservazione
\theoremstyle{definition}
\newtheorem{obs}{Osservazione}

\definecolor{coloros}{HTML}{6e5e89}

\tcolorboxenvironment{obs}{blanker,breakable,left=5mm,before skip=10pt,after skip=10pt, borderline west={1mm}{0pt}{coloros}}

\numberwithin{obs}{section}

% Nota
\newtheorem{nota}{Nota}

\definecolor{ncol}{HTML}{f9ebbe}

\tcolorboxenvironment{nota}{blanker,breakable,left=5mm,before skip=10pt,after skip=10pt, borderline west={1mm}{0pt}{ncol}}

\numberwithin{nota}{section}



%%%%%%%%%% Medie con integrali multipli
\def\Yint#1{\mathchoice
    {\YYint\displaystyle\textstyle{#1}}%
    {\YYint\textstyle\scriptstyle{#1}}%
    {\YYint\scriptstyle\scriptscriptstyle{#1}}%
    {\YYint\scriptscriptstyle\scriptscriptstyle{#1}}%
      \!\iint}
\def\YYint#1#2#3{{\setbox0=\hbox{$#1{#2#3}{\iint}$}
    \vcenter{\hbox{$#2#3$}}\kern-.51\wd0}}
\def\longdash{{-}\mkern-3.5mu{-}} 
   % consider using "\mkern-7.5mu" if esint package is loaded
\def\tiltlongdash{\rotatebox[origin=c]{15}{$\longdash$}}
\def\fiint{\Yint\tiltlongdash}

\def\Zint#1{\mathchoice
    {\YYint\displaystyle\textstyle{#1}}%
    {\YYint\textstyle\scriptstyle{#1}}%
    {\YYint\scriptstyle\scriptscriptstyle{#1}}%
    {\YYint\scriptscriptstyle\scriptscriptstyle{#1}}%
      \!\iiint}
      \def\tilongdash{\mkern6mu{-}\mkern-4mu{-}\mkern-5mu{-}} 
   % consider using "\mkern-7.5mu" if esint package is loaded
\def\titiltlongdash{\rotatebox[origin=c]{15}{$\tilongdash$}}
\def\fiiint{\Zint\titiltlongdash}

%Captions
\captionsetup[figure]{font=footnotesize,labelfont=footnotesize}
\captionsetup[table]{font=footnotesize,labelfont=footnotesize}
%Titlesec
\titleformat{\section}
{\fontsize{15}{20}\sffamily\scshape}
{\normalfont\color{gray}{\fontsize{20}{20}\selectfont\thesection}}
{0.7em}
{}
\hypersetup{colorlinks,breaklinks, linkcolor=[RGB]{74, 122, 164}}
\definecolor{asdf}{HTML}{4a7aa4}
% Personalizza la formattazione della subsection
\titleformat{\subsection}[block]{\fontsize{12}{20}\bfseries}{\normalfont\thesubsection}{.5em}{}


% Personalizza la formattazione della subsubsection
\titleformat{\subsubsection}[block]{\fontsize{10}{20}\bfseries}{\normalfont\thesubsubsection}{.5em}{}

% Maketitle customization
\renewcommand{\maketitle}{
\begin{center}
{\sffamily
{\fontsize{20}{20}\selectfont\MakeUppercase\thetitle}}

\vspace{0.2in}

{\large\scshape\sffamily\theauthor}
\end{center}
}

%Evaluate symbol
\DeclareMathOperator{\di}{d\!}
\newcommand*\Eval[3]{\left.#1\right\rvert_{#2}^{#3}}

%%%%%%% Numero delle equazioni in formato a.b
\numberwithin{equation}{subsection}
%%%%%

%%%%%%%%%% Personalizzazione numeri lista
\renewcommand{\theenumi}{(\arabic{enumi})}

%%%% Table of contents

\usepackage[titles]{tocloft}

\renewcommand{\cftdot}{}
\usepackage{titletoc}
%\setcounter{tocdepth}{2}

%%%%%%%%%%%%%%%% Toc style

% Personalizzazione scritta indice


% Font
\usepackage[osf]{newpxtext}
\usepackage{sansiwona}
\usepackage{cmupint}
\usepackage{newtxmath}



\begin{document}
\maketitle
\tableofcontents 
\newpage
\section{Introduzione -- Derivata e soluzioni deboli, spazi $\mathcal{E}, \ \mathcal{D}$}
\begin{definizione}{Definizione di derivata debole}{dd}
	Sia $f\in L^1_\text{loc}(\Omega )$ e $\alpha  \in \mathbb{N}_0^n$; allora si dice che esiste $\partial ^\alpha  f \in L^1_\text{loc}(\Omega ) $ \textit{in senso debole} se:
	\begin{equation}
		\exists g \in L^1_\text{loc}(\Omega ) : \int_{\Omega } g \varphi \ dx = (-1)^{\left\lvert \alpha  \right\rvert } \int_{\Omega } f \partial ^\alpha  \varphi  \ dx, \ \forall \varphi \in C_0^\infty (\Omega )
	\end{equation}
	Se questo \`e vero, allora $\partial ^\alpha  f =g$ e si dice che $g$ \`e la \textbf{derivata debole} di ordine $\alpha $ di $f$.
\end{definizione}

\begin{teorema}{Teorema di Riemann-Lebesgue}{rlth}

	Sia $g \in L^1_\text{loc}(\Omega )$ e $\int_{\Omega } g \varphi  \ dx = 0, \ \forall \varphi  \in C_0^\infty(\Omega )$; allora $g= 0$ quasi ovunque in $\Omega $.
\end{teorema}

\noindent Si applica il concetto di derivata debole alle edp; sia $P(x,\partial )$ un operatore differenziale lineare di ordine $m\in \mathbb{N}$ del tipo:
\begin{equation}
	P(x,\partial ) : = \sum_{\left\lvert \alpha  \right\rvert \le m}^{} a_\alpha  (x) \partial ^\alpha , \ a_\alpha  \in C^{\left\lvert \alpha  \right\rvert } (\Omega ) , \ \alpha \in \mathbb{N}_0^n, \ \left\lvert \alpha  \right\rvert \le m
\end{equation}
Si prende $u \in C^m (\Omega )$ come la soluzione classica di $P(x,\partial ) u = f, \ f \in C^0 (\Omega )$; integrando per parti per ogni $\varphi \in C_0^\infty(\Omega )$:
\begin{equation*}
	\int_{\Omega } f \varphi  \ dx = \int_{\Omega } \varphi \left[ \sum_{\left\lvert \alpha  \right\rvert \le m}^{} a_\alpha (\partial ^\alpha u)\right]  \ dx = \int_{\Omega } \left[ \sum_{\left\lvert \alpha  \right\rvert \le m}^{} (-1)^{\left\lvert \alpha  \right\rvert } \partial ^\alpha  (a_\alpha  \varphi ) \right] u \ dx
\end{equation*}
da cui si definisce:
\begin{equation}
	P^\top (x,\partial ) \varphi  : = \sum_{\left\lvert \alpha  \right\rvert \le m}^{} (-1)^{\left\lvert \alpha  \right\rvert } \partial ^\alpha  (a_\alpha  \varphi )
\end{equation}
e si chiama \textbf{operatore trasposto} di $P(x,\partial )$. Si arriva alla seguente definizione.
\begin{definizione}
	{Soluzione debole}{sdb}
	Siano $u,f \in L^1_\text{loc}(\Omega )$ e sia $P(x,\partial )$ come definito sopra; si dice che $P(x,\partial ) u = f$ \`e valida debolmente se:
	\begin{equation}
		\int_{\Omega } f\varphi  \ dx = \int_{\Omega } \left[ P^\top(x,\partial ) \varphi  \right] u \ dx, \ \forall \varphi \in C_0^\infty (\Omega )
	\end{equation}
\end{definizione}
Il problema con la definizione \ref{def:dd} \`e che mentre il lato di destra \`e sempre verificato, quello di sinistra potrebbe perdere senso perch\'e non \`e detto che una funzione $u \in L^1_\text{loc}$ sia derivabile; a questo proposito, ci si concentra sul lato di destra e, data $f\in L^1_\text{loc}(\Omega )$, si considera la mappa:
\begin{equation}\label{distschel}
	g_\alpha  : C_0^\infty(\Omega )\to \mathbb{C}, \ g_\alpha (\varphi ) := (-1)^{\left\lvert \alpha  \right\rvert } \int_{\Omega } f (\partial ^\alpha  \varphi ) \ dx , \ \forall \varphi  \in C_0^\infty(\Omega )
\end{equation}
Questo funzionale $g_\alpha $ \`e lineare e $\forall \varphi \in C_0^\infty(\Omega )$ se ne pu\`o stimare la norma:
\begin{equation}
	\left\lvert g_\alpha (\varphi ) \right\rvert \le \int_{K } \left\lvert f \right\rvert \left\lvert \partial ^\alpha \varphi  \right\rvert dx \le \sup_{x \in K} \left\lvert \partial ^\alpha \varphi (x) \right\rvert  \int_{K } \left\lvert f \right\rvert \ dx
\end{equation}
dove viene fissato $K$ compatto e tale che $K \subset \Omega, \ \operatorname{supp}\varphi \subseteq K  $ cos\`i da avere l'integrale di $f$ indipendente dal supporto di $\varphi $ e, quindi, costante. Questo fa pensare di dotare $C^\infty(\Omega )$ di una topologia\footnote{L'obiettivo per $C^\infty$ e poi per $C_0^\infty$ \`e quello di definire delle topologie rispetto alle quali gli operatori differenziali con i quali si avr\`a a che fare risulteranno essere continui.} $\tau $ si definisce $\mathcal{E}(\Omega ):= \left(C^\infty(\Omega ) , \tau \right) $. Allora:
\begin{itemize}
	\item una successione $\left\{ \varphi _j \right\} _{j\in \mathbb{N}} \subset C^\infty(\Omega )$ converge in $\mathcal{E}(\Omega )$ a $\varphi  \in C^\infty(\Omega )$ se e soltanto se:
		\begin{equation}
			\forall K \subset \Omega \text{ compatto }, \forall \alpha \in \mathbb{N}^n_0, \ \lim_{j \to \infty} \sup_{x \in K} \left\lvert \partial ^\alpha (\varphi _j -\varphi )(x) \right\rvert =0
		\end{equation}
	\item $\mathcal{E}(\Omega )$ \`e un localmente convesso, metrizzabile e completo spazio vettoriale topologico su $\mathbb{C}$.
\end{itemize}
Si vuole arrivare allo stesso risultato per $C_0^\infty(\Omega )$; si potrebbe pensare di usare la topologia indotta da $\mathcal{E}(\Omega $ in $C_0^\infty(\Omega )$, ma questa avrebbe il difetto che non assicura la compattezza di funzioni risultanti da serie di funzioni compatte.

La definizione di $\mathcal{D}(\Omega )$ si ottiene come segue: si prende $\mathcal{D}_K(\Omega )$ come lo spazio di funzioni $C^\infty(\Omega )$ a supporto in $K$ con la topologia indotta da $\mathcal{E}(\Omega )$; si considera in $C_0^\infty(\Omega )$ la topologia indotta limite dagli spazi $\left\{ \mathcal{D}_K(\Omega ) \right\}, \ K\subset \Omega    $ compatto e il risultante spazio vettoriale topologico \`e proprio $\mathcal{D}(\Omega )$. Presenta le seguenti caratteristiche:
\begin{itemize}
	\item $\mathcal{D}(\Omega )$ \`e uno spazio vettoriale topologico, localmente convesso e completo, su $\mathbb{C}$;
	\item una successione $\left\{ \varphi _j \right\} _{j\in \mathbb{N}} \subset C_0^\infty(\Omega )$ converge in $\mathcal{D}(\Omega )$ a $\varphi \in C_0^{\infty} (\Omega )$ se e soltanto se sono verificate le due seguenti condizioni:
		\begin{itemize}
			\item esiste $K \subset \Omega $ compatto tale che $\operatorname{supp} \varphi _j \subseteq K, \ \forall j \in \mathbb{N}$ e $\operatorname{supp} \subseteq K$;
			\item $\forall \alpha \in \mathbb{N}_0^n$ si ha:
				\begin{equation}
				\lim_{j \to \infty} \sup_{x \in K} \left\lvert \partial ^\alpha (\varphi _j - \varphi ) (x)  \right\rvert =0 
				\end{equation}
		\end{itemize}
\end{itemize}
La topologia definita su $\mathcal{D}(\Omega )$ \`e talmente pi\`u raffinata di quella su $\mathcal{E}(\Omega )$ che se anche la funzione limite di una successione di funzioni lisce a supporto compatto appartiene a $\mathcal{D}(\Omega )$, pu\`o ancora non avere limite nel senso di $\mathcal{D}(\Omega )$.

\newpage
\section{Distribuzioni}
\subsection{Caratterizzazione}
Riprendendo la mappa definita in $\ref{distschel}$, si d\`a la seguente definizione.
\begin{definizione}
	{Definizione di distribuzione}{defistr}
	La mappa $u:\mathcal{D}(\Omega ) \to \mathbb{C}$ \`e chiamata distribuzione su $\Omega $ se \`e lineare e continua.
\end{definizione}
Ogni funzionale lineare e continuo \`e, in generale, sequenzialmente continuo, ma non vale il contrario. Tuttavia, per un funzionale lineare su $\mathcal{D}(\Omega )$, continuit\`a e continuit\`a sequenziale si equivalgono, quindi: \textit{sia $u:\mathcal{D}(\Omega ) \to \mathbb{C}$ lineare; allora $u$ \`e una distribuzione se e soltanto se $\forall \left\{ \varphi _j \right\} _{j\in \mathbb{N}} \subset C_0^{\infty} (\Omega )$ t.c. $\varphi _j\to \varphi \in C_0^{\infty} (\Omega )$ in $\mathcal{D}(\Omega )$, vale $\lim_{j \to \infty} \langle u,\varphi _j \rangle=\left\langle u,\varphi  \right\rangle$}.
\begin{prop}
	{}{distk}
	Sia $u:\mathcal{D}(\Omega ) \to \mathbb{C}$ lineare; allora $u$ \`e una distribuzione se e soltanto se $\forall K \subset \Omega $ compatto, esiste $k \in \mathbb{N}_0$ e $C \in (0,\infty)$ tale che:
	\begin{equation}
		\left\lvert \left\langle u,\varphi  \right\rangle \right\rvert \le C \sup_{\substack{x \in  K \\ \left\lvert \alpha   \right\rvert \le  k}} \left\lvert \partial ^\alpha  \varphi (x) \right\rvert , \ \forall \varphi  \in  C_0^\infty(\Omega ), \ \operatorname{supp} \varphi \subseteq K
	\end{equation}
	\begin{proof}
		Si dimostra $(\Leftarrow)$, assumendo che $\forall K \subset \Omega $ compatto, esistano $k$ e $C$ che soddisfano la relazione riportata sopra. Per mostrare che $u$ \`e una distribuzione, si prende $\varphi _j \to 0$ in $\mathcal{D}(\Omega )$; allora esiste un compatto $K \subseteq \Omega $ t.c. $\operatorname{supp} (\varphi _j) \subseteq K, \ \forall j \in \mathbb{N}$ e $\partial ^\alpha \varphi _j \to 0 $ uniformemente in $K, \ \forall \alpha  \in \mathbb{N}_0^n$ per definizione di convergenza in $\mathcal{D}$.

		Per l'assunzione di partenza, ci sono $k,C$ per questo $K$ che soddisfano 
		\begin{equation}
			\left\lvert \left\langle u,\varphi _j \right\rangle \right\rvert \le C \sup_{\substack{x \in K \\ \left\lvert \alpha  \right\rvert \le k}} \left\lvert \partial ^\alpha  \varphi _j(x) \right\rvert \to 0
		\end{equation}
	\end{proof}
\end{prop}






\end{document}
