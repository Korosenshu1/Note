\documentclass[11pt, a4paper]{scrartcl}
% Packages
%\usepackage{stix}
\usepackage[margin=1in]{geometry}
\usepackage{index}
\makeindex
\usepackage[utf8]{inputenc}
\usepackage[T1]{fontenc}
\usepackage{varwidth}
\usepackage{amsbsy} % Bold math symbols
\usepackage{amsmath, amssymb}
\usepackage{esint}
\usepackage{titlesec}
\usepackage{xcolor}
\usepackage{titling}
\usepackage{braket}
\usepackage{tensor}
\usepackage[linktocpage]{hyperref}
\usepackage{pgfplots}
\usepackage{multicol}
\setlength{\columnsep}{2em}
\usepackage{caption}
\usepackage{amsthm}
\usepackage{import}
\usepackage{cancel}
\usepackage{caption}
\usepackage{tcolorbox}
\usepackage{nicematrix}
\usepackage{mathrsfs}
\usepackage{mathtools}
\usepackage{enumerate}
\usepackage{graphicx}
\usepackage{lipsum}
\usepackage[italian]{babel}
% To reset footnote numbering each page
\usepackage[perpage]{footmisc}
\usepackage{setspace}
\setstretch{1.1}

%Captions
\captionsetup[figure]{font=footnotesize,labelfont=footnotesize}
\captionsetup[table]{font=footnotesize,labelfont=footnotesize}
%Titlesec
\titleformat{\section}
{\fontsize{15}{20}\sffamily\scshape}
{\normalfont\color{gray}{\fontsize{20}{20}\selectfont\thesection}}
{0.7em}
{}
\hypersetup{colorlinks,breaklinks, linkcolor=[RGB]{74, 122, 164}}

\newcommand\vertarrowbox[3][6ex]{%
  \begin{array}[t]{@{}c@{}} #2 \
  \left\uparrow\vcenter{\hrule height #1}\right.\kern-\nulldelimiterspace\
  \makebox[0pt]{\scriptsize#3}
  \end{array}%
}
\definecolor{asdf}{HTML}{4a7aa4}
% Personalizza la formattazione della subsection
\titleformat{\subsection}[block]{\fontsize{12}{20}\bfseries}{\normalfont\thesubsection}{.5em}{}


% Personalizza la formattazione della subsubsection
\titleformat{\subsubsection}[block]{\fontsize{10}{20}\bfseries}{\normalfont\thesubsubsection}{.5em}{}

% Maketitle customization
\renewcommand{\maketitle}{
\begin{center}
{\sffamily
{\fontsize{20}{20}\selectfont\MakeUppercase\thetitle}}

\vspace{0.2in}

{\large\scshape\sffamily\theauthor}
\end{center}
}

% Titles 
\title{
Divergenza delle serie
perturbative in
meccanica quantistica\\}
\author{Manuel Deodato}
\date{}



%Evaluate symbol
\DeclareMathOperator{\di}{d\!}
\newcommand*\Eval[3]{\left.#1\right\rvert_{#2}^{#3}}

%%%%%%% Numero delle equazioni in formato a.b
\numberwithin{equation}{subsection}
%%%%%

%%%%%%%%%% Personalizzazione numeri lista
\renewcommand{\theenumi}{(\arabic{enumi})}

%%%%%%%%%% Medie con integrali multipli
\def\Yint#1{\mathchoice
    {\YYint\displaystyle\textstyle{#1}}%
    {\YYint\textstyle\scriptstyle{#1}}%
    {\YYint\scriptstyle\scriptscriptstyle{#1}}%
    {\YYint\scriptscriptstyle\scriptscriptstyle{#1}}%
      \!\iint}
\def\YYint#1#2#3{{\setbox0=\hbox{$#1{#2#3}{\iint}$}
    \vcenter{\hbox{$#2#3$}}\kern-.51\wd0}}
\def\longdash{{-}\mkern-3.5mu{-}} 
   % consider using "\mkern-7.5mu" if esint package is loaded
\def\tiltlongdash{\rotatebox[origin=c]{15}{$\longdash$}}
\def\fiint{\Yint\tiltlongdash}

\def\Zint#1{\mathchoice
    {\YYint\displaystyle\textstyle{#1}}%
    {\YYint\textstyle\scriptstyle{#1}}%
    {\YYint\scriptstyle\scriptscriptstyle{#1}}%
    {\YYint\scriptscriptstyle\scriptscriptstyle{#1}}%
      \!\iiint}
      \def\tilongdash{\mkern6mu{-}\mkern-4mu{-}\mkern-5mu{-}} 
   % consider using "\mkern-7.5mu" if esint package is loaded
\def\titiltlongdash{\rotatebox[origin=c]{15}{$\tilongdash$}}
\def\fiiint{\Zint\titiltlongdash}


%%%% Table of contents

\usepackage[titles]{tocloft}

\renewcommand{\cftdot}{}
\usepackage{titletoc}
%\setcounter{tocdepth}{2}

%%%%%%%%%%%%%%%% Toc style

% Personalizzazione scritta indice


% Font
\usepackage[osf]{newpxtext}

\usepackage{sansiwona}


% Ambienti
\newtheoremstyle{style1}% name of the style to be used
{15pt}% measure of space to leave above the theorem. E.g.: 3pt
{15pt}% measure of space to leave below the theorem. E.g.: 3pt
{\normalfont}% name of font to use in the body of the theorem
{}% measure of space to indent
{\sffamily\scshape\bfseries}% name of head font
{}% punctuation between head and body
{ }% space after theorem head; " " = normal interword space
{\thmname{#1}\thmnumber{ #2}{\thmnote{~--- #3}}.\newline}




\theoremstyle{style1}
\newtheorem{teorema}{Teorema}[section]
\newtheorem{corollario}{Corollario}[teorema]
\newtheorem{lemma}{Lemma}[teorema]
\newtheorem{definizione}{Definizione}[section]
\newtheorem{osservazione}{Osservazione}[section]
\newtheorem{notazione}{Notazione}[section]
\newtheorem{esempio}{Esempio}[section]
\newtheorem{esercizio}{Esercizio}[section]

\renewcommand\qedsymbol{$\blacksquare$}

\newenvironment{svolgimento}{\renewcommand\qedsymbol{$\spadesuit$}\begin{proof}[Svolgimento]}{\end{proof}}

%% Generic box
\newtcolorbox{eqbox}[1][]
{
colback=gray!10,
arc=0pt,
boxrule=0pt,
title=#1
}

 \newenvironment{boxenv}[1][]{
    \begin{eqbox}[#1]
    }{
   \end{eqbox}
}








%%%%%%%%%%%%%%%%%%%%%%%%%%%%%%%%%%%%%%%%%%%%%%%%%%%%%%%%%%%%%%%%%%%%%%%%

\begin{document}
\maketitle
\vspace{.5cm}
\begin{center}
	\Large\textit{LONG ABSTRACT} 
\end{center}
L'obiettivo della tesi \`e quello di investigare tali evenienze per capire le motivazioni che dietro tali comportamenti.
A scopo di esempio, si utilizzer\`a un oscillatore armonico soggetto a perturbazione quartica; in questo caso particolare, si mostrer\`a come il problema risiede nella dipendenza dal parametro infinitesimo di questa perturbazione.

Studiando l'esempio proposto, si pu\`o arrivare allo sviluppo perturbativo per lo stato fondamentale notando che ogni stato eccitato per l'oscillatore armonico \`e scrivibile come un polinomio per una gaussiana. 
Tramite argomenti di parit\`a, si conclude che la perturbazione, essendo pari, non pu\`o mischiare stati con parit\`a diversa, quindi lo stato fondamentale in particolare verr\`a collegato solo a stati pari.
Inoltre, scrivendo la potenza quartica del potenziale in termini degli operatori di salita e discesa, si nota che uno stato si collega con un altro il cui livello energetico dista massimo quattro livelli dal primo.
Usando queste regole di selezione per caratterizzare il polinomio e inserendo tutto nell'equazione di partenza, si ottiene una relazione per gli ordini perturbativi dell'energia; tramite calcolo numerico, si trova che il loro valore \`e legato al fattoriale dell'ordine di sviluppo a cui appartengono.

La divergenza di questo sviluppo \`e inevitabilmente sinonimo di una mancata convergenza nell'intorno dello zero del parametro infinitesimo della perturbazione: di fatto, il potenziale quartico in cui si immerge la particella, nel caso in cui il relativo parametro sia negativo, risulta non pi\`u limitato inferiormente, il che permetterebbe alla particella di liberarsi dal potenziale per effetto tunnel.
In questo caso, l'effetto tunnel -- il quale non \`e descrivibile perturbativamente -- \`e il motivo per cui lo sviluppo in serie delle energie attorno allo zero fallisce nel convergere.

\`E possibile caratterizzare ulteriormente questa divergenza tramite lo scaling di Symanzik: tramite una trasformazione unitaria, si trova una relazione che lega l'energia in funzione del parametro perturbativo con un'altra funzione con coefficiente la radice cubica di tale parametro. 
Questo permette di evidenziare ulteriormente la causa della divergenza dello sviluppo in serie attorno allo zero.

La divergenza \`e legata all'analiticit\`a del dominio, il quale \`e formato dagli stati per cui l'Hamiltoniano risulta Hermitiano.
Una condizione \`e che il potenziale ammetta valore medio sugli stati, cio\`e che l'integrale associato sia convergente. 
Prendendo una funzione d'onda che va come l'inverso di una potenza quadratica nella posizione, questa \`e valida dal punto di vista dell'Hamiltoniano imperturbato, ma non della perturbazione quartica perch\'e il prodotto tra il modulo quadro della funzione d'onda e il potenziale quartico si semplificherebbero, facendo divergere l'integrale.
Questa divergenza del valore medio della perturbazione sugli stati ammissibili dal punto di vista dell'Hamiltoniano imperturbato \`e sinonimo del fatto che, indipendentemente da quanto piccolo si prende il parametro associato alla perturbazione stessa, gli effetti che questa ha sul sistema non \`e mai tale.

Questo si generalizza tramite il teorema di Kato-Rellich, il quale stabilisce due condizioni che, se rispettate, assicurano tale sviluppo, e usando il teorema di Kato, che fornisce una condizione sufficiente per rientrare nelle condizioni del primo.

\end{document}
