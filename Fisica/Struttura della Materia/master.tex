\documentclass[10pt, a4paper]{scrartcl}
% Packages
\usepackage[margin=1.5in]{geometry}
\usepackage{index}
\makeindex
\usepackage[utf8]{inputenc}
\usepackage[T1]{fontenc}
\usepackage{varwidth}
\usepackage{amsmath, amssymb}
\usepackage{esint}
\usepackage{titlesec}
\usepackage{xcolor}
\usepackage{titling}
\usepackage{braket}
\usepackage{tensor}
\usepackage[linktocpage]{hyperref}
\usepackage{pgfplots}
\usepackage{multicol}
\setlength{\columnsep}{2em}
\usepackage{caption}
\usepackage{amsthm}
\usepackage{import}
\usepackage{cancel}
\usepackage{caption}
\usepackage{tcolorbox}
\usepackage{nicematrix}
\usepackage{mathrsfs}
\usepackage{mathtools}
\usepackage{enumerate}
\usepackage{graphicx}
\usepackage{lipsum}
\usepackage[italian]{babel}
% To reset footnote numbering each page
\usepackage[perpage]{footmisc}
\usepackage{cmupint}
\usepackage{newtxmath}
%Captions
\captionsetup[figure]{font=footnotesize,labelfont=footnotesize}
\captionsetup[table]{font=footnotesize,labelfont=footnotesize}
%Titlesec
\titleformat{\section}
{\fontsize{15}{20}\sffamily\scshape}
{\normalfont\color{gray}{\fontsize{20}{20}\selectfont\thesection}}
{0.7em}
{}
\hypersetup{colorlinks,breaklinks, linkcolor=[RGB]{74, 122, 164}}

\newcommand\vertarrowbox[3][6ex]{%
  \begin{array}[t]{@{}c@{}} #2 \
  \left\uparrow\vcenter{\hrule height #1}\right.\kern-\nulldelimiterspace\
  \makebox[0pt]{\scriptsize#3}
  \end{array}%
}
\definecolor{asdf}{HTML}{4a7aa4}
% Personalizza la formattazione della subsection
\titleformat{\subsection}[block]{\fontsize{12}{20}\bfseries}{\normalfont\thesubsection}{.5em}{}


% Personalizza la formattazione della subsubsection
\titleformat{\subsubsection}[block]{\fontsize{10}{20}\bfseries}{\normalfont\thesubsubsection}{.5em}{}

% Maketitle customization
\renewcommand{\maketitle}{
\begin{center}
{\sffamily
{\fontsize{20}{20}\selectfont\MakeUppercase\thetitle}}

\vspace{0.2in}

{\large\scshape\sffamily\theauthor}
\end{center}
}

% Titles 
\title{Note di\\\vspace{.2cm} Struttura della Materia}
\author{Manuel Deodato}
\date{}



%Evaluate symbol
\DeclareMathOperator{\di}{d\!}
\newcommand*\Eval[3]{\left.#1\right\rvert_{#2}^{#3}}

%%%%%%% Numero delle equazioni in formato a.b
\numberwithin{equation}{subsection}
%%%%%

%%%%%%%%%% Personalizzazione numeri lista
\renewcommand{\theenumi}{(\arabic{enumi})}

%%%%%%%%%% Medie con integrali multipli
\def\Yint#1{\mathchoice
    {\YYint\displaystyle\textstyle{#1}}%
    {\YYint\textstyle\scriptstyle{#1}}%
    {\YYint\scriptstyle\scriptscriptstyle{#1}}%
    {\YYint\scriptscriptstyle\scriptscriptstyle{#1}}%
      \!\iint}
\def\YYint#1#2#3{{\setbox0=\hbox{$#1{#2#3}{\iint}$}
    \vcenter{\hbox{$#2#3$}}\kern-.51\wd0}}
\def\longdash{{-}\mkern-3.5mu{-}} 
   % consider using "\mkern-7.5mu" if esint package is loaded
\def\tiltlongdash{\rotatebox[origin=c]{15}{$\longdash$}}
\def\fiint{\Yint\tiltlongdash}

\def\Zint#1{\mathchoice
    {\YYint\displaystyle\textstyle{#1}}%
    {\YYint\textstyle\scriptstyle{#1}}%
    {\YYint\scriptstyle\scriptscriptstyle{#1}}%
    {\YYint\scriptscriptstyle\scriptscriptstyle{#1}}%
      \!\iiint}
      \def\tilongdash{\mkern6mu{-}\mkern-4mu{-}\mkern-5mu{-}} 
   % consider using "\mkern-7.5mu" if esint package is loaded
\def\titiltlongdash{\rotatebox[origin=c]{15}{$\tilongdash$}}
\def\fiiint{\Zint\titiltlongdash}


%%%% Table of contents

\usepackage[titles]{tocloft}

\renewcommand{\cftdot}{}
\usepackage{titletoc}
%\setcounter{tocdepth}{2}

%%%%%%%%%%%%%%%% Toc style

% Personalizzazione scritta indice


% Font
\usepackage[osf]{newpxtext}

\usepackage{sansiwona}


% Ambienti
\newtheoremstyle{style1}% name of the style to be used
{15pt}% measure of space to leave above the theorem. E.g.: 3pt
{15pt}% measure of space to leave below the theorem. E.g.: 3pt
{\normalfont}% name of font to use in the body of the theorem
{}% measure of space to indent
{\sffamily\scshape\bfseries}% name of head font
{}% punctuation between head and body
{ }% space after theorem head; " " = normal interword space
{\thmname{#1}\thmnumber{ #2}{\thmnote{~--- #3}}.}




\theoremstyle{style1}
\newtheorem{teorema}{Teorema}[section]
\newtheorem{corollario}{Corollario}[teorema]
\newtheorem{lemma}{Lemma}[teorema]
\newtheorem{definizione}{Definizione}[section]
\newtheorem{osservazione}{Osservazione}[section]
\newtheorem{notazione}{Notazione}[section]
\newtheorem{esempio}{Esempio}[section]
\newtheorem{esercizio}{Esercizio}[section]

\renewcommand\qedsymbol{$\blacksquare$}

\newenvironment{svolgimento}{\renewcommand\qedsymbol{$\spadesuit$}\begin{proof}[Svolgimento]}{\end{proof}}

%% Generic box
\newtcolorbox{eqbox}[1][]
{
colback=gray!10,
arc=0pt,
boxrule=0pt,
title=#1
}

 \newenvironment{boxenv}[1][]{
    \begin{eqbox}[#1]
    }{
   \end{eqbox}
}







%%%%%%%%%%%%%%%%%%%%%%%%%%%%%%%%%%%%%%%%%%%%%%%%%%%%%%%%%%%%%%%%%%%%%%%%

\begin{document}
\maketitle
\newpage
\tableofcontents 
\newpage
\section{Nozioni di meccanica statistica e termodinamica}

\subsection{Gas di particelle}


Si considera gas di particelle non interagenti e puntiformi. Ciascuna particella soddisfa $\hat{H}\psi (\mathbf{r} ) = E  \psi (\mathbf{r} )$ con $\hat{H} = \frac{\hat{\mathbf{p} }}{2m}$ e $E = \frac{\hbar ^2}{2m} q ^2 $, quindi la soluzione generale \`e:
\begin{equation}
	\psi (\mathbf{r} ) = e^{ i \mathbf{q} \cdot \mathbf{r} } 
\end{equation}
Imponendo condizione di periodicit\`a al bordo della scatola $\Rightarrow q_i = \frac{2\pi}{L}l_i, \ l_i = 0, \pm 1,\pm 2,\ldots$. 

Si assumer\`a che particelle interagiscano abbastanza poco da rendere valida questa trattazione, e abbastanza tanto da permettere transizioni di fase.
\subsection{Principi della termodinamica}
\begin{enumerate}[(a).]
	\item \textbf{Primo principio}: per un sistema chiuso (niente scambio di particelle) e isolato, vi \`e conservazione dell'energia interna:
		\begin{equation}
			dE = \delta  Q + \delta L
		\end{equation}
	\item \textbf{Secondo principio}: l'entropia, data da $S = \kappa _B \log \Gamma$\footnote{Questa espressione \`e il caso limite della pi\`u generale $S = \kappa _B \sum_{i}^{} p_i \log p_i$ che si ha quando il sistema non \`e all'equilibrio, cio\`e quando i microstati non sono popolati uniformemente.} (con $\Gamma$ numero di microstati del sistema all'equilibrio), per un sistema isolato, soddisfa $\frac{d S}{d t} \ge 0$. L'uguaglianza vale quando \`e raggiunto l'equilibrio.
	\item \textbf{Terzo principio}: l'entropia tende a $0$ per sistemi perfettamente ordinati, cio\`e sistemi in cui tutte le particelle popolano un solo microstato $\Rightarrow S = \kappa _B \log 1 = 0$. Sistemi perfettamente ordinati sono cristalli perfetti a temperatura nulla; non tutti i materiali a $T=0$ risultano perfettamente ordinati e alcuni presentano entropia residua.
\end{enumerate}
\subsection{Potenziali termodinamici}
A seconda del caso, si usano diverse riscritture dell'energia.
\begin{itemize}
	\item \textbf{Energia libera di Helmholtz}: $F = E - TS \Rightarrow dF = -SdT - PdV$. La sua variazione a temperatura costante restituisce lavoro compiuto sul sistema: $\delta F |_T = - P \delta V |_T = \delta L$.
	\item \textbf{Energia libera di Gibbs}: $\Phi = E - TS + PV = F + PV \Rightarrow d\Phi=  VdP - SdT$. \`E adatta a descrivere transizioni di fase.
	\item \textbf{Entalpia}: $W = E + PV \Rightarrow dW=  TdS + VdP$. La sua variazione a pressione costante \`e il calore scambiato dal sistema: $\delta W|_P = T \delta S|_P = \delta Q$.
\end{itemize}
Se \`e possibile scambio di particelle, la dipendenza da $N$ nei potenziali si aggiunge con:
\begin{equation}
	\mu = \left(\frac{\partial E}{\partial N} \right) _{SV} = \left(\frac{\partial F}{\partial N} \right) _{TV} = \left(\frac{\partial \Phi}{\partial N} \right) _{SP} 
\end{equation}
$\mu $ \`e esso stesso un potenziale: $d\mu = - S / N dT + V / N dP = - s dT + vdP$\footnote{Aggiungendo particelle ferme ad un sistema, \`e ragionevole avere $\mu < 0$, visto che l'energia media diminuirebbe con l'aumentare di $N$.}. Un altro potenziale utile \`e il \textbf{potenziale di Landau}: $\Omega = F - \mu N\Rightarrow d\Omega = -SdT - PdV - N d\mu $.



\subsection{Calori specifici e compressibilit\`a}

Calori specifici a volume e pressione costante:
\begin{equation}
	\begin{split}
		&c_V = \left(\frac{\partial E}{\partial T} \right) _V = T \left(\frac{\partial S}{\partial T} \right) _V = -T \left(\frac{\partial ^2 F}{\partial T^2} \right) _V \\
		&c_P = \left(\frac{\partial E}{\partial T} \right) _P = T \left(\frac{\partial S}{\partial T} \right) _P = -T \left(\frac{\partial ^2 \Phi}{\partial T^2} \right) _P
	\end{split}
\end{equation}
Vale
\begin{equation}
	c_P \ge  c_V
\end{equation}
Compressibilit\`a per trasformazioni isoterma e adiabatica:
\begin{equation}
	k_T = -\frac{1}{V} \left(\frac{\partial V}{\partial P} \right) _T=-\frac{1}{V}\left(\frac{\partial ^2\Phi}{\partial P^2} \right) _T \hspace{.1cm} ; \hspace{.2cm} k_S = -\frac{1}{V } \left(\frac{\partial V}{\partial P} \right) _S = \left[ V \left(\frac{\partial ^2E}{\partial V^2} \right) _S \right] ^{-1} 
\end{equation}
\subsection{Diagrammi di fase}
Grafico che mostra stato fisico di una sostanza in funzione, solitamente, di temperatura e pressione. Assumendo di avere un sistema con due stati coesistenti $\Rightarrow N_1+N_2 = \text{cost.}\Rightarrow \delta N_1 = - \delta N_2$, all'equilibrio:
\[
\frac{\partial F}{\partial N_1} = \frac{\partial }{\partial N_1} (F_1+F_2) = \frac{\partial F_1}{\partial N_1} -\frac{\partial F_2}{\partial N_2} = \mu_1-\mu_2 = 0
\] 
da cui si ottiene relazione $\mu _1 (P,T) = \mu _2(P,T)$ che permette di tracciare grafico $P=f(T)$. Lo stesso si pu\`o fare per tre stati coesistenti, individuando \textit{punto triplo}. 

Da $d\mu_1 = d \mu _2$, si ha $-s_1 dT + v_1 dP = -s_2 dT - v_2dP$, quindi:
\begin{equation}
	\frac{d P}{d T} = \frac{s_2-s_1}{v_2-v_1}
\end{equation}
\subsection{Modello per sistemi statistici}

Si tratteranno i sistemi dividendo l'Universo in sistema in esame ($E,S,T$) $+$ parte complementare, chiamata \textbf{bagno termico} ($E',S',T$). Quest'ultimo sar\`a assunto essere \textit{sempre all'equilibrio e alla stessa temperatura del sistema}. 

La variazione di energia del bagno termico dipende solo da variazione dell'entropia $\Rightarrow \delta E' = T \delta S'$; inoltre essendo l'Universo sempre isolato, la sua variazione di energia \`e nulla $\Rightarrow \delta E + \delta E' =0$. 

Unendo le due, si ha $\delta S' = -\delta E / T$; per il secondo principio, $\delta S + \delta S' \ge 0 \Rightarrow T\delta S - \delta E \ge 0 \Rightarrow \delta (E-TS) \le 0$, da cui si deduce che un sistema \textit{a temperatura fissata} \`e all'equilibrio quando $F = E- TS $ \`e al minimo.

Consentendo scambio di particelle, vale lo stesso principio con $\delta \Omega \le 0$, quindi $\Omega = F - \mu N $ minimo.
\subsubsection{Sistema in bagno termico}

Si indica con $\mathscr{S}$ il sistema immerso in bagno termico $\mathscr{S}'$ e con $\mathscr{S}_0$ l'Universo. Questi hanno rispettivamente dipendenza dalle variabili $(V,N,E,S), \ (V',N',E',S') , \ (V_0,N_0,E_0,S_0)$.

$\mathscr{S}$ si trova in stato quantistico generico indicato tramite serie di numeri quantici $\alpha $; si assume che \textit{il volume sia fissato} e si richiede che: $E_\alpha \ll E_0$ e $N_\alpha  \ll N_0$; in questo modo \textit{temperatura e potenziale chimico del bagno termico sono costanti}.

I microstati dell'Universo sempre equiprobabili perch\'e \`e sempre all'equilibrio $\Rightarrow w_\text{eq} = 1 / \Gamma_0$, con $\Gamma_0$ numero di microstati. La probabilit\`a di avere uno stato $\alpha $ per il sistema, allora \`e $w_\alpha  = \Gamma'_\alpha  / \Gamma_0$, dove $\Gamma'_\alpha $ \`e il numero di microstati in cui $\mathscr{S}$ \`e in $\alpha $ e $\mathscr{S}' $ \`e in uno stato generico.

L'entropia di $\mathscr{S}'$ \`e:
\begin{equation}\label{spa}
	S'_\alpha = \kappa _B \log \Gamma'_\alpha  = S'(E_0 - E_\alpha , N_0 - N_\alpha )
\end{equation}
Inoltre:
\begin{equation}
	S_0 - S'_\alpha  = \kappa _B \log \Gamma_0 - \kappa _B \log \Gamma'_\alpha = - \kappa _B \log \frac{\Gamma'_\alpha }{\Gamma_0} = -\kappa _B \log w_\alpha 
\end{equation}
quindi
\begin{boxenv}[]
\begin{equation}
	w_\alpha  = \exp \left(-\frac{S_0-S'_\alpha }{\kappa _B}\right) \equiv A e^{S'_\alpha  / \kappa _B} 
\end{equation}
\end{boxenv}
\noindent In questo modo, si pu\`o calcolare valore medio dell'entropia per $\mathscr{S}$ (in genere $\alpha $ non \`e uno stato di equilibrio per $\mathscr{S}$):
\begin{equation}
	\langle S  \rangle \equiv \langle S_0 - S'_\alpha  \rangle = - \kappa _B \sum_{\alpha }^{} w_\alpha  \log w_\alpha 
\end{equation}
\subsubsection{Funzione di granpartizione}
Sviluppando in serie eq. \ref{spa}, si ha:
\begin{equation}
	S'_\alpha  \simeq S'(E_0,N_0) - \left(\frac{\partial S'}{\partial E'} \right) _{N'} E_\alpha - \left(\frac{\partial S'}{\partial N'} \right) _{E'} N_\alpha \Rightarrow S'_\alpha  = \text{cost.} - \frac{E_\alpha - \mu  N_\alpha }{T }
\end{equation}
perci\`o la probabilit\`a, comprensiva di normalizzazione, \`e:
\begin{boxenv}[]
\begin{equation}
	w_\alpha = \frac{\exp \left[ -(E_\alpha  - \mu  N_\alpha ) / \kappa _B T \right] }{\sum_{\alpha }^{} \exp \left[ - (E_\alpha  - \mu  N_\alpha ) / \kappa _B T \right] } \equiv \frac{1}{\mathscr{L}} \exp \left(- \frac{E_\alpha - \mu  N_\alpha }{\kappa _B T}\right) 
\end{equation}
\end{boxenv}
\noindent con $\mathscr{L}$ \textbf{funzione di granpartizione}. Nel limite di $N_\alpha  = N, \ \forall \alpha $, $w_\alpha $ tende al caso canonico con normalizzazione data dalla funzione di partizione $\mathscr{Z}$.

\subsubsection{Entropia e potenziali}
Ora si pu\`o calcolare $\langle S \rangle$:
\begin{equation}
	\langle S  \rangle = \kappa \log \mathscr{L} + \frac{1}{T} \sum_{\alpha }^{} w_\alpha  E_\alpha - \frac{\mu }{T} \sum_{\alpha }^{} w_\alpha  N_\alpha = \kappa _B \log\mathscr{L} + \frac{\langle E \rangle - \mu  \langle N \rangle}{T}
\end{equation}
Da questa si ottiene il potenziale di Landau:
\begin{equation}
	\begin{split}
		\Omega &= -\kappa _B T \log \mathscr{L}  = -\kappa _B T \log \sum_{\alpha }^{} \exp \left(- \frac{E_\alpha -\mu N_\alpha }{\kappa _B T}\right)  \\
		       &= -\mu  N - \kappa _B T \log \sum_{\alpha }^{} \exp \left( -\frac{E_\alpha }{\kappa _B T}\right) = -\mu  N- \kappa _B T \log\mathscr{Z}
	\end{split}
\end{equation}
dove si \`e imposto $N_\alpha  = N , \ \forall \alpha $. Conseguentemente $F = \Omega +\mu N = - \kappa _B T \log\mathscr{Z}$. 

\subsubsection{Degenerazione dei livelli energetici}

Ammettendo che \textit{diversi stati occupano stesso livello energetico}, continuando ad assumere $N_\alpha  = N , \ \forall \alpha $:
\begin{equation}
	w(E_\alpha )= \frac{1}{\mathscr{Z}} \rho (E_\alpha ) \exp \left( - \frac{E_\alpha }{\kappa _B T}\right) 
\end{equation}
con $\rho $ degenerazione relativa a energia $E_\alpha $. Passando al continuo:
\begin{equation}
	w(E_\alpha ) \to w (\mathscr{E}) = \frac{1}{\mathscr{Z}} \rho (\mathscr{E}) \exp \left(- \frac{\mathscr{E}}{\kappa _BT}\right) , \ \mathscr{Z}\to \int_{0} ^{+\infty} d \mathscr{E}\ \rho (\mathscr{E}) \exp\left(- \frac{\mathscr{E}}{\kappa _BT}\right) 
\end{equation}
Il numero di microstati si pu\`o riscrivere come:
\begin{equation}
	d\Gamma = \frac{d\Gamma}{d \mathscr{E}}d\mathscr{E} = \rho (\mathscr{E}) d \mathscr{E}
\end{equation}

\subsubsection{Applicazione -- Particelle non-interagenti}

Per $N$ particelle non-interagenti, ciascun grado di libert\`a fattorizza in $\mathscr{Z}$; per particelle \textbf{distinguibili} (distribuzioni diverse delle particelle in microstati individuano stati diversi), si ha $\mathscr{Z}_\text{tot} = \mathscr{Z}_{1p} ^N$; per particelle \textbf{indistinguibili}, una buona stima \`e: $\mathscr{Z}_\text{tot} = \frac{1}{N!}\mathscr{Z}_{1p} ^N$. Si considera il secondo caso.

Si ha $F = - \kappa _B T \log \mathscr{Z}= \kappa _B T \log N! - \kappa _B N T \log \mathscr{Z}_{1p} $. Ricordando che $E_{q_i}  = \frac{\hbar ^2 q_{i} ^2}{2m} $, con $q _ i= \frac{2\pi l_i}{L}$, quindi $E_q \propto L^{-2} = V^{-2 / 3} $, pertanto:
\[
\begin{split}
	P &= - \left(\frac{\partial F}{\partial V} \right) _T = \frac{N\kappa _B T}{\mathscr{Z}_{1p} }\frac{\partial \mathscr{Z}_{1p} }{\partial V} = - \frac{N\kappa _B T}{\mathscr{Z}_{1p} }\frac{1}{\kappa _BT} \sum_{i}^{} \frac{\partial E_{q_i} }{\partial V} \exp \left(- \frac{E_{q_i} }{\kappa _B T}\right) \\
	  &= \frac{2N}{3V}\frac{1}{\mathscr{Z}_{1p} } \sum_{i}^{} E_{q_i} \exp \left( - \frac{E_{q_i} }{\kappa _BT}\right) \equiv \frac{2N}{3V} \langle E \rangle
\end{split}
\] 


\subsubsection{Applicazione -- Sistema a due stati*}
\begin{boxenv}[]
\centering \textit{Sistema in cui particelle interagiscono solo tramite spin\\ Valutare se va scritto} 
\end{boxenv}
\subsection{Spazio delle fasi}

Per sistema di $N$ particelle, \`e uno spazio $6N$-dimensionale delle coordinate e impulsi. Fissare energia dell'Universo equivale a definire un'ipersuperficie $\Sigma_0$ a $(6N-1)$ dimensioni data da $\mathscr{E}_0\big(\left\{ x_i \right\} , \left\{ p_i \right\} \big) = E_0$.

Si discretizza lo spazio in celle che rispettano $\Delta x_k \Delta p_k = \tau $, con $\tau $ costante generica. Si assume che \textit{le celle siano piccoli a sufficienza da avere un solo stato in ciascuna}; allora numero di stati sar\`a area dell'ipersuperficie normalizzata con elemento di volume:
\begin{equation}
	\Gamma_0 = \frac{1}{\tau ^{f_0} } \iint_{\Sigma_0} \prod_{i=1} ^{f_0}  dx_i , dp_i , \ \text{ con } f_0 = 3N
\end{equation}
Allora l'entropia dell'Universo \`e:
\begin{equation}
	S_0 = \kappa _B \log \iint_{\Sigma_0} \prod_{i=1} ^{f_0} dx_i dp_i - \kappa _B f_0 \log \tau 
\end{equation}
Per $\Sigma'$ ipersuperificie data da $E'_\alpha  = E_0 - E_\alpha $, si pu\`o ripetere il discorso per il bagno termico:
\begin{equation}
	S'_\alpha  = \kappa _B \log \iint_{\Sigma'} \prod_{i=1} ^{f'} dx_i dp_i - \kappa _B f' \log \tau 
\end{equation}
In questo modo, l'entropia media del sistema \`e:
\begin{equation}
	\langle S \rangle = \kappa _B \log \iint_{\Sigma_0} \prod_{i=1} ^{f_0} dx_i dp_i - \left\langle \kappa _B \log \iint_{\Sigma'} \prod_{i=1} ^{f'} dx_i dp_i \right\rangle - \kappa _B (f_0-f') \log \tau 
\end{equation}
L'entropia \`e singolare per $\tau \to 0$, quindi deve essere un valore finito.

\subsubsection{Costante di Planck}
Si ricava per particella confinata in segmento $L$. I livelli energetici sono $E_q = \hbar ^2 q^2 / (2m)$ con $q = 2\pi l / L$, e $l \in \mathbb{Z}$. Il conteggio degli stati nella cella $L \Delta p$ \`e $\Delta l = L\Delta q / (2\pi) = L\Delta p /(2\pi\hbar )$; d'altra parte:
\[
\frac{1}{\tau }\int_{L} \int_{\Delta p} dxdp = \frac{L}{\tau }\Delta p \Rightarrow \tau  =2\pi \hbar  = h
\] 
\subsubsection{Applicazione -- Densit\`a di energia ed energia per singola particella}\label{1p}

Per singola particella libera, usando coordinate cilindriche per gli impulsi:
\begin{equation}
	\Gamma = \frac{1}{h^3} \iint d^3 x d^3 p = \frac{V}{h^3} \int 4\pi  p^2 \ dp
\end{equation}
Visto che $\mathscr{E} = p^2 / 2m$, tramite confronto:
\begin{equation}
	\rho (\mathscr{E})d \mathscr{E} = \frac{4\pi V }{h^3}p^2 \frac{d p}{d \mathscr{E}} d \mathscr{E}= \frac{4\pi V m^{3 / 2} }{h^3}\sqrt{2 \mathscr{E}} d \mathscr{E}
\end{equation}
Si pu\`o calcolare l'energia media:
\begin{equation*}
		\langle \mathscr{E} \rangle = \frac{\displaystyle \int_{0} ^{+\infty} \mathscr{E} e^{ - \mathscr{E} / \kappa _B T}  \rho (\mathscr{E}) d \mathscr{E}  }{\displaystyle \int_{0} ^{+\infty} e^{ - \mathscr{E} / \kappa _B T}  \rho (\mathscr{E}) d \mathscr{E}  }= \kappa _B T \frac{\displaystyle \int_{0} ^{+\infty} dx \ x^{3 / 2} e^{ - x} }{\displaystyle \int_{0} ^{+\infty}dx \ x^{1 / 2} } e ^{-x} = \kappa_B T \frac{\Gamma(5 / 2)}{\Gamma (3/2)} = \frac{3}{2} \kappa _B T
\end{equation*}
\subsubsection{Applicazione -- Gas interagente}

Gas non-relativistico immerso in potenziale generico dipendente solo dalle coordinate. Elemento differenziale dello spazio delle fasi \`e $d\Gamma = \rho (\mathscr{E}) d \mathscr{E} = \frac{1}{N!h^{3N} } \prod_{i=1} ^{3N} dx_i dp_i$, da cui essendo $\mathscr{E} = U\big(\left\{ x_i \right\} \big) + \sum_{i=1}^{3N} p_i^2 / 2m$
\[
	\begin{split}
	\mathscr{Z} &= \int e^{- \mathscr{E} / \kappa _B T}  \rho (\mathscr{E}) d \mathscr{E} = \frac{1}{N! h^{3N} }\iint \prod_{i=1} ^{3N} dx_i dp_i e^{- \mathscr{E} / \kappa _B T} \\
			    &= \frac{1}{N! h^{3N} } \iint \prod_{i=1} ^{3N} dx_i dp_i \ \exp \left[ - \frac{\sum_{i=1}^{3N} p_i^2}{2m\kappa _B T} - \frac{U \big(\left\{ x_i \right\} \big)}{\kappa _BT} \right] \\
			    &=\frac{1}{N! h^{3N} } \int \prod _{i=1} ^N d^3 p_i \exp \left[ - \frac{\sum_{i=1}^{N} p_i  ^2}{2m \kappa _B T} \right] \int \prod_{i=1} ^N d^3 x_i \exp\left[ - \frac{U\big(\left\{ x_i \right\} \big)}{\kappa _B T} \right] 
	\end{split}
\] 
Il primo integrale, insieme al prefattore, si pu\`o ricondurre a quello di un gas ideale, a meno di un $V^N$:
\[
		\mathscr{Z}_{IG}= \frac{1}{N!} \left[ \frac{1}{h^3} \iint d^3 x d^3 p \ \exp \left( - \frac{p^2}{2m\kappa _B T } \right) \right] ^N= \frac{V^N}{N! h^{3N} } \int \prod_{i=1} ^N d^3 p_i \ \exp \left(-\frac{\sum_{i=1}^{N} p_i^2}{2m\kappa _BT}\right) = \frac{1}{N!} \frac{V^N}{\Lambda ^{3N} }
\] 
con $\Lambda = 2\pi \hbar /\sqrt{2\pi m\kappa _B T} $ lunghezza d'onda termica di de Broglie. Il secondo dipende dalla forma del potenziale ed \`e detto \textbf{integrale delle configurazioni}:
\begin{equation}
	\mathscr{D} \equiv \int \prod_{i=1} ^N	d^3 x_i \ \exp \left(- \frac{U\big(\left\{ x_i \right\} \big)}{\kappa _BT}\right) 
\end{equation}
Quindi:
\begin{equation}
\mathscr{Z} = \frac{1}{N!} \frac{\mathscr{D}}{\Lambda ^{3N} }
\end{equation}
Per la funzione di granpartizione\footnote{Nella seconda uguagliamza, si spezza la somma, raggruppando i termini della somma stessa in base a $N_\alpha $, per questo $\alpha | N_\alpha $ indica una somma sugli $\alpha $ relativa a ciascun $N_\alpha $.}:
\begin{equation}
	\begin{split}
		\mathscr{L} &= \sum_{\alpha }^{} \exp \left[ - \frac{E_\alpha  - \mu  N_\alpha }{\kappa _B T} \right] = \sum_{N_\alpha }^{} \left[ \exp \left(\frac{\mu N_\alpha }{\kappa _B T}\right) \sum_{\alpha | N_\alpha }^{} \exp \left(- \frac{E_\alpha }{\kappa _B T}\right)  \right] \\
			    & =\sum_{N_\alpha }^{} \left\{ \left[ \exp \left(\frac{\mu }{\kappa _BT}\right)  \right] ^{N_\alpha } \sum_{\alpha | N_\alpha }^{} \exp \left(-\frac{E_\alpha }{\kappa _BT}\right)  \right\} \equiv \sum_{N_\alpha }^{} \left[ z^{N_\alpha } \sum_{\alpha | N_\alpha }^{} \exp\left(- \frac{E_\alpha }{\kappa _BT}\right)  \right] 
	\end{split}
\end{equation}
dove $z$ \`e detta \textbf{fugacit\`a}. Nel limite al continuo, si trova:
\begin{equation}
	\mathscr{L}= \sum_{N}^{} \frac{z^N \mathscr{D}_N}{N! \Lambda ^{3N} }
\end{equation}
con $\mathscr{D}_N$ integrale delle configurazioni relativo agli stati $\alpha $ con $N$ particelle.


\subsection{Gas perfetto}
Particelle confinate in scatola con autostati dell'energia individuati dagli impulsi $q$. Numero di particelle in uno stato \`e $n_q$. 

Per gas ideale, la maggior parte dei microstati sar\`a vuota, cio\`e $w(0) \approx 1$, e la probabilit\`a di avere pi\`u di una particella in un microstato \`e praticamente nulla, quindi $w(1) \ll 1$ e $w(n\ge 2) \approx 0$. Usando $\mathscr{L} = \exp \left[ - \Omega / \kappa _B T \right] $:
\begin{equation}
	w(n_q) = \exp \left[ \frac{\Omega _q - n_q (E_q - \mu )}{\kappa _B T} \right] 
\end{equation}
Allora le condizioni di popolazione dei microstati si traducono in:
\begin{equation}
	\begin{split}
		&w(0) \approx 1 \Rightarrow \exp \left(\frac{\Omega _q}{\kappa _B T}\right) \approx 1\\
		&w(n) = e^{\Omega _q / \kappa _B T} \left[\exp \left(- \frac{E_q - \mu }{\kappa _BT} \right)\right]  ^n\equiv w^n(1)  \ll 1, \forall q \iff \exp \left(\frac{\mu}{\kappa _BT }\right) \ll 1
	\end{split}
\end{equation}
quindi $\mu \to -\infty$. Da questo, il numero medio di particelle in uno stato $q$ \`e:
\begin{equation}
	\langle n_q \rangle = \frac{\sum_{n_q}^{} n_q \exp\left[ -n_q (E_q - \mu ) / \kappa _BT \right] }{\sum_{n_q}^{}\exp\left[ -n_q (E_q - \mu ) / \kappa _BT \right]  } \approx \exp \left(- \frac{E_q - \mu }{\kappa _BT }\right) 
\end{equation}
avendo usato $w(1) \ll 1$. Dal potenziale di Landau, si ottiene equazione di stato dei gas perfetti:
\[
\Omega _q \approx -\kappa _B T \log \left[ 1 + \exp\left(- \frac{E_q - \mu }{\kappa _B T }\right)  \right] = -\kappa _B T \log \Big(1 + \langle n_q \rangle\Big) \approx -\kappa _B T \langle n_q \rangle
\] 
Essendo $\Omega \approx - \kappa _B T \sum_{q}^{} \langle n_q \rangle  \equiv -\kappa _B T N$ e valendo allo stesso tempo $\Omega = - PV$, si ha $PV = \kappa _B NT$.

Ora si ricava $N$ in funzione di $T,V,\mu $. Usando la densit\`a di stati $\rho (\mathscr{E})$ trovata per singola particella in \S \ref{1p}, si ha:
\begin{equation}
	\begin{split}
		N &= \int \rho (\mathscr{E}) \exp\left(- \frac{\mathscr{E}-\mu }{\kappa _BT}\right) \ d \mathscr{E}= \frac{4\pi \sqrt{2} V m^{ 3 /2 } }{h^3}e^{ \mu  / \kappa _B T} \int d\mathscr{E} \ e^{ - \mathscr{E} / \kappa _B T}  \mathscr{E}^{1 / 2} \\
		  &= \frac{V}{\Lambda ^3}e^{\mu  / \kappa _B T} 
	\end{split}
\end{equation}
Usando $PV = N\kappa _B T$, si pu\`o scrivere
\begin{equation}
	\begin{split}
		&\mu = - \kappa _B T \log \frac{\kappa _B T}{P \Lambda ^3}\\
		&\Phi = N\mu  = - \kappa _B NT \log \frac{\kappa _B T}{P\Lambda ^3}
	\end{split}
\end{equation}
Quindi, espandendo il logaritmo del prodotto nelle somme dei logaritmi e nuovamente la legge $PV = N\kappa _B T$ per sostituire la pressione nel primo logaritmo:
\begin{equation}
	\begin{split}
		S &= - \left(\frac{\partial \Phi}{\partial T} \right) _{PN} = - N\kappa _B \log \frac{V}{N} + \frac{5}{2} N\kappa _B \log\kappa _B T + N\kappa _B \left(\frac{5}{2} + \frac{3}{2} \log \frac{m}{2\pi\hbar ^2}\right) 
	\end{split}
\end{equation}
Da questa trattazione, si ricavano tutti gli altri risultati, come:
\[
c_P = T \left(\frac{\partial S}{\partial T} \right) _P =  \frac{5}{2}N\kappa _B \hspace{.1cm} ; \hspace{.2cm} c_V = T \left(\frac{\partial S}{\partial T} \right) _V = \frac{3}{2} N \kappa _B 
\] 
Dalla formula per $\mu $, la condizione di gas ideale diventa:
\begin{equation}
	\frac{\kappa _B  T}{P\Lambda ^3}  \gg 1 \iff \frac{N\Lambda ^3}{V}\ll 1
\end{equation}
Infine, fissando $N$ (ensemble canonico):
\begin{equation}
	F = -N\kappa _BT \log \frac{V}{\Lambda ^3}  + \kappa _BT \log N!
\end{equation}
mentre fissando $\langle N \rangle$ (ensemble grancanonico):
\begin{equation}
	F = \Phi - PV = - N\kappa _B T  \log \frac{V}{\Lambda ^3} + \kappa _B T (N \log N - N)
\end{equation}
Per $N$ grandi, queste espressioni coincidono, essendo $\log N! \approx N \log N - N$.


\subsection{Distribuzione dell'energia}

In assenza di potenziale, vincolo sull'energia \`e fissato da $\mathscr{E} = \frac{1}{2m} \sum_{i=1}^{3N} p_i^2$; in questo, un elemento dello spazio delle fasi $d\Gamma$ sar\`a proporzionale ad un elemento di volume, a sua volta proporzionale al raggio: $\rho (\mathscr{E}) d \mathscr{E} \propto d V* (\mathscr{E}) \propto (p^*)^{3N } $, con $p^* = \sqrt{2 m \mathscr{E}} = \sqrt{\sum_{i=1}^{3N} p_i^2} $.

Ricordando che $w (\mathscr{E}) = \frac{1}{\mathscr{Z}}\rho (\mathscr{E}) e^{ - \mathscr{E } /\kappa _B T} $ (nel caso di degenerazione di un livello energetico e $N _\alpha  = N ,  \ \forall  \alpha $):
\[
\begin{split}
	&dV^* \propto \frac{\partial V^*}{\partial \mathscr{E}}  d \mathscr{E} \propto (p^*)^{3N -1 } \frac{\partial p^*}{\partial \mathscr{E}} d \mathscr{E}\propto \mathscr{E} ^{3N /2 -1} d \mathscr{E}\\
	&\Rightarrow  w (\mathscr{E}) \propto \mathscr{E}^{3N /2 -1 } \exp \left(- \frac{\mathscr{E}}{\kappa _B T}\right) \Rightarrow w(\mathscr{E}) = \frac{1}{\Gamma(3N / 2) } \left(\frac{\mathscr{E}}{\kappa _B T}\right) ^{3N / 2 -1} \frac{\exp(- \mathscr{E}/\kappa _B T)}{\kappa _B T}
\end{split}
\] 
L'energia pi\`u probabile si ottiene imponendo $\partial _\mathscr{E} w \stackrel{!}{=} 0$, da cui $\mathscr{E}_\text{max}= (3N / 2 -1) \kappa _B T$. D'altra parte, il valore medio \`e $E = \int d \mathscr{E} \ w(\mathscr{E}) \mathscr{E} = \frac{3}{2}N \kappa _B T $: i due differiscono per fattore additivo indipendnete da $N$, quindi per $N$ molto grandi, la distribuzione \`e piccata attorno al valore medio.

Per la varianza $\sigma _\mathscr{E}^2 = \langle \mathscr{E}^2 \rangle- E^2$, si usa $\partial _T^2 F = (E^2 - \langle \mathscr{E}^2 \rangle) / (\kappa _B T^3)$ e $c_V = \partial_T E = 3N\kappa _B / 2$, quindi:
\begin{equation}
	\sigma _\mathscr{E}^2 = - \kappa _B T^3 \frac{\partial ^2F}{\partial T^2} = \kappa _B T^2 c_V = \frac{3}{2} N\kappa _B^ 2 T^2 
\end{equation}
 Per la singola particella, allora: $\sigma _\mathscr{E} \sim \kappa _B T$.




\subsection{Incertezze quantistiche}
\begin{boxenv}[]
	\centering \textit{Valutare se aggiungere} 
\end{boxenv}






















\newpage
\section{Gas quantistici}
\subsection{Statistiche di Bose-Einstein e Fermi-Dirac}
Perch\'e valga indistinguibilit\`a delle particelle, a bassa temperatura si devono modificare gli stati occupabili. Per sistema di due particelle, deve risultare $\lvert \psi (1,2) \rvert ^2 = \lvert \psi (2,1) \rvert ^2$, cio\`e la probabilit\`a di trovare le particelle in un punto dello spazio deve essere uguale se si scambiano le due particelle.

Quindi $\psi (1,2) = \pm \psi (2,1)$. Si assume che le due particelle stiano o in $a$, o in $b$, si \textit{suppone che le funzioni d'onda delle singole particelle siano fattorizzate}; le uniche combinazioni che rispettano la condizione $\psi (1,2) = \pm \psi (2,1)$ sono una simmetrica e una antisimmetrica:
\begin{equation}
		\psi _S = \psi _a (1) \psi _b (2) + \psi _a(2) \psi _b(1); \ \psi _A = \psi _a(1) \psi _b(2) - \psi _a(2) \psi _b(1)
\end{equation}
Quando entrambe sono nello stesso stato ($a=b$), $\psi _A = 0 $; questo \`e il principio di esclusione di Pauli. 

Particelle con funzione d'onda antisimmetrica sono dette \textbf{fermioni}, mentre con funzione d'onda simmetrica sono dette \textbf{bosoni}.

Per principio di esclusione, i fermioni possono soddisfare $n_q = 0,1$ solamente, quindi:
\begin{equation}\label{ofd}
	\Omega _q = -\kappa _B T \log \left[ 1+ \exp\left(\frac{\mu  - E_q}{\kappa _B T}\right)  \right] 
\end{equation}
da cui si ricava la \textbf{statistica di Fermi-Dirac}:
\begin{boxenv}[]
\begin{equation}
	\langle n_q \rangle = - \frac{\partial \Omega _q}{\partial \mu } = \frac{1}{\exp \left[ (E_q - \mu ) / \kappa_B T \right]  + 1}
\end{equation}
\end{boxenv}
\noindent Per ottenere potenziale di Landau e numero di particelle totali, basta sommare su $q$. Per i bosoni, invece, tutti gli $n$ sono possibili e si deve calcolare la somma di una serie geometrica, \textit{che converge solamente se} $\mu \le  E_0$\footnote{Visto che $E_0$ \`e l'energia minore, la condizione pi\`u restrittiva \`e $\mu \le E_0$.}. In questa trattazione $E_0=0$, quindi $\mu $ deve essere negativo e
\begin{equation}\label{obe}
	\Omega _q = - \kappa _B T \log \sum_{n=0}^{+\infty} \exp \left[ \frac{n(\mu  - E_q)}{\kappa _B T} \right] = \kappa _B T \log \left[1 - \exp\left(\frac{\mu  - E_q}{\kappa _B T}\right) \right]
\end{equation}
da cui la \textbf{statistica di Bose-Einstein} \`e:
\begin{boxenv}[]
\begin{equation}
	\langle n_q \rangle = \frac{1}{\exp\left[ (E_q - \mu )/\kappa _B T \right] - 1}
\end{equation}
\end{boxenv}
\subsection{Gas perfetto debolmente degenere}
Si studia comportamento quantistico del gas perfetto. Per passare al continuo, \`e necessario che fluttuazioni statistiche siano maggiori della separazione tra i livelli, quindi
\begin{equation}
	\kappa _B T \gg \frac{\hbar ^2}{2m} \left(\frac{2\pi}{L}\right) ^2
\end{equation}
Se $L\sim 1 \text{ cm}$ e $m\sim 10^{-24} \text{ g}$ (massa atomo di idrogeno), si ha $T \gg 10^{-13} $ K; per elettroni $T\gg 10^{-10} $ K. Dal punto di vista pratico, queste sono sempre soddisfatte.

In assenza di campi, ogni spin $S$ ha $g = 2S + 1$ orientazioni possibili e si deve aggiungere nel conteggio dei microstati\footnote{Aumentando il numero di microstati di $g$, l'entropia subisce un aumento per un termine $\kappa _B \log g$.}. La correzione per misura dello spazio delle fasi \`e:
\[
d\Gamma= \frac{d^3 x d^3 p }{h^3}\to g \frac{d^3 xd^3p}{h^3}
\] 
Si sviluppano in serie le espressioni dei potenziali di Landau per ciascuna statistica (eq. \ref{ofd}, \ref{obe}) (segno superiore per FD, inferiore per BE):
\begin{equation*}
	\begin{split}
		\Omega &= \mp \kappa _B T \sum_{q}^{} \log \left[ 1\pm \exp\left( - \frac{E_q - \mu }{\kappa _B T}\right)  \right] \\
		       &\simeq -\kappa _B T \sum_{q}^{} \exp \left(-\frac{E_q - \mu }{\kappa _B T} \right) \pm \frac{\kappa _B T }{2}\sum_{q}^{} \exp \left(-2 \frac{E_q - \mu }{\kappa_B T }\right) \\
		       &= \Omega _\text{class} \pm \frac{\kappa _B T}{2} \sum_{q}^{} \exp \left( - 2 \frac{E_q - \mu }{\kappa _B T}\right) 
	\end{split}
\end{equation*}
Si esegue il passaggio al continuo e si sostituisce $\mathscr{E} \to 2\mathscr{E}$, usando $\rho (\mathscr{E}) \propto \sqrt{\mathscr{E}} $; inoltre si riscrive il potenziale classico:
\[
\begin{split}
	&\sum_{q}^{} \exp \left(- \frac{2 E_q}{\kappa _B T}\right)  \to \int_{0} ^{+\infty} \rho (\mathscr{E}) \exp \left(-\frac{2\mathscr{E}}{\kappa _B T}\right) \ d\mathscr{E} \to \left(\frac{1}{2}\right) ^{3 / 2} \int_{0} ^{\infty} \rho (\mathscr{E}) \exp\left(- \frac{\mathscr{E}}{\kappa _B T}\right)  d \mathscr{E}\\
	& \Omega _\text{class} \to - \kappa _B T \exp \left(\frac{\mu  }{\kappa _B T}\right)  \int_{0} ^{+\infty} \rho (\mathscr{E}) \exp \left(- \frac{\mathscr{E}}{\kappa _B T}\right) \ d \mathscr{E}
\end{split}
\] 
Da questi passaggi, si ottiene:
\begin{equation}
	\Omega \simeq \Omega _{\text{class}} \left(1 \mp \frac{e^{ \mu  / \kappa _B  T} }{2 ^{5 / 2} }\right) 
\end{equation}
Usando $\Omega  = - PV $ e $\Omega _\text{class} = - N \kappa _B T$:
\begin{equation}
	PV \simeq N\kappa _B T \left(1 \mp \frac{e^{ \mu  / \kappa _B T} }{2^{ 5 / 2} }\right) 
\end{equation}
La statistica agisce come sorta di forza sulle particelle; per capire se attrattiva o repulsiva, si deve passare da $\Omega $ a $F $\footnote{Usando $\Omega $, si \`e trovato risultato in cui $P / N$ ha $N$ variabile.}, in cui sono fissati $T,V,N$. Definendo $\delta \Omega = \Omega  - \Omega _\text{class}$ e $\delta F  = F - F_\text{class}$, si ha $(\delta \Omega ) _{T,V,\mu } = (\delta F) _{T,V,N} $. 

Per eliminare $\mu $ in $\delta \Omega  = \pm N\kappa _B Te^{\mu  / \kappa _B T} / 2^{5 / 2} $, si usa espressione classica $\mu _\text{class}= - \kappa _B T \log V /(N\Lambda ^3)$, da cui:
\begin{equation}
	\begin{split}
		&F = F_\text{class} \pm \frac{N^2 \kappa _B T \Lambda ^3}{2^{5 / 2}V }\\
		& \Rightarrow P = - \frac{\partial F }{\partial V } = \frac{N\kappa _B T}{V}\left(1 \pm \frac{N\Lambda ^3}{2^{5 / 2} V}\right) 
	\end{split}
\end{equation}
Quindi la statistica di Fermi corrisponde ad una forza repulsiva, mentre quella di Bose a una attrattiva.
\subsection{Gas di Fermi}

Si considera gas di fermioni nel limite $T \to 0 $, in cui le particelle occuperanno i livelli energetici pi\`u bassi consetiti dal principio di esclusione, inziando dal ground state a salire, fino a esaurimento particelle. 

Allora lo spazio delle fasi di singola particella\footnote{Per spazio delle fasi di singola particella, si fa riferimento all'insieme di tutti i possibili stati che una particella pu\`o occupare; in avanti, si menzioner\`a lo spazio delle fasi complessivo (quello di tutto il gas), il quale rappresenter\`a tutti gli stati occupabili dall'intero sistema. Essendo le particelle indistinguibili, quest'ultimo deve collassare a un punto perch\'e i fermioni si possono distribuire solo in stati di singola particella ad energia via via crescente.} avr\`a tutte le celle piene dall'origine fino a un'energia $E_f = p_f^2 / 2m$, con $E_f$ \textbf{energia di Fermi} e $p_f$ \textbf{impulso di Fermi}, relativa all'energia del pi\`u alto stato quantistico occupabile da una particella\footnote{In quanto tale, dipender\`a dal numero totale di particelle e dal volume in cui è confinato il gas.}; lo spazio delle fasi a molti corpi, invece, consiste in un solo punto.

Nel limite $T\to 0$, il grafico ($E_q$, $n_q$) (numero di occupazione in funzione dell'energia) \`e un gradino con $n_q = 1$ per $0\le E_q\le \mu_0$ e $0$ altrimenti, con $\mu (T=0) \equiv \mu_{0}\equiv E_f$.

Quest'ultimo \`e fissato dal numero totale di particelle dato da:
\begin{equation}
	\begin{split}
		N = \lim_{T \to 0} \sum_{q}^{} \frac{1}{\exp\left[ (E_q - \mu  ) / \kappa _B T \right]  + 1 } &\to \lim_{T \to 0 } \int_{0} ^{+\infty} \frac{\rho (\mathscr{E})d \mathscr{E}}{\exp\left[ (\mathscr{E}-\mu ) / \kappa _B T \right] + 1}\\
		  & =  \frac{gV}{h^3}\frac{4}{3}\pi p^3_f
	\end{split}
\end{equation}
con $g = 2S + 1$ degenerazione degli stati quantistici dovuta allo spin. L'ultima uguaglianza \`e verificata perch\'e, in questo caso, $N$ \`e \# di celle in una sfera di raggio $p_f^3$ nello spazio delle fasi di singola particella. 

Da questo, $p_f = h [3N / (4\pi V g)] ^{1 / 3}$, quindi il valore dell'energia di Fermi \`e:
\begin{equation}
	E_f = \frac{h^2}{2m} \left(\frac{N}{V}\right) ^{2/3} \left(\frac{3}{4\pi g} \right) ^{2 /3 }
\end{equation}
Da questa si ottiene \textbf{temperatura di Fermi} $\kappa _B T_f = E_f$. 

La sfera nello spazio di singola particella \`e detta \textbf{sfera di Fermi}, o \textbf{mare di Fermi}, mentre il guscio \`e detto \textbf{superficie di Fermi}.

Essendo $\rho (\mathscr{E}) \propto \mathscr{E}^{1 / 2} $, l'energia media per particella è:
\begin{equation}
	\langle \mathscr{E} \rangle = \frac{\displaystyle \int_{0} ^{E_f} \mathscr{E}\rho (\mathscr{E}) d \mathscr{E}}{\displaystyle \int_{0} ^{E_f} \rho (\mathscr{E}) d\mathscr{E}} = \frac{ \displaystyle \int_{0} ^{E_f} \mathscr{E}^{3 / 2}  d \mathscr{E}}{\displaystyle  \int_{0} ^{E_f} \mathscr{E}^{1 / 2}  d \mathscr{E}}= \frac{3}{5} E_f
\end{equation}
Quindi l'energia totale \`e $E = \frac{3}{5} N E_f$. 

Entropia del sistema \`e nulla (una sola possibile configurazione nello spazio delle fasi del gas) e l'energia complessiva del sistema coincide con l'energia interna (sempre perché il sistema si pu\`o trovare in un solo stato); allora la pressione \`e:
\begin{equation}\label{Pf}
P = - \left(\frac{\partial E}{\partial V} \right) _N = \frac{2}{3} \frac{E}{V}
\end{equation}
Pressione finita a temperatura nulla \`e conseguenza della forza repulsiva tra i fermioni.	
\subsubsection{Comportamento del gas per $T>0$}
Si considera cosa succede al gas quando la temperatura sale di poco sopra $0$, quindi nel limite $T \ll T_f$\footnote{La scala di grandezza delle temperature \`e data solo da $T_f$ in questo caso, quindi si usa questa come riferimento.}. \textit{Questo modello si user\`a per studiare comportamento degli elettroni nei metalli}, quindi si stima $T_f$ usando massa e spin dell'elettrone e densit\`a di elettroni di conduzione nel rame, ottenendo $T_f \approx 8.5 \times 10^4$ K; questa risulta due ordini di grandezza sopra la temperatura di fusione del rame stesso, quindi il gas di elettroni \`e sempre in limite di basse temperature.

Con aumento di $T$, le particelle sulla superficie di Fermi (nei livelli energetici pi\`u esterni) possono eccitarsi con energia $\sim \kappa _B T$, mentre quelli nel mare no perch\'e i livelli successivi sono occupati. 
Numero di elettroni eccitati $\sim T / T_f$ per il totale. 

Il grafico di $n(\mathscr{E})$ \`e un gradino consumato nell'intervallo attorno a $E_f$, di larghezza $\sim \kappa _B T$. 

Questo modello, per elettroni in metallo con background uniforme positivamente carico\footnote{Approssimazione in cui gli ioni si immaginano come una carica positiva uniformemente distribuita invece che come punti discreti facenti parte di un reticolo, per questo l'approssimazione \`e valida nella condizione riportata.}, \`e valido finch\'e $\lambda \gg a$, con $\lambda $ lunghezza d'onda elettroni e $a$ dimensione caratteristica del reticolo del metallo.

La condizione \`e verificata per stati a bassi impulsi ($p = \hbar  q$), per i quali si pu\`o assumere che $E \propto q^2$ (l'energia continua ad obbedire la legge di dispersione), mentre avr\`a una forma diversa fuori da questo regime. 
In realt\`a, anche in questo caso, \`e diversa: $E = \frac{\hbar ^2 q^2}{2m^*}$, con $m^*$ massa efficace dovuta all'interazione degli elettroni con gli ioni.

\subsubsection{Propriet\`a termiche del gas di Fermi}
Indicando con $\overline{n}(\mathscr{E})$ la statistica di Fermi-Dirac, si sa da eq. \ref{Pf} che $\Omega = -\frac{2}{3}E$, quindi:
\[
	\Omega = -\frac{2}{3} \int_{0} ^{+\infty} \mathscr{E}\rho (\mathscr{E}) \overline{n}(\mathscr{E}) \ d \mathscr{E}= -\frac{2}{3} \frac{4 \pi V g \sqrt{2} m^{2 / 3} }{h^3}\int_{0} ^{+\infty} \frac{\mathscr{E}^{3/2} }{\exp \big[(\mathscr{E}-\mu) / \kappa _BT \big] + 1} d \mathscr{E}
\] 
Si deve, quindi, risolvere integrale della forma $I = \int_{0} ^{+\infty} \frac{f(\mathscr{E}) d \mathscr{E}}{\exp\left[ (\mathscr{E}-\mu ) / \kappa _ BT \right] +1}$.
Visto che $\overline{n}(\mathscr{E})$ \`e una funzione gradino per $T \ll T_f$, il contributo maggiore nell'integrale sar\`a da $0$ a $E_f$ in cui $n(\mathscr{E}) = 1$.
L'integrale su questi estremi si chiama $I_0$. La correzione su $I$ \`e $\delta I$ in modo che $I = I_0 + \delta I$; essendo $I = I_0$ per $T=0$, il comportamento termico \`e incluso in $\delta I$.

Per ricavare $\delta I$, si calcola differenza tra integrale sul gradino e integrale esatto: $\delta I = I - I_0$.

Nel calcolo di $I_0$ al posto di $I$, si introduce una sovrastima prima di $E_f$ e una sottostima dopo per avere gradino perfetto. Si prende $z = (\mathscr{E} - \mu ) / \kappa _B T $. Si definiscono $g_0(z), g_1(z)$ come
\[
	g_1 (z) = \frac{1}{e^z +1} \hspace{.1cm} ; \hspace{.2cm} g_0(z) = 1 - \frac{1}{e^z +1} = \frac{1}{e^{-z}  +1}
\] 
Si nota che la loro differenza rappresenta esattamente la differenza tra la funzione gradino e la vera statistica; allora $\delta I$ \`e dato da\footnote{L'estremo inferiore \`e $-\infty$ perch\'e per temperature sufficientemente basse, si ha $- \mu  / \kappa _B T \approx - \infty$; questo permette di semplificare calcolo dell'integrale.}
\[
\delta I = \int_{0} ^{+\infty} d \mathscr{E} \ f(\mathscr{E}) \left[ g_1(z) - g_0(z) \right] = \int_{-\infty} ^{+\infty} \kappa _B T f(\mu +\kappa _B T z) \left[ g_1(z) - g_0(z) \right]  \ dz
\] 
Visto che $g_1-g_0 \neq 0$ solo intorno a $z=0$, si sviluppa, $f$ attorno a $z=0$:
\[
	\begin{split}
		&f(\mu  + \kappa _B T z) \simeq f(\mu ) + \kappa _B T z \left(\frac{\partial f}{\partial \mathscr{E}} \right) _{\mathscr{E} = \mu } = f(\mu ) + \kappa _B T zf '(\mu )\\
		& \Rightarrow \delta I = \kappa _B T f(\mu ) \int_{-\infty} ^{+\infty} dz \left[ g_1(z) - g_0(z) \right] + \kappa _B^2 T^2 f'(\mu ) \int_{-\infty} ^{+\infty} dz \left[ g_1(z) - g_0(z) \right] z
	\end{split}
\] 
Essendo $g_1(z)$ la parte di $\overline{n}$ per $z\ge 0$ e $g_0(z)=1 - \overline{n}$, per $z\le 0$:
Allora, mandando $z\to - z$ e considerando che $g_1(z) \neq 0 $ solo per $z\ge 0$ e $g_0(z) \neq 0$ solo per $z \le 0$, il primo termine in $\delta I$ \`e nullo, mentre il secondo \`e
\[
\int_{0} ^{+\infty} dz \ z g_1(z) - \int_{-\infty} ^0 dz \ z g_0(z) = 2 \int_{0} ^{+\infty} \frac{z dz}{e^z + 1}= \frac{\pi^2 }{6}
\] 
Complessivo di correzione quadratica in $T$, il potenziale di Landau \`e:
\begin{equation}
	\Omega = -\frac{2}{3} \frac{4 \pi \sqrt{2} V g m^{ 3 /2 } }{h^3}\left[ \frac{2}{5} \mu ^{5 / 2} + \frac{\pi^2 }{4} \sqrt{\mu } (\kappa _B  T )^2  \right] 
\end{equation}
quindi 
\begin{equation}
	N = - \frac{\partial \Omega }{\partial \mu } = N_0 \left[ 1 + \frac{\pi^2}{8} \left(\frac{\kappa _B T}{\mu }\right) ^2 \right] 
\end{equation}
Questo risultato è utile nel caso di esperimenti in cui vi \`e un gran numero di particelle e il potenziale chimico viene tenuto costante.
In altri esperimenti, capita di lavorare con $N$ fisso e conviene trovare un'espressione per $\mu $, che sarà funzione di $T$.

Si consiera stato iniziale a $T=0$ con $N =  N_0 \propto \big(\mu (T=0)\big)^{ 3 / 2} \equiv \mu_0^{3 / 2} $; per assunzione, $N$ \`e costante, quindi ad una certa temperatura $T \gtrsim 0$, si ha:
\[
	N' = N'_0 \left[1 + \frac{\pi^2}{8} \left(\frac{\kappa _B T}{\mu' }\right)^2\right]
\] 
con $\mu ' \equiv \mu (T)$. Ora, imponendo conservazione del numero di particelle, deve valere $N' / N=N' / N_0 \stackrel{!}{=} 1$, ossia:
\[
\left[ 1+ \frac{\pi^2}{8}\left(\frac{\kappa _B T}{\mu '}\right) ^2 \right] \left(\frac{\mu '}{\mu_0 }\right) ^{3 / 2} =1
\] 
Si risolve per $\mu '$, sostituendo $\mu' $ con $\mu_0$ al denominatore (che porta errore oltre secondo ordine) e si sviluppa in serie:
\begin{equation}
\mu ' = \frac{\mu_0 }{\left[ 1 + \pi^2 / 8 (\kappa _B T / \mu ')^2 \right] ^{2 / 3} }\simeq \mu_0 \left[ 1 - \frac{\pi^2}{12} \left(\frac{\kappa _B T}{\mu_0 }\right) ^2 \right] 
\end{equation}
Sostituendo in $\Omega $, si possono trovare entropia e calore specifico:
\[
\begin{split}
	&S = -\left(\frac{\partial \Omega }{\partial T}\right)_{V,\mu }    = \frac{4 \pi \sqrt{2} V g m^{3 / 2} }{h^3}\frac{\pi^2 }{3}\mu ^{1 / 2} \kappa _B^2 T \simeq \frac{\pi^2}{2} N \kappa _B \frac{T}{T_f}\\
	&c_V = T \left(\frac{\partial S}{\partial T} \right) _V \simeq \frac{\pi^2}{ 2} N\kappa _B \frac{T}{T_f}
\end{split}
\] 
Stima per calore specifico in accordo con dati sperimentali, ma stime migliori si ottengono per $m\to m^*$. 
La massa effettiva si pu\`o misurare tramite campi magnetici ed \`e legata a frequenza di ciclotrone: $\omega_c = e \lvert \mathbf{H}  \rvert / (m^* c)$.

\subsubsection{Paramagnetismo di Pauli}

Immergendo sistema in campo magnetico, la degenerazione dovuta allo spin \`e rotta e si ha $\mathscr{E}_{\pm} = \frac{p^2}{2m} \pm \mu B$, con $\mu $ momento magnetico della particella.
Questo termine aggiuntivo trasla energia e va considerato in calcolo del numero di occupazione e densit\`a di stati.

Se $T= 0 , B=0$, densit\`a di popolazioni con spin up e spin down \`e uguale per entrambe e pari a $\rho (\mathscr{E}) / 2$ e lo spazio delle fasi \`e riempito dall'origine a $E_f$. 

Quando $B \neq 0 $, questo interagisce con momento magnetico intrinseco delle particelle e l'energia per gli stati dei due tipi di spin cambia secondo energia di Zeeman:
\[
	E_\uparrow = E - \mu  B \hspace{.1cm} ; \hspace{.2cm} E _\downarrow = E + \mu B
\] 
quindi stati $\uparrow$ avranno energia minore e quindi saranno pi\`u popolati, nonostante l'energia di Fermi rimanga invariata.

La differenza di popolazione $\Delta N$ si pu\`o approssimare con $ E_\downarrow\rho (E_f) / 2 - E_\uparrow\rho (E_f) / 2  = \Delta E \rho (E_f) / 2$, assumendo $\rho (\mathscr{E})$ uniforme.
In questo modo:
\[
\Delta N = 2\mu B \frac{\rho (E_f) }{2} = 2\mu  _B \frac{4 \pi V m^{3 / 2} }{\sqrt{2} h^3}\sqrt{E_f} = \frac{3}{2} \frac{\mu  B }{E_f}N
\] 
con $N$ calcolato sulla densit\`a per $B=0 , T=0$:
\[
N = \int_{0} ^{E_f} \rho  (\mathscr{E}) \ d \mathscr{E} = \int_{0} ^{E_f} \frac{4\pi \sqrt{2} V m^{3 / 2} }{h^3} \mathscr{E}^{1 / 2}  \ d \mathscr{E} = \frac{4\pi \sqrt{2} V m^{3/2} }{h^3} \frac{2}{3} E_f^{ 3 / 2} 
\] 
Questa forma di paramagnetismo, con magnetizzazione
\begin{equation}
	M = \mu  \Delta N = \frac{3N}{2} \frac{\mu ^2}{E_f}B
\end{equation}
\`e detta \textbf{paramagnetismo di Pauli}.

\subsubsection{Emissione termoionica}

Emissione di elettroni indotta termicamente nei conduttori. Si fanno le seguenti ipotesi:
\begin{itemize}
	\item il metallo \`e una buca di potenziale alta $W$;
	\item il rate di emissione \`e basso $\Rightarrow $ numero di elettroni nel metallo \`e $\sim$ costante;
	\item \`e presente campo elettrico esterno che rimuove elettroni emessi (altrimenti il rate sarebbe nullo);
	\item \`e il compleanno di Stefano (tanti auguri).
\end{itemize}
Affinch\'e vi sia effettiva emissione, si impone che in direzione $z$ (arbitraria) valga $p_z > \sqrt{2m W} $. 
Se $dt$ tempo di emissione, si deve avere $dz = v_z dt = (p_z / m) dt$.

Il numero di elettroni emessi sar\`a proporzionale all'integrale del numero di occupazione sulla parte di spazio delle fasi in cui si pu\`o verificare l'emissione;
dividendo per $dt$, si ottiene il rate.
Definendo rate per unit\`a di superficie, si taglia l'integrale su $dS = dxdy$:
\begin{equation*}
			dR = \frac{g \overline{n} d\Gamma}{dS dt} = \frac{2 \overline{n}}{h^3} \frac{dxdy v_z dt d^3 p}{dS dt}= \frac{2\overline{n}}{h^3}\frac{p_z }{m} d^3p
\end{equation*}
quindi:
\begin{equation*}
  			\begin{split}
			 R &= \frac{2}{h^3} \int_{\sqrt{2m W} } ^{+\infty} \frac{p_z dp_z}{m} \iint_{\mathbb{R}^2}  \frac{dp_x dp_y}{\exp \left[ (p^2 / 2m - \mu ) / \kappa _ B T \right] + 1} \\
					      &= \frac{2}{h^3} \int_{\sqrt{2m W} } ^{+\infty} \frac{p_z dp_z}{m} \int_{0} ^{+\infty} \frac{2\pi p ' dp'}{\exp \left[ \big(p'^2 + p_z^2)/ 2m - \mu \big) / \kappa _ B T \right] + 1} \\
					      &= \frac{4\pi \kappa _B T }{h^3}\int_{\sqrt{2mW} } ^{+\infty} p_z\log \left[ 1+ \exp\left(\frac{\mu  - p_z^2 / 2m}{\kappa _B T}\right)  \right] dp_z  =  \frac{4 \pi m \kappa _B T }{h^3} \int_{W} ^{+\infty} d \mathscr{E}_z \ \log \left[ 1+ \exp \left(\frac{\mu - \mathscr{E}_z}{\kappa _B T}\right)  \right] 
			\end{split}
\end{equation*}
Infine, bisogna imporre che $W - \mu  \gg \kappa _B T$\footnote{Si richiede che il potenziale di estrazione sia molto maggiore del potenziale chimico, altrimenti elettroni fuggirebbero spontaneamente.}, per cui $\exp\left(\frac{\mu  - \mathscr{E}_z}{\kappa _B T}\right) \ll 1$; allora sviluppando:
\begin{equation}
	R \simeq \frac{4\pi m \kappa _B T }{h^3}\int_{W} ^{+\infty} d \mathscr{E}_z\ \exp\left( \frac{\mu  - \mathscr{E}_z}{\kappa _B T}\right) = \frac{4\pi m \kappa _B ^2 T^2}{h^3} \exp \left(\frac{\mu  - W}{\kappa _B T}\right) 
\end{equation}
Dal rate, si ottiene la densit\`a di corrente:
\begin{equation}
	J = eR = \frac{4 \pi e m_e \kappa _B ^2 T^2}{h^3} \exp \left(\frac{\mu  - W}{\kappa _B T}\right) 
\end{equation}
\subsubsection{Effetto fotoelettrico}

Gas di elettroni nel metallo colpito da fotoni di energia $hv$. La condizione in direzione di fuga diventa: $\frac{p^2_z}{2m} + hv > W$, quindi:
\[
R = \frac{4 \pi m \kappa _B T}{h^3} \int_{W- hv} ^{+\infty}  d \mathscr{E}_z \log \left[ 1 + \exp \left(\frac{\mu - \mathscr{E}_z}{\kappa _B T}\right)  \right] 
\] 
Non si pu\`o sviluppare in serie come prima perch\'e potrebbe essere $hv \sim W$. Si prende $x = \frac{\mathscr{E}_z - W + hv}{\kappa _B T}$ e $hv_0 = W - \mu  \approx W - E_f = \phi $:
\[
R = \frac{4 \pi m \kappa _B^2 T^2}{h^3}  \int_{0} ^{+\infty} dx \ \log \left[ 1+ \exp \left(\frac{h(v-v_0)}{\kappa _B T} - x\right)  \right] 
\] 
Integrali del genere hanno soluzioni della forma
\[
\int_{0} ^{+\infty} dx \ \log \left(1 + e ^{ \delta  - x } \right) = f_2 (e^\delta )
\] 
Per trovare espressione di $f_2$ si considerano i casi limite, rispettivamente radiazione molto energetica e poco energetica:
\[
\begin{split}
	&h(v-v_0) \gg \kappa _B T \Rightarrow  e^\delta  \gg 1 \Rightarrow  f_2 (e^\delta ) \simeq \frac{\delta ^2}{2}\\
	& v< v_0 \Rightarrow  h|v-v_0| \gg \kappa _B T \Rightarrow e^\delta \ll 1 \Rightarrow  f_2 (e^\delta ) \simeq e^\delta 
\end{split}
\] 
Nel primo caso, l'espressione della corrente \`e 
\begin{equation}
	J \simeq \frac{me}{\hbar } (v-v_0)^2
\end{equation}
cio\`e la corrente non ha dipendenza dalla temperatura perch\'e l'emissione degli elettroni \`e prevalentemente dovuta all'incisione di fotoni ad alta energia.

Nel secondo caso, invece:
\begin{equation}
	J \simeq \frac{4 \pi m e \kappa _B^2 T^2}{h^3} \exp \left(\frac{hv - \phi }{\kappa _B T}\right) 
\end{equation}
che \`e una correzione alla corrente termoionica.
\subsection{Gas di Bose}
\subsubsection{Condensato di Bose-Einstein}
Bosoni descritti dalla statistica
\[
\overline{n} (\mathscr{E}) = \frac{1}{\exp\left[ (\mathscr{E}-\mu ) / \kappa _B T \right] - 1}
\] 
e deve valere $\mu  < 0$. Nel limite $\mu \to 0$:
\begin{itemize}
	\item a $T $ costante, esiste massimo numero di particelle consentito, sopra cui si dovrebbe avere $\mu >0$;
	\item a $N$ fissato, esiste limite inferiore per $T$, imposto sempre dal segno di $\mu $.
\end{itemize}
Ci si aspetterebbe, per\`o, di poter osservare un gas a qualsi temperatura con qualsiasi numero di particelle. 
Si cerca il motivo di questo risultato.

Si calcola $N$ a $T$ costante passando da somma a integrale:
\[
N = \sum_{q}^{} \overline{n}_q = \frac{4\pi V g \sqrt{2} m^{3 / 2} }{h^3}\int_{0} ^{+\infty} \frac{\mathscr{E}^{1 / 2} d\mathscr{E}}{\exp\left[ (\mathscr{E}-\mu ) / \kappa _B T \right] - 1}
\] 
Si manda $\mu \to 0$ e si prende $\mathscr{E} = \kappa _B T x$, quindi:
\[
N = \frac{4\pi V g \sqrt{2} m^{3 / 2} }{h^3} (\kappa _B T)^{3 / 2} \int_{0} ^{+\infty}  \frac{\sqrt{x} dx}{e^x - 1}
\] 
L'integrale ha soluzione generale: $\int_{0} ^{+\infty} \frac{x^n dx}{e^x - 1} = \Gamma(n+1) \zeta(n+1)$;
il risultato corrisponderebbe alle aspettative se l'integrale divergesse, mentre $\int_{0} ^{+\infty} \frac{\sqrt{x} dx}{e^x - 1} =\frac{\sqrt{\pi} }{2}2.612= 2.31$. 
Si definiscono, quindi, una densit\`a critica e una temperatura critica, rispettivamente, sopra cui e sotto cui sorgono problemi:
\begin{equation}
	\begin{split}
		& \left(\frac{N}{V}\right) _c = \frac{2.612}{\Lambda ^3}\\
		& T_c = \frac{1}{2.31} \left(\frac{N}{V}\right) _c^{2 / 3} \frac{h^2}{(4\pi \sqrt{2} )^{2/3} m\kappa _B }
	\end{split}
\end{equation}
L'errore \`e dovuto nel passaggio al continuo: per quanto la condizione $\kappa _B T \gg \frac{1}{2m} (h / L)^2\Rightarrow N^{2 / 3} \gg 2.31 \frac{(4\pi \sqrt{2} )^{2/3} }{2} \approx 7.8$ sia solitamente verificata, si ha, contemporaneamente al passaggio al continuo, anche $\lim_{\mathscr{E} \to 0} \rho (\mathscr{E}) =0$, quindi nei conti precedenti, si sono trascurate le particelle nello stato fondamentale.

Queste, per\`o, tendono a popolare sempre pi\`u lo stato fondamentale pi\`u si sale sopra la densit\`a critica a $T$ fissato, o si scende sotto temperatura critica a $N$ fissato.

Quello che si verifica \`e una \textbf{transizione di fase} in uno stato conosciuto come \textbf{condensato di Bose-Einstein}, con numero di particelle nello stato fondamentale dato da:
\begin{boxenv}[]
\begin{equation}
	\overline{n}_0 = \frac{1}{\exp(-\mu / \kappa _B T) - 1}
\end{equation}
\end{boxenv}
\noindent che coerentemente diverge per $\mu \to 0$.

Valore di $N$ calcolato prima, in realt\`a, \`e $N^* = N - N_0 = N (T / T_c)^{3 / 2} $, da cui:
\begin{equation}
	N_0 = N \left[1 - \left(\frac{T}{T_c}\right) ^{3 / 2} \right] 
\end{equation}
\begin{osservazione}
	In due dimensioni non vi pu\`o essere condensazione perch\'e la densit\`a di stati \`e costante in energia; infatti per $p^2=p_i^2+p_j^2$:
	\[
	d \mathscr{E} = \frac{p dp }{m} \Rightarrow dp_i dp_j = 2 \pi p dp = 2\pi m d\mathscr{E}
	\] 
\end{osservazione}
\noindent Per $\mu \to 0$ e per $T< T_c$, ricordando che le particelle nel condensato hanno energia nulla, l'energia media \`e:
\begin{equation}
		E = \int_{0} ^{+\infty}\frac{\rho  ( \mathscr{E}) d \mathscr{E}}{\exp(\mathscr{E} / \kappa _B T) - 1} = \frac{4\pi V g \sqrt{2} m^{3 / 2} }{h^3} (\kappa _BT)^{ 3 / 2} \kappa _B T \int_{0} ^{+\infty} \frac{x ^{3 / 2} dx}{e^x - 1}\approx 0.77 \cdot N\kappa _B T \left(\frac{T}{T_c}\right) ^{3 / 2} \propto T^{5 / 2} 
\end{equation}
da cui la capacit\`a termica \`e:
\begin{equation}
	c_V \approx 1.9 \cdot  N\kappa _B \left(\frac{T}{T_c}\right) ^{3 / 2} = 1.9 \cdot  N^* \kappa _B \propto T^{3 / 2} 
\end{equation}
Per $T$ grande, per\`o, $c_V $ deve tenere a $3N\kappa _B / 2$; l'andamento trovato sopra cambia bruscamente per $T \sim T_c$.

Infine, la pressione si ottiene a partire dalla formula valida per tutti i gas perfetti a dispersione quadratica:
\begin{equation}
	P = \frac{2}{3} \frac{E}{V} \approx 0.513 \cdot  \frac{N\kappa _B T}{V} \left(\frac{T}{T_c}\right) ^{3/2} 
\end{equation}
Essendo calcolata a $\mu (P,T) = 0$, questa identifica la curva di coesistenza tra stato gassoso non degenere e del condensato di Bose-Einstein.

\subsubsection{Oscillatori in una scatola}

L'Hamiltoniano di singolo oscillatore \`e:
\begin{equation}
	H = \frac{\mathbf{p} ^2}{2m} + \frac{m\omega ^2 \mathbf{x} ^2}{2}		
\end{equation}
L'energia media classica \`e data dal principio di equipartizione ed \`e $E = 3N\kappa _B T$, mentre l'energia quantizzata per singola particella \`e $E_n=(n+1 / 2) \hbar \omega$.

La funzione di partizione di singola particella, quindi, \`e:
\begin{equation}
	Z_1 = \sum_{n=0}^{+\infty} \exp \left[ - \left(n + \frac{1}{2}\right) \frac{\hbar \omega}{\kappa _B T} \right] = \frac{1}{2 \operatorname{sinh}(\hbar  \omega / \kappa _B T) }
\end{equation}
Da questa, si ottiene energia media per oscillatore:
\begin{equation}
	E = \frac{1}{2}\hbar  \omega + \frac{\hbar  \omega}{\exp(\hbar \omega / \kappa _B T) - 1} = \left(\overline{n} + \frac{1}{2}\right) \hbar \omega
\end{equation}
Dall'ultima, si ha che gli oscillatori seguono la statistica di Bose-Einstein con $\mu =0$:
\begin{equation}
	\overline{n}(\omega) = \frac{1}{\exp(\hbar \omega / \kappa _B T) - 1}
\end{equation}
\subsubsection{Corpo nero}
Si usa gas di oscillatori come modello per campo elettromagnetico: si vede campo come un oscillatore con diversi livelli popolati secondo la statistica $\overline{n}$, o come unico livello popolato da $\overline{n}$ fotoni.

Essendo $\mu  =0 $, non esiste legge di conservazione per numero totale di quasiparticelle: se ne possono creare e distruggere a piacimento.

Inserendo gas in scatola con pareti perfettamente assorbenti (quindi con condizione di annullamento ai bordi), la legge di dispersione \`e $\omega = ck$, quindi $\mathscr{E} = cp$.

Allora elemento di spazio delle fasi \`e:
\begin{equation}
	d\Gamma = 2V \frac{4 \pi p^2 dp}{h^3} = V \frac{\omega^2 d\omega }{\pi^2 c^3} \rho (\omega) d\omega
\end{equation}
dove il fattore $2$ \`e perch\'e esistono due modi possibili di propagazione, linearmente indipendenti fra loro, per il campo (circolare destra e circolare sinsitra), associata allo spin del fotone.

La densit\`a di energia per unit\`a di volume \`e ottenuta moltiplicando energia del singolo fotone per numero di occupazione e densit\`a (a meno di $V$), trascurando energia $\hbar  \omega / 2$ che non \`e misurabile e farebbe divergere energia totale. 
Quindi:
\begin{boxenv}[]
\begin{equation}
	u(\omega ) d\omega = \frac{\rho (\omega)}{V} \overline{n}(\omega) \hbar \omega \ d\omega= \frac{\hbar  \omega^3}{\exp(\hbar \omega / \kappa _B T ) - 1} \frac{d\omega}{\pi^2 c^3}
\end{equation}
\end{boxenv}
\noindent Questa \`e la \textbf{legge di Planck} per radiazione di corpo nero. 

Integrando sulle frequenze, si ha energia totale del corpo nero in funzione della temperatura:
\begin{equation}
	E = \int_{0} ^{+\infty} u(\omega) d\omega = \sigma T^4
\end{equation}
con $\sigma $ \textbf{costante di Stefan-Boltzmann}. Il calore specifico \`e 
\begin{equation}
	c_ V = \frac{\partial E}{\partial T} \propto T^3		
\end{equation}
e diverge con $T$  perch\'e $N$ non \`e fissato e $\omega$ non ha limite superiore, quindi si possono aggiungere quasiparticelle con energia grande a piacere. 
Il limite classico per alte $T$ non si osserva perch\'e gli oscillatori trattati sono infiniti.

\newpage

\section{Vibrazioni, trasporto e rumore nei cristalli}
\subsection{Modelli di Einstein e Debeye}
Si modellano atomi di un cristallo come oscillatori armonici indipendeti; classciamente $c_V = 3 N \kappa _B$ (in base a quanto trovato nel capitolo precedente)
che \`e la \textbf{legge di Dulong-Petit}.

Sperimentalmente, però, si vede che $c_V \to 0 $ per $T \to 0$.
Per gas di oscillatori, si \`e visto che:
\[
E = E_0 + 3N \frac{\hbar \omega}{\exp(\hbar \omega / \kappa _B T) -1 }
\] 
che soddisfa $c_V \to 0$ per $T \to 0$ e $c_V \to 3N \kappa _B $ per $T \to + \infty$. 
Questo \`e il \textbf{modello di Einstein.} 
Anche questo \`e in disaccordo con gli esperimenti perch\'e $c_V \to 0$ proporzionalmente a $T^3$, non esponenzialmente.
L'errore \`e trattare oscillatori indipendenti: cos\`i facendo, $\omega$ non dipende dalla densit\`a, quindi l'energia non dipende dal volume $\Rightarrow $ compressibilit\`a infinita.

Nel \textbf{modello di Debeye}, gli oscillatori sono accoppiati, quindi ognuno osciller\`a con una propria frequenza $\omega_\alpha $ e influenzer\`a gli oscillatori vicino. In questo modello, i \textit{fononi} saranno i modi di oscillazione collettiva degli atomi.
La lunghezza d'onda dei fononi \`e \textit{limitata superiormente} dalla dimensione del solido e \textit{inferiormente} dal passo del reticolo del cristallo.

\subsubsection{Modello di Debeye continuo}

Si tratta modello di Debeye per $k\to 0$ perch\'e si \`e interessati al comportamento del modello per basse temperature\footnote{Bassi $k$ implicano basse energie che implicano basse temperature.}, essendo in questo regime l'andamento da correggere. 
Inoltre, errori a temperature pi\`u alte, gli stati sarebbero molto poco popolati (visto che seguono una Bose-Einstein), quindi compiere degli errori ad alte $T$ non cambia il risultato.
In questa approssimazione \`e anche possibile trattare il reticolo come continuo: se $k\to 0 \Rightarrow  \lambda \to \infty$, quindi non si \`e sensibili alla discretizzazione del cristallo.

Per mezzi continui: $\partial _t \rho  + \nabla (\rho  \mathbf{v} ) = 0$ e $\rho  \partial _t \mathbf{ v}  + \nabla P = 0$\footnote{La prima \`e l'equazione di continuit\`a per la corrente di massa, mentre la seconda deriva da $F=ma$. Indicando con $\mathcal{F}$ una forza per unit\`a di volume, l'uguaglianza soddisfatta \`e: $\mathcal{F} = \rho \partial _t v$. Infine, usando $\mathcal{F}=-\nabla P$, si ottiene proprio $\nabla P +\rho \partial _t v = 0$.}.
Se pressione e densit\`a si discostano di poco dall'equilibrio, cio\`e $P = P_0 + P'$ e $\rho  = \rho _0 + \rho '$: $\nabla P' + \rho _0 \partial _t \mathbf{v} = 0$ e $\partial _t \rho ' + \rho _0 \nabla \cdot \mathbf{v} =0$.
Si assume che $\rho $ dipenda solo da $P$; sviluppando: $\rho \simeq \rho (P_0) + P'\partial _P \rho $, con $\rho _0 \equiv \rho (P_0)$. Quindi deve valere $\rho ' = P'\partial _P \rho  \equiv P ' / c^2$, dove $c^2 \equiv \partial _\rho P = 1 / k\rho $, in cui $k = k_T = k_S$\footnote{Avendo assunto indipendenza dalla temperatura nello sviluppo di $\rho $, non ha senso distinguere tra $k _T$ e $k_S$.} \`e la compressibilit\`a.

Le equazioni differenziali diventano:
\begin{equation}
	\begin{cases}
		\displaystyle c^2 \nabla  \rho  ' + \rho _0 \frac{\partial \mathbf{v} }{\partial t} = 0 \\
		\\
		 \displaystyle \frac{\partial \rho '}{\partial t} + \rho _0 \nabla \cdot \mathbf{v} = 0 
	\end{cases} \Longrightarrow \frac{1}{c^2}\frac{\partial ^2 \rho '}{\partial t^2} - \nabla ^2 \rho ' = 0
\end{equation}
L'equazione di d'Alambert \`e ottenuta sottraendo la divergenza della prima e la derivata parziale temporale della seconda.
Una base per le soluzioni sono le onde piane con
\begin{equation}
	\rho '(t,\mathbf{x} ) \propto e^{- (\omega t - \mathbf{q} \cdot \mathbf{x} )} , \ \omega = cq
\end{equation}
I fononi sono bosoni con legge di disperione uguale a quella dei fotoni in una scatola.
La loro densit\`a spettrale \`e identica a meno di fattore di molteplicit\`a.
Nel conteggio, bisogna includere modi di propagazione trasversi e propagazione longitudinale, la quale avr\`a, in generale, velocit\`a diversa:
\begin{equation}
	\begin{split}
		u(\omega) d\omega &= \frac{\hbar \omega}{\exp(\hbar \omega/\kappa _B T) - 1} \rho (\omega) d\omega = \left(\frac{1}{c_L^3  }+\frac{2}{c_T ^3}\right) \frac{V\hbar \omega^3}{\exp(\hbar \omega / \kappa _B T)-1}\frac{d\omega}{2\pi^2}\\
		&= \frac{3V}{2\pi^2 \overline{c}^3} \frac{\hbar  \omega^3 d\omega}{\exp(\hbar \omega / \kappa _B T ) -1}
	\end{split}
\end{equation}
Nell'ultima uguaglianza, si \`e considerata velocit\`a media $\overline{c}$ per semplificare trattazione. 
Nell'integrare questa densit\`a di energia, l'integrale non si estende pi\`u fino a infinito come per campo em, ma esiste un'energia massima associata ad una frequenz a massima $\omega_\text{max}$, a sua volta associata ad un vettore d'onda massimo $q_D$ che \`e il \textbf{vettore d'onda di Debeye}. Questo \`e presente perch\'e, al contrario dei fotoni del campo elettromagnetico, i fononi sono oscillazioni di un reticolo cristallino, quindi non possono avere una lunghezza d'onda inferiore al passo reticolare.

Si calcola energia totale per basse temperature, ossia $\kappa _B T \ll \hbar \omega_\text{max} = \frac{hc}{\lambda _\text{min}}$; \`e chiaro che in questo regime, si pu\`o approssimare il limite superiore dell'energia come infinito e si ottiene lo stesso integrale ottenuto per il campo elettromagnetico:
\begin{equation}
	E = \int_{0} ^{+\infty} u(\omega) \ d\omega \propto T^4 \Rightarrow c_V \propto T^3
\end{equation}
Il risultato ottenuto \`e in accordo con l'andamento misurato sperimentalmente.

Per studiare il limite opposto (alte temperature), si ha il problema che $\rho (\omega)$ usata finora non \`e pi\`u valida perch\'e \`e stata ricavata in assunzione di $k\to 0$; 
per temperature alte, ci si avvicina a lunghezze d'onda dell'ordine del passo reticolare, quindi l'assunzione di mezzo continuo non \`e pi\`u valida.
Prendendo la solita $\omega=ck$ lineare derivante da tale approssimazione, si finirebbero per contare pi\`u modi di quelli effettivamente presenti.

Per continuare a usare la stessa $\rho (\omega)$, ma contare il numero giusto di modi si ridefinisce la frequenza massima, imponendo che il numero sia sempre $3N$:
\begin{equation}
	3N = \int_{0} ^{\omega_\text{max}}  \rho (\omega) \ d\omega= \frac{12 \pi V}{(2\pi \overline{c})^3}\int_{0} ^{\omega_\text{max}} \omega^2 \ d\omega  \implies \omega_\text{max}=2\pi \overline{c} \left(\frac{3N}{4\pi V}\right) ^{ 1/3} 
\end{equation}
dove $3N$ sono le possibili degenerazioni, avendo tre possibili modi di propagazione (uno longitudinale e due trasversali) e $N$ oscillatori accoppiati. Da questa, si ottiene la temperatura di Debeye come $\hbar \omega_\text{max} = \kappa _B \theta _D$ e l'energia \`e:
\begin{equation}
	\begin{split}
		E &= \int_{0} ^{\kappa _B \theta _D} \mathscr{E} \rho (\mathscr{E}) \overline{n}(\mathscr{E}) \ d\mathscr{E} = 9 N\kappa _B T \left(\frac{T}{\theta _D}\right) ^3 \int_{0} ^{\theta _D / T} \frac{x^3}{e^x - 1} \ dx \\
		  &\simeq 9 N\kappa _B T \left(\frac{T}{\theta _D}\right) ^3 \int_{0} ^{\theta _D / T} \frac{x^3}{x}=3N\kappa _B T
	\end{split}
\end{equation}
in accordo con Dulong-Petit. Si nota che avendo preso $\theta _D / T \ll 1$, si \`e potuto sviluppare $e^x -1 \simeq x$.
\subsection{Equazione del trasporto di Boltzmann}
Se $f$ distribuzione nello spazio delle fasi per sistema conservativo, il teorema di Liouville permette di scrivere:
\[
f(t+ dt , \mathbf{x} + d\mathbf{x} , \mathbf{v} +d \mathbf{v} ) = f(t,\mathbf{x} ,\mathbf{v} )
\] 
Si cerca quantit\`a che tenga conto del lasciare poco fuori equlibrio il sistema; questa deve essere t.c.:
\begin{itemize}
	\item  sia proporzionale alla distanza del sistema dall'equilibrio;
	\item faccia tendere sistema verso equilibrio;
	\item tenga conto del tempo caratteristico $\tau _c$ degli urti tra particelle che ristabiliscono l'equilibrio.
\end{itemize}
La forma pi\`u semplice \`e quella della derivata collisionale:
\begin{equation}
	\left(\frac{\partial f}{\partial t} \right) _\text{coll} = - \frac{f-f_0}{\tau _c}
\end{equation}
con $f_0$ distribuzione all'equilibrio. In assenza di capi, questa ha soluzione
\begin{equation}
	f(t) - f_0 = \left[ f(0) - f_0 \right] e^{-t  / \tau _c} 
\end{equation}
Il teorema di Liouville si generalizza come:
\begin{equation}
	f(t + dt, \mathbf{x} + d\mathbf{x} , \mathbf{v} + d\mathbf{v} ) - f(t, \mathbf{x} , \mathbf{v} ) = \left(\frac{\partial f}{\partial t} \right) _\text{coll} dt
\end{equation}
Da questa si ottiene l'\textbf{equazione del trasporto di Boltzmann}:
\begin{boxenv}[]
\begin{equation}
	\frac{\partial f}{\partial t} + \frac{d \mathbf{r} }{d t} \cdot \nabla _\mathbf{r} f + \frac{d \mathbf{v} }{d t} \cdot \nabla _\mathbf{v} f = \left(\frac{\partial f}{\partial t} \right) _\text{coll}
\end{equation}
\end{boxenv}
\noindent Le soluzioni stazionarie si ottengono potenndo $\partial _t f = 0$.
Unidimensionalmente, lungo $x$, si ha:
\[
v_x \frac{\partial f }{\partial x}  + \dot{v}_x \frac{\partial f}{\partial v_x}  = \frac{f- f_0}{\tau _c}\Rightarrow f = f_0 - \tau _c \left(a_x \frac{\partial f}{\partial v_x} + v_x \frac{\partial f}{\partial x} \right) 
\] 
con $a_x = \dot{v}_x$. Si assume sistema poco fuori equilibrio, cio\`e $f \simeq f_0 + \Delta $, quindi, compiendo errore al secondo ordine nelle derivate\footnote{Questo si pu\`o fare perch\'e il termine con le derivate \`e gi\`a una correzione a $f$, quindi aggiungendo $\Delta $ nelle derivate, si otterrebbe una correzione al secondo ordine, che si trascura.}:
\begin{equation}
	f = f_0 - \tau _c \left(a \frac{\partial f_0}{\partial v_x}  + v \frac{\partial f_0}{\partial x} \right) 
\end{equation}
\subsubsection{Conducibilit\`a elettrica}
Gas di elettroni portato di poco fuori equilibrio da debole campo elettrico.
Si usa equazione del trasporto per calcolare la conducibilit\`a elettrica.
La trattazione \`e semi-classica perch\'e si tratteranno elettroni come particelle, sotto l'assunzione che la lunghezza d'onda
di de Broglie degli elettroni sia molto pi\`u piccola della lunghezza caratteristica su cui variano statistica e potenziale elettrico.
Da argomenti di parit\`a, si mostra che la correzione alle autoenergie svanisce al primo ordine in teoria delle perturbazioni, cio\`e per piccoli campi,
la struttura dei livelli energetivi \`e invariata.
Si \textit{assume che gli elettroni siano localmente distribuiti secondo la Fermi-Dirac}:
\begin{equation}
	f_0 = \frac{1}{\exp\left[ (\mathscr{E}-\mu ) /\kappa _B T  \right] +1}
\end{equation}
Inoltre, \textit{si assume che potenziale chimico e temperatura siano uniformi nello spazio}: $\partial _\mathbf{x} \mu = \partial _\mathbf{x} T = 0$.
Si ricorda ancora che l'energia \`e indipendente da $\mathbf{x} $ al primo ordine perturbativo.
Si ha:
\[
	\partial _x f_0 = \frac{\partial f_0}{\partial \mu } \frac{\partial \mu }{\partial x} + \frac{\partial f_0}{\partial T} \frac{\partial T}{\partial x} = 0 \hspace{.1cm} ; \hspace{.2cm} \partial _{v_x} f_0 = \frac{\partial f_0}{\partial \mathscr{E}} \frac{\partial \mathscr{E}}{\partial v_x} = mv_x \frac{\partial f_0}{\partial \mathscr{E}} 
\] 
Usando $a = eE / m$, si trova:
\begin{equation}
	f = f_0- \tau _c e E v_x \frac{\partial f_0}{\partial \mathscr{E}} 
\end{equation}
Per basse temperature, cio\`e $T\ll T_f$, la Fermi-Dirac \`e un gradino, quindi $\partial _\mathscr{E} f_0 = -\delta  (\mathscr{E}-E_f)$\footnote{Questo perch\'e la distribuzione \`e un gradino per $T\sim 0 $ e decresce solamente in $\mathscr{E} = E_f$; il segno negativo \`e perch\'e decresce.};
la densit\`a di corrente \`e:
\begin{equation}
	\begin{split}
		&dJ = (2s+1) ev_x f \frac{d \Gamma}{d ^3 x}  = 2ev_x f \frac{d^3 p}{h^3}\\
		&\Rightarrow J = 2e \int_{-\infty} ^{+\infty}  v_x f \frac{d^3 p}{h^3} = 2e \int_{-\infty} ^{+\infty}  v_x f_0 \frac{d^3 p}{h^3} + 2e^2 E \tau _c \int_{-\infty} ^{+\infty}  v_x^2 \delta (\mathscr{E}-E_f) \frac{d^3p}{ h^3} 
	\end{split}
\end{equation}
Il primo integrale svanisce per integranda dispari in $v_x$ (essendo la Fermi-Dirac $f_0$ pari perch\'e dipendente da $p^2 / 2m$)\footnote{Ci si aspetta che questo faccia $0$ perch\'e \`e la parte in assenza di campo elettrico, quindi la corrente netta deve essere nulla.}, quindi:
\begin{equation}
	\begin{split}
		J &= 2e^2 E \tau _c \int v^2 \cos^2 \theta  \delta (\mathscr{E} -E_f) \frac{2\pi p ^2 dp \sin \theta  d\theta }{h^3}\\
		  & = \frac{2e^2 E \tau _c }{4\pi^2 \hbar ^3} \int \frac{2\mathscr{E}}{m} \cos^2 \theta \delta (\mathscr{E}-E_f) p^2 \ dp d\cos \theta = \frac{2e^2 E \tau _c}{3 m \pi^2 \hbar ^3} \int \mathscr{E} \delta (\mathscr{E} - E_f) p^2 \ dp\\
		  &= \frac{2 e^2 E \tau _c}{3 \pi ^2 \hbar ^3} \sqrt{ 2m}  \int \mathscr{E }^{3 / 2}  \delta (\mathscr{E}-E_f) \ d\mathscr{E} = \frac{2e^2 E \tau _c}{3 \pi^2 \hbar ^3} \sqrt{2m}  E_f ^{3 / 2} = \frac{e^2 \tau _c N}{mV } E
	\end{split}
\end{equation}
con $E $ campo elettrico, dove nell'ultima uguaglianza si \`e sostituito $E_f$ ottenuto da $\frac{N}{V}=\int_{0} ^{E_f} \rho (\mathscr{E}) d  \mathscr{E}$. Da questa, si ottiene la \textbf{conducibilit\`a elettrica}:
\begin{equation}
	\sigma  = \frac{J}{E} = \frac{e^2 \tau _c}{m} \frac{N}{V}
\end{equation}
\subsubsection{Conducibilit\`a termica}
\textbf{Conduciblit\`a termica} $\kappa _T$ \`e definita da:
\begin{equation}
	J_Q = - \kappa _T \frac{d T}{d x} 
\end{equation}
Essendo questa una corrente di calore, quindi di energia, si potrebbe pensare di ispirarsi al conto precedente, scrivendo $J_Q = \int \mathscr{E} f v_x \frac{d^3 p}{h^3}$, ma questo coincide col flusso di energia.
Essendo interessati al solo trasporto di calore e non anche al moto stesso delle particelle, si toglie il potenziale chimico all'energia totale, visto che $\delta Q = T dS = dE -\mu dN$; per singola particella, si divide per $dN$, quindi si ottiene $\mathscr{E}-\mu $.
Allora:
\begin{equation}
	J_Q = \int (\mathscr{E} -\mu ) f v_x \frac{d^3 }{h^3}
\end{equation}
In questo caso, non si pu\`o avere $T $ uniforme altrimenti non si avrebbe corrente termica.
In assenza adi campi, si ha $a=0$, quindi l'equazione del trasporto \`e:
\begin{equation}
	f = f_0 - \tau _c v_x \frac{\partial f_0}{\partial T}  \frac{\partial T}{\partial x} 
\end{equation}
Non si pu\`o fare neanche pi\`u uso dell'approssimazione della Fermi-Dirac come gradino perch\'e l'integrale farebbe $0$;
tramite conti con espressione completa della statistica, si ha:
\begin{equation}
	J_Q = -\frac{d T}{d x} \kappa _B ^2 T \frac{\pi^2}{3}\frac{N}{V}\frac{\tau _c}{m}
\end{equation}
da cui:
\begin{equation}
	\kappa _T = \frac{\pi^2}{3} \frac{N}{mV}\kappa _B ^2 T \tau _c
\end{equation}
Il rapporto tra le due conduciblit\`a, a meno di $T$, dipende solo da costanti fondamentali;
la relazione che esprime tale rapporto \`e la \textbf{relazione di Wiedmanan-Franz}:
\begin{equation}
	\frac{\kappa _T}{T \sigma } = \frac{\pi^2}{3} \left(\frac{\kappa _B}{e}\right) ^2
\end{equation}

\subsection{Fluttuazioni}

In generale, fluttuazioni del numero di particelle per gas perfetto sono date da:
\begin{equation}
	\left\langle (\Delta N)^2 \right\rangle = - \kappa _B T \left(\frac{\partial ^2 \Omega }{\partial \mu ^2} \right) _{TV} 
\end{equation}
mentre per numero di occupazione del singolo stato si ha:
\begin{equation}
	\left\langle (\Delta n) ^2 \right\rangle= - \kappa _B T \left(\frac{\partial^2 \Omega _q}{\partial \mu ^2} \right) _{TV} 
\end{equation}
Visto che $\overline{n}_q = -\partial _\mu  \Omega _q$, si ha:
\begin{equation}
	\left\langle (\Delta n)^2 \right\rangle = \kappa _B T \left(\frac{\partial \overline{n}_q}{\partial \mu  } \right) _{TV} 
\end{equation}
Nel caso particolare di statistiche di Fermi-Dirac o Bose-Einstein:
\begin{equation}
	\left\langle (\Delta n)^2 \right\rangle = \kappa _B T \frac{\partial \overline{n}}{\partial \mu } = \frac{\exp \left[ (\mathscr{E}-\mu ) \kappa _B T \right] + {\color{asdf}( \pm 1 \mp 1 )} }{\big(\exp\left[ (\mathscr{E}-\mu ) / \kappa _B T \right] \pm 1\big)^2} = \overline{n} \mp \overline{n}^2
\end{equation}
dove il $-$ \`e per i fermioni, mentre $+$ per i bosoni.

\begin{esempio}
	[Il conto per i fotoni]
	Si mostra che per fotoni (bosoni) vale $\left\langle (\Delta n)^2 \right\rangle \propto \overline{n}^2$ come trovato sopra.


	Per farlo, si considerano $N$ sorgenti che emettono radiazione non-coerente, che fa interferenza in un punto dello spazio. 
Il campo elettrico, per ampiezze uguali, \`e $E = \sum_{j=1}^{N} \varepsilon e^{i \phi _j} $. 
Numero di fotoni proporzionale a intensit\`a media della radiazione, a sua volta proporzionale a $\langle \lvert E \rvert ^2 \rangle$; si ha:
\[
		\lvert E \rvert ^2 = \sum_{j=1}^{N} \varepsilon  e^{i\phi _j} \sum_{l=1}^{N} \varepsilon e^{-i \phi _l} = \varepsilon ^2 \left[N + \sum_{l\neq j}^{N} e^{ i (\phi _j - \phi _l)} \right] = \varepsilon ^2 \left[ N + 2 \sum_{j>l}^{N} \cos(\phi _j - \phi _l) \right] 
\] 
Visto che il secondo termine ha media nulla, si ha: $\overline{n} \propto \langle \lvert E \rvert ^2 \rangle=N \varepsilon ^2$. 
Infine, essendo che
\begin{equation*}
	\begin{split}
		\langle n^2 \rangle \propto \langle \lvert E \rvert^4  \rangle &= \varepsilon  \left\langle \left(N + 2 \sum_{j>l}^{N} \cos(\phi _j - \phi _{l}) \right) ^2 \right\rangle = \varepsilon ^4 \left(N^2 + 4N \left\langle \sum_{j>l}^{N} \cos(\phi _j - \phi l) \right\rangle + 4 \left\langle \sum_{j>l}^{N} \cos^2 (\phi _j - \phi _l) \right\rangle\right) \\
				    &= \varepsilon ^4 \left(N^2 + 4 \frac{1}{2} \frac{N(N-1)}{2}\right) = \varepsilon ^4 (2N^2-N) \sim 2 \varepsilon ^4 N^2
	\end{split}
\end{equation*}
si riottiene effettivamente $\langle (\Delta n)^2 \rangle \propto \varepsilon ^4 N^2 = \overline{n}^2$.
\end{esempio}
\noindent Si ripete ragionamento per $g$ celle dello spazio delle fasi pi\`u che per singolo stato.
Assumendo che celle vicine abbiano energie simili, quindi numeri di occupazione simili, il numero di occupazione del gruppo di celle \`e $\overline{N} = g \overline{n}$.

Le fluttuazioni sono date da:
\[
\langle (\Delta N)^2 \rangle = \kappa _B T \left(\frac{\partial \overline{N}}{\partial \mu } \right) _{TV}  = \kappa _B T g \left(\frac{\partial \overline{n}}{\partial \mu } \right) _{TV}  =  g(\overline{n} \pm \overline{n}^2) = \overline{N} \pm \frac{\overline{N}^2}{g}
\] 
Per stimare $g$ in situazione sperimentale, si indica con $S$ area della sorgente e misurando entro angolo solido $\Delta \Omega $.
Per $T $ tempo di msura, il volume dello spazio interessato dai fotoni \`e $V = ScT$, mentre la larghezza degli impulsi \`e $\Delta p = h \Delta \nu  / c \sim h / (c \tau _c)$ con $\Delta \nu $ larghezza di banda e $\tau _c $ tempo di coerenza.
Allora:
\begin{equation}
	g = \frac{Vp^2 \Delta p \Delta \Omega }{h^3} = \frac{S\Delta \Omega }{\lambda ^2}\frac{T}{\tau _c} \equiv \frac{S'}{A_c}\frac{T}{\tau _c}
\end{equation}
Si nota che $g$ \`e il prodotto di coerenza spaziale per coerenza temporale. 
\subsubsection{Esperimento di Hanbury, Brown e Twiss}
Siano $R_{1},R_2$ due rivelatori a distanza $d$ variabile in modo che possano entrare e uscire dall'area di coerenza.
Siano $N_1,N_2$ due contatori che misurano in un certo intervallo temporale.
Il segnale emesso \`e inviato a correlatore che restituisce $\overline{c} = \langle (N_1-\overline{N}_1)(N_2-\overline{N}_2) \rangle$.

Assumendo di essere in area di coerenza ($\Rightarrow $ $g$ \`e ottenuto interamente da $T / \tau _c$) ed essendo i fotoni dei bosoni: $\langle (N - \overline{N})^2 \rangle= N + \overline{N}^2 / g$; per $N = N_1+N_2$:
\[
\begin{split}
	&\big\langle (N_1+N_2 - \overline{N}_1 - \overline{N}_2)^2 \big\rangle = \overline{N}_1 + \overline{N}_2 + \frac{(\overline{N}_1 + \overline{N}_2)^2}{g}\\
	&\Rightarrow \overline{c} = \left\langle (N_1-\overline{N}_1) (N_2 - \overline{N}_2) \right\rangle = \frac{\overline{N}_1 \overline{N}_2}{g}
\end{split}
\] 
\subsubsection{Introduzione al rumore Johnson e teorema di Wiener-Chin\v cin}

All'equilibrio termodinamico, il movimento collettivo dei portatori di carica deve avere media nulla.
La differenza di potenziale termica residua in assenza di campi esterni \`e, quindi, dovuta a fluttuazioni statistiche espresse da $\langle (\Delta V)^2 \rangle = 4 R \kappa _B T \Delta f$, con $\Delta f$ larghezza di banda.

Si considera linea di trasmissione di lunghezza $L$ e chiusra da due resistenze $R $ uguali, con un ramo a terra; 
questa irraggia per effetto Joule la potenza ottenuta dal campo em che percepisce. 
I modi equispaziati del campo sono dati da $L = n \lambda  / 2$, $\lambda  = c / f$ e $f = c n / (2L)$.

Per ampiezza $\Delta f$, si osservano $2L \Delta f / c$ modi; all'equilibrio termico, ciascuno di questi ha energia termica data da:
\begin{equation}
	\overline{E} = h f \overline{n}= \frac{hf}{\exp(hf / \kappa _B T) -1 }
\end{equation}
Sviluppando per basse frequenze ($hf \ll\kappa _B T$), si ha $E \sim \kappa _B T$ e l'energia media dei modi \`e:
\begin{equation}
	\overline{E}_{\Delta f}  = \kappa _B T \frac{2L}{c}\Delta f
\end{equation}
Ogni resistenza riceve met\`a energia in tempo $L / c$, quindi la potenza che ciascuna riceve \`e $P_{\Delta f} = \kappa _B T \Delta f$,
che sar\`a riemessa per effetto Joule: $RI^2 = \overline{V}^2 / 4R = \kappa _B T \Delta f$. 
Da questa si ottiene espressione per fluttuazioni statistiche riportata all'inizio.	
Si vuole applicare questo al gas di elettroni.

Si considera potenziale $V(t)$ che fluttua nel tempo, misurato per tempo $T$; sviluppando in serie di Fourier:
\begin{equation}
	V(t) = \sum_{n=1}^{+\infty} \big(a_n \cos(2 \pi f_n t) + b_n \sin (2\pi f_n t) \big), \ f_n = \frac{n}{T}
\end{equation}
Nota: $\langle \cdot  \rangle_t$ indica media temporale, mentre $\langle \cdot  \rangle_s$ indica media statistica su ensemble di misure.

La potenza dissipata per componetne spettrale da un generico elemento resistivo \`e:
\begin{equation}
	\begin{split}
		&P_n = \frac{1}{R} \left\langle \big(a_n \cos(2\pi f_n t) + b_n \sin(2\pi f_n t)\big)^2 \right\rangle_t = \frac{a_n^2 + b_n^2}{2R}\\
		&\Rightarrow \langle P_n \rangle_s = \frac{\langle a_n^2 \rangle_s + \langle b_n^2 \rangle_s}{2R}
	\end{split}
\end{equation}
Le componenti con diverso $n$ si mediano a $0$\footnote{I doppi prodotti sono tutti nulli perch\'e quelli con diverso $n$ contribuiscono in maniera indipendente.}; 
complessivamente, la potenza dissipata \`e:
\begin{equation}
	\frac{\left\langle \langle V^2(t) \rangle_s \right\rangle_t}{R} = \sum_{n}^{} \langle P_n \rangle_s = \frac{1}{R}\sum_{n}^{} \frac{\langle a_n^2 \rangle_s + \langle b_n^2 \rangle_s}{2} \equiv \sum_{n}^{} G(f_n) \Delta f
\end{equation}
con $G(f_n)$ \textbf{densit\`a spettrale}. 
Per trovare la sua forma esplicita, si considera trasformata di Fourier, su intervallo temporale $T$ di campionamento, di una generica grandezza dipendete dal tempo $x(t)$:
\begin{equation*}
	x(t) = \int_{-\infty} ^{+\infty} df \ \widetilde{x}_t (f) e^{2 \pi i f t} 
\end{equation*}
con $x(t) \neq 0$ solamente in $\left[ -T / 2, T / 2 \right] $.
Se $x$ \`e reale, il modulo quadro della sua trasformata \`e pari e, per Parseval:
\begin{equation*}
	\int_{-T / 2} ^{+T / 2} x^2 (t) \ dt = \int_{-\infty} ^{+\infty} \lvert \widetilde{x}_t \rvert ^2 \ df = 2 \int_{0} ^{+\infty} \lvert \widetilde{x} \rvert ^2 \ df
\end{equation*}
La media temporale \`e data da:
\[
\langle x^2 \rangle_t = \lim_{T \to +\infty} \frac{1}{ T } \int_{-T / 2} ^{+T / 2}  x^2 (t) \ dt = \lim_{T \to +\infty} \frac{2}{T} \int_{0} ^{+\infty} \lvert \widetilde{x}_T \rvert ^2 \ df
\] 
Mediando anche sull'ensemble:
\[
\left\langle \langle x^2 \rangle_t \right\rangle_s = \int_{0} ^{+\infty} \lim_{T \to +\infty} \frac{2}{T} \left\langle \lvert \widetilde{x}_T \rvert ^2 \right\rangle \ df = \int_{0} ^{+\infty} G(f)  \ df
\] 
D'altra parte si pu\`o scrivere la correlazione temporale come:
\[
\begin{split}
	c(\tau ) &= \left\langle \langle x(t) x(t+\tau ) \rangle_t  \right\rangle_s = \left\langle \lim_{T \to +\infty} \frac{1}{T}\int_{-T / 2} ^{+ T / 2} x(t) x(t+\tau ) \ dt \right\rangle_s \\
		 &= \left\langle \lim_{T \to +\infty} \frac{1}{T}\int_{-\infty} ^{+\infty} \int_{-\infty} ^{+\infty} \int_{-T / 2} ^{+T / 2} \widetilde{x}_T (f) e ^{2\pi i ft} \widetilde{x}_T (f') e ^{2\pi i f (t+\tau )}  \ dfdf'dt \right\rangle_s\\
		 &= \left\langle \lim_{T \to +\infty} \frac{1}{T}\int_{-\infty} ^{+\infty} df ' \ e^{2 \pi i f' \tau } \int_{-\infty } ^{+\infty}df\ \widetilde{x}_T (f) \widetilde{x}_T (f') \int_{-T / 2} ^{+ T/2}  dt\ e^{2\pi i (f+f') t} \right\rangle_s
\end{split}
\] 
Visto che l'ultimo integrale \`e una $\delta $ di Dirac:
\begin{equation}
	c(\tau ) = \left\langle \lim_{T \to +\infty} \int_{-\infty} ^{+\infty} df\ \widetilde{x}_T (f) \widetilde{x}_T (-f) e^{2\pi i f \tau }  \right\rangle _s= \left\langle \lim_{T \to +\infty} \frac{2}{T}\int_{0} ^{+\infty} df \ \lvert \widetilde{x}_T f) \rvert ^2 \cos(2\pi f \tau ) \right\rangle_s
\end{equation}
Questo \`e il \textbf{teorema di Wiener-Chin\v cin}, che afferma che la densit\`a spettrale di energia di un segnale coincide con la coseno-trasformata della funzione di autocorrelazione del segnale stesso.

In genere, per sistemi sottoposti a perdita di coerenza nell'evoluzione temporale, si ha $c(\tau ) = c(0) e^{ - \tau  / \tau _c} $, quindi:
\begin{equation}
	G(f) = 4 \int_{0} ^{+\infty} c(0) e ^{- \tau  / \tau _c} \cos (2\pi f \tau ) \ d\tau = \frac{4 c(0) \tau _c}{1+ (2\pi f\tau _c)^2}
\end{equation}
che \`e circa costante fino a frequenze $\sim 1 / (2 \pi \tau _c)$ e trascurabile sopra.


\subsubsection{Rumore Johnson}

Si considera filo con $N = S l n$ elettroni (area $\times $ lunghezza $\times $ densit\`a); la corrente e resistenza sono:
\[
	J = \frac{e^2 n \tau _c }{m}E \hspace{.1cm} ; \hspace{.2cm} R = \frac{l}{S\sigma _s} = \frac{lm}{ne^2 \tau _c S}
\] 
Allora la differenza di potenziale \`e:
\[
V = RSJ = RSnev_x = \frac{Re}{l} \sum_{i=1}^{N} \langle \Delta v_{x,i}  \rangle_{(f,f+\Delta f)} 
\] 
Per la funzione di autocorrelazione, si assume la fomra:
\[
c(\tau ) = \left\langle \langle v_{x,i} (t) v_{x,i} (t+\tau ) \rangle_s \right\rangle_t \stackrel{!}{=} v_{x,i} ^2 e^{ - \tau  / \tau _c} 
\] 
Usando teorema di Wiener-Chin\v cin e teorema di equipartizione:
\begin{equation}
	G_i = \frac{4 \langle v_{x,i} ^2 \rangle\tau _c}{1+ 2 \pi f \tau _c} \approx 4 \frac{\kappa _B T}{m}\tau _c
\end{equation}
Per $f\tau _c \ll 1$, la distribuzione in ptenza \`e costante in frequenza (spettro bianco), quindi in $(f,f+\Delta f)$:
\[
\langle \Delta v_{x,i}  \rangle_{f,f+\Delta f)} \approx G_i \Delta f
\] 
Unendo tutto, si ottiene:
\begin{equation}
	\left\langle V^2_{\Delta f}  \right\rangle_t = \frac{R^2 e^2}{l^2} \sum_{i=1}^{N} 4 \frac{\kappa _B T}{m}\tau_c \Delta f = 4 \kappa _B T R \Delta f
\end{equation}
Il rumore termico \`e predominante nello studio delle propriet\`a del trasporto di un conduttore.

\subsubsection{Rumore shot}

Per basse correnti, si osserva natura discreta dei portatori di carica e si parla di \textbf{rumore shot}. 
In tempo di misura $T$, la corrente \`e $\langle I \rangle = \langle n \rangle \frac{e}{T}$.
Le fluttuazioni nella corrente dipendono dalle fluttuazioni del numero di elettroni medio misurato.

La probabilit\`a di misurare $n$ elettroni in intervallo temporale $T$ diviso in $N$ intervalli uguali, dove $N$ t.c. in intervallo $T / N$ si misura solo un elettrone con probabilit\`a $p$, \`e:
\begin{equation}
	P(n,T) = \binom{N}{n} p^n (1-p)^{N-n} 
\end{equation}
che si approssima con Poissoniana per $p\ll 1 $ e $N \gg 1$:
\begin{equation}
	P(n,T) = \frac{\langle n \rangle^n}{n!}e^{- \langle n \rangle} ,\ \langle (\Delta n)^2 \rangle = \langle n \rangle
\end{equation}
La densit\`a spettrale del ruimore shot \`e:
\begin{equation}
	G(f) = 2e \langle I \rangle
\end{equation}
Il rumore shot \`e pi\`u significativo del Johnson a temperatura ambiente nel caso in cui $\Delta V = RI > 2\kappa _BT / e \approx 50$ mV.







































\newpage
\section{Solidi cristallini}
\subsection{Livelli energetici di un cristallo}


Si risolve equazione di Shr\"odinger pi\`u completa per livelli elettronici; l'Hamiltoniano deve contenere parte cinetica di elettroni e nuclei e componente di interazione Coulomniana tra di essi:
\begin{equation}
	\begin{split}
		H_0 &= T_e + T_N + V_{ee} + V_{NN} + V_{eN}   \\
		    &= - \frac{\hbar ^2}{2m} \sum_{i}^{} \nabla _i^2 - \frac{\hbar ^2 }{2M} \sum_{i}^{} \nabla _i^2 + \frac{1}{2} \sum_{i\neq j}^{} \frac{e^2}{\lvert \mathbf{r} _i - \mathbf{r} _j \rvert } + \frac{1}{2}\sum_{l\neq m }^{} \frac{Z_l Z_m e^2 }{\lvert \mathbf{R} _l - \mathbf{R}_m  \rvert } - \sum_{i,l}^{} \frac{Z_l e^2}{\lvert \mathbf{r} _i - \mathbf{R} _l \rvert }
	\end{split}
\end{equation}
Quindi si risolve $H_0 \psi  = E \psi $ in approssimazione non-relativistica.
Oltre ad approssimazione non-relativistica, si assume che la funzione d'onda fattorizzi come $\psi (\vec{\mathbf{r} }, \vec{\mathbf{R} }) = F( \vec{\mathbf{R} }) \phi _n (\vec{\mathbf{r} }, \vec{\mathbf{R} })$ e che $\vec{\mathbf{R} }$ sia un parametro in $\phi _n$\footnote{Il pedice $n$ in $\phi _n$ sta ad indicare il generico livello energetico $n$-esimo a cui tale funzione d'onda fa riferimento.} e non variabile.
Questa \`e l'\textbf{approssimazione di Born-Oppenheimer} ed \`e giustificata dal fatto che i nuclei, molto pi\`u pensati degli elettroni, sono approssimabili come fermi nello studio del moto degli elettroni stessi. Allora:
\begin{equation}
	\nabla ^2 _\mathbf{R} \psi  \equiv \phi _n \nabla ^2 _\mathbf{R} F + 2 \nabla _\mathbf{R} \phi _n \cdot \nabla _\mathbf{R} F + F \nabla ^2_\mathbf{R} \phi _n = \phi _n \nabla ^2_\mathbf{R} F
\end{equation}
Complessivamente, l'equazione di Shr\"odinger \`e data da:
\begin{equation}
	\begin{split}
		-\frac{\hbar ^2}{2m} F( \vec{\mathbf{R} }) \nabla ^2_\mathbf{r} \phi _n (\vec{\mathbf{R} }, \vec{\mathbf{r} }) &+ ( V_{ee}  + V_{eN}+V_{N N} ) F(\vec{\mathbf{R} }) \phi_n (\vec{\mathbf{R} }, \vec{\mathbf{r} }) + \operatorname{} \\
				&- \frac{\hbar ^2}{2M} \phi _n (\vec{\mathbf{R} }, \vec{\mathbf{r} }) \nabla ^2_\mathbf{R} F(\vec{\mathbf{R} })  = E F(\vec{\mathbf{R} }) \phi _n(\vec{\mathbf{R} }, \vec{\mathbf{r} })
	\end{split}
\end{equation}
Dividendo ambo i membri per $\psi $ e includendo termini nucleari nell'autoenergia:
\begin{equation}
-\frac{\hbar ^2}{2m}\frac{\nabla ^2 _\mathbf{r} \phi _n}{\phi _n} + V_{ee}  + V_{eN} = E_n(\vec{\mathbf{R} })
\end{equation}
Trovate le autoenergie, si pu\`o scrivere equazione di Shr\"odinger per il nucleo:
\begin{equation}
	-\frac{\hbar ^2}{2M} \frac{\nabla ^2 _\mathbf{R} F(\vec{\mathbf{R} })}{F(\vec{\mathbf{R} })} + V_{N N } + E_n(\vec{\mathbf{R} }) = E
\end{equation}
Tuttavia non si è ancora in grado di risolvere quella per gli elettroni, nonostante le semplificazioni.
Si assume, ulteriormente, che gli elettroni siano indipendenti, ossia che la funzione d'onda elettronica si scriva come prodotto delle funzioni d'onda per singolo elettrone.
Avendo a che fare con fermioni, questa scomposizione deve risultare antisimmetrica per scambio delle particelle. 
Questa si ricava dal \textbf{determinante di Slater}:
\begin{equation}
	\phi _n (\vec{\mathbf{r} }) = \frac{1}{\sqrt{n!} } \begin{vmatrix} \phi _1(\mathbf{r} _1) & \phi _1(\mathbf{r} _2) & \cdots & \phi _1 (\mathbf{r} _n) \\ 
	\phi_2 (\mathbf{r}_1 )& \phi _2 ( \mathbf{r} _2) & \cdots & \phi _2 (\mathbf{r} _n)\\
	\vdots & \vdots& \ddots & \vdots\\
\phi _n (\mathbf{r} _1) & \phi _n (\mathbf{r} _2) & \cdots &\phi _n(\mathbf{r} _n)\end{vmatrix} 
\end{equation}
L'Hamiltoniano ottenuto per la funzione d'onda elettronica \`e dato da:
\begin{equation}
	\hat{H} = -\frac{\hbar^2}{2m} \nabla ^2_\mathbf{r}  + \frac{1}{2}\sum_{i\neq j}^{} \frac{e^2}{\lvert \mathbf{r} _i - \mathbf{r} _j \rvert } - \sum_{i,l}^{} \frac{Z_l e^2}{\lvert \mathbf{r} _i - \mathbf{R} _l \rvert }
\end{equation}
Applicandolo alla funzione d'onda fattorizzata, si ottiene l'\textbf{Hamiltoniano di Hartree-Fock}:
\begin{equation}
	\begin{split}
		\hat{H}_\text{HF} \phi _i (\vec{\mathbf{r} } ) = &- \frac{\hbar ^2}{2m} \nabla _\mathbf{r} ^2 \phi _i(\vec{\mathbf{r} })-\sum_{l}^{} \frac{Z_l e^2}{\lvert \vec{\mathbf{r} } - \mathbf{R} _l \rvert } \phi _i (\vec{\mathbf{r} }) + e^2 \sum_{j\neq i }^{} \int \frac{\phi _j^*(\vec{\mathbf{r} }') \phi _j(\vec{\mathbf{r} }')}{\lvert \vec{\mathbf{r} }-\vec{\mathbf{r} }' \rvert } d^3 r' \ \phi _i (\vec{\mathbf{r} })\\
							   &-e^2 \sum_{j\neq i}^{} \int \frac{\phi _j^*(\vec{\mathbf{r} }') \phi _i (\vec{\mathbf{r} }')}{\lvert \vec{\mathbf{r} }- \vec{\mathbf{r} }' \rvert } d^3 r' \ \phi _j (\vec{\mathbf{r} })
	\end{split}
\end{equation}
I primi due termini rappresentano parte cinetica e interazione Coulombiana con i nuclei; il terzo termine \`e \textbf{termine di Hartree} e rappresenta la parte puramente moltiplicativa dell'interazione Coulombiana tra elettroni; l'ultimo termine, detto \textbf{di scambio}, assicura antisimmetria della funzione d'onda imponendo cambiamento di segno quando si scambiano due elettroni.

L'ultimo \`e problematico perch\'e contiene la $\phi _i$ per cui si vuole risolvere;
un metodo per trattare il problema \`e tramite \textbf{local density approximation}, in cui si media il potenziale nel termine di scambio sommando sulle $\phi _i$ e rinormalizzando. 
Cos\`i facendo, l'interazione Coulombiana si pu\`o scrivere tramite singolo potenziale $V(\vec{\mathbf{r} })$ e l'equazione per le autoenergie si scrive in una forma standard:
\begin{equation}\label{eser}
	-\frac{\hbar ^2}{2m} \nabla ^2_\mathbf{r} \phi _i(\vec{\mathbf{r} }) + V(\vec{\mathbf{r} }) \phi _i (\vec{\mathbf{r} }) = E_n \phi _i (\vec{\mathbf{r} })
\end{equation}
L'idea ora non \`e di risolvere l'equazione in quanto $V$ pu\`o essere molto complicato, ma piuttosto se ne vogliono studiare le propriet\`a.
\subsection{Reticoli e teorema di Bloch}
In solidi cristallini, le posizioni di equilibrio degli atomi sono periodiche, quindi \`e periodico il potenziale a cui sono soggetti gli elettroni: $V(\mathbf{r} +\mathbf{R} ) = V(\mathbf{r} )$, con $\left\{ \mathbf{R}  \right\} $ set di vettori che definiscono il reticolo cristallino, dove ogni elemento \`e esprimibile come $\mathbf{R} = n_1\mathbf{a} _1 + n_2 \mathbf{a} _2 + n_3 \mathbf{a} _3$. 
Gli $\mathbf{a} _i$ definiscono una \textbf{cella unitaria}, definita come la pi\`u piccola struttura periodica identificabile in un cristallo. 
L'insieme $\{\mathbf{R} \}$ \`e detto \textbf{reticolo di Bravais}; non \`e assicurato che questo coincida con la forma del reticolo atomico, n\'e che ogni cella unitaria contenga un singolo atomo.
Per ogni set $\left\{ \mathbf{R}  \right\} $, si definisce \textbf{reticolo reciproco} il set $\left\{ \mathbf{K}  \right\} $ tale che
\[
e^{i \mathbf{K} \cdot \mathbf{R} }  =1 
\] 
dove ogni vettore in $\left\{ \mathbf{K}  \right\} $ si potr\`a scrivere come: $\mathbf{K}  = n_1 \mathbf{b} _1 + n_2 \mathbf{b} _2 + n_3 \mathbf{b} _3$.
Dovendo soddisfare $e^{i \mathbf{K} \cdot \mathbf{R} } = 1$, deve valere $\mathbf{a} _i \cdot \mathbf{b} _j = 2\pi \delta _{ij} $.
Ad esempio, per reticolo cubico di lato $a$, con $\mathbf{a} _i = a_i \hat{\mathbf{x} }_i\equiv a \hat{\mathbf{x} }_i$, il reticolo reciproco \`e dato da $\mathbf{b} _i = 2\pi \hat{\mathbf{x} }_i / a$.

Si usa eq. \ref{eser} con ipotesi di periodicit\`a per $V(\mathbf{r} )$. 
L'impulso non va bene come numero quantico perch\'e l'Hamiltoniano non commuta con $\mathbf{p} $ sempre, ma solo per un generico sottogruppo di traslazioni. 
Quindi $V(\mathbf{r} )$ accoppia onde piane con impulsi diversi: per reticolo unidimensionale di passo $a$, gli elettroni entrano come onde piane di impulso $\mathbf{q} _0$ ed escono con impulso $|\mathbf{q} | = | \mathbf{q} _0|$\footnote{Questo \`e assicurato dal fatto che, essendo il reticolo di massa molto grande, l'impulso trasferito dall'urto non \`e sufficiente da farlo spostare significativamente dall'equilibrio ed \`e possibile assumere l'urto come elastico.} ma con angolo\footnote{Immaginando reticolo disposto lungo asse $\hat{x}$, se gli elettroni arrivano dal semipiano negativo, questi individuano angolo $\alpha_0$ con asse $\hat{x}$; l'angolo in uscita $\alpha $ \`e formato simmetricamente rispetto ad asse $\hat{y}$ con lo stesso asse $\hat{x}$.} $\alpha \neq \alpha _0$. 
La differenza di cammino ottico \`e $\delta = a(\cos \alpha - \cos \alpha _0)$, con $\cos \alpha  = q_x / \lvert \mathbf{q}  \rvert $ e $\cos \alpha _0 = q_{0,x} / \lvert \mathbf{q} _0 \rvert $.

Per avere interferenza costruttiva, si impone che $\delta $ sia multiplo della lunghezza d'onda degli elettroni:
\[
	\delta = \frac{\mathbf{a} \cdot  \Delta \mathbf{q} }{\lvert \mathbf{q}  \rvert } \stackrel{!}{=} l \lambda  \implies \Delta \mathbf{q} \cdot \mathbf{a}  = 2\pi l  
\] 
dove si \`e usato che $\lvert \mathbf{q}  \rvert = \lvert \mathbf{q} _0 \rvert = 2\pi /\lambda $ e $\mathbf{a } = a \hat{\mathbf{x} }$.
Essendo $\Delta \mathbf{q} \cdot \mathbf{a}  = 2\pi l$\footnote{Chiaramente, si fosse trattato un reticolo tridimensionale, si sarebbe avuto un risultato simile: $\mathbf{a } \cdot  \Delta \mathbf{q} = 2\pi l_1, \ \mathbf{b} \cdot  \Delta \mathbf{q} = 2\pi l_2, \ \mathbf{c} \cdot \Delta \mathbf{q} = 2\pi l_3$.}, le variazioni tra vettore d'onda in entrata e uscita soddisfa le regole date per i vettori del reticolo reciproco, quindi $\Delta \mathbf{q} \in \left\{ \mathbf{K}  \right\} $.
Questo metodo viene usato per determinare la struttura dei reticoli di un cristallo: quando si osserva interferenza costruttiva, si sa che la differenza tra vettore d'onda in entrata e uscita \`e un vettore di $\left\{ \mathbf{K}  \right\} $, quindi si fa il reciproco del reticolo reciproco per ottenere la struttura interna del reticolo data dai vettori di $\left\{ \mathbf{R}  \right\} $.
\subsubsection{Cella di Wigner-Seitz e prima zona di Brillouin}
La \textbf{cella di Wigner-Seitz} \`e una cella primitiva che rispetta tutte le propriet\`a di simmetria della struttura cristallina. Questa \`e definita attorno a un punto (nodo) del reticolo di Bravais ed \`e formata dai punti che sono pi\`u vicini a quel nodo rispetto a qualunque altro. Geometricamente, la regione corrisponde al pi\`u piccolo poliedro individuato dall'intersezione dei piani che bisecano ortogonalmente i segmenti che congiungono il nodo a ciascuno dei \textit{primi pi\`u vicini} ad esso.

La \textbf{prima zona di Brillouin} \`e l'analogo della cella di Wigner-Seitz per il reticolo reciproco e si ottiene con gli stessi identici passaggi, solo che nel reticolo reciproco.
 
\subsubsection{Teorema di Bloch}
Il teorema afferma che se il potenziale $V(\mathbf{r} )$ in
\[
\left( - \frac{\hbar ^2}{2m}\nabla ^2 + V(\mathbf{r} )\right) \psi (\mathbf{r} ) = E \psi (\mathbf{r} )
\] 
\`e periodico ($V(\mathbf{r} +\mathbf{R} )=V(\mathbf{r} )$), proprio come nel caso considerato, allora le soluzioni all'equazione di Shr\"odinger sono date dalle \textbf{funzioni di Bloch}:
\begin{equation}
	\phi _{n, \mathbf{k} } (\mathbf{r} ) = e^{i \mathbf{k} \cdot \mathbf{r} } u _{n, \mathbf{k} } (\mathbf{r} )
\end{equation}
con $u_{n, \mathbf{k} } $ funzione periodica sul reticolo, che rende periodica sul reticolo stesso anche  $\phi _{n,\mathbf{k} } $, a meno di una fase:
\[
u_{n,\mathbf{k} } (\mathbf{r} + \mathbf{R} ) = u_{n,\mathbf{k} } (\mathbf{r} ) \implies \phi _{n,\mathbf{k} } (\mathbf{r} + \mathbf{R} ) = e^{i \mathbf{k} \cdot \mathbf{R} } \phi _{n,\mathbf{k} } (\mathbf{r} ) 
\] 
Si nota che nel caso di $V(\mathbf{r}) =0$, si ha invarianza traslazionale completa e le soluzioni all'equazione di Shr\"odinger sono onde piane; l'aggiunta di questo potenziale porta una simmetria traslazionale parziale, quindi il risultato dell'equazione non sono esattamente onde piane, ma hanno una forma pi\`u complessa. 
In questo caso, si nota applicando $\hat{p}$ alla soluzione $\psi _{n,\mathbf{k} } $, che queste non sono autofunzioni dell'impulso, quindi impulso ed Hamiltoniano non commutano.
Per questa ragione, l'impulso in s\'e non \`e pi\`u un buon numero quantico, ma lo \`e se ridefinito includendo un vettore del reticolo reciproco.
\begin{proof}
	Si  definisce set di operatori che rappresentano traslazioni discrete sul reticolo, per cui $[\hat{T}_\mathbf{R} , \hat{H}] = 0$\footnote{Questo si dimostra a partire dal fatto che $V(\mathbf{r} ) = V(\mathbf{r} +\mathbf{R} )$.}. Per le autofunzioni simultanee dei due varr\`a: $\hat{T}_\mathbf{R} \phi  = C(\mathbf{R} ) \phi $; inoltre, gli operatori di traslazione godono della propriet\`a di composizione: $\hat{T}_\mathbf{R} \hat{T}_{\mathbf{R}'}  = \hat{T}_{\mathbf{R} '}  \hat{T}_\mathbf{R} = \hat{T}_{\mathbf{R} + \mathbf{R} '} $, per cui vale anche $C(\mathbf{R} ) C(\mathbf{R} ') = C(\mathbf{R} +\mathbf{R} ')$. 

	Si definiscono questi autovalori per traslazioni di $\mathbf{a} _1, \mathbf{a} _2, \mathbf{a}_3 $; essendo numeri complessi in generale, si scrivono come esponenziali:
	\[
	C(\mathbf{a} _1) = e^{2\pi i x_1} ; \ C(\mathbf{a} _2) = e^{2\pi i x_2} ; \ C(\mathbf{a} _3) e^{2\pi i x_3} 
	\] 
	Sfruttando questa propriet\`a considerando che un generico $\mathbf{R} = n_1 \mathbf{a} _1 + n_2 \mathbf{a} _2 + n_3 \mathbf{a}_3 $, ogni autovalore si pu\`o scrivere come:
	\begin{equation}
		C( \mathbf{R} ) = C( \mathbf{a} _1) ^{n_1} C(\mathbf{a} _2)^{n_2} C(\mathbf{a} _3) ^{n_3} = e^{ i \mathbf{k} \cdot \mathbf{R} } 
	\end{equation}
dove $\mathbf{k} = x_1 \mathbf{b} _1 + x_2 \mathbf{b} _2  + x_3 \mathbf{b} _3$ e $\mathbf{b} _i \cdot \mathbf{a}_j = 2\pi \delta _{ij} $. 

Rimane da mostrare che gli $x_i$ sono reali. 
Questo si ottiene da ipotesi su condizioni al contorno. Si \`e interessati a descrivere cristallo idealmente infinito, cio\`e non si vuole che le dimensioni del cristallo rientrino nella trattazione, se non per il numero totale di celle del cristallo. Si immagina di avere $N$ cellette totali, con $N_1$ nella prima direzione, $N_2$ nella seconda e $N_3$ nella terza in modo da scrivere che $N = N_1N_2N_3$. Quindi, per descrivere cristallo infinito, \`e comodo prendere condizioni al bordo periodiche\footnote{Note col nome di \textbf{condizioni di Born-von Karman}.} t.c. $\phi (\mathbf{r} + N_i \mathbf{a}_i) = \phi (\mathbf{r} ) $; allo stesso tempo si ha l'ipotesi per cui
\[
\phi _{n, \mathbf{k} } (\mathbf{r} + N_i \mathbf{a} _i) = e^{i \mathbf{k} \cdot N_i \mathbf{a} _i} \phi _{n , \mathbf{k} } (\mathbf{r} )
\] 
per cui $e^{i \mathbf{k} \cdot N_i \mathbf{a}_i } \phi _{n, \mathbf{k} } (\mathbf{r} ) = \phi _{n , \mathbf{k} } (\mathbf{r} )$, che \`e verificata per $\mathbf{k} \cdot N_i \mathbf{a} _i = 2\pi m_i $, quindi $x_i = m_i /  N_i$, con $m_i\in \mathbb{Z}$.
\end{proof}
Gli impulsi permessi per le autofunzioni di $\hat{H}, \hat{T}_\mathbf{R} $ sono un insieme approssimabile come continuo per $N_i$ grandi, cio\`e cristalli infiniti e:
\begin{equation}
	\mathbf{k} = \sum_{i=1}^{3} \frac{m_i}{N_i} \mathbf{b} _i
\end{equation}
\begin{osservazione}
	Finch\'e $m_i < N_i$ si \`e nella stessa cella unitaria; quando vale $m_i > N_i$, ci si \`e spostati in una cella adiacente. Per questo motivo, $N$ \`e sia il numero di celle del cristallo nello spazio reale, sia il numero di autostati in ogni cella unitaria del reticolo reciproco.
\end{osservazione}
Il teorema permette di restringersi alla cella unitaria del reticolo reciproco per la caratterizzazione degli autostati. Tipicamente, la cella unitaria si identifica con la prima zona di Brillouin.
\begin{proof}
	Si considera $\mathbf{k} ' = \mathbf{k} + \mathbf{K} $, con $\mathbf{k} $ nella prima zona di Brillouin. Allora $\psi _{n, \mathbf{k}' } (\mathbf{r} ) = e^{ i \mathbf{k} ' \cdot \mathbf{r} } u_{n, \mathbf{k} '} (\mathbf{r} )=e^{i \mathbf{k} \cdot \mathbf{r} } e^{i \mathbf{K} \cdot \mathbf{r} } u_{n ,\mathbf{k} '} (\mathbf{r} ) $. Per\`o, mandando $\mathbf{r} \to \mathbf{r } + \mathbf{R} $, si ha:
	\[
		e^{i \mathbf{K} \cdot \mathbf{r} } u_{n,\mathbf{k} '} (\mathbf{r} )\to e^{i \mathbf{K} \cdot \mathbf{r} } \underbracket{e^{i \mathbf{K} \cdot \mathbf{R} }}_{=1}  u_{n,\mathbf{k} '} (\mathbf{r} +\mathbf{R}) = e^{i \mathbf{K} \cdot \mathbf{r} } u_{n,\mathbf{k} '} (\mathbf{r} )
	\] 
	Essendo $e^{i \mathbf{K} \cdot \mathbf{r} } u_{n,\mathbf{k} '} (\mathbf{r} )$ periodico, allora si sceglie $e^{i\mathbf{K} \cdot \mathbf{r} } u_{n,\mathbf{k} '} (\mathbf{r} ) \equiv u_{n,\mathbf{k} } (\mathbf{r} )$, con $\mathbf{k} $ nella prima zona di Brillouin. Quindi:
	\begin{equation}
		\psi _{n, \mathbf{k}' }(\mathbf{r} ) \equiv \psi _{n,\mathbf{k} }(\mathbf{r} ) = e^{i \mathbf{k} \cdot \mathbf{r} } u_{n,\mathbf{k} } (\mathbf{r} )
	\end{equation}
\end{proof}
\subsection{Bande energetiche}
Per introdurle, si considera la presenza di potenziale $V(\mathbf{r} )$ che non influisce sulla forma dei livelli energetici. Permette di scrivere ancora che $\mathscr{E}_\mathbf{k}  = \frac{\hbar ^2 k^2}{2m}$.

Per cristallo unidimensionale con passo reticolare $a$, la prima zona di Brillouin va da $- \pi / a$ a $ + \pi / a$. Il grafico di $\mathscr{E}(k)$ sar\`a una parabola e si traslano all'interno dell'intervallo $[-\pi/a,\pi / a]$ tutti quei $k$ fuori dalla zona di Brillouin. Le soluzioni all'equazione di Shr\"odinger unidimesionale per particella libera sono onde piane $e^{i \mathbf{k} \cdot \mathbf{r} } $; se $k\in [-\pi / a, \pi /a]$, allora si sta studiando la prima banda energetica con $u_{1,\mathbf{k} } (\mathbf{r} ) = 1$. Per $k\in [\pm\pi / a, \pm2 \pi /a]$, si \`e nella seconda banda energetica e $\psi _{2,\mathbf{k} }(\mathbf{r} ) = e^{i \mathbf{k} \cdot \mathbf{r} }  e^{\pm i 2\pi r / a }  \Rightarrow u_{2,\mathbf{k}}(\mathbf{r} ) =e^{\pm i 2\pi r / a}  $. Considerazioni analoghe valgono per bande energetiche superiori. L'indice $n$ permette di individuare la banda energetica in esame.

\subsubsection{Approccio perturbativo}


Si studia sistema con potenziale diverso considerandolo come perturbazione dalla particella libera per caratterizzarne le bande e vedere in che modo si distingue dal caso appena trattato. Nello sviluppo, si devono tenere solo onde piane che corrispondono ad un certo $\mathbf{k} $ nella zona di Brillouin, altrimenti la propriet\`a di periodicit\`a viene meno, quindi:
\begin{equation}
	\psi _{\mathbf{k} } = \sum_{\mathbf{K} }^{} c_{\mathbf{k} - \mathbf{K} } e^{i(\mathbf{k} -\mathbf{K} ) \cdot  \mathbf{r} } 
\end{equation}
Essendo potenziale periodico, si sviluppa in serie di Fourier:
\begin{equation}
	V(\mathbf{r} ) = \sum_{\mathbf{K} }^{} V_\mathbf{K} e^{i \mathbf{K} \cdot \mathbf{ r} } \Rightarrow V_\mathbf{K} = \frac{1}{V}\int_{\text{celle}} d\mathbf{r}  \ V(\mathbf{r} ) e^{-i\mathbf{K} \cdot \mathbf{r} } 
\end{equation}
Per semplificare i conti, si assume
\[
V_0 = \frac{1}{V} \int_{\text{celle}} d\mathbf{r} \ V(\mathbf{r} ) \stackrel{!}{=} 0
\] 
Inserendo correzione nel problema originale e proiettando lungo ciascuna onda piana, si ottengono $n$ equazioni (una per ciascun termine dello sviluppo) della forma:
\begin{equation}
	\left[ \frac{\hbar ^2}{2m} (\mathbf{k}-\mathbf{K} )^2 - E \right] c_{\mathbf{k} -\mathbf{K} } + \sum_{\mathbf{K} '}^{} V_{\mathbf{K} ' - \mathbf{K} } c_{\mathbf{k} -\mathbf{K} '} =0
\end{equation}
\subsubsection{Caso non-degenere}
Si considerano $\mathbf{k} $ lontani da $\mathbf{k} + \mathbf{K} $\footnote{Questi saranno, relativamente alla parabola del caso semplice considerato prima, quelli lontani dagli estremi della prima zona di Brillouin.} in modo da trattare caso non-degenere. Si definisce 
\begin{equation}
	\mathscr{E}^0_{\mathbf{k} -\mathbf{K} } := \frac{\hbar ^2}{2m}(\mathbf{k} -\mathbf{K} )^2
\end{equation}
In assunzione di non-degenerazione:
\[
\lvert \mathscr{E}^0_{\mathbf{k} -\mathbf{K} } - \mathscr{E}^0_{\mathbf{k} -\mathbf{K} _1}  \rvert \gg V_\mathbf{K} , \ \forall \mathbf{K} \neq \mathbf{K} _1
\] 
Ricordando che $V_0=0$, l'equazione si scrive come:
\begin{equation}
	(E-\mathscr{E}^0_{\mathbf{k}-\mathbf{K} _1 }) c_{\mathbf{k} -\mathbf{K} _1} = \sum_{\mathbf{K} \neq\mathbf{K}_1 }^{} c_{\mathbf{k} -\mathbf{K} } V_{\mathbf{K} -\mathbf{K} _1} 
\end{equation}
I coefficienti si possono determinare dalle altre $n-1$ equazioni:
\begin{equation}
	c_{\mathbf{k} -\mathbf{K} } = \frac{1}{E - \mathscr{E}^0_{\mathbf{k} - \mathbf{K} } } \sum_{\mathbf{K} '}^{} V_{\mathbf{K}' - \mathbf{K} } c_{\mathbf{k} - \mathbf{K} '}=\frac{V_{\mathbf{K} _1 - \mathbf{K} } c_{\mathbf{k} -\mathbf{K} _1} }{E-\mathscr{E}^0_{\mathbf{k} -\mathbf{K} } } + \frac{1}{E-\mathscr{E}^0_{\mathbf{k} -\mathbf{K} } } \sum_{\mathbf{K} ' \neq \mathbf{K} _1}^{} V_{\mathbf{K} '-\mathbf{K} }c_{\mathbf{k} -\mathbf{K} '} 
	\end{equation}
Studiando ordini di grandezza, il primo termine qui \`e di ordine $V$ per via del potenziale (cio\`e ordine molto piccolo visto che si sta assumendo potenziale come piccola perturbazione); il termine con la somma, invece, deve essere di ordine $V^2$ perch\'e i coefficienti sono relativi a $\mathbf{K} $ superiori a $\mathbf{K} _1$. Inserendo nell'equazione 4.3.5, si trascurano termini di ordine $V^3$ (\`e presente un ulteriore $V$ a moltiplicare) e si rimane con:
\begin{equation}
	E \simeq \mathscr{E}^0_{\mathbf{k} -\mathbf{K} _1}  + \sum_{\mathbf{K} }^{} \frac{\lvert V_{\mathbf{K}-\mathbf{K} _1 }   \rvert ^2}{\mathscr{E}^0_{\mathbf{k} -\mathbf{K} _1 } - \mathscr{E}^0 _{\mathbf{k} -\mathbf{K} } }
\end{equation}
\subsubsection{Caso degenere}

In questo caso, si considerano vettori d'onda vicino agli spigoli. Nel caso bidimensionale, la degenerazione pu\`o essere massimo due, mentre in pi\`u dimensioni ce ne possono stare molti raggruppati in un intorno. Questo caso \`e contraddistinto da:
\begin{equation}
	E \sim \mathscr{E}^0_{\mathbf{k} - \mathbf{K} _1} , \ldots, \mathscr{E}^0_{\mathbf{k} -\mathbf{K} _m} \implies \lvert \mathscr{E}^0_{\mathbf{k} - \mathbf{K} _i} - \mathscr{E}^0_{\mathbf{k} - \mathbf{K} _j}  \rvert \ll V ,  \ \forall j=1,\ldots,m  
\end{equation}
Cio\`e le energie sono molto vicine per questi $m$ $\mathbf{k} _j$. L'equazione, separando degeneri e non-degeneri, diventa:
\begin{equation}
	(E-\mathscr{E}^0_{\mathbf{k} -\mathbf{K} _i}) c_{\mathbf{k} -\mathbf{K} _i} = \sum_{j=1}^{m} V_{\mathbf{K} _j - \mathbf{K} _i} c_{\mathbf{k} -\mathbf{K} _j} + \sum_{\mathbf{K} \neq \mathbf{K_m} }^{} V_{\mathbf{K} -\mathbf{K} _i} c_{\mathbf{k} -\mathbf{K} } 
\end{equation}
dove per $\mathbf{K} _m$ si intende tutti quelli degeneri. Analogamente al caso precedente, si possono trovare delle espressioni per i coefficienti. Nel caso degenere, questo pu\`o essere fatto solamente per quelli della parte non-degenere perch\'e si deve dividere per la differenza delle energie al primo membro e si \`e sicuri che \`e diversa da zero solamente nel caso non-degenere.
\begin{equation}
	c_{\mathbf{k} -\mathbf{K} } = \frac{1}{E- \mathscr{E}^0_{\mathbf{k} - \mathbf{K} } } \Bigg[ \sum_{j=1}^{m}\underbracket{ V_{\mathbf{K} _j - \mathbf{K}} c_{\mathbf{k} -\mathbf{K} _j}}_{=o(U)}   + \sum_{\mathbf{K}' \neq \mathbf{K} _m}^{} \underbracket{V_{\mathbf{K} ' - \mathbf{K} } c_{\mathbf{k} -\mathbf{K} '} }_{= o(U^2)}   \Bigg] \simeq \frac{1}{E - \mathscr{E}^0_{\mathbf{k} - \mathbf{K} } } \sum_{j=1}^{m} V_{\mathbf{K} _j - \mathbf{K} } c_{\mathbf{k} - \mathbf{K} _j} 
\end{equation}
dove si sono tenuti solamente quelli di ordine $o(U)$. Sostituendo nell'equazione e tenendo solo quelli di ordine $o(U)$ come nel caso non-degenere:
\begin{equation}
	(E-\mathscr{E}^0_{\mathbf{k} - \mathbf{K} _i} ) c_{\mathbf{k} - \mathbf{K} _i} = \sum_{j=1}^{m} V_{\mathbf{K} _j - \mathbf{K} _i} c_{\mathbf{k} -\mathbf{K} _j} + o(U^2) + o(U^3) \simeq \sum_{j=1}^{m} V_{\mathbf{K} _j - \mathbf{K} _i} c_{\mathbf{k} -\mathbf{K} _j}
\end{equation}
In due dimensioni, si hanno due equazioni:
\[
\begin{cases}
	\displaystyle (E-\mathscr{E}^0_{\mathbf{k} - \mathbf{K} _1} )c_{\mathbf{k} -\mathbf{K} _1} = V_{\mathbf{K}_2 - \mathbf{K} _1} c_{\mathbf{k}-\mathbf{K}_2  } \\
	\\
	\displaystyle (E-\mathscr{E}^0_{\mathbf{k} - \mathbf{K} _2} )c_{\mathbf{k} -\mathbf{K} _2} = V_{\mathbf{K}_1 - \mathbf{K} _2} c_{\mathbf{k}-\mathbf{K}_1  } 
\end{cases}
\] 
Si sostituiscono $\mathbf{q} = \mathbf{k} -\mathbf{K} _1$ e $\mathbf{K} = \mathbf{K}_2 -\mathbf{K} _1$, quindi:
\begin{equation}
	\begin{cases}
		(E-\mathscr{E}^0_\mathbf{q} ) c_\mathbf{q}  = V_\mathbf{K} c_{q - \mathbf{K} } \\
		(E-\mathscr{E}^0_\mathbf{q-\mathbf{K} } ) c_\mathbf{q-\mathbf{K} }  = V_\mathbf{-K} c_{\mathbf{q} }\equiv V^*_\mathbf{K} c_\mathbf{q}  \\
	\end{cases}
\end{equation}
Questi sono degeneri, quindi hanno circa la stessa energia: $\mathscr{E}^0_{\mathbf{q} } \sim \mathscr{E}^0_{\mathbf{q} -\mathbf{K} } $, quindi $\lvert \mathbf{q}  \rvert \sim \lvert \mathbf{q} -\mathbf{K}  \rvert $. Le soluzioni non banali del sistema si ottengono imponendo determinante nullo:
\begin{equation}
	E = \frac{\mathscr{E}^0 _\mathbf{q}  + \mathscr{E}^0_{\mathbf{q- \mathbf{K} } } }{2} \pm \sqrt{\left(\frac{\mathscr{E}^0_\mathbf{q} - \mathscr{E}^0_{\mathbf{q} - \mathbf{K} } }{2}\right) ^2 + \lvert V_\mathbf{K}  \rvert ^2} 
\end{equation}
Nel caso fossero esattamente degeneri: $E  = \mathscr{E}^0_\mathbf{q} \pm \lvert V_\mathbf{K}  \rvert $.
\subsection{Propriet\`a dei cristalli}
Con trattazione precedente, si \`e trovata la possibilit\`a di descrivere livelli energetici non tramite il vettore d'onda in s\'e, ma quasi: si usa vettore d'onda e i vettori nel reticolo reciproco. Inoltre, dalla teoria perturbativa, si \`e visto che queste energie si distribuiscono a formare dei \textit{gap} in cui non sono presenti livelli energetici accessibili.

Le bande sono riferite a energie degli elettroni, che soddisfano la Fermi-Dirac; si distinguono diversi casi: per ogni banda, saranno possibili $2N$ stati, dove $N$ deriva dal numero di impulsi nella prima zona di Brillouin, mentre il $2$ \`e per lo spin. 
Si distinguono diversi casi.
\begin{enumerate}[(a).]
	\item L'energia di Fermi cade in un gap: in questo caso si ha un \textbf{isolante} perch\'e l'azione di un campo elettrico esterno non \`e sufficiente per far aumentare l'energia a tal punto da colmare la differenza tra l'energia di Fermi e l'energia pi\`u bassa della banda superiore. In questo caso, gli elettroni si distribuiscono nelle bande sotto il livello definito dall'energia di Fermi.
	\item L'energia di Fermi cade all'interno di una banda: in questo caso si ha un \textbf{conduttore} perch\'e l'azione di un campo elettrico \`e sufficiente a portare gli elettroni in stati superiori a quelli sotto l'energia di Fermi.
	\item L'energia cade in un gap, ma tramite drogaggio, si riesce a far avvicinare a sufficienza il livello energetico ad una banda cosicch\'e l'applicazione di un campo elettrico farebbe comportare il materiale come conduttore. 

		Gli atomi accettori o donori creano dei nuovi stati energetici per la disposizione degli elettroni, rispettivamente vicino alla banda inferiore e quella superiore.
\end{enumerate}
\begin{boxenv}[]
\centering Riprendere Lezione 15 da 49:13 
\end{boxenv}


























\end{document}


