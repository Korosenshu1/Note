\documentclass[10pt, a4paper]{scrartcl}
% Packages
%\usepackage{stix}
\usepackage[margin=1.5in]{geometry}
\usepackage{index}
\makeindex
\usepackage[utf8]{inputenc}
\usepackage[T1]{fontenc}
\usepackage{varwidth}
\usepackage{amsmath, amssymb}
\usepackage{esint}
\usepackage{titlesec}
\usepackage{xcolor}
\usepackage{titling}
\usepackage{braket}
\usepackage{tensor}
\usepackage[linktocpage]{hyperref}
\usepackage{pgfplots}
\usepackage{multicol}
\setlength{\columnsep}{2em}
\usepackage{caption}
\usepackage{amsthm}
\usepackage{import}
\usepackage{cancel}
\usepackage{caption}
\usepackage{tcolorbox}
\usepackage{nicematrix}
\usepackage{mathrsfs}
\usepackage{mathtools}
\usepackage{enumerate}
\usepackage{graphicx}
\usepackage{lipsum}
\usepackage[italian]{babel}
% To reset footnote numbering each page
\usepackage[perpage]{footmisc}

%Captions
\captionsetup[figure]{font=footnotesize,labelfont=footnotesize}
\captionsetup[table]{font=footnotesize,labelfont=footnotesize}
%Titlesec
\titleformat{\section}
{\fontsize{15}{20}\sffamily\scshape}
{\normalfont\color{gray}{\fontsize{20}{20}\selectfont\thesection}}
{0.7em}
{}
\hypersetup{colorlinks,breaklinks, linkcolor=[RGB]{74, 122, 164}}

\newcommand\vertarrowbox[3][6ex]{%
  \begin{array}[t]{@{}c@{}} #2 \
  \left\uparrow\vcenter{\hrule height #1}\right.\kern-\nulldelimiterspace\
  \makebox[0pt]{\scriptsize#3}
  \end{array}%
}
\definecolor{asdf}{HTML}{4a7aa4}
% Personalizza la formattazione della subsection
\titleformat{\subsection}[block]{\fontsize{12}{20}\bfseries}{\normalfont\thesubsection}{.5em}{}


% Personalizza la formattazione della subsubsection
\titleformat{\subsubsection}[block]{\fontsize{10}{20}\bfseries}{\normalfont\thesubsubsection}{.5em}{}

% Maketitle customization
\renewcommand{\maketitle}{
\begin{center}
{\sffamily
{\fontsize{20}{20}\selectfont\MakeUppercase\thetitle}}

\vspace{0.2in}

{\large\scshape\sffamily\theauthor}
\end{center}
}

% Titles 
\title{Note di\\\vspace{.2cm} Struttura della Materia}
\author{Manuel Deodato}
\date{}



%Evaluate symbol
\DeclareMathOperator{\di}{d\!}
\newcommand*\Eval[3]{\left.#1\right\rvert_{#2}^{#3}}

%%%%%%% Numero delle equazioni in formato a.b
\numberwithin{equation}{subsection}
%%%%%

%%%%%%%%%% Personalizzazione numeri lista
\renewcommand{\theenumi}{(\arabic{enumi})}

%%%%%%%%%% Medie con integrali multipli
\def\Yint#1{\mathchoice
    {\YYint\displaystyle\textstyle{#1}}%
    {\YYint\textstyle\scriptstyle{#1}}%
    {\YYint\scriptstyle\scriptscriptstyle{#1}}%
    {\YYint\scriptscriptstyle\scriptscriptstyle{#1}}%
      \!\iint}
\def\YYint#1#2#3{{\setbox0=\hbox{$#1{#2#3}{\iint}$}
    \vcenter{\hbox{$#2#3$}}\kern-.51\wd0}}
\def\longdash{{-}\mkern-3.5mu{-}} 
   % consider using "\mkern-7.5mu" if esint package is loaded
\def\tiltlongdash{\rotatebox[origin=c]{15}{$\longdash$}}
\def\fiint{\Yint\tiltlongdash}

\def\Zint#1{\mathchoice
    {\YYint\displaystyle\textstyle{#1}}%
    {\YYint\textstyle\scriptstyle{#1}}%
    {\YYint\scriptstyle\scriptscriptstyle{#1}}%
    {\YYint\scriptscriptstyle\scriptscriptstyle{#1}}%
      \!\iiint}
      \def\tilongdash{\mkern6mu{-}\mkern-4mu{-}\mkern-5mu{-}} 
   % consider using "\mkern-7.5mu" if esint package is loaded
\def\titiltlongdash{\rotatebox[origin=c]{15}{$\tilongdash$}}
\def\fiiint{\Zint\titiltlongdash}


%%%% Table of contents

\usepackage[titles]{tocloft}

\renewcommand{\cftdot}{}
\usepackage{titletoc}
%\setcounter{tocdepth}{2}

%%%%%%%%%%%%%%%% Toc style

% Personalizzazione scritta indice


% Font
\usepackage[osf]{newpxtext}

\usepackage{sansiwona}


% Ambienti
\newtheoremstyle{style1}% name of the style to be used
{15pt}% measure of space to leave above the theorem. E.g.: 3pt
{15pt}% measure of space to leave below the theorem. E.g.: 3pt
{\normalfont}% name of font to use in the body of the theorem
{}% measure of space to indent
{\sffamily\scshape\bfseries}% name of head font
{}% punctuation between head and body
{ }% space after theorem head; " " = normal interword space
{\thmname{#1}\thmnumber{ #2}{\thmnote{~--- #3}}.\newline}




\theoremstyle{style1}
\newtheorem{teorema}{Teorema}[section]
\newtheorem{corollario}{Corollario}[teorema]
\newtheorem{lemma}{Lemma}[teorema]
\newtheorem{definizione}{Definizione}[section]
\newtheorem{osservazione}{Osservazione}[section]
\newtheorem{notazione}{Notazione}[section]
\newtheorem{esempio}{Esempio}[section]
\newtheorem{esercizio}{Esercizio}[section]

\renewcommand\qedsymbol{$\blacksquare$}

\newenvironment{svolgimento}{\renewcommand\qedsymbol{$\spadesuit$}\begin{proof}[Svolgimento]}{\end{proof}}

%% Generic box
\newtcolorbox{eqbox}[1][]
{
colback=gray!10,
arc=0pt,
boxrule=0pt,
title=#1
}

 \newenvironment{boxenv}[1][]{
    \begin{eqbox}[#1]
    }{
   \end{eqbox}
}








%%%%%%%%%%%%%%%%%%%%%%%%%%%%%%%%%%%%%%%%%%%%%%%%%%%%%%%%%%%%%%%%%%%%%%%%

\begin{document}
\maketitle
\newpage
\tableofcontents 
\newpage
\section{Nozioni di meccanica statistica e termodinamica}

\subsection{Gas di particelle}


Si considera gas di particelle non interagenti e puntiformi. Ciascuna particella soddisfa $\hat{H}\psi (\mathbf{r} ) = E  \psi (\mathbf{r} )$ con $\hat{H} = \frac{\hat{\mathbf{p} }}{2m}$ e $E = \frac{\hbar ^2}{2m} q ^2 $, quindi la soluzione generale \`e:
\begin{equation}
	\psi (\mathbf{r} ) = e^{ i \mathbf{q} \cdot \mathbf{r} } 
\end{equation}
Imponendo condizione di periodicit\`a al bordo della scatola $\Rightarrow q_i = \frac{2\pi}{L}l_i, \ l_i = 0, \pm 1,\pm 2,\ldots$. 

Si assumer\`a che particelle interagiscano abbastanza poco da rendere valida questa trattazione, e abbastanza tanto da permettere transizioni di fase.
\subsection{Principi della termodinamica}
\begin{enumerate}[(a).]
	\item \textbf{Primo principio}: per un sistema chiuso (niente scambio di particelle) e isolato, vi \`e conservazione dell'energia interna:
		\begin{equation}
			dE = \delta  Q + \delta L
		\end{equation}
	\item \textbf{Secondo principio}: l'entropia, data da $S = \kappa _B \log \Gamma$\footnote{Questa espressione \`e il caso limite della pi\`u generale $S = \kappa _B \sum_{i}^{} p_i \log p_i$ che si ha quando il sistema non \`e all'equilibrio, cio\`e quando i microstati non sono popolati uniformemente.} (con $\Gamma$ numero di microstati del sistema all'equilibrio), per un sistema isolato, soddisfa $\frac{d S}{d t} \ge 0$. L'uguaglianza vale quando \`e raggiunto l'equilibrio.
	\item \textbf{Terzo principio}: l'entropia tende a $0$ per sistemi perfettamente ordinati, cio\`e sistemi in cui tutte le particelle popolano un solo microstato $\Rightarrow S = \kappa _B \log 1 = 0$. Sistemi perfettamente ordinati sono cristalli perfetti a temperatura nulla; non tutti i materiali a $T=0$ risultano perfettamente ordinati e alcuni presentano entropia residua.
\end{enumerate}
\subsection{Potenziali termodinamici}
A seconda del caso, si usano diverse riscritture dell'energia.
\begin{itemize}
	\item \textbf{Energia libera di Helmholtz}: $F = E - TS \Rightarrow dF = -SdT - PdV$. La sua variazione a temperatura costante restituisce lavoro compiuto sul sistema: $\delta F |_T = - P \delta V |_T = \delta L$.
	\item \textbf{Energia libera di Gibbs}: $\Phi = E - TS + PV = F + PV \Rightarrow d\Phi=  VdP - SdT$. \`E adatta a descrivere transizioni di fase.
	\item \textbf{Entalpia}: $W = E + PV \Rightarrow dW=  TdS + VdP$. La sua variazione a pressione costante \`e il calore scambiato dal sistema: $\delta W|_P = T \delta S|_P = \delta Q$.
\end{itemize}
Se \`e possibile scambio di particelle, la dipendenza da $N$ nei potenziali si aggiunge con:
\begin{equation}
	\mu = \left(\frac{\partial E}{\partial N} \right) _{SV} = \left(\frac{\partial F}{\partial N} \right) _{TV} = \left(\frac{\partial \Phi}{\partial N} \right) _{SP} 
\end{equation}
$\mu $ \`e esso stesso un potenziale: $d\mu = - S / N dT + V / N dP = - s dT + vdP$\footnote{Aggiungendo particelle ferme ad un sistema, \`e ragionevole avere $\mu < 0$, visto che l'energia media diminuirebbe con l'aumentare di $N$.}. Un altro potenziale utile \`e il \textbf{potenziale di Landau}: $\Omega = F - \mu N\Rightarrow d\Omega = -SdT - PdV - N d\mu $.



\subsection{Calori specifici e compressibilit\`a}

Calori specifici a volume e pressione costante:
\begin{equation}
	\begin{split}
		&c_V = \left(\frac{\partial E}{\partial T} \right) _V = \frac{1}{T} \left(\frac{\partial S}{\partial T} \right) _V = -T \left(\frac{\partial ^2 F}{\partial T^2} \right) _V \\
		&c_P = \left(\frac{\partial E}{\partial T} \right) _P = T \left(\frac{\partial S}{\partial T} \right) _P = -T \left(\frac{\partial ^2 \Phi}{\partial T^2} \right) _P
	\end{split}
\end{equation}
Vale
\begin{equation}
	c_P \ge  c_V
\end{equation}
Compressibilit\`a per trasformazioni isoterma e adiabatica:
\begin{equation}
	k_T = -\frac{1}{V} \left(\frac{\partial V}{\partial P} \right) _T=-\frac{1}{V}\left(\frac{\partial ^2\Phi}{\partial P^2} \right) _T \hspace{.1cm} ; \hspace{.2cm} k_S = -\frac{1}{V } \left(\frac{\partial V}{\partial P} \right) _S = \left[ V \left(\frac{\partial ^2E}{\partial V^2} \right) _S \right] ^{-1} 
\end{equation}
\subsection{Diagrammi di fase}
Grafico che mostra stato fisico di una sostanza in funzione, solitamente, di temperatura e pressione. Assumendo di avere un sistema con due stati coesistenti $\Rightarrow N_1+N_2 = \text{cost.}\Rightarrow \delta N_1 = - \delta N_2$, all'equilibrio:
\[
\frac{\partial F}{\partial N_1} = \frac{\partial }{\partial N_1} (F_1+F_2) = \frac{\partial F_1}{\partial N_1} -\frac{\partial F_2}{\partial N_2} = \mu_1-\mu_2 = 0
\] 
da cui si ottiene relazione $\mu _1 (P,T) = \mu _2(P,T)$ che permette di tracciare grafico $P=f(T)$. Lo stesso si pu\`o fare per tre stati coesistenti, individuando \textit{punto triplo}. 

Da $d\mu_1 = d \mu _2$, si ha $-s_1 dT + v_1 dP = -s_2 dT - v_2dP$, quindi:
\begin{equation}
	\frac{d P}{d T} = \frac{s_2-s_1}{v_2-v_1}
\end{equation}
\subsection{Modello per sistemi statistici}

Si tratteranno i sistemi dividendo l'Universo in sistema in esame ($E,S,T$) $+$ parte complementare, chiamata \textbf{bagno termico} ($E',S',T$). Quest'ultimo sar\`a assunto essere \textit{sempre all'equilibrio e alla stessa temperatura del sistema}. 

La variazione di energia del bagno termico dipende solo da variazione dell'entropia $\Rightarrow \delta E' = T \delta S'$; inoltre essendo l'Universo sempre isolato, la sua variazione di energia \`e nulla $\Rightarrow \delta E + \delta E' =0$. 

Unendo le due, si ha $\delta S' = -\delta E / T$; per il secondo principio, $\delta S + \delta S' \ge 0 \Rightarrow T\delta S - \delta E \ge 0 \Rightarrow \delta (E-TS) \le 0$, da cui si deduce che un sistema \textit{a temperatura fissata} \`e all'equilibrio quando $F = E- TS $ \`e al minimo.

Consentendo scambio di particelle, vale lo stesso principio con $\delta \Omega \le 0$, quindi $\Omega = F - \mu N $ minimo.
\subsubsection{Sistema in bagno termico}

Si indica con $\mathscr{S}$ il sistema immerso in bagno termico $\mathscr{S}'$ e con $\mathscr{S}_0$ l'Universo. Questi hanno rispettivamente dipendenza dalle variabili $(V,N,E,S), \ (V',N',E',S') , \ (V_0,N_0,E_0,S_0)$.

$\mathscr{S}$ si trova in stato quantistico generico indicato tramite serie di numeri quantici $\alpha $; si assume che \textit{il volume sia fissato} e si richiede che: $E_\alpha \ll E_0$ e $N_\alpha  \ll N_0$; in questo modo \textit{temperatura e potenziale chimico del bagno termico sono costanti}.

I microstati dell'Universo sempre equiprobabili perch\'e \`e sempre all'equilibrio $\Rightarrow w_\text{eq} = 1 / \Gamma_0$, con $\Gamma_0$ numero di microstati. La probabilit\`a di avere uno stato $\alpha $ per il sistema, allora \`e $w_\alpha  = \Gamma'_\alpha  / \Gamma_0$, dove $\Gamma'_\alpha $ \`e il numero di microstati in cui $\mathscr{S}$ \`e in $\alpha $ e $\mathscr{S}' $ \`e in uno stato generico.

L'entropia di $\mathscr{S}'$ \`e:
\begin{equation}\label{spa}
	S'_\alpha = \kappa _B \log \Gamma'_\alpha  = S'(E_0 - E_\alpha , N_0 - N_\alpha )
\end{equation}
Inoltre:
\begin{equation}
	S_0 - S'_\alpha  = \kappa _B \log \Gamma_0 - \kappa _B \log \Gamma'_\alpha = - \kappa _B \log \frac{\Gamma'_\alpha }{\Gamma_0} = -\kappa _B \log w_\alpha 
\end{equation}
quindi
\begin{boxenv}[]
\begin{equation}
	w_\alpha  = \exp \left(-\frac{S_0-S'_\alpha }{\kappa _B}\right) \equiv A e^{S'_\alpha  / \kappa _B} 
\end{equation}
\end{boxenv}
\noindent In questo modo, si pu\`o calcolare valore medio dell'entropia per $\mathscr{S}$ (in genere $\alpha $ non \`e uno stato di equilibrio per $\mathscr{S}$):
\begin{equation}
	\langle S  \rangle \equiv \langle S_0 - S'_\alpha  \rangle = - \kappa _B \sum_{\alpha }^{} w_\alpha  \log w_\alpha 
\end{equation}
\subsubsection{Funzione di granpartizione}
Sviluppando in serie eq. \ref{spa}, si ha:
\begin{equation}
	S'_\alpha  \simeq S'(E_0,N_0) - \left(\frac{\partial S'}{\partial E'} \right) _{N'} E_\alpha - \left(\frac{\partial S'}{\partial N'} \right) _{E'} N_\alpha \Rightarrow S'_\alpha  = \text{cost.} - \frac{E_\alpha - \mu  N_\alpha }{T }
\end{equation}
perci\`o la probabilit\`a, comprensiva di normalizzazione, \`e:
\begin{boxenv}[]
\begin{equation}
	w_\alpha = \frac{\exp \left[ -(E_\alpha  - \mu  N_\alpha ) / \kappa _B T \right] }{\sum_{\alpha }^{} \exp \left[ - (E_\alpha  - \mu  N_\alpha ) / \kappa _B T \right] } \equiv \frac{1}{\mathscr{L}} \exp \left(- \frac{E_\alpha - \mu  N_\alpha }{\kappa _B T}\right) 
\end{equation}
\end{boxenv}
\noindent con $\mathscr{L}$ \textbf{funzione di granpartizione}. Nel limite di $N_\alpha  = N, \ \forall \alpha $, $w_\alpha $ tende al caso canonico con normalizzazione data dalla funzione di partizione $\mathscr{Z}$.

\subsubsection{Entropia e potenziali}
Ora si pu\`o calcolare $\langle S_\alpha  \rangle$:
\begin{equation}
	\langle S  \rangle = \kappa \log \mathscr{L} + \frac{1}{T} \sum_{\alpha }^{} w_\alpha  E_\alpha - \frac{\mu }{T} \sum_{\alpha }^{} w_\alpha  N_\alpha = \kappa _B \log\mathscr{L} + \frac{\langle E \rangle - \mu  \langle N \rangle}{T}
\end{equation}
Da questa si ottiene il potenziale di Landau:
\begin{equation}
	\begin{split}
		\Omega &= -\kappa _B T \log \mathscr{L}  = -\kappa _B T \log \sum_{\alpha }^{} \exp \left(- \frac{E_\alpha -\mu N_\alpha }{\kappa _B T}\right)  \\
		       &= -\mu  N - \kappa _B T \log \sum_{\alpha }^{} \exp \left( -\frac{E_\alpha }{\kappa _B T}\right) = -\mu  N- \kappa _B T \log\mathscr{Z}
	\end{split}
\end{equation}
dove si \`e imposto $N_\alpha  = N , \ \forall \alpha $. Conseguentemente $F = \Omega +\mu N = - \kappa _B T \log\mathscr{Z}$. 

\subsubsection{Degenerazione dei livelli energetici}

Ammettendo che \textit{diversi stati occupano stesso livello energetico}, continuando ad assumere $N_\alpha  = N , \ \forall \alpha $:
\begin{equation}
	w(E_\alpha )= \frac{1}{\mathscr{Z}} \rho (E_\alpha ) \exp \left( - \frac{E_\alpha }{\kappa _B T}\right) 
\end{equation}
con $\rho $ degenerazione relativa a energia $E_\alpha $. Passando al continuo:
\begin{equation}
	w(E_\alpha ) \to w (\mathscr{E}) = \frac{1}{\mathscr{Z}} \rho (\mathscr{E}) \exp \left(- \frac{\mathscr{E}}{\kappa _BT}\right) , \ \mathscr{Z}\to \int_{0} ^{+\infty} d \mathscr{E}\ \rho (\mathscr{E}) \exp\left(- \frac{\mathscr{E}}{\kappa _BT}\right) 
\end{equation}
Il numero di microstati si pu\`o riscrivere come:
\begin{equation}
	d\Gamma = \frac{d\Gamma}{d \mathscr{E}}d\mathscr{E} = \rho (\mathscr{E}) d \mathscr{E}
\end{equation}

\subsubsection{Applicazione -- Particelle non-interagenti}

Per $N$ particelle non-interagenti, ciascun grado di libert\`a fattorizza in $\mathscr{Z}$; per particelle \textbf{distinguibili} (distribuzioni diverse delle particelle in microstati individuano stati diversi), si ha $\mathscr{Z}_\text{tot} = \mathscr{Z}_{1p} ^N$; per particelle \textbf{indistinguibili}, una buona stima \`e: $\mathscr{Z}_\text{tot} = \frac{1}{N!}\mathscr{Z}_{1p} ^N$. Si considera il secondo caso.

Si ha $F = - \kappa _B T \log \mathscr{Z}= \kappa _B T \log N! - \kappa _B N T \log \mathscr{Z}_{1p} $. Ricordando che $E_{q_i}  = \frac{\hbar ^2 q_{i} ^2}{2m} $, con $q _ i= \frac{2\pi l_i}{L}$, quindi $E_q \propto L^{-2} = V^{-2 / 3} $, pertanto:
\[
\begin{split}
	P &= - \left(\frac{\partial F}{\partial V} \right) _T = \frac{N\kappa _B T}{\mathscr{Z}_{1p} }\frac{\partial \mathscr{Z}_{1p} }{\partial V} = - \frac{N\kappa _B T}{\mathscr{Z}_{1p} }\frac{1}{\kappa _BT} \sum_{i}^{} \frac{\partial E_{q_i} }{\partial V} \exp \left(- \frac{E_{q_i} }{\kappa _B T}\right) \\
	  &= \frac{2N}{3V}\frac{1}{\mathscr{Z}_{1p} } \sum_{i}^{} E_{q_i} \exp \left( - \frac{E_{q_i} }{\kappa _BT}\right) \equiv \frac{2N}{3V} \langle E \rangle
\end{split}
\] 


\subsubsection{Applicazione -- Gas perfetto}
Particelle confinate in scatola con autostati dell'energia individuati dagli impulsi $q$. Numero di particelle in uno stato \`e $n_q$. 

Per gas ideale, la maggior parte dei microstati sar\`a vuota, cio\`e $w(0) \approx 1$, e la probabilit\`a di avere pi\`u di una particella in un microstato \`e praticamente nulla, quindi $w(1) \ll 1$ e $w(n\ge 2) \approx 0$. Usando $\mathscr{L} = \exp \left[ - \Omega / \kappa _B T \right] $:
\begin{equation}
	w(n_q) = \exp \left[ \frac{\Omega _q - n_q (E_q - \mu )}{\kappa _B T} \right] 
\end{equation}
Allora le condizioni di popolazione dei microstati si traducono in:
\begin{equation}
	\begin{split}
		&w(0) \approx 1 \Rightarrow \exp \left(\frac{\Omega _q}{\kappa _B T}\right) \approx 1\\
		&w(n) = e^{\Omega _q / \kappa _B T} \left[\exp \left(- \frac{E_q - \mu }{\kappa _BT} \right)\right]  ^n\equiv w^n(1)  \ll 1, \forall q \iff \exp \left(\frac{\mu}{\kappa _BT }\right) \ll 1
	\end{split}
\end{equation}
quindi $\mu \to -\infty$. Da questo, il numero medio di particelle in uno stato $q$ \`e:
\begin{equation}
	\langle n_q \rangle = \frac{\sum_{n_q}^{} n_q \exp\left[ -n_q (E_q - \mu ) / \kappa _BT \right] }{\sum_{n_q}^{}\exp\left[ -n_q (E_q - \mu ) / \kappa _BT \right]  } \approx \exp \left(- \frac{E_q - \mu }{\kappa _BT }\right) 
\end{equation}
avendo usato $w(1) \ll 1$. Dal potenziale di Landau, si ottiene equazione di stato dei gas perfetti:
\[
\Omega _q \approx -\kappa _B T \log \left[ 1 + \exp\left(- \frac{E_q - \mu }{\kappa _B T }\right)  \right] = -\kappa _B T \log \Big(1 + \langle n_q \rangle\Big) \approx -\kappa _B T \langle n_q \rangle
\] 
Essendo $\Omega \approx - \kappa _B T \sum_{q}^{}a \langle b_q \rangle  \equiv -\kappa _B T N$ e valendo allo stesso tempo $\Omega = - PV$, si ha $PV = \kappa _B NT$.

\subsubsection{Applicazione -- Sistema a due stati*}
\begin{boxenv}[]
\centering \textit{Sistema in cui particelle interagiscono solo tramite spin\\ Valutare se va scritto} 
\end{boxenv}
\subsection{Spazio delle fasi}

Per sistema di $N$ particelle, \`e uno spazio $6N$-dimensionale delle coordinate e impulsi. Fissare energia dell'Universo equivale a definire un'ipersuperficie $\Sigma_0$ a $(6N-1)$ dimensioni data da $\mathscr{E}_0\big(\left\{ x_i \right\} , \left\{ p_i \right\} \big) = E_0$.

Si discretizza lo spazio in celle che rispettano $\Delta x_k \Delta p_k = \tau $, con $\tau $ costante generica. Si assume che \textit{le celle siano piccoli a sufficienza da avere un solo stato in ciascuna}; allora numero di stati sar\`a area dell'ipersuperficie normalizzata con elemento di volume:
\begin{equation}
	\Gamma_0 = \frac{1}{\tau ^{f_0} } \iint_{\Sigma_0} \prod_{i=1} ^{f_0}  dx_i , dp_i , \ \text{ con } f_0 = 3N
\end{equation}
Allora l'entropia dell'Universo \`e:
\begin{equation}
	S_0 = \kappa _B \log \iint_{\Sigma_0} \prod_{i=1} ^{f_0} dx_i dp_i - \kappa _B f_0 \log \tau 
\end{equation}
Per $\Sigma'$ ipersuperificie data da $E'_\alpha  = E_0 - E_\alpha $, si pu\`o ripetere il discorso per il bagno termico:
\begin{equation}
	S'_\alpha  = \kappa _B \log \iint_{\Sigma'} \prod_{i=1} ^{f'} dx_i dp_i - \kappa _B f' \log \tau 
\end{equation}
In questo modo, l'entropia media del sistema \`e:
\begin{equation}
	\langle S \rangle = \kappa _B \log \iint_{\Sigma_0} \prod_{i=1} ^{f_0} dx_i dp_i - \left\langle \kappa _B \log \iint_{\Sigma'} \prod_{i=1} ^{f'} dx_i dp_i \right\rangle - \kappa _B (f_0-f') \log \tau 
\end{equation}
L'entropia \`e singolare per $\tau \to 0$, quindi deve essere un valore finito.

\subsubsection{Costante di Planck}
Si ricava per particella confinata in segmento $L$. I livelli energetici sono $E_q = \hbar ^2 q^2 / (2m)$ con $q = 2\pi l / L$, e $l \in \mathbb{Z}$. Il conteggio degli stati nella cella $L \Delta p$ \`e $\Delta l = L\Delta q / (2\pi) = L\Delta p /(2\pi\hbar )$; d'altra parte:
\[
\frac{1}{\tau }\int_{L} \int_{\Delta p} dxdp = \frac{L}{\tau }\Delta p \Rightarrow \tau  =2\pi \hbar  = h
\] 
\subsubsection{Applicazione -- Densit\`a di energia ed energia per singola particella}

Per singola particella libera, usando coordinate cilindriche per gli impulsi:
\begin{equation}
	\Gamma = \frac{1}{h^3} \iint d^3 x d^3 p = \frac{V}{h^3} \int 4\pi  p^2 \ dp
\end{equation}
Visto che $\mathscr{E} = p^2 / 2m$, tramite confronto:
\begin{equation}
	\rho (\mathscr{E})d \mathscr{E} = \frac{4\pi V }{h^3}p^2 \frac{d p}{d \mathscr{E}} d \mathscr{E}= \frac{4\pi V m^{3 / 2} }{h^3}\sqrt{2 \mathscr{E}} d \mathscr{E}
\end{equation}
Si pu\`o calcolare l'energia media:
\begin{equation*}
		\langle \mathscr{E} \rangle = \frac{\displaystyle \int_{0} ^{+\infty} \mathscr{E} e^{ - \mathscr{E} / \kappa _B T}  \rho (\mathscr{E}) d \mathscr{E}  }{\displaystyle \int_{0} ^{+\infty} e^{ - \mathscr{E} / \kappa _B T}  \rho (\mathscr{E}) d \mathscr{E}  }= \kappa _B T \frac{\displaystyle \int_{0} ^{+\infty} dx \ x^{3 / 2} e^{ - x} }{\displaystyle \int_{0} ^{+\infty}dx \ x^{1 / 2} } e ^{-x} = \kappa_B T \frac{\Gamma(5 / 2)}{\Gamma (3/2)} = \frac{3}{2} \kappa _B T
\end{equation*}
\subsubsection{Applicazione -- Gas interagente}

Gas non-relativistico immerso in potenziale generico dipendente solo dalle coordinate. Elemento differenziale dello spazio delle fasi \`e $d\Gamma = \rho (\mathscr{E}) d \mathscr{E} = \frac{1}{N!h^{3N} } \prod_{i=1} ^{3N} dx_i dp_i$, da cui essendo $\mathscr{E} = U\big(\left\{ x_i \right\} \big) + \sum_{i=1}^{3N} p_i^2 / 2m$
\[
	\begin{split}
	\mathscr{Z} &= \int e^{- \mathscr{E} / \kappa _B T}  \rho (\mathscr{E}) d \mathscr{E} = \frac{1}{N! h^{3N} }\iint \prod_{i=1} ^{3N} dx_i dp_i e^{- \mathscr{E} / \kappa _B T} \\
			    &= \frac{1}{N! h^{3N} } \iint \prod_{i=1} ^{3N} dx_i dp_i \ \exp \left[ - \frac{\sum_{i=1}^{3N} p_i^2}{2m\kappa _B T} - \frac{U \big(\left\{ x_i \right\} \big)}{\kappa _BT} \right] \\
			    &=\frac{1}{N! h^{3N} } \int \prod _{i=1} ^N d^3 p_i \exp \left[ - \frac{\sum_{i=1}^{N} p_i  ^2}{2m \kappa _B T} \right] \int \prod_{i=1} ^N d^3 x_i \exp\left[ - \frac{U\big(\left\{ x_i \right\} \big)}{\kappa _B T} \right] 
	\end{split}
\] 
Il primo integrale, insieme al prefattore, si pu\`o ricondurre a quello di un gas ideale, a meno di un $V^N$:
\[
		\mathscr{Z}_{IG}= \frac{1}{N!} \left[ \frac{1}{h^3} \iint d^3 x d^3 p \ \exp \left( - \frac{p^2}{2m\kappa _B T } \right) \right] ^N= \frac{V^N}{N! h^{3N} } \int \prod_{i=1} ^N d^3 p_i \ \exp \left(-\frac{\sum_{i=1}^{N} p_i^2}{2m\kappa _BT}\right) = \frac{1}{N!} \frac{V^N}{\Lambda ^{3N} }
\] 
con $\Lambda = 2\pi \hbar /\sqrt{2\pi m\kappa _B T} $ lunghezza d'onda termica di de Broglie. Il secondo dipende dalla forma del potenziale ed \`e detto \textbf{integrale delle configurazioni}:
\begin{equation}
	\mathscr{D} \equiv \int \prod_{i=1} ^N	d^3 x_i \ \exp \left(- \frac{U\big(\left\{ x_i \right\} \big)}{\kappa _BT}\right) 
\end{equation}
Quindi:
\begin{equation}
\mathscr{Z} = \frac{1}{N!} \frac{\mathscr{D}}{\Lambda ^{3N} }
\end{equation}
Per la funzione di granpartizione\footnote{Nella seconda uguagliamza, si spezza la somma, raggruppando i termini della somma stessa in base a $N_\alpha $, per questo $\alpha | N_\alpha $ indica una somma sugli $\alpha $ relativa a ciascun $N_\alpha $.}:
\begin{equation}
	\begin{split}
		\mathscr{L} &= \sum_{\alpha }^{} \exp \left[ - \frac{E_\alpha  - \mu  N_\alpha }{\kappa _B T} \right] = \sum_{N_\alpha }^{} \left[ \exp \left(\frac{\mu N_\alpha }{\kappa _B T}\right) \sum_{\alpha | N_\alpha }^{} \exp \left(- \frac{E_\alpha }{\kappa _B T}\right)  \right] \\
			    & =\sum_{N_\alpha }^{} \left\{ \left[ \exp \left(\frac{\mu }{\kappa _BT}\right)  \right] ^{N_\alpha } \sum_{\alpha | N_\alpha }^{} \exp \left(-\frac{E_\alpha }{\kappa _BT}\right)  \right\} \equiv \sum_{N_\alpha }^{} \left[ z^{N_\alpha } \sum_{\alpha | N_\alpha }^{} \exp\left(- \frac{E_\alpha }{\kappa _BT}\right)  \right] 
	\end{split}
\end{equation}
dove $z$ \`e detta \textbf{fugacit\`a}. Nel limite al continuo, si trova:
\begin{equation}
	\mathscr{L}= \sum_{N}^{} \frac{z^N \mathscr{D}_N}{N! \Lambda ^{3N} }
\end{equation}
con $\mathscr{D}_N$ integrale delle configurazioni relativo agli stati $\alpha $ con $N$ particelle.

\subsection{Distribuzione delle variabili termodinamiche*}
\begin{boxenv}[]
\centering \textit{Valutare se va aggiunto} 
\end{boxenv}
\subsection{Incertezze quantistiche*}
\begin{boxenv}[]
\centering \textit{Valutare se va aggiunto} 
\end{boxenv}












\end{document}
