\documentclass[10pt, a4paper]{scrartcl}
% Packages
%\usepackage{stix}
\usepackage[margin=1.5in]{geometry}
\usepackage{index}
\makeindex
\usepackage[utf8]{inputenc}
\usepackage[T1]{fontenc}
\usepackage{varwidth}
\usepackage{amsmath, amssymb}
\usepackage{esint}
\usepackage{titlesec}
\usepackage{xcolor}
\usepackage{titling}
\usepackage{braket}
\usepackage{tensor}
\usepackage[linktocpage]{hyperref}
\usepackage{pgfplots}
\usepackage{multicol}
\setlength{\columnsep}{2em}
\usepackage{caption}
\usepackage{amsthm}
\usepackage{import}
\usepackage{cancel}
\usepackage{caption}
\usepackage{tcolorbox}
\usepackage{nicematrix}
\usepackage{mathrsfs}
\usepackage{mathtools}
\usepackage{enumerate}
\usepackage{graphicx}
\usepackage{lipsum}
\usepackage[italian]{babel}
% To reset footnote numbering each page
\usepackage[perpage]{footmisc}

%Captions
\captionsetup[figure]{font=footnotesize,labelfont=footnotesize}
\captionsetup[table]{font=footnotesize,labelfont=footnotesize}
%Titlesec
\titleformat{\section}
{\fontsize{15}{20}\sffamily\scshape}
{\normalfont\color{gray}{\fontsize{20}{20}\selectfont\thesection}}
{0.7em}
{}
\hypersetup{colorlinks,breaklinks, linkcolor=[RGB]{74, 122, 164}}

\newcommand\vertarrowbox[3][6ex]{%
  \begin{array}[t]{@{}c@{}} #2 \
  \left\uparrow\vcenter{\hrule height #1}\right.\kern-\nulldelimiterspace\
  \makebox[0pt]{\scriptsize#3}
  \end{array}%
}
\definecolor{asdf}{HTML}{4a7aa4}
% Personalizza la formattazione della subsection
\titleformat{\subsection}[block]{\fontsize{12}{20}\bfseries}{\normalfont\thesubsection}{.5em}{}


% Personalizza la formattazione della subsubsection
\titleformat{\subsubsection}[block]{\fontsize{10}{20}\bfseries}{\normalfont\thesubsubsection}{.5em}{}

% Maketitle customization
\renewcommand{\maketitle}{
\begin{center}
{\sffamily
{\fontsize{20}{20}\selectfont\MakeUppercase\thetitle}}

\vspace{0.2in}

{\large\scshape\sffamily\theauthor}
\end{center}
}

% Titles 
\title{Note di\\\vspace{.2cm} Struttura della Materia}
\author{Manuel Deodato}
\date{}



%Evaluate symbol
\DeclareMathOperator{\di}{d\!}
\newcommand*\Eval[3]{\left.#1\right\rvert_{#2}^{#3}}

%%%%%%% Numero delle equazioni in formato a.b
\numberwithin{equation}{subsection}
%%%%%

%%%%%%%%%% Personalizzazione numeri lista
\renewcommand{\theenumi}{(\arabic{enumi})}

%%%%%%%%%% Medie con integrali multipli
\def\Yint#1{\mathchoice
    {\YYint\displaystyle\textstyle{#1}}%
    {\YYint\textstyle\scriptstyle{#1}}%
    {\YYint\scriptstyle\scriptscriptstyle{#1}}%
    {\YYint\scriptscriptstyle\scriptscriptstyle{#1}}%
      \!\iint}
\def\YYint#1#2#3{{\setbox0=\hbox{$#1{#2#3}{\iint}$}
    \vcenter{\hbox{$#2#3$}}\kern-.51\wd0}}
\def\longdash{{-}\mkern-3.5mu{-}} 
   % consider using "\mkern-7.5mu" if esint package is loaded
\def\tiltlongdash{\rotatebox[origin=c]{15}{$\longdash$}}
\def\fiint{\Yint\tiltlongdash}

\def\Zint#1{\mathchoice
    {\YYint\displaystyle\textstyle{#1}}%
    {\YYint\textstyle\scriptstyle{#1}}%
    {\YYint\scriptstyle\scriptscriptstyle{#1}}%
    {\YYint\scriptscriptstyle\scriptscriptstyle{#1}}%
      \!\iiint}
      \def\tilongdash{\mkern6mu{-}\mkern-4mu{-}\mkern-5mu{-}} 
   % consider using "\mkern-7.5mu" if esint package is loaded
\def\titiltlongdash{\rotatebox[origin=c]{15}{$\tilongdash$}}
\def\fiiint{\Zint\titiltlongdash}


%%%% Table of contents

\usepackage[titles]{tocloft}

\renewcommand{\cftdot}{}
\usepackage{titletoc}
%\setcounter{tocdepth}{2}

%%%%%%%%%%%%%%%% Toc style

% Personalizzazione scritta indice


% Font
\usepackage[osf]{newpxtext}

\usepackage{sansiwona}


% Ambienti
\newtheoremstyle{style1}% name of the style to be used
{15pt}% measure of space to leave above the theorem. E.g.: 3pt
{15pt}% measure of space to leave below the theorem. E.g.: 3pt
{\normalfont}% name of font to use in the body of the theorem
{}% measure of space to indent
{\sffamily\scshape\bfseries}% name of head font
{}% punctuation between head and body
{ }% space after theorem head; " " = normal interword space
{\thmname{#1}\thmnumber{ #2}{\thmnote{~--- #3}}.\newline}




\theoremstyle{style1}
\newtheorem{teorema}{Teorema}[section]
\newtheorem{corollario}{Corollario}[teorema]
\newtheorem{lemma}{Lemma}[teorema]
\newtheorem{definizione}{Definizione}[section]
\newtheorem{osservazione}{Osservazione}[section]
\newtheorem{notazione}{Notazione}[section]
\newtheorem{esempio}{Esempio}[section]
\newtheorem{esercizio}{Esercizio}[section]

\renewcommand\qedsymbol{$\blacksquare$}

\newenvironment{svolgimento}{\renewcommand\qedsymbol{$\spadesuit$}\begin{proof}[Svolgimento]}{\end{proof}}

%% Generic box
\newtcolorbox{eqbox}[1][]
{
colback=gray!10,
arc=0pt,
boxrule=0pt,
title=#1
}

 \newenvironment{boxenv}[1][]{
    \begin{eqbox}[#1]
    }{
   \end{eqbox}
}








%%%%%%%%%%%%%%%%%%%%%%%%%%%%%%%%%%%%%%%%%%%%%%%%%%%%%%%%%%%%%%%%%%%%%%%%

\begin{document}
\maketitle
\newpage
\tableofcontents 
\newpage
\section{Nozioni di meccanica statistica e termodinamica}

\subsection{Gas di particelle}


Si considera gas di particelle non interagenti e puntiformi. Ciascuna particella soddisfa $\hat{H}\psi (\mathbf{r} ) = E  \psi (\mathbf{r} )$ con $\hat{H} = \frac{\hat{\mathbf{p} }}{2m}$ e $E = \frac{\hbar ^2}{2m} q ^2 $, quindi la soluzione generale \`e:
\begin{equation}
	\psi (\mathbf{r} ) = e^{ i \mathbf{q} \cdot \mathbf{r} } 
\end{equation}
Imponendo condizione di periodicit\`a al bordo della scatola $\Rightarrow q_i = \frac{2\pi}{L}l_i, \ l_i = 0, \pm 1,\pm 2,\ldots$. 

Si assumer\`a che particelle interagiscano abbastanza poco da rendere valida questa trattazione, e abbastanza tanto da permettere transizioni di fase.
\subsection{Principi della termodinamica}
\begin{enumerate}[(a).]
	\item \textbf{Primo principio}: per un sistema chiuso (niente scambio di particelle) e isolato, vi \`e conservazione dell'energia interna:
		\begin{equation}
			dE = \delta  Q + \delta L
		\end{equation}
	\item \textbf{Secondo principio}: l'entropia, data da $S = \kappa _B \log \Gamma$\footnote{Questa espressione \`e il caso limite della pi\`u generale $S = \kappa _B \sum_{i}^{} p_i \log p_i$ che si ha quando il sistema non \`e all'equilibrio, cio\`e quando i microstati non sono popolati uniformemente.} (con $\Gamma$ numero di microstati del sistema all'equilibrio), per un sistema isolato, soddisfa $\frac{d S}{d t} \ge 0$. L'uguaglianza vale quando \`e raggiunto l'equilibrio.
	\item \textbf{Terzo principio}: l'entropia tende a $0$ per sistemi perfettamente ordinati, cio\`e sistemi in cui tutte le particelle popolano un solo microstato $\Rightarrow S = \kappa _B \log 1 = 0$. Sistemi perfettamente ordinati sono cristalli perfetti a temperatura nulla; non tutti i materiali a $T=0$ risultano perfettamente ordinati e alcuni presentano entropia residua.
\end{enumerate}
\subsection{Potenziali termodinamici}
A seconda del caso, si usano diverse riscritture dell'energia.
\begin{itemize}
	\item \textbf{Energia libera di Helmholtz}: $F = E - TS \Rightarrow dF = -SdT - PdV$. La sua variazione a temperatura costante restituisce lavoro compiuto sul sistema: $\delta F |_T = - P \delta V |_T = \delta L$.
	\item \textbf{Energia libera di Gibbs}: $\Phi = E - TS + PV = F + PV \Rightarrow d\Phi=  VdP - SdT$. \`E adatta a descrivere transizioni di fase.
	\item \textbf{Entalpia}: $W = E + PV \Rightarrow dW=  TdS + VdP$. La sua variazione a pressione costante \`e il calore scambiato dal sistema: $\delta W|_P = T \delta S|_P = \delta Q$.
\end{itemize}
Se \`e possibile scambio di particelle, la dipendenza da $N$ nei potenziali si aggiunge con:
\begin{equation}
	\mu = \left(\frac{\partial E}{\partial N} \right) _{SV} = \left(\frac{\partial F}{\partial N} \right) _{TV} = \left(\frac{\partial \Phi}{\partial N} \right) _{SP} 
\end{equation}
$\mu $ \`e esso stesso un potenziale: $d\mu = - S / N dT + V / N dP = - s dT + vdP$\footnote{Aggiungendo particelle ferme ad un sistema, \`e ragionevole avere $\mu < 0$, visto che l'energia media diminuirebbe con l'aumentare di $N$.}. Un altro potenziale utile \`e il \textbf{potenziale di Landau}: $\Omega = F - \mu N\Rightarrow d\Omega = -SdT - PdV - N d\mu $.

\subsection{Modello per sistemi statistici}

Si tratteranno i sistemi dividendo l'Universo in sistema in esame ($E,S,T$) $+$ parte complementare, chiamata \textbf{bagno termico} ($E',S',T$). Quest'ultimo sar\`a assunto essere \textit{sempre all'equilibrio e alla stessa temperatura del sistema}. 

La variazione di energia del bagno termico dipende solo da variazione dell'entropia $\Rightarrow \delta E' = T \delta S'$; inoltre essendo l'Universo sempre isolato, la sua variazione di energia \`e nulla $\Rightarrow \delta E + \delta E' =0$. 

Unendo le due, si ha $\delta S' = -\delta E / T$; per il secondo principio, $\delta S + \delta S' \ge 0 \Rightarrow T\delta S - \delta E \ge 0 \Rightarrow \delta (E-TS) \le 0$, da cui si deduce che un sistema \textit{a temperatura fissata} \`e all'equilibrio quando $F = E- TS $ \`e al minimo.

Consentendo scambio di particelle, vale lo stesso principio con $\delta \Omega \le 0$, quindi $\Omega = F - \mu N $ minimo.

\subsection{Calori specifici e compressibilit\`a}

Calori specifici a volume e pressione costante:
\begin{equation}
	\begin{split}
		&c_V = \left(\frac{\partial E}{\partial T} \right) _V = \frac{1}{T} \left(\frac{\partial S}{\partial T} \right) _V = -T \left(\frac{\partial ^2 F}{\partial T^2} \right) _V \\
		&c_P = \left(\frac{\partial E}{\partial T} \right) _P = T \left(\frac{\partial S}{\partial T} \right) _P = -T \left(\frac{\partial ^2 \Phi}{\partial T^2} \right) _P
	\end{split}
\end{equation}
Vale
\begin{equation}
	c_P \ge  c_V
\end{equation}
Compressibilit\`a per trasformazioni isoterma e adiabatica:
\begin{equation}
	k_T = -\frac{1}{V} \left(\frac{\partial V}{\partial P} \right) _T=-\frac{1}{V}\left(\frac{\partial ^2\Phi}{\partial P^2} \right) _T \hspace{.1cm} ; \hspace{.2cm} k_S = -\frac{1}{V } \left(\frac{\partial V}{\partial P} \right) _S = \left[ V \left(\frac{\partial ^2E}{\partial V^2} \right) _S \right] ^{-1} 
\end{equation}
\subsection{Diagrammi di fase}
Grafico che mostra stato fisico di una sostanza in funzione, solitamente, di temperatura e pressione. Assumendo di avere un sistema con due stati coesistenti $\Rightarrow N_1+N_2 = \text{cost.}\Rightarrow \delta N_1 = - \delta N_2$, all'equilibrio:
\[
\frac{\partial F}{\partial N_1} = \frac{\partial }{\partial N_1} (F_1+F_2) = \frac{\partial F_1}{\partial N_1} -\frac{\partial F_2}{\partial N_2} = \mu_1-\mu_2 = 0
\] 
da cui si ottiene relazione $\mu _1 (P,T) = \mu _2(P,T)$ che permette di tracciare grafico $P=f(T)$. Lo stesso si pu\`o fare per tre stati coesistenti, individuando \textit{punto triplo}. 

Da $d\mu_1 = d \mu _2$, si ha $-s_1 dT + v_1 dP = -s_2 dT - v_2dP$, quindi:
\begin{equation}
	\frac{d P}{d T} = \frac{s_2-s_1}{v_2-v_1}
\end{equation}

\subsection{Sistema in bagno termico}









\end{document}
