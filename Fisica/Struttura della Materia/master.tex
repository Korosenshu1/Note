\documentclass[10pt, a4paper]{scrartcl}
% Packages
\usepackage[margin=1.5in]{geometry}
\usepackage{index}
\makeindex
\usepackage[utf8]{inputenc}
\usepackage[T1]{fontenc}
\usepackage{varwidth}
\usepackage{amsmath, amssymb}
\usepackage{esint}
\usepackage{titlesec}
\usepackage{xcolor}
\usepackage{titling}
\usepackage{braket}
\usepackage{tensor}
\usepackage[linktocpage]{hyperref}
\usepackage{pgfplots}
\usepackage{multicol}
\setlength{\columnsep}{2em}
\usepackage{caption}
\usepackage{amsthm}
\usepackage{import}
\usepackage{cancel}
\usepackage{caption}
\usepackage{tcolorbox}
\usepackage{nicematrix}
\usepackage{mathrsfs}
\usepackage{mathtools}
\usepackage{enumerate}
\usepackage{graphicx}
\usepackage{lipsum}
\usepackage[italian]{babel}
% To reset footnote numbering each page
\usepackage[perpage]{footmisc}
\usepackage{cmupint}
\usepackage{newtxmath}
%Captions
\captionsetup[figure]{font=footnotesize,labelfont=footnotesize}
\captionsetup[table]{font=footnotesize,labelfont=footnotesize}
%Titlesec
\titleformat{\section}
{\fontsize{15}{20}\sffamily\scshape}
{\normalfont\color{gray}{\fontsize{20}{20}\selectfont\thesection}}
{0.7em}
{}
\hypersetup{colorlinks,breaklinks, linkcolor=[RGB]{74, 122, 164}}

\newcommand\vertarrowbox[3][6ex]{%
  \begin{array}[t]{@{}c@{}} #2 \
  \left\uparrow\vcenter{\hrule height #1}\right.\kern-\nulldelimiterspace\
  \makebox[0pt]{\scriptsize#3}
  \end{array}%
}
\definecolor{asdf}{HTML}{4a7aa4}
% Personalizza la formattazione della subsection
\titleformat{\subsection}[block]{\fontsize{12}{20}\bfseries}{\normalfont\thesubsection}{.5em}{}


% Personalizza la formattazione della subsubsection
\titleformat{\subsubsection}[block]{\fontsize{10}{20}\bfseries}{\normalfont\thesubsubsection}{.5em}{}

% Maketitle customization
\renewcommand{\maketitle}{
\begin{center}
{\sffamily
{\fontsize{20}{20}\selectfont\MakeUppercase\thetitle}}

\vspace{0.2in}

{\large\scshape\sffamily\theauthor}
\end{center}
}

% Titles 
\title{Note di\\\vspace{.2cm} Struttura della Materia}
\author{Manuel Deodato}
\date{}



%Evaluate symbol
\DeclareMathOperator{\di}{d\!}
\newcommand*\Eval[3]{\left.#1\right\rvert_{#2}^{#3}}

%%%%%%% Numero delle equazioni in formato a.b
\numberwithin{equation}{subsection}
%%%%%

%%%%%%%%%% Personalizzazione numeri lista
\renewcommand{\theenumi}{(\arabic{enumi})}

%%%%%%%%%% Medie con integrali multipli
\def\Yint#1{\mathchoice
    {\YYint\displaystyle\textstyle{#1}}%
    {\YYint\textstyle\scriptstyle{#1}}%
    {\YYint\scriptstyle\scriptscriptstyle{#1}}%
    {\YYint\scriptscriptstyle\scriptscriptstyle{#1}}%
      \!\iint}
\def\YYint#1#2#3{{\setbox0=\hbox{$#1{#2#3}{\iint}$}
    \vcenter{\hbox{$#2#3$}}\kern-.51\wd0}}
\def\longdash{{-}\mkern-3.5mu{-}} 
   % consider using "\mkern-7.5mu" if esint package is loaded
\def\tiltlongdash{\rotatebox[origin=c]{15}{$\longdash$}}
\def\fiint{\Yint\tiltlongdash}

\def\Zint#1{\mathchoice
    {\YYint\displaystyle\textstyle{#1}}%
    {\YYint\textstyle\scriptstyle{#1}}%
    {\YYint\scriptstyle\scriptscriptstyle{#1}}%
    {\YYint\scriptscriptstyle\scriptscriptstyle{#1}}%
      \!\iiint}
      \def\tilongdash{\mkern6mu{-}\mkern-4mu{-}\mkern-5mu{-}} 
   % consider using "\mkern-7.5mu" if esint package is loaded
\def\titiltlongdash{\rotatebox[origin=c]{15}{$\tilongdash$}}
\def\fiiint{\Zint\titiltlongdash}


%%%% Table of contents

\usepackage[titles]{tocloft}

\renewcommand{\cftdot}{}
\usepackage{titletoc}
%\setcounter{tocdepth}{2}

%%%%%%%%%%%%%%%% Toc style

% Personalizzazione scritta indice


% Font
\usepackage[osf]{newpxtext}

\usepackage{sansiwona}


% Ambienti
\newtheoremstyle{style1}% name of the style to be used
{15pt}% measure of space to leave above the theorem. E.g.: 3pt
{15pt}% measure of space to leave below the theorem. E.g.: 3pt
{\normalfont}% name of font to use in the body of the theorem
{}% measure of space to indent
{\sffamily\scshape\bfseries}% name of head font
{}% punctuation between head and body
{ }% space after theorem head; " " = normal interword space
{\thmname{#1}\thmnumber{ #2}{\thmnote{~--- #3}}.}




\theoremstyle{style1}
\newtheorem{teorema}{Teorema}[section]
\newtheorem{corollario}{Corollario}[teorema]
\newtheorem{lemma}{Lemma}[teorema]
\newtheorem{definizione}{Definizione}[section]
\newtheorem{osservazione}{Osservazione}[section]
\newtheorem{notazione}{Notazione}[section]
\newtheorem{esempio}{Esempio}[section]
\newtheorem{esercizio}{Esercizio}[section]

\renewcommand\qedsymbol{$\blacksquare$}

\newenvironment{svolgimento}{\renewcommand\qedsymbol{$\spadesuit$}\begin{proof}[Svolgimento]}{\end{proof}}

%% Generic box
\newtcolorbox{eqbox}[1][]
{
colback=gray!10,
arc=0pt,
boxrule=0pt,
title=#1
}

 \newenvironment{boxenv}[1][]{
    \begin{eqbox}[#1]
    }{
   \end{eqbox}
}







%%%%%%%%%%%%%%%%%%%%%%%%%%%%%%%%%%%%%%%%%%%%%%%%%%%%%%%%%%%%%%%%%%%%%%%%

\begin{document}
\maketitle
\newpage
\tableofcontents 
\newpage
\section{Nozioni di meccanica statistica e termodinamica}

\subsection{Gas di particelle}


Si considera gas di particelle non interagenti e puntiformi. Ciascuna particella soddisfa $\hat{H}\psi (\mathbf{r} ) = E  \psi (\mathbf{r} )$ con $\hat{H} = \frac{\hat{\mathbf{p} }}{2m}$ e $E = \frac{\hbar ^2}{2m} q ^2 $, quindi la soluzione generale \`e:
\begin{equation}
	\psi (\mathbf{r} ) = e^{ i \mathbf{q} \cdot \mathbf{r} } 
\end{equation}
Imponendo condizione di periodicit\`a al bordo della scatola $\Rightarrow q_i = \frac{2\pi}{L}l_i, \ l_i = 0, \pm 1,\pm 2,\ldots$. 

Si assumer\`a che particelle interagiscano abbastanza poco da rendere valida questa trattazione, e abbastanza tanto da permettere transizioni di fase.
\subsection{Principi della termodinamica}
\begin{enumerate}[(a).]
	\item \textbf{Primo principio}: per un sistema chiuso (niente scambio di particelle) e isolato, vi \`e conservazione dell'energia interna:
		\begin{equation}
			dE = \delta  Q + \delta L
		\end{equation}
	\item \textbf{Secondo principio}: l'entropia, data da $S = \kappa _B \log \Gamma$\footnote{Questa espressione \`e il caso limite della pi\`u generale $S = \kappa _B \sum_{i}^{} p_i \log p_i$ che si ha quando il sistema non \`e all'equilibrio, cio\`e quando i microstati non sono popolati uniformemente.} (con $\Gamma$ numero di microstati del sistema all'equilibrio), per un sistema isolato, soddisfa $\frac{d S}{d t} \ge 0$. L'uguaglianza vale quando \`e raggiunto l'equilibrio.
	\item \textbf{Terzo principio}: l'entropia tende a $0$ per sistemi perfettamente ordinati, cio\`e sistemi in cui tutte le particelle popolano un solo microstato $\Rightarrow S = \kappa _B \log 1 = 0$. Sistemi perfettamente ordinati sono cristalli perfetti a temperatura nulla; non tutti i materiali a $T=0$ risultano perfettamente ordinati e alcuni presentano entropia residua.
\end{enumerate}
\subsection{Potenziali termodinamici}
A seconda del caso, si usano diverse riscritture dell'energia.
\begin{itemize}
	\item \textbf{Energia libera di Helmholtz}: $F = E - TS \Rightarrow dF = -SdT - PdV$. La sua variazione a temperatura costante restituisce lavoro compiuto sul sistema: $\delta F |_T = - P \delta V |_T = \delta L$.
	\item \textbf{Energia libera di Gibbs}: $\Phi = E - TS + PV = F + PV \Rightarrow d\Phi=  VdP - SdT$. \`E adatta a descrivere transizioni di fase.
	\item \textbf{Entalpia}: $W = E + PV \Rightarrow dW=  TdS + VdP$. La sua variazione a pressione costante \`e il calore scambiato dal sistema: $\delta W|_P = T \delta S|_P = \delta Q$.
\end{itemize}
Se \`e possibile scambio di particelle, la dipendenza da $N$ nei potenziali si aggiunge con:
\begin{equation}
	\mu = \left(\frac{\partial E}{\partial N} \right) _{SV} = \left(\frac{\partial F}{\partial N} \right) _{TV} = \left(\frac{\partial \Phi}{\partial N} \right) _{SP} 
\end{equation}
$\mu $ \`e esso stesso un potenziale: $d\mu = - S / N dT + V / N dP = - s dT + vdP$\footnote{Aggiungendo particelle ferme ad un sistema, \`e ragionevole avere $\mu < 0$, visto che l'energia media diminuirebbe con l'aumentare di $N$.}. Un altro potenziale utile \`e il \textbf{potenziale di Landau}: $\Omega = F - \mu N\Rightarrow d\Omega = -SdT - PdV - N d\mu $.



\subsection{Calori specifici e compressibilit\`a}

Calori specifici a volume e pressione costante:
\begin{equation}
	\begin{split}
		&c_V = \left(\frac{\partial E}{\partial T} \right) _V = T \left(\frac{\partial S}{\partial T} \right) _V = -T \left(\frac{\partial ^2 F}{\partial T^2} \right) _V \\
		&c_P = \left(\frac{\partial E}{\partial T} \right) _P = T \left(\frac{\partial S}{\partial T} \right) _P = -T \left(\frac{\partial ^2 \Phi}{\partial T^2} \right) _P
	\end{split}
\end{equation}
Vale
\begin{equation}
	c_P \ge  c_V
\end{equation}
Compressibilit\`a per trasformazioni isoterma e adiabatica:
\begin{equation}
	k_T = -\frac{1}{V} \left(\frac{\partial V}{\partial P} \right) _T=-\frac{1}{V}\left(\frac{\partial ^2\Phi}{\partial P^2} \right) _T \hspace{.1cm} ; \hspace{.2cm} k_S = -\frac{1}{V } \left(\frac{\partial V}{\partial P} \right) _S = \left[ V \left(\frac{\partial ^2E}{\partial V^2} \right) _S \right] ^{-1} 
\end{equation}
\subsection{Diagrammi di fase}
Grafico che mostra stato fisico di una sostanza in funzione, solitamente, di temperatura e pressione. Assumendo di avere un sistema con due stati coesistenti $\Rightarrow N_1+N_2 = \text{cost.}\Rightarrow \delta N_1 = - \delta N_2$, all'equilibrio:
\[
\frac{\partial F}{\partial N_1} = \frac{\partial }{\partial N_1} (F_1+F_2) = \frac{\partial F_1}{\partial N_1} -\frac{\partial F_2}{\partial N_2} = \mu_1-\mu_2 = 0
\] 
da cui si ottiene relazione $\mu _1 (P,T) = \mu _2(P,T)$ che permette di tracciare grafico $P=f(T)$. Lo stesso si pu\`o fare per tre stati coesistenti, individuando \textit{punto triplo}. 

Da $d\mu_1 = d \mu _2$, si ha $-s_1 dT + v_1 dP = -s_2 dT - v_2dP$, quindi:
\begin{equation}
	\frac{d P}{d T} = \frac{s_2-s_1}{v_2-v_1}
\end{equation}
\subsection{Modello per sistemi statistici}

Si tratteranno i sistemi dividendo l'Universo in sistema in esame ($E,S,T$) $+$ parte complementare, chiamata \textbf{bagno termico} ($E',S',T$). Quest'ultimo sar\`a assunto essere \textit{sempre all'equilibrio e alla stessa temperatura del sistema}. 

La variazione di energia del bagno termico dipende solo da variazione dell'entropia $\Rightarrow \delta E' = T \delta S'$; inoltre essendo l'Universo sempre isolato, la sua variazione di energia \`e nulla $\Rightarrow \delta E + \delta E' =0$. 

Unendo le due, si ha $\delta S' = -\delta E / T$; per il secondo principio, $\delta S + \delta S' \ge 0 \Rightarrow T\delta S - \delta E \ge 0 \Rightarrow \delta (E-TS) \le 0$, da cui si deduce che un sistema \textit{a temperatura fissata} \`e all'equilibrio quando $F = E- TS $ \`e al minimo.

Consentendo scambio di particelle, vale lo stesso principio con $\delta \Omega \le 0$, quindi $\Omega = F - \mu N $ minimo.
\subsubsection{Sistema in bagno termico}

Si indica con $\mathscr{S}$ il sistema immerso in bagno termico $\mathscr{S}'$ e con $\mathscr{S}_0$ l'Universo. Questi hanno rispettivamente dipendenza dalle variabili $(V,N,E,S), \ (V',N',E',S') , \ (V_0,N_0,E_0,S_0)$.

$\mathscr{S}$ si trova in stato quantistico generico indicato tramite serie di numeri quantici $\alpha $; si assume che \textit{il volume sia fissato} e si richiede che: $E_\alpha \ll E_0$ e $N_\alpha  \ll N_0$; in questo modo \textit{temperatura e potenziale chimico del bagno termico sono costanti}.

I microstati dell'Universo sempre equiprobabili perch\'e \`e sempre all'equilibrio $\Rightarrow w_\text{eq} = 1 / \Gamma_0$, con $\Gamma_0$ numero di microstati. La probabilit\`a di avere uno stato $\alpha $ per il sistema, allora \`e $w_\alpha  = \Gamma'_\alpha  / \Gamma_0$, dove $\Gamma'_\alpha $ \`e il numero di microstati in cui $\mathscr{S}$ \`e in $\alpha $ e $\mathscr{S}' $ \`e in uno stato generico.

L'entropia di $\mathscr{S}'$ \`e:
\begin{equation}\label{spa}
	S'_\alpha = \kappa _B \log \Gamma'_\alpha  = S'(E_0 - E_\alpha , N_0 - N_\alpha )
\end{equation}
Inoltre:
\begin{equation}
	S_0 - S'_\alpha  = \kappa _B \log \Gamma_0 - \kappa _B \log \Gamma'_\alpha = - \kappa _B \log \frac{\Gamma'_\alpha }{\Gamma_0} = -\kappa _B \log w_\alpha 
\end{equation}
quindi
\begin{boxenv}[]
\begin{equation}
	w_\alpha  = \exp \left(-\frac{S_0-S'_\alpha }{\kappa _B}\right) \equiv A e^{S'_\alpha  / \kappa _B} 
\end{equation}
\end{boxenv}
\noindent In questo modo, si pu\`o calcolare valore medio dell'entropia per $\mathscr{S}$ (in genere $\alpha $ non \`e uno stato di equilibrio per $\mathscr{S}$):
\begin{equation}
	\langle S  \rangle \equiv \langle S_0 - S'_\alpha  \rangle = - \kappa _B \sum_{\alpha }^{} w_\alpha  \log w_\alpha 
\end{equation}
\subsubsection{Funzione di granpartizione}
Sviluppando in serie eq. \ref{spa}, si ha:
\begin{equation}
	S'_\alpha  \simeq S'(E_0,N_0) - \left(\frac{\partial S'}{\partial E'} \right) _{N'} E_\alpha - \left(\frac{\partial S'}{\partial N'} \right) _{E'} N_\alpha \Rightarrow S'_\alpha  = \text{cost.} - \frac{E_\alpha - \mu  N_\alpha }{T }
\end{equation}
perci\`o la probabilit\`a, comprensiva di normalizzazione, \`e:
\begin{boxenv}[]
\begin{equation}
	w_\alpha = \frac{\exp \left[ -(E_\alpha  - \mu  N_\alpha ) / \kappa _B T \right] }{\sum_{\alpha }^{} \exp \left[ - (E_\alpha  - \mu  N_\alpha ) / \kappa _B T \right] } \equiv \frac{1}{\mathscr{L}} \exp \left(- \frac{E_\alpha - \mu  N_\alpha }{\kappa _B T}\right) 
\end{equation}
\end{boxenv}
\noindent con $\mathscr{L}$ \textbf{funzione di granpartizione}. Nel limite di $N_\alpha  = N, \ \forall \alpha $, $w_\alpha $ tende al caso canonico con normalizzazione data dalla funzione di partizione $\mathscr{Z}$.

\subsubsection{Entropia e potenziali}
Ora si pu\`o calcolare $\langle S \rangle$:
\begin{equation}
	\langle S  \rangle = \kappa \log \mathscr{L} + \frac{1}{T} \sum_{\alpha }^{} w_\alpha  E_\alpha - \frac{\mu }{T} \sum_{\alpha }^{} w_\alpha  N_\alpha = \kappa _B \log\mathscr{L} + \frac{\langle E \rangle - \mu  \langle N \rangle}{T}
\end{equation}
Da questa si ottiene il potenziale di Landau:
\begin{equation}
	\begin{split}
		\Omega &= -\kappa _B T \log \mathscr{L}  = -\kappa _B T \log \sum_{\alpha }^{} \exp \left(- \frac{E_\alpha -\mu N_\alpha }{\kappa _B T}\right)  \\
		       &= -\mu  N - \kappa _B T \log \sum_{\alpha }^{} \exp \left( -\frac{E_\alpha }{\kappa _B T}\right) = -\mu  N- \kappa _B T \log\mathscr{Z}
	\end{split}
\end{equation}
dove si \`e imposto $N_\alpha  = N , \ \forall \alpha $. Conseguentemente $F = \Omega +\mu N = - \kappa _B T \log\mathscr{Z}$. 

\subsubsection{Degenerazione dei livelli energetici}

Ammettendo che \textit{diversi stati occupano stesso livello energetico}, continuando ad assumere $N_\alpha  = N , \ \forall \alpha $:
\begin{equation}
	w(E_\alpha )= \frac{1}{\mathscr{Z}} \rho (E_\alpha ) \exp \left( - \frac{E_\alpha }{\kappa _B T}\right) 
\end{equation}
con $\rho $ degenerazione relativa a energia $E_\alpha $. Passando al continuo:
\begin{equation}
	w(E_\alpha ) \to w (\mathscr{E}) = \frac{1}{\mathscr{Z}} \rho (\mathscr{E}) \exp \left(- \frac{\mathscr{E}}{\kappa _BT}\right) , \ \mathscr{Z}\to \int_{0} ^{+\infty} d \mathscr{E}\ \rho (\mathscr{E}) \exp\left(- \frac{\mathscr{E}}{\kappa _BT}\right) 
\end{equation}
Il numero di microstati si pu\`o riscrivere come:
\begin{equation}
	d\Gamma = \frac{d\Gamma}{d \mathscr{E}}d\mathscr{E} = \rho (\mathscr{E}) d \mathscr{E}
\end{equation}

\subsubsection{Applicazione -- Particelle non-interagenti}

Per $N$ particelle non-interagenti, ciascun grado di libert\`a fattorizza in $\mathscr{Z}$; per particelle \textbf{distinguibili} (distribuzioni diverse delle particelle in microstati individuano stati diversi), si ha $\mathscr{Z}_\text{tot} = \mathscr{Z}_{1p} ^N$; per particelle \textbf{indistinguibili}, una buona stima \`e: $\mathscr{Z}_\text{tot} = \frac{1}{N!}\mathscr{Z}_{1p} ^N$. Si considera il secondo caso.

Si ha $F = - \kappa _B T \log \mathscr{Z}= \kappa _B T \log N! - \kappa _B N T \log \mathscr{Z}_{1p} $. Ricordando che $E_{q_i}  = \frac{\hbar ^2 q_{i} ^2}{2m} $, con $q _ i= \frac{2\pi l_i}{L}$, quindi $E_q \propto L^{-2} = V^{-2 / 3} $, pertanto:
\[
\begin{split}
	P &= - \left(\frac{\partial F}{\partial V} \right) _T = \frac{N\kappa _B T}{\mathscr{Z}_{1p} }\frac{\partial \mathscr{Z}_{1p} }{\partial V} = - \frac{N\kappa _B T}{\mathscr{Z}_{1p} }\frac{1}{\kappa _BT} \sum_{i}^{} \frac{\partial E_{q_i} }{\partial V} \exp \left(- \frac{E_{q_i} }{\kappa _B T}\right) \\
	  &= \frac{2N}{3V}\frac{1}{\mathscr{Z}_{1p} } \sum_{i}^{} E_{q_i} \exp \left( - \frac{E_{q_i} }{\kappa _BT}\right) \equiv \frac{2N}{3V} \langle E \rangle
\end{split}
\] 


\subsubsection{Applicazione -- Sistema a due stati*}
\begin{boxenv}[]
\centering \textit{Sistema in cui particelle interagiscono solo tramite spin\\ Valutare se va scritto} 
\end{boxenv}
\subsection{Spazio delle fasi}

Per sistema di $N$ particelle, \`e uno spazio $6N$-dimensionale delle coordinate e impulsi. Fissare energia dell'Universo equivale a definire un'ipersuperficie $\Sigma_0$ a $(6N-1)$ dimensioni data da $\mathscr{E}_0\big(\left\{ x_i \right\} , \left\{ p_i \right\} \big) = E_0$.

Si discretizza lo spazio in celle che rispettano $\Delta x_k \Delta p_k = \tau $, con $\tau $ costante generica. Si assume che \textit{le celle siano piccoli a sufficienza da avere un solo stato in ciascuna}; allora numero di stati sar\`a area dell'ipersuperficie normalizzata con elemento di volume:
\begin{equation}
	\Gamma_0 = \frac{1}{\tau ^{f_0} } \iint_{\Sigma_0} \prod_{i=1} ^{f_0}  dx_i , dp_i , \ \text{ con } f_0 = 3N
\end{equation}
Allora l'entropia dell'Universo \`e:
\begin{equation}
	S_0 = \kappa _B \log \iint_{\Sigma_0} \prod_{i=1} ^{f_0} dx_i dp_i - \kappa _B f_0 \log \tau 
\end{equation}
Per $\Sigma'$ ipersuperificie data da $E'_\alpha  = E_0 - E_\alpha $, si pu\`o ripetere il discorso per il bagno termico:
\begin{equation}
	S'_\alpha  = \kappa _B \log \iint_{\Sigma'} \prod_{i=1} ^{f'} dx_i dp_i - \kappa _B f' \log \tau 
\end{equation}
In questo modo, l'entropia media del sistema \`e:
\begin{equation}
	\langle S \rangle = \kappa _B \log \iint_{\Sigma_0} \prod_{i=1} ^{f_0} dx_i dp_i - \left\langle \kappa _B \log \iint_{\Sigma'} \prod_{i=1} ^{f'} dx_i dp_i \right\rangle - \kappa _B (f_0-f') \log \tau 
\end{equation}
L'entropia \`e singolare per $\tau \to 0$, quindi deve essere un valore finito.

\subsubsection{Costante di Planck}
Si ricava per particella confinata in segmento $L$. I livelli energetici sono $E_q = \hbar ^2 q^2 / (2m)$ con $q = 2\pi l / L$, e $l \in \mathbb{Z}$. Il conteggio degli stati nella cella $L \Delta p$ \`e $\Delta l = L\Delta q / (2\pi) = L\Delta p /(2\pi\hbar )$; d'altra parte:
\[
\frac{1}{\tau }\int_{L} \int_{\Delta p} dxdp = \frac{L}{\tau }\Delta p \Rightarrow \tau  =2\pi \hbar  = h
\] 
\subsubsection{Applicazione -- Densit\`a di energia ed energia per singola particella}\label{1p}

Per singola particella libera, usando coordinate cilindriche per gli impulsi:
\begin{equation}
	\Gamma = \frac{1}{h^3} \iint d^3 x d^3 p = \frac{V}{h^3} \int 4\pi  p^2 \ dp
\end{equation}
Visto che $\mathscr{E} = p^2 / 2m$, tramite confronto:
\begin{equation}
	\rho (\mathscr{E})d \mathscr{E} = \frac{4\pi V }{h^3}p^2 \frac{d p}{d \mathscr{E}} d \mathscr{E}= \frac{4\pi V m^{3 / 2} }{h^3}\sqrt{2 \mathscr{E}} d \mathscr{E}
\end{equation}
Si pu\`o calcolare l'energia media:
\begin{equation*}
		\langle \mathscr{E} \rangle = \frac{\displaystyle \int_{0} ^{+\infty} \mathscr{E} e^{ - \mathscr{E} / \kappa _B T}  \rho (\mathscr{E}) d \mathscr{E}  }{\displaystyle \int_{0} ^{+\infty} e^{ - \mathscr{E} / \kappa _B T}  \rho (\mathscr{E}) d \mathscr{E}  }= \kappa _B T \frac{\displaystyle \int_{0} ^{+\infty} dx \ x^{3 / 2} e^{ - x} }{\displaystyle \int_{0} ^{+\infty}dx \ x^{1 / 2} } e ^{-x} = \kappa_B T \frac{\Gamma(5 / 2)}{\Gamma (3/2)} = \frac{3}{2} \kappa _B T
\end{equation*}
\subsubsection{Applicazione -- Gas interagente}

Gas non-relativistico immerso in potenziale generico dipendente solo dalle coordinate. Elemento differenziale dello spazio delle fasi \`e $d\Gamma = \rho (\mathscr{E}) d \mathscr{E} = \frac{1}{N!h^{3N} } \prod_{i=1} ^{3N} dx_i dp_i$, da cui essendo $\mathscr{E} = U\big(\left\{ x_i \right\} \big) + \sum_{i=1}^{3N} p_i^2 / 2m$
\[
	\begin{split}
	\mathscr{Z} &= \int e^{- \mathscr{E} / \kappa _B T}  \rho (\mathscr{E}) d \mathscr{E} = \frac{1}{N! h^{3N} }\iint \prod_{i=1} ^{3N} dx_i dp_i e^{- \mathscr{E} / \kappa _B T} \\
			    &= \frac{1}{N! h^{3N} } \iint \prod_{i=1} ^{3N} dx_i dp_i \ \exp \left[ - \frac{\sum_{i=1}^{3N} p_i^2}{2m\kappa _B T} - \frac{U \big(\left\{ x_i \right\} \big)}{\kappa _BT} \right] \\
			    &=\frac{1}{N! h^{3N} } \int \prod _{i=1} ^N d^3 p_i \exp \left[ - \frac{\sum_{i=1}^{N} p_i  ^2}{2m \kappa _B T} \right] \int \prod_{i=1} ^N d^3 x_i \exp\left[ - \frac{U\big(\left\{ x_i \right\} \big)}{\kappa _B T} \right] 
	\end{split}
\] 
Il primo integrale, insieme al prefattore, si pu\`o ricondurre a quello di un gas ideale, a meno di un $V^N$:
\[
		\mathscr{Z}_{IG}= \frac{1}{N!} \left[ \frac{1}{h^3} \iint d^3 x d^3 p \ \exp \left( - \frac{p^2}{2m\kappa _B T } \right) \right] ^N= \frac{V^N}{N! h^{3N} } \int \prod_{i=1} ^N d^3 p_i \ \exp \left(-\frac{\sum_{i=1}^{N} p_i^2}{2m\kappa _BT}\right) = \frac{1}{N!} \frac{V^N}{\Lambda ^{3N} }
\] 
con $\Lambda = 2\pi \hbar /\sqrt{2\pi m\kappa _B T} $ lunghezza d'onda termica di de Broglie. Il secondo dipende dalla forma del potenziale ed \`e detto \textbf{integrale delle configurazioni}:
\begin{equation}
	\mathscr{D} \equiv \int \prod_{i=1} ^N	d^3 x_i \ \exp \left(- \frac{U\big(\left\{ x_i \right\} \big)}{\kappa _BT}\right) 
\end{equation}
Quindi:
\begin{equation}
\mathscr{Z} = \frac{1}{N!} \frac{\mathscr{D}}{\Lambda ^{3N} }
\end{equation}
Per la funzione di granpartizione\footnote{Nella seconda uguagliamza, si spezza la somma, raggruppando i termini della somma stessa in base a $N_\alpha $, per questo $\alpha | N_\alpha $ indica una somma sugli $\alpha $ relativa a ciascun $N_\alpha $.}:
\begin{equation}
	\begin{split}
		\mathscr{L} &= \sum_{\alpha }^{} \exp \left[ - \frac{E_\alpha  - \mu  N_\alpha }{\kappa _B T} \right] = \sum_{N_\alpha }^{} \left[ \exp \left(\frac{\mu N_\alpha }{\kappa _B T}\right) \sum_{\alpha | N_\alpha }^{} \exp \left(- \frac{E_\alpha }{\kappa _B T}\right)  \right] \\
			    & =\sum_{N_\alpha }^{} \left\{ \left[ \exp \left(\frac{\mu }{\kappa _BT}\right)  \right] ^{N_\alpha } \sum_{\alpha | N_\alpha }^{} \exp \left(-\frac{E_\alpha }{\kappa _BT}\right)  \right\} \equiv \sum_{N_\alpha }^{} \left[ z^{N_\alpha } \sum_{\alpha | N_\alpha }^{} \exp\left(- \frac{E_\alpha }{\kappa _BT}\right)  \right] 
	\end{split}
\end{equation}
dove $z$ \`e detta \textbf{fugacit\`a}. Nel limite al continuo, si trova:
\begin{equation}
	\mathscr{L}= \sum_{N}^{} \frac{z^N \mathscr{D}_N}{N! \Lambda ^{3N} }
\end{equation}
con $\mathscr{D}_N$ integrale delle configurazioni relativo agli stati $\alpha $ con $N$ particelle.


\subsection{Gas perfetto}
Particelle confinate in scatola con autostati dell'energia individuati dagli impulsi $q$. Numero di particelle in uno stato \`e $n_q$. 

Per gas ideale, la maggior parte dei microstati sar\`a vuota, cio\`e $w(0) \approx 1$, e la probabilit\`a di avere pi\`u di una particella in un microstato \`e praticamente nulla, quindi $w(1) \ll 1$ e $w(n\ge 2) \approx 0$. Usando $\mathscr{L} = \exp \left[ - \Omega / \kappa _B T \right] $:
\begin{equation}
	w(n_q) = \exp \left[ \frac{\Omega _q - n_q (E_q - \mu )}{\kappa _B T} \right] 
\end{equation}
Allora le condizioni di popolazione dei microstati si traducono in:
\begin{equation}
	\begin{split}
		&w(0) \approx 1 \Rightarrow \exp \left(\frac{\Omega _q}{\kappa _B T}\right) \approx 1\\
		&w(n) = e^{\Omega _q / \kappa _B T} \left[\exp \left(- \frac{E_q - \mu }{\kappa _BT} \right)\right]  ^n\equiv w^n(1)  \ll 1, \forall q \iff \exp \left(\frac{\mu}{\kappa _BT }\right) \ll 1
	\end{split}
\end{equation}
quindi $\mu \to -\infty$. Da questo, il numero medio di particelle in uno stato $q$ \`e:
\begin{equation}
	\langle n_q \rangle = \frac{\sum_{n_q}^{} n_q \exp\left[ -n_q (E_q - \mu ) / \kappa _BT \right] }{\sum_{n_q}^{}\exp\left[ -n_q (E_q - \mu ) / \kappa _BT \right]  } \approx \exp \left(- \frac{E_q - \mu }{\kappa _BT }\right) 
\end{equation}
avendo usato $w(1) \ll 1$. Dal potenziale di Landau, si ottiene equazione di stato dei gas perfetti:
\[
\Omega _q \approx -\kappa _B T \log \left[ 1 + \exp\left(- \frac{E_q - \mu }{\kappa _B T }\right)  \right] = -\kappa _B T \log \Big(1 + \langle n_q \rangle\Big) \approx -\kappa _B T \langle n_q \rangle
\] 
Essendo $\Omega \approx - \kappa _B T \sum_{q}^{} \langle n_q \rangle  \equiv -\kappa _B T N$ e valendo allo stesso tempo $\Omega = - PV$, si ha $PV = \kappa _B NT$.

Ora si ricava $N$ in funzione di $T,V,\mu $. Usando la densit\`a di stati $\rho (\mathscr{E})$ trovata per singola particella in \S \ref{1p}, si ha:
\begin{equation}
	\begin{split}
		N &= \int \rho (\mathscr{E}) \exp\left(- \frac{\mathscr{E}-\mu }{\kappa _BT}\right) \ d \mathscr{E}= \frac{4\pi \sqrt{2} V m^{ 3 /2 } }{h^3}e^{ \mu  / \kappa _B T} \int d\mathscr{E} \ e^{ - \mathscr{E} / \kappa _B T}  \mathscr{E}^{1 / 2} \\
		  &= \frac{V}{\Lambda ^3}e^{\mu  / \kappa _B T} 
	\end{split}
\end{equation}
Usando $PV = N\kappa _B T$, si pu\`o scrivere
\begin{equation}
	\begin{split}
		&\mu = - \kappa _B T \log \frac{\kappa _B T}{P \Lambda ^3}\\
		&\Phi = N\mu  = - \kappa _B NT \log \frac{\kappa _B T}{P\Lambda ^3}
	\end{split}
\end{equation}
Quindi, espandendo il logaritmo del prodotto nelle somme dei logaritmi e nuovamente la legge $PV = N\kappa _B T$ per sostituire la pressione nel primo logaritmo:
\begin{equation}
	\begin{split}
		S &= - \left(\frac{\partial \Phi}{\partial T} \right) _{PN} = - N\kappa _B \log \frac{V}{N} + \frac{5}{2} N\kappa _B \log\kappa _B T + N\kappa _B \left(\frac{5}{2} + \frac{3}{2} \log \frac{m}{2\pi\hbar ^2}\right) 
	\end{split}
\end{equation}
Da questa trattazione, si ricavano tutti gli altri risultati, come:
\[
c_P = T \left(\frac{\partial S}{\partial T} \right) _P =  \frac{5}{2}N\kappa _B \hspace{.1cm} ; \hspace{.2cm} c_V = T \left(\frac{\partial S}{\partial T} \right) _V = \frac{3}{2} N \kappa _B 
\] 
Dalla formula per $\mu $, la condizione di gas ideale diventa:
\begin{equation}
	\frac{\kappa _B  T}{P\Lambda ^3}  \gg 1 \iff \frac{N\Lambda ^3}{V}\ll 1
\end{equation}
Infine, fissando $N$ (ensemble canonico):
\begin{equation}
	F = -N\kappa _BT \log \frac{V}{\Lambda ^3}  + \kappa _BT \log N!
\end{equation}
mentre fissando $\langle N \rangle$ (ensemble grancanonico):
\begin{equation}
	F = \Phi - PV = - N\kappa _B T  \log \frac{V}{\Lambda ^3} + \kappa _B T (N \log N - N)
\end{equation}
Per $N$ grandi, queste espressioni coincidono, essendo $\log N! \approx N \log N - N$.


\subsection{Distribuzione dell'energia}

In assenza di potenziale, vincolo sull'energia \`e fissato da $\mathscr{E} = \frac{1}{2m} \sum_{i=1}^{3N} p_i^2$; in questo, un elemento dello spazio delle fasi $d\Gamma$ sar\`a proporzionale ad un elemento di volume, a sua volta proporzionale al raggio: $\rho (\mathscr{E}) d \mathscr{E} \propto d V* (\mathscr{E}) \propto (p^*)^{3N } $, con $p^* = \sqrt{2 m \mathscr{E}} = \sqrt{\sum_{i=1}^{3N} p_i^2} $.

Ricordando che $w (\mathscr{E}) = \frac{1}{\mathscr{Z}}\rho (\mathscr{E}) e^{ - \mathscr{E } /\kappa _B T} $ (nel caso di degenerazione di un livello energetico e $N _\alpha  = N ,  \ \forall  \alpha $):
\[
\begin{split}
	&dV^* \propto \frac{\partial V^*}{\partial \mathscr{E}}  d \mathscr{E} \propto (p^*)^{3N -1 } \frac{\partial p^*}{\partial \mathscr{E}} d \mathscr{E}\propto \mathscr{E} ^{3N /2 -1} d \mathscr{E}\\
	&\Rightarrow  w (\mathscr{E}) \propto \mathscr{E}^{3N /2 -1 } \exp \left(- \frac{\mathscr{E}}{\kappa _B T}\right) \Rightarrow w(\mathscr{E}) = \frac{1}{\Gamma(3N / 2) } \left(\frac{\mathscr{E}}{\kappa _B T}\right) ^{3N / 2 -1} \frac{\exp(- \mathscr{E}/\kappa _B T)}{\kappa _B T}
\end{split}
\] 
L'energia pi\`u probabile si ottiene imponendo $\partial _\mathscr{E} w \stackrel{!}{=} 0$, da cui $\mathscr{E}_\text{max}= (3N / 2 -1) \kappa _B T$. D'altra parte, il valore medio \`e $E = \int d \mathscr{E} \ w(\mathscr{E}) \mathscr{E} = \frac{3}{2}N \kappa _B T $: i due differiscono per fattore additivo indipendnete da $N$, quindi per $N$ molto grandi, la distribuzione \`e piccata attorno al valore medio.

Per la varianza $\sigma _\mathscr{E}^2 = \langle \mathscr{E}^2 \rangle- E^2$, si usa $\partial _T^2 F = (E^2 - \langle \mathscr{E}^2 \rangle) / (\kappa _B T^3)$ e $c_V = \partial_T E = 3N\kappa _B / 2$, quindi:
\begin{equation}
	\sigma _\mathscr{E}^2 = - \kappa _B T^3 \frac{\partial ^2F}{\partial T^2} = \kappa _B T^2 c_V = \frac{3}{2} N\kappa _B^ 2 T^2 
\end{equation}
 Per la singola particella, allora: $\sigma _\mathscr{E} \sim \kappa _B T$.




\subsection{Incertezze quantistiche}
\begin{boxenv}[]
	\centering \textit{Valutare se aggiungere} 
\end{boxenv}






















\newpage
\section{Gas quantistici}
\subsection{Statistiche di Bose-Einstein e Fermi-Dirac}
Perch\'e valga indistinguibilit\`a delle particelle, a bassa temperatura si devono modificare gli stati occupabili. Per sistema di due particelle, deve risultare $\lvert \psi (1,2) \rvert ^2 = \lvert \psi (2,1) \rvert ^2$, cio\`e la probabilit\`a di trovare le particelle in un punto dello spazio deve essere uguale se si scambiano le due particelle.

Quindi $\psi (1,2) = \pm \psi (2,1)$. Si assume che le due particelle stiano o in $a$, o in $b$, si \textit{suppone che le funzioni d'onda delle singole particelle siano fattorizzate}; le uniche combinazioni che rispettano la condizione $\psi (1,2) = \pm \psi (2,1)$ sono una simmetrica e una antisimmetrica:
\begin{equation}
		\psi _S = \psi _a (1) \psi _b (2) + \psi _a(2) \psi _b(1); \ \psi _A = \psi _a(1) \psi _b(2) - \psi _a(2) \psi _b(1)
\end{equation}
Quando entrambe sono nello stesso stato ($a=b$), $\psi _A = 0 $; questo \`e il principio di esclusione di Pauli. 

Particelle con funzione d'onda antisimmetrica sono dette \textbf{fermioni}, mentre con funzione d'onda simmetrica sono dette \textbf{bosoni}.

Per principio di esclusione, i fermioni possono soddisfare $n_q = 0,1$ solamente, quindi:
\begin{equation}\label{ofd}
	\Omega _q = -\kappa _B T \log \left[ 1+ \exp\left(\frac{\mu  - E_q}{\kappa _B T}\right)  \right] 
\end{equation}
da cui si ricava la \textbf{statistica di Fermi-Dirac}:
\begin{boxenv}[]
\begin{equation}
	\langle n_q \rangle = - \frac{\partial \Omega _q}{\partial \mu } = \frac{1}{\exp \left[ (E_q - \mu ) / \kappa_B T \right]  + 1}
\end{equation}
\end{boxenv}
\noindent Per ottenere potenziale di Landau e numero di particelle totali, bastsa sommare su $q$. Per i bosoni, invece, tutti gli $n$ sono possibili e si deve calcolare la somma di una serie geometrica, \textit{che converge solamente se} $\mu \le  E_0$. In questa trattazione $E_0=0$, quindi $\mu $ deve essere negativo e
\begin{equation}\label{obe}
	\Omega _q = - \kappa _B T \log \sum_{n=0}^{+\infty} \exp \left[ \frac{n(\mu  - E_q)}{\kappa _B T} \right] = \kappa _B T \log \left[1 - \exp\left(\frac{\mu  - E_q}{\kappa _B T}\right) \right]
\end{equation}
da cui la \textbf{statistica di Bose-Einstein} \`e:
\begin{boxenv}[]
\begin{equation}
	\langle n_q \rangle = \frac{1}{\exp\left[ (E_q - \mu )/\kappa _B T \right] - 1}
\end{equation}
\end{boxenv}
\subsection{Gas perfetto debolmente degenere}
Si studia comportamento quantistico del gas perfetto. Per passare al continuo, \`e necessario che fluttuazioni statistiche siano maggiori della separazione tra i livelli, quindi
\begin{equation}
	\kappa _B T \gg \frac{\hbar ^2}{2m} \left(\frac{2\pi}{L}\right) ^2
\end{equation}
Se $L\sim 1 \text{ cm}$ e $m\sim 10^{-24} \text{ g}$ (massa atomo di idrogeno), si ha $T \gg 10^{-13} $ K; per elettroni $T\gg 10^{-10} $ K. Dal punto di vista pratico, queste sono sempre soddisfatte.

In assenza di campi, ogni spin $S$ ha $g = 2S + 1$ orientazioni possibili e si deve aggiungere nel conteggio dei microstati\footnote{Aumentando il numero di microstati di $g$, l'entropia subisce un aumento per un termine $\kappa _B \log g$.}. La correzione per misura dello spazio delle fasi \`e:
\[
d\Gamma= \frac{d^3 x d^3 p }{h^3}\to g \frac{d^3 xd^3p}{h^3}
\] 
Si sviluppano in serie le espressioni dei potenziali di Landau per ciascuna statistica (eq. \ref{ofd}, \ref{obe}) (segno superiore per FD, inferiore per BE):
\begin{equation*}
	\begin{split}
		\Omega &= \mp \kappa _B T \sum_{q}^{} \log \left[ 1\pm \exp\left( - \frac{E_q - \mu }{\kappa _B T}\right)  \right] \\
		       &= -\kappa _B T \sum_{q}^{} \exp \left(-\frac{E_q - \mu }{\kappa _B T} \right) \pm \frac{\kappa _B T }{2}\sum_{q}^{} \exp \left(-2 \frac{E_q - \mu }{\kappa_B T }\right) \\
		       &= \Omega _\text{class} \pm \frac{\kappa _B T}{2} \sum_{q}^{} \exp \left( - 2 \frac{E_q - \mu }{\kappa _B T}\right) 
	\end{split}
\end{equation*}
Si esegue il passaggio al continuo e si sostituisce $\mathscr{E} \to 2\mathscr{E}$, usando $\rho (\mathscr{E}) \propto \sqrt{\mathscr{E}} $; inoltre si riscrive il potenziale classico:
\[
\begin{split}
	&\sum_{q}^{} \exp \left(- \frac{2 E_q}{\kappa _B T}\right)  \to \int_{0} ^{+\infty} \rho (\mathscr{E}) \exp \left(-\frac{2\mathscr{E}}{\kappa _B T}\right) \ d\mathscr{E} \to \left(\frac{1}{2}\right) ^{3 / 2} \int_{0} ^{\infty} \rho (\mathscr{E}) \exp\left(- \frac{\mathscr{E}}{\kappa _B T}\right)  d \mathscr{E}\\
	& \Omega _\text{class} \to - \kappa _B T \exp \left(\frac{\mu  }{\kappa _B T}\right)  \int_{0} ^{+\infty} \rho (\mathscr{E}) \exp \left(- \frac{\mathscr{E}}{\kappa _B T}\right) \ d \mathscr{E}
\end{split}
\] 
Da questi passaggi, si ottiene:
\begin{equation}
	\Omega \simeq \Omega _{\text{class}} \left(1 \mp \frac{e^{ \mu  / \kappa _B  T} }{2 ^{5 / 2} }\right) 
\end{equation}
Usando $\Omega  = - PV $ e $\Omega _\text{class} = - N \kappa _B T$:
\begin{equation}
	PV \simeq N\kappa _B T \left(1 \mp \frac{e^{ \mu  / \kappa _B T} }{2^{ 5 / 2} }\right) 
\end{equation}
La statistica agisce come sorta di forza sulle particelle; per capire se attrattiva o repulsiva, si deve passare da $\Omega $ a $F $\footnote{Usando $\Omega $, si \`e trovato risultato in cui $P / N$ ha $N$ variabile.}, in cui sono fissati $T,V,N$. Definendo $\delta \Omega = \Omega  - \Omega _\text{class}$ e $\delta F  = F - F_\text{class}$, si ha $(\delta \Omega ) _{T,V,\mu } = (\delta F) _{T,V,N} $. 

Per eliminare $\mu $ in $\delta \Omega  = \pm N\kappa _B Te^{\mu  / \kappa _B T} / 2^{5 / 2} $, si usa espressione classica $\mu _\text{class}= - \kappa _B T \log V /(N\Lambda ^3)$, da cui:
\begin{equation}
	\begin{split}
		&F = F_\text{class} \pm \frac{N^2 \kappa _B T \Lambda ^3}{2^{5 / 2}V }\\
		& \Rightarrow P = - \frac{\partial F }{\partial V } = \frac{N\kappa _B T}{V}\left(1 \pm \frac{N\Lambda ^3}{2^{5 / 2} V}\right) 
	\end{split}
\end{equation}
Quindi la statistica di Fermi corrisponde ad una forza repulsiva, mentre quella di Bose a una attrattiva.
\subsection{Gas di Fermi}

Si considera gas di fermioni nel limite $T \to 0 $, in cui le particelle occuperanno i livelli energetici pi\`u bassi consetiti dal principio di esclusione, inziando dal ground state a salire, fino a esaurimento particelle. 

Allora lo spazio delle fasi di singola particella\footnote{Per spazio delle fasi di singola particella, si fa riferimento all'insieme di tutti i possibili stati che una particella pu\`o occupare; in avanti, si menzioner\`a lo spazio delle fasi complessivo (quello di tutto il gas), il quale rappresenter\`a tutti gli stati occupabili dall'intero sistema. Essendo le particelle indistinguibili, quest'ultimo deve collassare a un punto perch\'e i fermioni si possono distribuire solo in stati di singola particella ad energia via via crescente.} avr\`a tutte le celle piene dall'origine fino a un'energia $E_f = p_f^2 / 2m$, con $E_f$ \textbf{energia di Fermi} e $p_f$ \textbf{impulso di Fermi}, relativa all'energia del pi\`u alto stato quantistico occupabile da una particella\footnote{In quanto tale, dipender\`a dal numero totale di particelle e dal volume in cui è confinato il gas.}; lo spazio delle fasi a molti corpi, invece, consiste in un solo punto.

Nel limite $T\to 0$, il grafico ($E_q$, $n_q$) (numero di occupazione in funzione dell'energia) \`e un gradino con $n_q = 1$ per $0\le E_q\le \mu_0$ e $0$ altrimenti, con $\mu (T=0) \equiv \mu_{0}\equiv E_f$.

Quest'ultimo \`e fissato dal numero totale di particelle dato da:
\begin{equation}
	\begin{split}
		N = \lim_{T \to 0} \sum_{q}^{} \frac{1}{\exp\left[ (E_q - \mu  ) / \kappa _B T \right]  + 1 } &\to \lim_{T \to 0 } \int_{0} ^{+\infty} \frac{\rho (\mathscr{E})}{\exp\left[ (\mathscr{E}-\mu ) / \kappa _B T \right] + 1}\\
		  & =  \frac{gV}{h^3}\frac{4}{3}\pi p^3_f
	\end{split}
\end{equation}
con $g = 2S + 1$ degenerazione degli stati quantistici dovuta allo spin. L'ultima uguaglianza \`e verificata perch\'e, in questo caso, $N$ \`e \# di celle in una sfera di raggio $p_f^3$ nello spazio delle fasi di singola particella. 

Da questo, $p_f = h [3N / (4\pi g)] ^{1 / 3}$, quindi il valore dell'energia di Fermi \`e:
\begin{equation}
	E_f = \frac{h^2}{2m} \left(\frac{N}{V}\right) ^{2/3} \left(\frac{3}{4\pi g} \right) ^{2 /3 }
\end{equation}
Da questa si ottiene \textbf{temperatura di Fermi} $\kappa _B T_f = E_f$. 

La sfera nello spazio di singola particella \`e detta \textbf{sfera di Fermi}, o \textbf{mare di Fermi}, mentre il guscio \`e detto \textbf{superficie di Fermi}.

Essendo $\rho (\mathscr{E}) \propto \mathscr{E}^{1 / 2} $, l'energia media per particella è:
\begin{equation}
	\langle \mathscr{E} \rangle = \frac{\displaystyle \int_{0} ^{E_f} \mathscr{E}\rho (\mathscr{E}) d \mathscr{E}}{\displaystyle \int_{0} ^{E_f} \rho (\mathscr{E}) d\mathscr{E}} = \frac{ \displaystyle \int_{0} ^{E_f} \mathscr{E}^{3 / 2}  d \mathscr{E}}{\displaystyle  \int_{0} ^{E_f} \mathscr{E}^{1 / 2}  d \mathscr{E}}= \frac{3}{5} E_f
\end{equation}
Quindi l'energia totale \`e $E = \frac{3}{5} N E_f$. 

Entropia del sistema \`e nulla (una sola possibile configurazione nello spazio delle fasi del gas) e l'energia complessiva del sistema coincide con l'energia interna (sempre perché il sistema si pu\`o trovare in un solo stato); allora la pressione \`e:
\[
P = - \left(\frac{\partial E}{\partial V} \right) _N = \frac{2}{3} \frac{E}{V}
\] 
Pressione finita a temperatura nulla \`e conseguenza della forza repulsiva tra i fermioni.	
\subsubsection{Comportamento del gas per $T>0$}
Si considera cosa succede al gas quando la temperatura sale di poco sopra $0$, quindi nel limite $T \ll T_f$\footnote{La scala di grandezza delle temperature \`e data solo da $T_f$ in questo caso, quindi si usa questa come riferimento.}. \textit{Questo modello si user\`a per studiare comportamento degli elettroni nei metalli}, quindi si stima $T_f$ usando massa e spin dell'elettrone e densit\`a di elettroni di conduzione nel rame, ottenendo $T_f \approx 8.5 \times 10^4$ K; questa risulta due ordini di grandezza sopra la temperatura di fusione del rame stesso, quindi il gas di elettroni \`e sempre in limite di basse temperature.

Con aumento di $T$, le particelle sulla superficie di Fermi (nei livelli energetici pi\`u esterni) possono eccitarsi con energia $\sim \kappa _B T$, mentre quelli nel mare no perch\'e i livelli successivi sono occupati. 
Numero di elettroni eccitati $\sim T / T_f$ per il totale. 

Il grafico di $n(e)$ \`e un gradino consumato: l'intervallo attorno a $E_f$, di larghezza $\sim \kappa _B T$. 

Questo modello per elettroni in metallo con background uniforme postiivamente carico\footnote{Approssimazione in cui gli ioni si immaginano come una carica positiva uniformemente distribuita invece che come punti discreti facenti parte di un reticolo, per questo l'approssimazione \`e valida nella condizione riportata.}, quindi finch\'e $\lambda \gg a$, con $\lambda $ lunghezza d'onda elettroni e $a$ dimensione caratteristica del reticolo del metallo.

La condizione \`e verificata per stati a bassi impulsi ($p = \hbar  q$), per i quali si pu\`o assumere che $E \propto q^2$ (l'energia continua ad obbedire la legge di dispersione), mentre avr\`a una forma diversa fuori da questo regime. 
In realt\`a, anche in questo caso, \`e diversa: $E = \frac{\hbar ^2 q^2}{2m^*}$, con $m^*$ massa efficace dovuta all'interazione degli elettroni con gli ioni.

\subsubsection{Propriet\`a termiche del gas di Fermi}
Indicando con $\overline{n}(\mathscr{E})$ la statistica di Fermi-Dirac, \textit{si sa} che per gas perfetti con dispersione quadratica $\Omega = -\frac{2}{3}E$, quindi:
\[
	\Omega = -\frac{2}{3} \int_{0} ^{+\infty} \rho (\mathscr{E}) \overline{n}(\mathscr{E}) \ d \mathscr{E}= -\frac{2}{3} \frac{4 \pi V g \sqrt{2} m^{2 / 3} }{h^3}\int_{0} ^{+\infty} \frac{\mathscr{E}^{3/2} }{\exp \big[(\mathscr{E}-\mu) / \kappa _BT \big] + 1} d \mathscr{E}
\] 
Si deve, quindi, risolvere integrale della forma $I = \int_{0} ^{+\infty} \frac{f(\mathscr{E}) d \mathscr{E}}{\exp\left[ (\mathscr{E}-\mu ) / \kappa _ BT \right] +1}$.
Visto che $\overline{n}(\mathscr{E})$ \`e una funzione gradino per $T \ll T_f$, il contributo maggiore nell'integrale sar\`a da $0$ a $E_f$ in cui $n(\mathscr{E}) = 1$.
L'integrale su questi estremi si chiama $I_0$. La correzione su $I$ \`e $\delta I$ in modo che $I = I_0 + \delta I$; essendo $I = I_0$ per $T=0$, il comportamento termico \`e incluso in $\delta I$.

Per ricavare $\delta I$, si calcola differenza tra integrale sul gradino e integrale esatto: $\delta I = I - I_0$. Si introduce una sovrastima prima di $E_f$ e una sottostima dopo per avere gradino perfetto; questi due si compensano a vicenda e termine lineare \`e nullo.

Per correzioni successive, si prende $z = (\mathscr{E} - \mu ) / \kappa _B T $. Si definiscono $g_0(z), g_1(z)$ non nulle attorno $z=0$ e si esprime la correzione con\footnote{\color{red} Vedere perch\'e estremo inferiore \`e $-\infty$.}
\[
\delta I = \int_{-\infty} ^{+\infty} d \mathscr{E} \ f(\mathscr{E}) \left[ g_1(z) - g_0(z) \right] = \int_{-\infty} ^{+\infty} \kappa _B T f(\mu +\kappa _B T z) \left[ g_1(z) - g_0(z) \right]  \ dz
\] 
Essendo $g_0,g_1 \neq 0$ solo intorno a $z=0$, si sviluppa attorno a $z=0$:
\[
	\begin{split}
		&f(\mu  + \kappa _B T z) \simeq f(\mu ) + \kappa _B T z \left(\frac{\partial f}{\partial \mathscr{E}} \right) _{\mathscr{E} = \mu } = f(\mu ) + \kappa _B T zf '(\mu )\\
		& \Rightarrow \delta I = \kappa _B T f(\mu ) \int_{-\infty} ^{+\infty} dz \left[ g_1(z) - g_0(z) \right] + \kappa _B^2 T^2 f'(\mu ) \int_{-\infty} ^{+\infty} dz \left[ g_1(z) - g_0(z) \right] z
	\end{split}
\] 
Essendo $g_1(z)$ la parte di $\overline{n}$ per $z\ge 0$ e $g_0(z)=1 - \overline{n}$, per $z\le 0$:
\[
	g_1 (z) = \frac{1}{e^z +1} \hspace{.1cm} ; \hspace{.2cm} g_0(z) = 1 - \frac{1}{e^z +1} = \frac{1}{e^{-z}  +1}
\] 
Allora il primo termine in $\delta I$ \`e nullo, mentre il secondo \`e
\[
\int_{0} ^{+\infty} dz \ z g_1(z) - \int_{-\infty} ^0 dz \ z g_0(z) = 2 \int_{0} ^{+\infty} \frac{z dz}{e^z + 1}= \frac{\pi^2 }{6}
\] 
Complessivo di correzione quadratica in $T$, il potenziale di Landau \`e:
\begin{equation}
	\Omega = -\frac{2}{3} \frac{4 \pi \sqrt{2} V g m^{ 3 /2 } }{h^3}\left[ \frac{2}{5} \mu ^{5 / 2} + \frac{\pi^2 }{4} \sqrt{\mu } (\kappa _B  T )^2  \right] 
\end{equation}
quindi 
\begin{equation}
	N = - \frac{\partial \Omega }{\partial \mu } = N_0 \left[ 1 + \frac{\pi^2}{8} \left(\frac{\kappa _B T}{\mu }\right) ^2 \right] 
\end{equation}
Ora si vuole capire come si distribuiscono gli $N$ elettroni nei livelli al variare della temperatura. 
Per questo si considerano sistemi $\mathscr{S}, \mathscr{S}'$ a $T \neq 0$ e $N_0\neq N'_0$, $\mu  \neq \mu '$.
All'aumentare di $T$, $N,N'$ seguono la legge sopra. Se $N$ rimanesse invariato, $\mu $ dovrebbe cambiare di conseguenza, quindi basta imporre $N' = N_0(\mu )$. 

Dalla stessa legge per $N$, si vede che $N_0 \propto \mu ^{3 / 2} $ ({\color{red}?}), quindi si pu\`o sostituire rapporto $N'_0 / N_0$ in favore di $\mu ' / \mu $:
\[
\left[ 1+ \frac{\pi^2}{8}\left(\frac{\kappa _B T}{\mu '}\right) ^2 \right] \left(\frac{\mu '}{\mu }\right) ^{3 / 2} =1
\] 
Si risolve per $\mu '$, sostituendo $\mu $ con $\mu '$ al denominatore (che porta errore oltre secondo ordine) e si sviluppa in serie:
\[
\mu ' = \frac{\mu }{\left[ 1 + \pi^2 / 8 (\kappa _B T / \mu ')^2 \right] ^{2 / 3} }\simeq \mu \left[ 1 - \frac{\pi^2}{12} \left(\frac{\kappa _B T}{\mu }\right) ^2 \right] 
\] 
Sostituendo in $\Omega $, si possono trovare entropia e calore specifico:
\[
\begin{split}
	&S = -\frac{\partial \Omega }{\partial T}  = \frac{4 \pi \sqrt{2} V g m^{3 / 2} }{h^3}\frac{\pi^2 }{3}\mu ^{1 / 2} \kappa _B^2 T \simeq \frac{\pi^2}{2} N \kappa _B \frac{T}{T_f}\\
	&c_V = T \left(\frac{\partial S}{\partial T} \right) _V \simeq \frac{\pi^2}{ 2} N\kappa _B \frac{T}{T_f}
\end{split}
\] 
Stima per calore specifico in accordo con dati sperimentali, ma stime migliori si ottengono per $m\to m^*$. 
La massa effettiva si pu\`o misurare tramite campi magnetici ed \`e legata a frequenza di ciclotrone: $\omega_c = e \lvert \mathbf{H}  \rvert / (m^* c)$.

\subsubsection{Paramagnetismo di Pauli}

Immergendo sistema in campo magnetico, la degenerazione dovuta allo spin \`e rotta e si ha $\mathscr{E}_{\pm} = \frac{p^2}{2m} \pm \mu B$, con $\mu $ momento magnetico della particella.
Questo termine aggiuntivo trasla energia e va considerato in calcolo del numreo di occupazione e densit\`a di stati.

Se $T\sim 0 , B=0$, densit\`a di popolazioni con spin up e spin down \`e uguale per entrambe e pari a $\rho (\mathscr{E}) / 2$ e lo spazio delle fasi \`e riempito dall'origine a $E_f$. 

Quando $B \neq 0 $, questo interagisce con momento magnetico intrinseco delle particelle e l'energia per gli stati dei due tipi di spin cambia secondo energia di Zeeman:
\[
	E_\uparrow = E - \mu  B \hspace{.1cm} ; \hspace{.2cm} E _\downarrow = E + \mu B
\] 
quindi stati $\uparrow$ avranno energia minore e quindi saranno pi\`u popolati, nonostante l'energia di Fermi rimanga invariata.

La differenza di popolazione $\Delta N$ si pu\`o approssimare con $\rho (E_f) / 2 E_\downarrow - \rho (E_f) / 2 E\uparrow = \Delta E \rho (E_f) / 2$, assumendo $\rho (\mathscr{E})$ costante.
In questo modo:
\[
\Delta N = 2\mu B \frac{\rho (E_f) }{2} = 2\mu  _B \frac{4 \pi V m^{3 / 2} }{\sqrt{2} h^3}\sqrt{E_f} = \frac{3}{2}N \frac{\mu  B }{E_f}
\] 
con $N$ calcolato sulla densit\`a per $B=0 , T=0$:
\[
N = \int_{0} ^{E_f} \rho  (\mathscr{E}) \ d \mathscr{E} = \int_{0} ^{E_f} \frac{4\pi \sqrt{2} V m^{3 / 2} }{h^3} \mathscr{E}^{1 / 2}  \ d \mathscr{E} = \frac{4\pi \sqrt{2} V m^{3/2} }{h^3} \frac{2}{3} E_f^{ 3 / 2} 
\] 
Questa forma di paramagnetismo, con magnetizzazione
\begin{equation}
	M = \mu  \Delta N = \frac{3N}{2} \frac{\mu ^2}{E_f}B
\end{equation}
\`e detta \textbf{paramagnetismo di Pauli}.

\subsubsection{Emissione termoionica}

Emissione di elettroni indotta termicamente nei conduttori. Si fanno le seguenti ipotesi:
\begin{itemize}
	\item il metallo \`e una buca di potenziale alta $W$;
	\item il rate di emissione \`e basso $\Rightarrow $ numero di elettroni nel metallo \`e $\sim$ costante;
	\item \`e presente campo elettrico esterno che rimuove elettroni emessi (altrimenti il rate sarebbe nullo);
	\item \`e il compleanno di Stefano (tanti auguri).
\end{itemize}
Affinch\'e vi sia effettiva emissione, si impone che in direzione $z$ (arbitraria) valga $p_z > \sqrt{2m W} $. 
Se $dt$ tempo di emissione, si deve avere $dz = v_z dt = p_z / m dt$.

Il numero di elettroni emessi sar\`a proporzionale all'integrale del numero di occupazione sulla parte di spazio delle fasi in cui si pu\`o verificare l'emissione;
dividendo per $dt$, si ottiene il rate.

Definendo rate per unit\`a di superficie, si tagli ala'integrale su $dS = dxdy$:
\begin{equation*}
			dR = \frac{g \overline{n} d\Gamma}{dS dt} = \frac{2 \overline{n}}{h^3} \frac{dxdy v_z dt d^3 p}{dS dt}= \frac{2\overline{n}}{h^3}\frac{p_z }{m} d^3p
\end{equation*}
quindi:
\begin{equation*}
  			\begin{split}
			 R &= \frac{2}{h^3} \int_{\sqrt{2m W} } ^{+\infty} \frac{p_z dp_z}{m} \iint_{\mathbb{R}^2}  \frac{dp_x dp_y}{\exp \left[ (p^2 / 2m - \mu ) / \kappa _ B T \right] + 1} \\
					      &= \frac{2}{h^3} \int_{\sqrt{2m W} } ^{+\infty} \frac{p_z dp_z}{m} \int_{0} ^{+\infty} \frac{2\pi p ' dp'}{\exp \left[ \big(p'^2 + p_z^2)/ 2m - \mu \big) / \kappa _ B T \right] + 1} \\
					      &= \frac{4\pi \kappa _B T }{h^3}\int_{\sqrt{2mW} } ^{+\infty} p_z\log \left[ 1+ \exp\left(\frac{\mu  - p_z^2 / 2m}{\kappa _B T}\right)  \right] dp_z  =  \frac{4 \pi m \kappa _B T }{h^3} \int_{W} ^{+\infty} d \mathscr{E}_z \ \log \left[ 1+ \exp \left(\frac{\mu - \mathscr{E}_z}{\kappa _B T}\right)  \right] 
			\end{split}
\end{equation*}
Infine, bisogna imporre che $W - \mu  \gg \kappa _B T$\footnote{Si richiede che il potenziale di estrazione sia molto maggiore del potenziale chimico, altrimenti elettroni fuggirebbero spontaneamente.}, per cui $\exp\left(\frac{\mu  - \mathscr{E}_z}{\kappa _B T}\right) \ll 1$; allora sviluppando:
\begin{equation}
	R \simeq \frac{4\pi m \kappa _B T }{h^3}\int_{W} ^{+\infty} d \mathscr{E}_z\ \exp\left( \frac{\mu  - \mathscr{E}_z}{\kappa _B T}\right) = \frac{4\pi m \kappa _B ^2 T^2}{h^3} \exp \left(\frac{\mu  - W}{\kappa _B T}\right) 
\end{equation}
Dal rate, si ottiene la densit\`a di corrente:
\begin{equation}
	J = eR = \frac{4 \pi e m_e \kappa _B ^2 T^2}{h^3} \exp \left(\frac{\mu  - W}{\kappa _B T}\right) 
\end{equation}
\subsubsection{Effetto fotoelettrico}

Gas di elettroni nel metallo colpito da fotoni di energia $hv$. La condizione in direzione di fuga diventa: $\frac{p^2_z}{2m} + hv > W$, quindi:
\[
R = \frac{4 \pi m \kappa _B T}{h^3} \int_{W- hv} ^{+\infty}  d \mathscr{E}_z \log \left[ 1 + \exp \left(\frac{\mu - \mathscr{E}_z}{\kappa _B T}\right)  \right] 
\] 
Non si pu\`o sviluppare in serie come prima perch\'e potrebbe essere $hv \sim W$. Si prende $x = \frac{\mathscr{E}_z - W + hv}{\kappa _B T}$ e $hv_0 = W - \mu  \approx W - E_f = \phi $:
\[
R = \frac{4 \pi m \kappa _B^2 T^2}{h^3}  \int_{0} ^{+\infty} dx \ \log \left[ 1+ \exp \left(\frac{h(v-v_0)}{\kappa _B T} - x\right)  \right] 
\] 
Integrali del genere hanno soluzioni della forma
\[
\int_{0} ^{+\infty} dx \ \log \left(1 + e ^{ \delta  - x } \right) = f_2 (e^\delta )
\] 
Per trovare espressione di $f_2$ si considerano i casi limite, rispettivamente radiazione molto energetica e poco energetica:
\[
\begin{split}
	&h(v-v_0) \gg \kappa _B T \Rightarrow  e^\delta  \gg 1 \Rightarrow  f_2 (e^\delta ) \simeq \frac{\delta ^2}{2}\\
	& v< v_0 \Rightarrow  h|v-v_0| \gg \kappa _B T \Rightarrow e^\delta \ll 1 \Rightarrow  f_2 (e^\delta ) \simeq e^\delta 
\end{split}
\] 
Nel primo caso, l'espressione della corrente \`e 
\begin{equation}
	J \simeq \frac{me}{\hbar } (v-v_0)^2
\end{equation}
cio\`e la corrente non ha dipendenza dalla temperatura perch\'e l'emissione degli elettroni \`e prevalentemente dovuta all'incisione di fotoni ad alta energia.

Nel secondo caso, invece:
\begin{equation}
	J \simeq \frac{4 \pi m e \kappa _B^2 T^2}{h^3} \exp \left(\frac{hv - \phi }{\kappa _B T}\right) 
\end{equation}
che \`e una correzione alla corrente termoionica.
\subsection{Gas di Bose}
\subsubsection{Condensato di Bose-Einstein}
Bosoni descritti dalla statistica
\[
\overline{n} (\mathscr{E}) = \frac{1}{\exp\left[ (\mathscr{E}-\mu ) / \kappa _B T \right] - 1}
\] 
e deve valere $\mu  < 0$. Nel limite $\mu \to 0$:
\begin{itemize}
	\item a $T $ costante, esiste massimo numero di particelle consentito, sopra cui si dovrebbe avere $\mu >0$;
	\item a $N$ fissato, esiste limite inferiore per $T$, imposto sempre dal segno di $\mu $.
\end{itemize}
Ci si aspetterebbe, per\`o, di poter osservare un gas a qualsi temperatura con qualsiasi numero di particelle. 
Si cerca il motivo di questo risultato.

Si calcola $N$ a $T$ costante passando da somma a integrale:
\[
N = \sum_{q}^{} \overline{n}_q = \frac{4\pi V g \sqrt{2} m^{3 / 2} }{h^3}\int_{0} ^{+\infty} \frac{\mathscr{E}^{1 / 2} d\mathscr{E}}{\exp\left[ (\mathscr{E}-\mu ) / \kappa _B T \right] - 1}
\] 
Si manda $\mu \to 0$ e si prende $\mathscr{E} = \kappa _B T x$, quindi:
\[
N = \frac{4\pi V g \sqrt{2} m^{3 / 2} }{h^3} (\kappa _B T)^{3 / 2} \int_{0} ^{+\infty}  \frac{\sqrt{x} dx}{e^x - 1}
\] 
L'integrale ha soluzione generale: $\int_{0} ^{+\infty} \frac{x^n dx}{e^x - 1} = \Gamma(n+1) \zeta(n+1)$;
il risultato corrisponderebbe alle aspettative se l'integrale divergesse, mentre $\int_{0} ^{+\infty} \frac{\sqrt{x} dx}{e^x - 1} = 2.31$. 
Si definiscono, quindi, una densit\`a critica e una temperatura critica, rispettivamente, sopra cui e sotto cui sorgono problemi:
\begin{equation}
	\begin{split}
		& \left(\frac{N}{V}\right) _c = \frac{2.612}{\Lambda ^3}\\
		& T_c = \frac{1}{2.31} \left(\frac{N}{V}\right) _c^{2 / 3} \frac{h^2}{(4\pi \sqrt{2} )^{2/3} m\kappa _B }
	\end{split}
\end{equation}
L'errore \`e dovuto nel passaggio al continuo: per quanto la condizione $\kappa _B T \gg \frac{1}{2m} (h / L)^2\Rightarrow N^{2 / 3} \gg 2.31 \frac{(4\pi \sqrt{2} )^{2/3} }{2} \approx 7.8$ sia solitamente verificata, si ha, contemporaneamente al passaggio al continuo, anche $\lim_{\mathscr{E} \to 0} \rho (\mathscr{E}) =0$, quindi nei conti precedendi, si sono trascurate le particelle nello stato fondamentale.

Queste, per\`o, tendono a popolare sempre pi\`u lo stato fondamentale pi\`u si sale sopra la densit\`a critica a $T$ fissato, o si scende sotto temperatura critica a $N$ fissato.

Quello che si verifica \`e una \textbf{transizione di fase} in uno stato conosciuto come \textbf{condensato di Bose-Einstein}, con numero di particelle nello stato fondamentale dato da:
\begin{boxenv}[]
\begin{equation}
	\overline{n}_0 = \frac{1}{\exp(-\mu / \kappa _B T) - 1}
\end{equation}
\end{boxenv}
\noindent che coerentemente converge per $\mu \to 0$.

Valore di $N$ calcolato prima, in realt\`a, \`e $N^* = N - N_0 = N (T / T_c)^{3 / 2} $, da cui:
\begin{equation}
	N_0 = N \left[1 - \left(\frac{T}{T_c}\right) ^{3 / 2} \right] 
\end{equation}
\begin{osservazione}
	In due dimensioni non vi pu\`o essere condensazione perch\'e la densit\`a di stati \`e costante in energia:
	\[
	d \mathscr{E} = \frac{p dp }{m} \Rightarrow d^2 p = 2 \pi p dp = 2\pi m d\mathscr{E}
	\] 
\end{osservazione}
\noindent Per $\mu \to 0$ e per $T< T_c$, ricordando che le particelle nel condensato hanno energia nulla, l'energia media \`e:
\begin{equation}
		E = \int_{0} ^{+\infty}\frac{\rho  ( \mathscr{E}) d \mathscr{E}}{\exp(\mathscr{E} / \kappa _B T) - 1} = \frac{4\pi V g \sqrt{2} m^{3 / 2} }{h^3} (\kappa _BT)^{ 3 / 2} \kappa _B T \int_{0} ^{+\infty} \frac{x ^{3 / 2} dx}{e^x - 1}\approx 0.77 \cdot N\kappa _B T \left(\frac{T}{T_c}\right) ^{3 / 2} \propto T^{5 / 2} 
\end{equation}
da cui la capacit\`a termica \`e:
\begin{equation}
	c_V \approx 1.9 \cdot  N\kappa _B \left(\frac{T}{T_c}\right) ^{3 / 2} = 1.9 \cdot  N^* \kappa _B \propto T^{3 / 2} 
\end{equation}
Per $T$ grande, per\`o, $c_V $ deve tenere a $3N\kappa _B / 2$; l'andamento trovato sopra cambia bruscamente per $T \sim T_c$.

Infine, la pressione si ottiene a partire dalla formula valida per tutti i gas perfetti a dispersione quadratica:
\begin{equation}
	P = \frac{2}{3} \frac{E}{V} \approx 0.513 \cdot  \frac{N\kappa _B T}{V} \left(\frac{T}{T_c}\right) ^{3/2} 
\end{equation}
Essendo calcolata a $\mu (P,T) = 0$, questa identifica la curva di coesistenza tra stato gassoso non degenere e del condensato di Bose-Einstein.

\subsubsection{Oscillatori in una scatola}

L'Hamiltoniano di singolo oscillatore \`e:
\begin{equation}
	H = \frac{\mathbf{p} ^2}{2m} + \frac{m\omega ^2 \mathbf{x} ^2}{2}		
\end{equation}
L'energia media classica \`e data dal principio di equipaartizione ed \`e $E = 3N\kappa _B T$, mentre l'energia quantizzata per singola particella \`e $E_n=(n+1 / 2) \hbar \omega$.

La funzione di partizione di singola particella, quindi, \`e:
\begin{equation}
	Z_1 = \sum_{n=0}^{+\infty} \exp \left[ - \left(n + \frac{1}{2}\right) \frac{\hbar \omega}{\kappa _B T} \right] = \frac{1}{2 \operatorname{sinh}(\hbar  \omega / \kappa _B T) }
\end{equation}
Da questa, si ottiene energia media per oscillatore:
\begin{equation}
	E = \frac{1}{2}\hbar  \omega + \frac{\hbar  \omega}{\exp(hta \omega / \kappa _B T) - 1} = \left(\overline{n} + \frac{1}{2}\right) \hbar \omega
\end{equation}
Dall'ultima, si ha che gli oscillatori seguono la statistica di Bose-Einstein con $\mu =0$:
\begin{equation}
	\overline{n}(\omega) = \frac{1}{\exp(\hbar \omega / \kappa _B T) - 1}
\end{equation}
\subsubsection{Corpo nero}
Si usa gas di oscillatori come modello per campo elettromagnetico: si vede campo come un oscillatore con diversi livelli popolati secondo la statistica $\overline{n}$, o come unico livello popolato da $\overline{n}$ fotoni.

Essendo $\mu  =0 $, non esiste legge di conservazione per numero totale di quasiparticelle: se ne possono creare e distruggere a piacimento.

Inserendo gas in scsatola con pareti perfettamente assorbenti (quindi con condizione di annullamento ai bordi), la legge di dispersione \`e $\omega = ck$, quindi $\mathscr{E} = cp$.

Allora elemento di spazio delle fasi \`e:
\begin{equation}
	d\Gamma = 2V \frac{4 \pi p^2 dp}{h^3} = V \frac{\omega^2 d\omega }{\pi^2 c^3} \rho (\omega) d\omega
\end{equation}
dove il fattore $2$ \`e perch\'e esistono due modi possibili di propagazione, linearmente indipendenti fra loro, per il campo (circolare destra e circolare sinsitra), associata allo spin del fotone.

La densit\`a di energia per unit\`a di volume \`e ottenuta moltiplicando energia del singolo fotone per numero di occupazione e densit\`a (a meno di $V$), trascurando energia $\hbar  \omega / 2$ che non \`e misurabile e farebbe divergere energia totale. 
Quindi:
\begin{boxenv}[]
\begin{equation}
	u(\omega ) d\omega = \frac{\rho (\omega)}{V} \overline{n}(\omega) \hbar \omega \ d\omega= \frac{\hbar  \omega^3}{\exp(\hbar \omega / \kappa _B T ) - 1} \frac{d\omega}{\pi^2 c^3}
\end{equation}
\end{boxenv}
\noindent Questa \`e la \textbf{legge di Planck} per radiazione di corpo nero. 

Integrando sulle frequenze, si ha energia totale del corpo nero in funzione della temperatura:
\begin{equation}
	E = \int_{0} ^{+\infty} u(\omega) d\omega = \sigma T^4
\end{equation}
con $\sigma $ \textbf{costante di Stefan-Boltzmann}. Il calore specifico \`e 
\begin{equation}
	c_ V = \frac{\partial E}{\partial T} \propto T^3		
\end{equation}
e diverge con $T$  perch\'e $N$ non \`e fissato e $\omega$ non ha limite superiore, quindi si possono aggiungere quasiparticelle con energia grande a piacere. 
Il limite classico per alte $T$ non si osserva perch\'e gli oscillatori trattati sono infiniti.











\end{document}
